\documentclass[12pt,oneside]{book}

\usepackage{mybook}
\usepackage{mybookcover}


\title{新约圣经}
\author{和合本}

\begin{document}
\bookcover{book_cover.png}
\flypage{感谢上帝}

\frontmatter


\addchtoc{目录}
\setcounter{tocdepth}{2}    
\tableofcontents

\mainmatter

\chapter{马太福音第1章}

亚伯拉罕的后裔,大卫的子孙,耶稣基督的家谱。(后裔子孙原文都作儿子下同)

亚伯拉罕生以撒。以撒生雅各。雅各生犹大和他的弟兄。

犹大从他玛氏生法勒斯和谢拉。法勒斯生希斯仑。希斯仑生亚兰。

亚兰生亚米拿达。亚米拿达生拿顺。拿顺生撒门。

撒门从喇合氏生波阿斯。波阿斯从路得氏生俄备得。俄备得生耶西。

耶西生大卫王。大卫从乌利亚的妻子生所罗门。

所罗门生罗波安。罗波安生亚比雅。亚比雅生亚撒。

亚撒生约沙法。约沙法生约兰。约兰生乌西亚。

乌西亚生约坦。约坦生亚哈斯。亚哈斯生希西家。

希西家生玛拿西。玛拿西生亚们。亚们生约西亚。

百姓被迁到巴比伦的时候,约西亚生耶哥尼雅和他的弟兄。

迁到巴比伦之后,耶哥尼雅生撒拉铁。撒拉铁生所罗巴伯。

所罗巴伯生亚比玉。亚比玉生以利亚敬。以利亚敬生亚所。

亚所生撒督。撒督生亚金。亚金生以律。

以律生以利亚撒。以利亚撒生马但。马但生雅各。

雅各生约瑟,就是马利亚的丈夫。那称为基督的耶稣,是从马利亚生的。

这样,从亚伯拉罕到大卫,共有十四代。从大卫到迁至巴比伦的时候,也有十四代。从迁至巴比伦的时候到基督,又有十四代。

耶稣基督降生的事,记在下面。他母亲马利亚已经许配了约瑟,还没有迎娶,马利亚就从圣灵怀了孕。

他丈夫约瑟是个义人,不愿意明明的羞辱他,想暗暗的把他休了。

正思念这事的时候,有主的使者向他梦中显现,说大卫的子孙约瑟,不要怕,只管娶过你的妻子马利亚来。因他所怀的孕,是从圣灵来的。

他将要生一个儿子。你要给他起名叫耶稣。因他要将自己的百姓从罪恶里救出来。

这一切的事成就,是要应验主藉先知所说的话,

说,必有童女,怀孕生子,人要称他的名为以马内利。(以马内利翻出来,就是神与我们同在

约瑟醒了,起来,就遵着主使者的吩咐,把妻子娶过来。

只是没有和他同房,等他生了儿子,(有古卷作等他生了头胎的儿子)就给他起名叫耶稣。

\chapter{马太福音第2章}
当希律王的时候,耶稣生在犹太的伯利恒。有几个博士从东方来到耶路撒冷,说,

那生下来作犹太人之王的在那里。我们在东方看见他的星,特来拜他。

希律王听见了,就心里不安。耶路撒冷合城的人,也都不安。

他就召齐了祭司长和民间的文士,问他们说,基督当生在何处。

他们回答说,在犹太的伯利恒。因为有先知记着说,

犹大地的伯利恒阿,你在犹大诸城中,并不是最小的。因为将来有一位君王,要从你那里出来,牧养我以色列民。

当下希律暗暗的召了博士来,细问那星是什么时候出现的。

就差他们往伯利恒去,说,你们去仔细寻访那小孩子。寻到了,就来报信,我也好去拜他。

他们听见王的话,就去了。在东方所看见的那星,忽然在他们前头行,直行到小孩子的地方,就在上头停住了。

他们看见那星,就大大的欢喜。

进了房子,看见小孩子和他母亲马利亚,就俯伏拜那小孩子,揭开宝盒,拿黄金,乳香,没药为礼物献给他。

博士因为在梦中被主指示,不要回去见希律,就从别的路回本地去了。

他们去后,有主的使者向约瑟梦中显现,说,起来,带着小孩子同他母亲,逃往埃及,住在那里,等我吩咐你。因为希律必寻梢小孩子要除灭他。

约瑟就起来,夜间带着小孩子和他母亲往埃及去。

住在那里,直到希律死了。这是要应验主藉着先知所说的话,说,我从埃及召出我的儿子来。

希律见自己被博士愚弄,就大大发怒,差人将伯利恒城里,并四境所有的男孩,照着他向博士仔细查问的时候,凡两岁以里的,都杀尽了。

这就应了先知耶利米的话,说,

在拉玛听见号??大哭的声音,是拉结哭他儿女,不肯受安慰,因为他们都不在了。

希律死了以后,有主的使者,在埃及向约瑟梦中显现,说,

起来,带着小孩子和他母亲往以色列地去。因为要害小孩子性命的人已经死了。

约瑟就起来,把小孩子和他母亲带到以色列地去。

只因听见亚基老接着他父亲希律作了犹太王,就怕往那里去。又在梦中被主指示,便往加利利境内去了。

到了一座城,名叫拿撒勒,就住在那里。这要应验先知所说,他将为拿撒勒人的话了。

\chapter{马太福音第3章}

那时,有施洗的约翰出来,在犹太的旷野传道,说,

天国近了,你们应当悔改。

这人就是先知以赛亚所说的,他说,在旷野有人声喊着说,豫备主的道,修直他的路。

这约翰身穿骆驼毛的衣服,腰束皮带,吃的是蝗虫野蜜。

那时,耶路撒冷和犹太全地,并约旦河一带地方的人,都出去到约翰那里。

承认他们的罪,在约旦河里受他的洗。

约翰看见许多法利赛人和撒都该人,也来受洗,就对他们说,毒蛇的种类,谁指示你们逃避将来的忿怒呢。

你们要结出果子来,与悔改的心相称。

不要自己心里说,有亚伯拉罕为我们的祖宗。我告诉你们,神能从这些石头中给亚伯拉罕兴起子孙来。

现在斧子已经放在树根上,凡不结好果子的树,就砍下来,丢在火里。

我是用水给你们施洗,叫你们悔改。但那在我以后来的,能力比我更大,我就是给他提鞋,也不配。他要用圣灵与火给你们施洗。

他手里拿着簸箕,要扬净他的场,把麦子收在仓里,把糠用不灭的火烧尽了。

当下,耶稣从加利利来到约旦河,见了约翰,要受他的洗。

约翰想要拦住他,说,我当受你的洗,你反倒上我这里来吗。

耶稣回答说,你暂且许我。因为我们理当这样尽诸般的义。(或作礼)于是约翰许了他。

耶稣受了洗,随既从水里上来。天忽然为他开了,就看见神的灵,彷佛鸽子降下,落在他身上。

从天上有声音说,这是我的爱子,我所喜悦的。

\chapter{马太福音第4章}
当时,耶稣被圣灵引到旷野,受魔鬼的试探。

他禁食四十昼夜,后来就饿了。

那试探人的进前来对他说,你若是神的儿子,可以吩咐这些石头变成食物。

耶稣却回答说,经上记着说,人活着,不是单靠食物,乃是靠神口里所出的一切话。

魔鬼就带他进了圣城,叫他站在殿顶上,(顶原文作翅)

对他说,你若是神的儿子,可以跳下去。因为经上记着说,主要为你吩咐他的使者,用手托着你,免得你的脚碰在石头上。

耶稣对他说,经上又记着说,不可试探主你的神。

魔鬼又带他上了一座最高的山,将世上的万国,与万国的荣华,都指给他看,

对他说,你若俯伏拜我,我就把这一切都赐给你。

耶稣说,撒但退去吧。(撒但就是抵当的意思乃魔鬼的别名)因为经上记着说,当拜主你的神,单要事奉他。

于是魔鬼离了耶稣,有天使来伺候他。

耶稣听见约翰下了监,就退到加利利去。

后又离开拿撒勒,往迦百农去,就住在那里。那地方靠海,在西布伦和拿弗他利的边界上。

这是要应验先知以赛亚的话,

说,西布伦地,拿弗他利地,就是沿海的路,约旦河外,外邦人的加利利地。

那坐在黑暗里的百姓,看见了大光,坐在死荫之地的人,有光发现照着他们。

从那时候耶稣就传起道来,说,天国近了,你们应当悔改。

耶稣在加利利海边行走,看见弟兄二人,就是那称呼彼得的西门,和他的兄弟安得烈,在海里撒网。他们本是打鱼的。

耶稣对他们说,来跟从我,我要叫你们得人如得鱼一样。

他们就立刻舍了网,跟从了他。

从那里往前走,又看见弟兄二人,就是西庇太的儿子雅各,和他兄弟约翰,同他们的父亲西庇太在船上补网。耶稣就招呼他们。

他们立刻舍了船,别了父亲,跟从了耶稣。

耶稣走遍加利利,在各会堂里教训人,传天国的福音,医治百姓各样的病症。

他的名声就传遍了叙利亚。那里的人把一切害病的,就是害各样疾病,各样疼痛的,和被鬼附的,癫痫的,瘫痪的,都带了来,耶稣就治好了他们。

当下,有许多人从加利利,低加波利,耶路撒冷,犹太,约旦河外,来跟从他。

\chapter{马太福音第5章}
耶稣看见这许多人,就上了山,既已坐下,门徒到他跟前来。

他就开口教训他们说,

虚心的人有福了,因为天国是他们的。

哀恸的人有福了,因为他们必得安慰。

温柔的人有福了,因为他们必承受地土。

饥渴慕义的人有福了,因为他们必得饱足。

怜恤人的人有福了,因为他们必蒙怜恤。

清心的人有福了,因为他们必得见神。

使人和睦的人有福了,因为他们必称为神的儿子。

为义受逼迫的人有福了,因为天国是他们的。

人若因我辱骂你们,逼迫你们,捏造各样坏话毁谤你们,你们就有福了。

应当欢喜快乐,因为你们在天上的赏赐是大的。在你们以前的先知,人也是这样逼迫他们。

你们是世上的盐。盐若失了味,怎能叫他再咸呢。以后无用,不过丢在外面,被人践踏了。

你们是世上的光。城造在山上,是不能隐藏的。

人点灯,不放在斗底下,是放在灯台上,就照亮一家的人。

你们的光也当这样照在人前,叫他们看见你们的好行为,便将荣耀归给你们在天上的父。

弄想我来要废掉律法和先知。我来不是要废掉,乃是要成全。

我实在告诉你们,就是到天地都废去了,律法的一点一画也不能废去,都要成全。

所以无论何人废掉这诫命中最小的一条,又教训人这样作,他在天国要称为最小的。但无论何人遵行这诫命,又教训人遵行,他在天国要称为大的。

我告诉你们,你们的义,若不胜于文士和法利赛人的义,断不能进天国。

你们听见有吩咐古人的话,说,不可杀人,又说,凡杀人的,难免受审判。

只是我告诉你们,凡向弟兄动怒的,难免受审判。(有古卷在凡字下添无缘无故的五字)凡骂弟兄是拉加的,难免公会的审断。凡骂弟兄是魔利的,难免地狱的火。

所以你在祭坛上献礼物的时候,若想起弟兄向你怀怨,

就把礼物留在坛前,先去同弟兄和好,然后来献礼物。

你同告你的对头还在路上,就赶紧与他和息。恐怕他把你送给审判官,审判官交付衙役,你就下在监里了。

我实在告诉你,若有一文钱没有还清,你断不能从那里出来。

你们听见有话说,不可奸淫。

只是我告诉你们,凡看见妇女就动淫念的,这人心里已经与他犯奸淫了。

若是你的右眼叫你跌倒,就挖出来丢掉。宁可失去百体中的一体,不叫全身丢在地狱里。

若是右手叫你跌倒,就砍下来丢掉。宁可失去百体中的一体,不叫全身下地狱。

又有话说,人若休妻,就当给他休书。

只是我告诉你们,凡休妻的,若不是为淫乱的缘故,就是叫他作淫妇了。人若娶这被休的妇人,也是犯奸淫了。

你们又听见有吩咐古人的话,说,不可背誓,所起的誓,总要向主谨守。

只是我告诉你们,什么誓都不可起,不可指着天起誓,因为天是神的座位。

不可指着地起誓,因为地是他的脚凳。也不可指着耶路撒冷起誓,因为耶路撒冷是大君的京城。

又不可指着你的头起誓,因为你不能使一根头发变黑变白了。

你们的话,是,就说是,不是,就说不是。若再多说,就是出于那恶者。(或作是从恶里出来的)

你们听见有话说,以眼还眼,以牙还牙。

只是我告诉你们,不要与恶人作对。有人打你的右脸,连左脸也转过来由他打。

有人想要告你,要拿你的里衣,连外衣也由他拿去。

有人强逼你走一里路,你就同他走二里。

有求你的,就给他。有向你借货的,不可推辞。

你们听见有话说,当爱你的邻舍,恨你的仇敌。

只是我告诉你们,要爱你们的仇敌。为那逼迫你们的祷告。

这样,就可以作你们天父的儿子。因为他叫日头照好人,也照歹人,降雨给义人,也给不义的人。

你们若单爱那爱你们的人。有什么赏赐呢。就是税吏不也是这样行吗。

你们若单请你弟兄的安,比人有什么长处呢。就是外邦人不也是这样行吗。

所以你们要完全,像你们的天父完全一样。

\chapter{马太福音第6章}
你们要小心,不可将善事行在人的面前,故意叫他们看见。若是这样,就不能得你们天父的赏赐了。

所以你施舍的时候,不可在你前面吹号,像那假冒为善的人,在会堂里和街道上所行的,故意要得人的荣耀。我实在告诉你们,他们已经得了他们的赏赐。

你施舍的时候,不要叫左手知道右手所作的。

要叫你施舍的事行在暗中,你父在暗中察看,必报答你。(有古卷作必在明处报答你)

你们祷告的时候,不可像那假冒为善的人,爱站在会堂里,和十字路口上祷告,故意叫人看见。我实在告诉你们,他们已经得了他们的赏赐。

你祷告的时候,要进你的内屋,关上门,祷告你在暗中的父,你父在暗中察看,必然报答你。

你们祷告,不可像外邦人,用许多重复话。他们以为话多了必蒙垂听。

你们不可效法他们。因为你们没有祈求以先,你们所需用的,你们的父早已知道了。

所以你们祷告,要这样说,我们在天上的父,愿人都尊你的名为圣。

愿你的国降临,愿你的旨意行在地上,如同行在天上。

我们日用的饮食,今日赐给我们。

免我们的债,如同我们免了人的债。

不叫我们遇见试探,救我们脱离凶恶,(或作脱离恶者)因为国度,权柄,荣耀,全是你的,直到永远,阿们。(有古卷无因为至阿们等字)

你们饶恕人的过犯,你们的天父也必饶恕你们的过犯。

你们不饶恕人的过犯,你们的天父也必不饶恕你们的过犯。

你们禁食的时候,不可像那假冒为善的人,脸上带着愁容。因为他们把脸弄得难看,故意叫人看出他们是禁食。我实在告诉你们,他们已经得了他们的赏赐。

你们禁食的时候,要梳头洗脸,

不要叫人看出你禁食来,只叫你暗中的父看见。你父在暗中察看,必然报答你。

不要为自己积攒财宝在地上,地上有虫子咬,能锈坏,也有贼挖窟窿来偷。

只要积攒财宝在天上,天上没有虫子咬,不能锈坏,也没有贼挖窟窿来偷。

因为你的财宝在那里,你的心也在那里。

眼睛就是身上的灯。你的眼睛若了亮,全身就光明。

你的眼睛若昏花,全身就黑暗。你里头的光若黑暗了,那黑暗是何等大呢。

一个人不能事奉两个主。不是恶这个爱那个,就是重这个轻那个。你们不能又事奉神,又事奉玛门。(玛门是财利的意思)

所以我告诉你们,不要为生命忧虑,吃什么,喝什么。为身体忧虑,穿什么。生命不胜于饮食吗,身体不胜于衣裳吗。

你们看那天上的飞鸟,也不种,也不收,也不积蓄在仓里,你们的天父尚且养活他。你们不比飞鸟贵重得多吗。

你们那一个能用思虑,使寿数多加一刻呢。(或作使身量多加一肘呢)

何必为衣裳忧虑呢。你想野地里的百合花,怎样长起来,他也不劳苦,也不纺线。

然而我告诉你们,就是所罗门极荣华的时候,那他所穿戴的,还不如这花一朵呢。

你们这小信的人哪,野地的草,今天还在,明天就丢在炉里,神还给他这样的妆饰,何况你们呢。

所以不要忧虑,说,吃什么,喝什么,穿什么。

这都是外邦人所求的。你们需用的这一切东西,你们的天父是知道的。

你们要先求他的国,和他的义。这些东西都要加给你们了。

所以不要为明天忧虑。因为明天自有明天的忧虑。一天的难处一天当就够了。

\chapter{马太福音第7章}
你们不要论断人,免得你们被论断。

因为你们怎样论断人,也必怎样被论断。你们用什么量器量给人,也必用什么量器量给你们。

为什么看见你弟兄眼中有刺,却不想自己眼中有梁木呢。

你自己眼中有梁木,怎能对你弟兄说,容我去掉你眼中的刺呢。

你这假冒为善的人,先去掉自己眼中的梁木,然后才能看得清楚,去掉你弟兄眼中的刺。

不要把圣物给狗,也不要把你们的珍珠丢在猪前,恐怕他践踏了珍珠,转过来咬你们。

你们祈求,就给你们。寻梢,就寻见。叩门,就给你们开门。

因为凡祈求的,就得着。寻梢的,就寻见。叩门的,就给他开门。

你们中间,谁有儿子求饼,反给他石头呢。

求鱼,反给他蛇呢。

你们虽然不好,尚且知道拿好东西给儿女,何况你们在天上的父,岂不更把好东西给求他的人吗。

所以无论何事,你们愿意人怎样待你们,你们也要怎样待人。因为这就是律法和先知的道理。

你们要进窄门。因为引到灭亡,那门是宽的,路是大的,进去的人也多。

引到永生,那门是窄的,路是小的,找着的人也少。

你们要防备假先知。他们到你们这里来,外面披着羊皮,里面却是残暴的狼。

凭着他们的果子,就可以认出他们来。荆棘上岂能摘葡萄呢。蒺藜里岂能摘无花果呢。

这样,凡好树都结好果子,惟独坏树结坏果子。

好树不能结坏果子,坏树不能结好果子。

凡不结好果子的树,就砍下来,丢在火里。

所以凭着他们的果子,就可以认出他们来。

凡称呼我主阿,主阿的人,不能都进天国。惟独遵行我天父旨意的人,才能进去。

当那日必有许多人对我说,主阿,主阿,我们不是奉你的名传道,奉你的名赶鬼,奉你的名行许多异能吗。

我就明明的告诉他们说,我从来不认识你们,你们这些作恶的人,离开我去吧。

所以凡听见我这话就去行的,好比一个聪明人,把房子盖在磐石上。

雨淋,水冲,风吹,撞着那房子,房子总不倒塌。因为根基立在磐石上。

凡听见我这话不去行的,好比一个无知的人,把房子盖在沙土上。

雨淋,水冲,风吹,撞着那房子,房子就倒塌了。并且倒塌得很大。

耶稣讲完了这些话,众人都希奇他的教训。

因为他教训他们,正像有权柄的人,不像他们的文士。

\chapter{马太福音第8章}
耶稣下了山,有许多人跟着他。

有一个长大麻疯的来拜他说,主若肯,必能叫我洁净了。

耶稣伸手摸他说,我肯,你洁净了吧。他的大麻疯立刻就洁净了。

耶稣对他说,你切不可告诉人。只要去把身体给祭司察看,献上摩西所吩咐的礼物,对众人作证据。

耶稣进了迦百农,有一个百夫长进前来,求他说,

主阿,我的仆人害瘫痪病,躺在家里,甚是疼苦。

耶稣说,我去医治他。

百夫长回答说,主阿,你到我舍下,我不敢当。只要你说一句话,我的仆人就必好了。

因为我在人的权下,也有兵在我以下。对这个说,去,他就去。对那个说,来,他就来。对我的仆人说,你作这事,他就去作。

耶稣听见就希奇,对跟从他的人说,我实在告诉你们,这吗大的信心,就是在以色列中,我也没有遇见过。

我又告诉你们,从东从西,将有许多人来,在天国里与亚伯拉罕,以撒,雅各,一同坐席。

惟有本国的子民,竟被赶到外边黑暗里去。在那里必要哀哭切齿了。

耶稣对百夫长说,你回去吧。照你的信心,给你成全了。那时,他的仆人就好了。

耶稣到了彼得家里,见彼得的岳母害热病躺着。

耶稣把他的手一摸,热就退了。他就起来服事耶稣。

到了晚上,有人带着许多被鬼附的,来到耶稣跟前,他只用一句话,就把鬼都赶出去。并且治好了一切有病的人。

这是要应验先知以赛亚的话,说,他代替我们的软弱,担当我们的疾病。

耶稣见许多人围着他,就吩咐渡到那边去。

有一个文士来,对他说,夫子,你无论往那里去,我要跟从你。

耶稣说,狐狸有洞,天空的飞鸟有窝,人子却没有枕头的地方。

又有一个门徒对耶稣说,主阿,容我先回去埋葬我的父亲。

耶稣说,任凭死人埋葬他们的死人,你跟从我吧。

耶稣上了船,门徒跟着他。

海里忽然起了暴风,甚至船被波浪掩盖。耶稣却睡着了。

门徒来叫醒了他,说,主阿,救我们,我们丧命喇。

耶稣说,你们这小信的人哪,为什么胆怯呢。于是起来,斥责风和海,风和海就大大的平静了。

众人希奇说,这是怎样的人,连风和海也听从他了。

耶稣既渡到那边去,来到加大拉人的地方,就有两个被鬼附的人,从坟茔里出来迎着他,极其凶猛,甚至没有人能从那条路上经过。

他们喊着说,神的儿子,我们与你有什么相干。时候还没有到,你就上这里来叫我们受苦吗。

离他们很远,有一大群猪吃食。

鬼就央求耶稣说,若把我们赶出去,就打发我们进入猪群去吧。

耶稣说,去吧。鬼就出来,进入猪群。全群忽然闯下山崖,投在海里淹死了。

放猪的就逃跑进城,将这一切事,和被鬼附的人所遭遇的,都告诉人。

合城的人,都出来迎见耶稣。既见了,就央求他礼开他们的境界。

\chapter{马太福音第9章}
耶稣上了船,渡过海,来到自己的城里。

有人用褥子抬着一个瘫子,到耶稣跟前来。耶稣见他们的信心,就对瘫子说,小子,放心吧。你的罪赦了。

有几个文士心里说,这个人说僭妄的话了。

耶稣知道他们的心意,就说,你为什么心里怀着恶念呢。

或说,你的罪赦了。或说,你起来行走。那一样容易呢。

但要叫你们知道人子在地上有赦罪的权柄,就对瘫子说,起来,拿你的褥子回家去吧。

那人就起来,回家去了。

众人看见都惊奇,就归荣耀与神。因为他将这样的权柄赐给人。

耶稣从那里往前走,看见一个人名叫马太,坐在税关上,就对他说,你跟从我来。他就起来跟从了耶稣。

耶稣在屋里坐席的时候,有好些税吏和罪人来,与耶稣和他的门徒一同坐席。

法利赛人看见,就对耶稣的门徒说,你们的先生为什么和税吏并罪人一同吃饭呢。

耶稣听见,就说,康健的人用不着医生,有病的人才用得着。

经上说,我喜爱怜恤,不喜爱祭祀。这句话的意思,你们且去揣摩。我来,本不是召义人,乃是召罪人。

那时,约翰的门徒来见耶稣说,我们和法利赛人常常禁食,你的门徒倒不禁食,这是为什么呢。

耶稣对他们说,新郎和陪伴之人同在的时候,陪伴之人岂能哀恸呢。但日子将到,新郎要离开他们,那时候他们就要禁食。

没有人把新布补在旧衣服上。因为所补上的,反带坏了那衣服,破的就更大了。

也没人把新酒装在旧皮袋里。若是这样,皮袋就裂开,酒漏出来,连皮袋也坏了。惟独把新酒装在新皮袋里,两样就都保全了。

耶稣说这话的时候,有一个管会堂的来拜他说,我女儿刚才死了,求你去按手在他身上,他就必活了。

耶稣便起来,跟着他去,门徒也跟了去。

有一个女人,患了十二年的血漏来到耶稣背后,摸他的衣裳??子。

因为他心里说,我只摸他的衣裳,就必痊愈。

耶稣转过来看见他,就说,女儿,放心,你的信救了你。从那时候,女人就痊愈了。

耶稣到了管会堂的家里,看见有吹手,又有许多人乱囔。

就说,退去吧。这闺女不是死了,是睡着了。他们就嗤笑他。

众人既被撵出,耶稣就进去,拉着闺女的手,闺女便起来了。

于是这风声传遍了那地方。

耶稣从那里往前走,有两个瞎子跟着他,喊叫说,大卫的子孙,可怜我们吧。

耶稣进了房子,瞎子就来到他跟前,耶稣说,你们信我能作这事吗。他们说,主阿,我们信。

耶稣就摸他们的眼睛,说,照着你们的信给你们成全了吧。

他们的眼睛就开了。耶稣切切的嘱咐他们说,你们要小心,不可叫人知道。

他们出去,竟把他的名声传遍了那地方。

他们出去的时候,有人将鬼所附的一个哑吧,带到耶稣跟前来。

鬼被赶出去,哑吧就说出话来。众人都希奇说,在以色列中,从来没有见过这样的事。

法利赛人却说,他是靠着鬼王赶鬼。

耶稣走遍各城各乡,在会堂里教训人,宣讲天国的福音,又医治各样的病症。

他看见许多的人,就怜悯他们。因为他们困苦流离,如同羊没有牧人一般。

于是对门徒说,要收的庄稼多,作工的人少。

所以你们当求庄稼的主,打发工人出去,收他的庄稼。

\chapter{马太福音第10章}
耶稣叫了十二个门徒来,给他们权柄,能赶逐污鬼,并医治各样的病症。

这十二使徒的名,头一个叫西门,又称彼得,还有他兄弟安得烈。西庇太的儿子雅各,和雅各的兄弟约翰。

腓力,和巴多罗买,多马,和税吏马太,亚勒腓的儿子雅各,和达太。

奋锐党的西门,还有卖耶稣的加略人犹大。

耶稣差这十二个人去,吩咐他们说,外邦人的路,你们不要走。撒玛利亚人的城,你们不要进。

宁可往以色列家迷失的羊那里去。

随走随传,说,天国近了。

医治病人,叫死人复活,叫长大麻疯的洁净,把鬼赶出去。你们白白的得来,也要白白的舍去。

腰袋里,不要带金银铜钱。

行路不要带口袋,不要带两件褂子,也不要带鞋和拐杖。因为工人得饮食,是应当的。

你们无论进那一城,那一村,要打听那里谁是好人,就住在他家,直住到走的时候。

进他家里去,要请他的安。

那家若配得平安,你们所求的平安,就必临到那家。若不配得,你们所求的平安仍归你们。

凡不接待你们,不听你们的话的人,你们离开那家,或是那城的时候,就把脚上的尘土跺下去。

我实在告诉你们,当审判的日子,所多玛和蛾摩拉所受的,比那城还容易受呢。

我差你们去,如同羊进入狼群。所以你们要灵巧像蛇,驯良像鸽子。

你们要防备人。因为他们要把你们交给公会,也要在会堂鞭打你们。

并且你们要为我的缘故,被送到诸侯君王面前,对他们和外邦人作见证。

你们被交的时候,不要思虑怎样说话或说什么话。到那时候,必赐给你们当说的话。

因为不是你们自己说的,乃是你们父的灵在你们里头说的。

弟兄要把弟兄,父亲要把儿子,送到死地。儿女要与父母为敌,害死他们。

并且你们要为我的名,被众人恨恶,惟有忍耐到底的,必然得救。

有人在这城逼迫你们,就逃到那城去。我实在告诉你们,以色列的城邑,你们还没有走遍,人子就到了。

学生不能高过先生,仆人不能高过主人。

学生和先生一样,仆人和主人一样,也就吧了。人既骂家主是别西卜,何况他的家人呢。(别西卜是鬼王的名)

所以不要怕他们。因为掩盖的事,没有不露出来的。隐藏的事,没有不被人知道的。

我在暗中告诉你们的,你们要在明处说出来。你们耳中所听的,要在房上宣扬出来。

那杀身体不能杀灵魂的,不要怕他们。惟有能把身体和灵魂都灭在地狱里的,正要怕他。

两个麻雀,不是卖一分银子吗。若是你们的父不许,一个也不能掉在地上。

就是你们的头发,也都被数过了。

所以不要惧怕。你们比许多麻雀还贵重。

凡在人面前认我的,我在我天上的父面前,也必认他。

凡在人面前不认我的,我在我天上的父面前,也必不认他。

你们不要想我来,是叫地上太平。我来并不是叫地上太平,乃是叫地上动刀兵。

因为我来,是叫人与父亲生疏,女儿与母亲生疏,媳妇与婆婆生疏。

人的仇敌,就是自己家里的人。

爱父母过于爱我的,不配作我的门徒,爱儿女过于爱我的,不配作我的门徒。

不背着他的十字架跟从我的,也不配作我的门徒。

得着生命的,将要失丧生命。为我失丧生命的,将要得着生命。

人接待你们,就是接待我。接待我,就是接待那差我来的。

人因为先知的名接待先知,必得先知所得的赏赐,人因为义人的名接待义人,必得义人所得的赏赐。

无论何人,因为门徒的名,只把一杯凉水给这小子里的一个喝,我实在告诉你们,这人不能不得赏赐。

\chapter{马太福音第11章}
耶稣吩咐完了十二个门徒,就离开那里,往各城去传道教训人。

约翰在监里听见基督所作的事,就打发两个门徒去,

问他说,那将要来的是你吗,还是我们等候别人呢。

耶稣回答说,你们去,把所听见所看见的事告诉约翰。

就是瞎子看见,瘸子行走,长大麻疯的洁净,聋子听见。死人复活,穷人有福音传给他们。

凡不因我跌倒的,就有福了。

他们走的时候,耶稣就对众人讲论约翰说,你们从前出到旷野,是要看什么呢,要看风吹动的芦苇吗。

你们出去,到底是要看什么,要看穿细软衣服的人吗,那穿细软衣服的人,是在王宫里。

你们出去,究竟是为什么,是要看先知吗。我告诉你们,是的,他比先知大多了。

经上记着说,我要差遣我的使者在你面前,豫备道路。所说的就是这个人。

我实在告诉你们,凡妇人所生的,没有一个兴起来大过施洗约翰的。然而天国里最小的,比他还大。

从施洗约翰的时候到如今,天国是努力进入的,努力的人就得着了。

因为众先知和律法说预言,到约翰为止。

你们若肯领受,这人就是应当来的以利亚。

有耳可听的,就应当听。

我可用什么比这世代呢。好像孩童坐在街市上,招呼同伴,说,

我们向你们吹笛,你们不跳舞。我们向你们举哀,你们不捶胸。

约翰来了,也不吃,也不喝,人就说他是被鬼附着的。

人子来了,也吃,也喝,人又说他是贪食好酒的人,是税吏罪人的朋友。但智慧之子,总以智慧为是。(有古卷作但智慧在行为上就显为是)

耶稣在诸城中行了许多异能,那些城的人终不悔改,就在那时候责备他们说,

哥拉汛哪,你有祸了,伯赛大阿,你有祸了,因为在你们中间所行的异能,若行在推罗西顿,他们早已披麻蒙灰悔改了。

但我告诉你们,当审判的日子,推罗西顿所受的,比你们还容易呢。

迦百农阿,你已经升到天上。(或作你将要升到天上吗)将来必坠落阴间。因为在你那里所行的异能,若行在所多玛,他还可以存到今日。

但我告诉你们,当审判的日子,所多玛所受的,比你还容易呢。

那时,耶稣说,父阿,天地的主,我感谢你,因为你将这些事,向聪明通达人,就藏起来,向婴孩,就显出来。

父阿,是的,因为你的美意本是如此。

一切所有的,都是我父交付我的。除了父,没有人知道子。除了子和子所愿意指示的,没有人知道父。

凡劳苦担重担的人,可以到我这里来,我就使你们得安息。

我心里柔和谦卑,你们当负我的轭,学我的样式,这样,你们心里就必得享安息。

因为我的轭是容易的,我的担子是轻省的。

\chapter{马太福音第12章}
那时,耶稣在安息日,从麦地经过。他的门徒饿了,就掐起麦穗来吃。

法利赛人看见,就对耶稣说,看哪,你的门徒作安息日不可作的事了。

耶稣对他们说,经上记着大卫和跟从他的人肌饿之时所作的事,你们没有念过吗。

他怎吗进了神的殿,吃了陈设饼,这饼不是他和跟从他的人可以吃的,惟独祭司才可以吃。

再者,律法上所记的,当安息日,祭司在殿里犯了安息日,还是没有罪,你们没有念过吗。

但我告诉你们,在这里有一人比殿更大。

我喜爱怜恤,不喜爱祭祀。你们若明白这话的意思,就不将无罪的,当作有罪的了。

因为人子是安息日的主。

耶稣离开那地方,进了一个会堂。

那里有一个人,枯乾了一只手。有人问耶稣说,安息日治病,可以不可以。意思是要控告他。

耶稣说,你们中间谁有一只羊,当安息日掉在坑里,不把他抓住拉上来呢。

人比羊何等贵重呢。所以在安息日作善事是可以的。

于是对那人说,伸出手来。他把手一伸,手就复了原,和那只手一样。

法利赛人出去,商议怎样可以除灭耶稣。

耶稣知道了,就离开那里,有许多人跟着他,他把其中有病的人都治好了。

又嘱咐他们,不要给他传名。

这是要应验先知以赛亚的话,说,

看哪,我的仆人,我所拣选,所亲爱,心里所喜悦的,我要将我的灵赐给他,他必将公理传给外邦。

他不争竞,不喧囔。街上也没有人听见他的声音。

压伤的芦苇,他不折断。经残的灯火,他不吹灭。等他施行公理,叫公理得胜。

外邦人都要仰望他的名。

当下有人将一个被鬼附着,又瞎又哑的人,带到耶稣那里。耶稣就医治他,甚至那哑吧又能说话,又能看见。

众人都惊奇,说,这不是大卫的子孙吗。

但法利赛人听见,就说,这个人赶鬼,无非是靠着鬼王别西卜阿。

耶稣知道他们的意念,就对他们说,凡一国自相分争,就成为荒场,一城一家自相分争,必站立不住。

若撒但赶逐撒但,就是自相分争,他的国怎能站得住呢。

我若靠着别西卜赶鬼,你们的子弟赶鬼,又靠着谁呢。这样,他们就要断定你们的是非。

我若靠着神的灵赶鬼,这就是神的国临到你们了。

人怎能进壮士家里,抢夺他的家具呢,除非先捆住那壮士,才可以抢夺他的家财。

不与我相合的,就是敌我的,不同我收聚的,就是分散的。

所以我告诉你们,人一切的罪,和亵渎的话,都可得赦免。惟独亵渎圣灵,总不得赦免。

凡说话干犯人子的,还可得赦免。惟独说话干犯圣灵的,今世来世总不得赦免。

你们或以为树好,果子也好。树坏,果子也坏。因为看果子,就可以知道树。

毒蛇的种类,你们既是恶人,怎能说出好话来呢。因为心里所充满的,口里就说出来。

善人从他心里所存的善,就发出善来。恶人从他心里所存的恶,就发出恶来。

我又告诉你们,凡人所说的闲话,当审判的日子,必要句句供出来。

因为要凭你的话定你为义,也要凭你的话,定你有罪。

当时有几个文士和法利赛人,对耶稣说,夫子,我们愿意你显个神迹给我们看。

耶稣回答说,一个邪恶淫乱的世代求看神迹。除了先知约拿的神迹以外,再没有神迹给他们看。

约拿三日三夜在大鱼肚腹中,人子也要这样三日三夜在地里头。

当审判的时候,尼尼微人,要起来定这世代的罪,因为尼尼微人听了约拿所传的,就悔改了。看哪,在这里有一人比约拿更大。

当审判的时候,南方的女王,要起来定这世代的罪,因为他从地极而来,要听所罗门的智慧话。看哪,在这里有一人比所罗门更大。

污鬼离了人身,就在无水之地,过来过去,寻求安歇之处,却寻不着。

于是说,我要回到我所出来的屋里去。到了,就看见里面空闲,打扫乾净,修饰好了。

便去另带了七个比自己更恶的鬼来,都进去住在那里。那人末后的景况,比先前更不好了。这邪恶的世代,也要如此。

耶稣还对众人说话的时候,不料,他母亲和他弟兄站在外边,要与他说话。

有人告诉他说,看哪,你母亲和你弟兄站在外边,要与你说话。

他却回答那人说,谁是我的母亲。谁是我的弟兄。

就伸手指着门徒说,看哪,我的母亲,我的弟兄。

凡遵行我天父旨意的人,就是我的弟兄姐妹和母亲了。

\chapter{马太福音第13章}
当那一天,耶稣从房子里出来,坐在海边。

有许多人到他那里聚集,他只得上船坐下。众人都站在岸上。

他用比喻对他们讲许多道里,说,有一个撒种的出去撒种。

撒种的时候,有落在路旁的,飞鸟来吃尽了。

有落在土浅石头地上的。土既不深,发苗最快。

日头出来一晒,因为没有根,就枯乾了。

有落在荆棘里的。荆棘长起来,把他济住了。

又有落在好土里的,就结实,有一百倍的,有六十倍的,有三十倍的。

有耳可听的,就应当听。

门徒进前来,问耶稣说,对众人讲话,为什么用比喻呢。

耶稣回答说,因为天国的奥秘,只叫你们知道,不叫他们知道。

凡有的,还要加给他,叫他有馀。凡没有的,连他所有的,也要夺去。

所以我用比喻对他们讲,是因他们看也看不见,听也听不见,也不明白。

在他们身上,正应了以赛亚的预言说,你们听是要听见,却不明白。看是要看见,却不晓得。

因为这百姓油蒙了心,耳朵发沉,眼睛闭着。恐怕眼睛看见,耳朵听见,心里明白,回转过来,我就医治他们。

但你们的眼睛是有福的,因为看见了。你们的耳朵也是有福的,因为听见了。

我实在告诉你们,从前有许多先知和义人,要看你们所看的,却没有看见。要听你们所听见的,却没有听见。

所以你们当听这撒种的比喻。

凡听见天国道里不明白的,那恶者就来,把所撒在他心里的夺去了。这就是撒在路旁的了。

撒在石头地上的,就是人听了道,当下欢喜领受。

只因心里没有根,不过是暂时的。及至为道遭了患难,或是受了逼迫,立刻就倒了。

撒在荆棘里的,就是人听了道,后来有世上的思虑,钱财的迷惑,把道挤住了,不能结实。

撒在好土地上的,就是人听了道明白了,后来结实,有一百倍的,有六十倍的,有三十倍的。

耶稣又设个比喻对他们说,天国好像人撒好种在田里。

及至人睡觉的时候,有仇敌来,将稗子撒在麦子里,就走了。

到长苗吐穗的时候,稗子也显出来。

田主的仆人来告诉他说,主阿,你不是撒好种在田里吗。从那里来的稗子呢。

主人说,这是仇敌作的。仆人说,你要我们去薅出来吗。

主人说,不必,恐怕薅稗子,连麦子也拔出来。

这两样一齐长,等着收割。当收割的时候,我要对收割的人说,先将稗子薅出来,捆成捆,留着烧。惟有麦子,要收在仓里。

他又设个比喻对他们说,天国好像一粒芥菜种,有人拿去种在田里。

这原是百种里最小的。等到长起来,却比各样的菜都大,且成了树。天上的飞鸟来宿在他的枝上。

他又对他们讲个比喻说,天国好像面酵,有妇人拿来,藏在三斗面里,直等全团都发起来。

这都是耶稣用比喻对众人说的话。若不用比喻,就不对他们说什么。

这是要应验先知的话说,我要开口用比喻,把创世以来所隐藏的事发明出来。

当下耶稣离开众人,进了房子。他的门徒进前来说,请把田间稗子的比喻,讲给我们听。

他回答说,那撒好种的,就是人子。

田地,就是世界。好种,就是天国之子。稗子,就是那恶者之子。

撒稗子的仇敌,就是魔鬼。收割的时候,就是世界的末了。收割的人,就是天使。

将稗子薅出来,用火焚烧。世界的末了,也要如此。

人子要差遣使者,把一切叫人跌倒的,和作恶的,从他国里挑出来,

丢在火炉里。在那里必要哀哭切齿了。

那时义人在他们父的国里,要发出光来,像太阳一样。有耳可听的,就应当听。

天国好像宝贝藏在地里。人遇见了,就把他藏起来。欢欢喜喜的去变卖一切所有的买这块地。

天国又好像买卖人,寻梢好珠子。

遇见一颗重价的珠子,就去变卖他一切所有的,买了这颗珠子。

天国又好像网撒在海里,聚拢各样水族。

网既满了,人就拉上岸来。坐下,拣好的收在器具里,将不好的丢弃了。

世界的末了,也要这样。天使要出来,从义人中,把恶人分别出来,

丢在火炉里。在那里必要哀哭切齿了。

耶稣说,这一切的话,你们都明白了吗。他们说,我们明白了,

他说,凡文士受教作天国的门徒,就像一个家主,从他库里拿出新旧的东西来。

耶稣说完了这些比喻,就离开那里,

来到自己的家乡,在会堂里教训人,甚至他们都希奇说,这人从那里有这等智慧,和异能呢。

这不是木匠的儿子吗。他母亲不是叫马利亚吗。他弟兄们不是叫雅各,约西,(有古卷作约瑟),西门,犹大吗。

他妹妹们不是都在我们这里吗。这人从那里有这一切的事呢。

他们就厌弃他。(厌弃他原文作因他跌倒)耶稣对他们说,大凡先知,除了本地本家之外,没有不被人尊敬的。

耶稣因为他们不信,就在那里不多行异能了。

\chapter{马太福音第14章}
那时分封的王希律,听见耶稣的名声,

就对臣仆说,这是施洗约翰从死里复活,所以这些异能从他里面发出来。

起先希律为他兄弟腓力的妻子希罗底的缘故,把约翰拿住锁在监里。

因为约翰曾对他说,你娶这妇人是不合理的。

希律就想要杀他,只是怕百姓。因为他们以约翰为先知。

到了希律的生日,希罗底的女儿,在众人面前跳舞,使希律欢喜。

希律就起誓,应许随他所求的给他。

女儿被母亲所使,就说,请把施洗约翰的头,放在盘子里拿来给我。

王便忧愁,但因他所起的誓,又因同席的人,就吩咐给他。

于是打发人去,在监里斩了约翰。

把头放在盘子里,拿来给了女子。女子拿去给他母亲。

约翰的门徒来,把尸首领去,埋葬了。就去告诉耶稣。

耶稣听见了,就上船从那里独自退到野地里去。众人听见,就从各城里步行跟随他。

耶稣出来,见有许多的人,就怜悯他们,治好了他们的病人。

天将晚的时候,门徒进前来说,这是野地,时候已经过了。请叫众人散开,他们好往村子里去,自己买吃的。

耶稣说,不用他们去,你们给他们吃吧。

门徒说,我们这里只有五个饼,两条鱼。

耶稣说,拿过来给我。

于是吩咐众人坐在草地上。就拿着这五个饼,两条鱼,望着天,祝福,擘开饼,递给门徒。门徒又递给众人。

他们都吃,并且吃饱了。把剩下的零碎收拾起来,装满了十二个篮子。

吃的人,除了妇女孩子,约有五千。

耶稣随既催门徒上船,先渡到那边去,等他叫众人散开。

散了众人以后,他就独自上山祷告。到了晚上,只有他一人在那里。

那时船在海中,因风不顺,被浪摇撼。

夜里四更天,耶稣在海面上走,往门徒那里去。

门徒看见他在海面上走,就惊慌了,说,是个鬼怪。便害怕,喊叫起来。

耶稣连忙对他们说,你们放心。是我,不要怕。

彼得说,主,如果是你,请叫我从水面上走到你那里去。

耶稣说,你来吧。彼得就从船上下去,在水面上走,要到耶稣那里去。

只因见风甚大,就害怕。将要沉下去,便喊着说,主阿,救我。

耶稣赶紧伸手拉住他,说,你这小信的人哪,为什么疑惑呢。

他们上了船,风就住了。

在船上的人都拜他说,你真是神的儿子了。

他们过了海,来到革尼撒勒地方。

那里的人,一认出是耶稣,就打发人到周围地方去,把所有的病人,带到他那里。

只求耶稣准他们摸他的衣裳??子,摸着的人,就都好了。

\chapter{马太福音第15章}
那时有法利赛人和文士,从耶路撒冷来见耶稣说,

你的门徒为什么犯古人的遗传呢。因为吃饭的时候,他们不洗手。

耶稣回答说,你们为什么因着你们的遗传,犯神的诫命呢。

神说,当孝敬父母。又说,咒骂父母的,必治死他。

你们倒说,无论何人对父母说,我所当奉给你的,已经作了供献。

他就可以不孝敬父母。这就是你们藉着遗传,废了神的诫命。

假冒为善的人哪,以赛亚指着你们说的预言,是不错的。他说,

这百姓用嘴唇尊敬我,心却远离我。

他们将人的吩咐,当作道理教导人,所以拜我也是枉然。

耶稣就叫了众人来,对他们说,你们要听,也要明白。

入口的不能污秽人,出口的乃能污秽人。

当时,门徒进前来对他说,法利赛人听见这话,不服,你知道吗。(不服原文作跌倒)

耶稣回答说,凡栽种的物,若不是我父栽种的,必要拔出来。

任凭他们吧。他们是瞎眼领路的。若是瞎子领瞎子,两个人都要掉在坑里。

彼得对耶稣说,请将这比喻讲给我们听。

耶稣说,你们到如今还不明白吗。

岂不知凡入口的,是运到肚子里,又落在茅厕里吗。

惟独出口的,是从心里发出来的,这才污秽人。

因为从心里发出来的,有恶念,凶杀,奸淫,苟合,偷盗,妄证,谤??。

这都是污秽人的。至于不洗手吃饭,那却不污秽人。

耶稣离开那里,退到推罗西顿的境内去。

有一个迦南妇人,从那地方出来,喊着说,主啊,大卫的子孙,可怜我。我女儿被鬼附得甚苦。

耶稣一言不答。门徒进前来,求他说,这妇人在我们后头喊叫。请打发他走吧。

耶稣说,我奉差遣,不过是到以色列家迷失的羊那里去。

那妇人来拜他,说,主啊,帮助我。

他回答说,不好拿儿女的饼,丢给狗吃。

妇人说,主啊,不错。但是狗也吃它主人桌子上掉下来的啐渣儿。

耶稣说,妇人,你的信心是大的。照你所要的,给你成全了吧。从那时候,他女儿就好了。

耶稣离开那地方,来到靠近加利利的海边,就上山坐下。

有许多人到他那里,带着瘸子,瞎子,哑吧,有残疾的,和好些别的病人,都放在他脚前。他就治好了他们。

甚至众人都希奇。因为看见哑吧说话,残疾的痊愈,瘸子行走,瞎子看见,他们就归荣耀给以色列的神。

耶稣叫门徒来说,我怜悯这众人,因为他们同我在这里已经三天,也没有吃的了。我不愿意叫他们饿着回去,恐怕在路上困乏。

门徒说,我们在野地,那里有这吗多的饼,叫这许多人吃饱呢。

耶稣说,你们有多少饼。他们说,有七个,还有几条小鱼。

他就吩咐众人坐在地上。

拿着这七个饼和几条鱼,祝谢了,擘开,递给门徒。门徒又递给众人。

众人都吃并且吃饱了。收拾剩下的零碎,装满了七个筐子。

吃的人,除了妇女孩子,共有四千。

耶稣叫众人散去,就上船,来到马加丹的境界。

\chapter{马太福音第16章}
法利赛人和撒都该人,来试探耶稣,请他从天上显个神迹给他们看。

耶稣回答说,晚上天发红,你们就说,天必要晴。

早晨天发红,又发黑,你们就说,今日必有风雨。你们知道分辨天上的气色,倒不能分辨这时候的神迹。

一个邪恶淫乱的世代求神迹,除了约拿的神迹以外,再没有神迹给他看。耶稣就离开他们去。

门徒渡到那边去,忘了带饼。

耶稣对他们说,你们要谨慎,防备法利赛人和撒都该人的酵。

门徒彼此议论说,这是因我们没有带饼吧。

耶稣看出来,就说,你们这小信的人,为什么因为没有饼彼此议论呢。

你们还不明白吗,不记得那五个饼,分给五千人,又收拾了多少篮子的零碎吗。

也不记得那七个饼,分给四千人,又收拾了多少筐子的零碎吗。

我对你们说,要防备法利赛人和撒都该人的酵,这话不是指着饼说的。你们怎吗不明白呢。

门徒这才晓得他说的,不是叫他们防备饼的酵,乃是防备法利赛人和撒都该人的教训。

耶稣到了凯撒利亚腓力比的境内,就问门徒说,人说我人子是谁。(有古卷无我字)

他们说,有人说是施洗的约翰。有人说是以利亚。又有人说是耶利米,或是先知里的一位。

耶稣说,你们说我是谁。

西门彼得回答说,你是基督,是永生神的儿子。

耶稣对他说,西门巴约拿,你是有福的。因为这不是属血肉的指示你的,乃是我在天上的父指示的。

我还告诉你,你是彼得,我要把我的教会建造在这磐石上,阴间的权柄,不能胜过他。(权柄原文作门)

我要把天国的钥匙给你。凡你在地上所捆绑的,在天上也要捆绑。凡你在地上所释放的,在天上也要释放。

当下,耶稣嘱咐门徒,不可对人说他是基督。

从此耶稣才指示门徒,他必须上耶路撒冷去,受长老祭司长文士许多的苦,并且被杀,第三日复活。

彼得就拉着他说,主阿,万不可如此,这事必不临到你身上。

耶稣转过来,对彼得说,撒但退我后边去吧。你是绊我脚的。因为你不体贴神的意思,只体贴人的意思。

于是耶稣对门徒说,若有人要跟从我,就当舍己,背起他的十字架,来跟从我。

因为凡要救自己生命的,(生命或作灵魂下同)必丧掉生命。凡为我丧掉生命的,必得着生命。

人若赚得全世界,赔上自己的生命,有什么益处呢。人还能拿什么换生命呢。

人子要在他父的荣耀里,同着众使者降临。那时候,他要照各人的行为报应各人。

我实在告诉你们,站在这里的人,有人在没尝死味以前,必看见人子降临在他的国里。

\chapter{马太福音第17章}
过了六天,耶稣带着彼得,雅各,和雅各的兄弟约翰,暗暗的上了高山。

就在他们面前变了形像。脸面明亮如日头,衣裳洁白如光。

忽然有摩西,以利亚,向他们显现,同耶稣说话。

彼得对耶稣说,主阿,我们在这里真好。你若愿意,我就在这里搭三座棚,一座为你,一座为摩西,一座为以利亚。

说话之间,忽然有一朵光明的云彩遮盖他们。且有声音从云彩里出来说,这是我的爱子,我所喜悦的。你们要听他。

门徒听见,就俯伏在地,极其害怕。

耶稣进前来,摸他们说,起来,不要害怕。

他们举目不见一人,只见耶稣在那里。

下山的时候,耶稣吩咐他们说,人子还没有从死里复活,你们不要将所看见的告诉人。

门徒问耶稣说,文士为什么说,以利亚必须先来。

耶稣回答说,以利亚固然先来,并要复兴万事。

只是我告诉你们,以利亚已经来了,人却不认识他,竟任意待他。人子也将要这样受他们的害。

门徒这才明白耶稣所说的,是指着施洗的约翰。

耶稣和门徒到了众人那里,有一个人来见耶稣,跪下,说,

主阿,怜悯我的儿子。他害癫痫的病很苦,屡次跌在火里,屡次跌在水里。

我带他到你门徒那里,他们却不能医治他。

耶稣说,嗳,这又不信又悖谬的世代阿,我在你们这里要到几时呢。我忍耐你们要到几时呢。把他带到我这里来吧。

耶稣斥责那鬼,鬼就出来。从此孩子就痊愈了。

门徒暗暗的到耶稣跟前说,我们为什么不能赶出那鬼呢。

耶稣说,是因你们的信心小。我实在告诉你们,你们若有信心像一粒芥菜种,就是对这座山说,你从这边挪到那边,他也必挪去。并且你们没有一件不能作的事了。

至于这一类的鬼,若不祷告禁食,他就不出来。(或作不能赶他出来)

他们还住在加利利的时候,耶稣对门徒说,人子将要被交在人手里。

他们要杀害他,第三日他要复活。门徒就大大的忧愁。

到了迦百农,有收丁税的人来见彼得说,你们的先生不纳丁税吗。(丁税约有半块钱)

彼得说,纳。他进了屋子,耶稣先向他说,西门,你的意思如何。世上的君王,向谁徵收关税丁税。是向自己的儿子呢,是向外人呢。

彼得说,是向外人。耶稣说,既然如此,儿子就可以免税了。

但恐怕触犯他们,(触犯原文作绊倒)你且往海边去钓鱼,把先钓上来的鱼拿起来,开了他的口,必得一块钱,可以拿去给他们,作你我的税银。

\chapter{马太福音第18章}
当时门徒进前来,问耶稣说,天国里谁是最大的。

耶稣便叫一个小孩子来,使他站在他们当中,

说,我实在告诉你们,你们若不回转,变成小孩子的样式,断不得进天国。

所以凡自己谦卑像这小孩子的,他在天国里就是最大的。

凡为我的名,接待一个像这小孩子的,就是接待我。

凡使这信我的一个小子跌倒的,倒不如把大磨石拴在这人的颈项上,沉在深海里。

这世界有祸了,因为将人绊倒。绊倒人的事是免不了的,但那绊倒人的有祸了。

倘若你一只手,或是一只脚,叫你跌倒,就砍下来丢掉。你缺一只手,或是一只脚,进入永生,强如有两手两脚,被丢在永火里。

倘若你一只眼叫你跌倒,就把他挖出来丢掉。你只有一只眼进入永生,强如有两只眼被丢在地狱的火里。

你们要小心,不可轻看这小子里的一个。我告诉你们,他们的使者在天上,常见我天父的面。(有古卷在此有

人子来为要拯救失丧的人)

一个人若有一百只羊,一只走迷了路,你们的意思如何。他岂不撇下这九十九只,往山里去找那只迷路的羊吗。

若是找着了,我实在告诉你们,他为这一只羊欢喜,比为那没有迷路的九十九只欢喜还大呢。

你们在天上的父,也是这样不愿意这小子里失丧一个。

倘若你的弟兄得罪你,你就去趁着只有他和你在一处的时候,指出他的错来。他若听你,你便得了你的弟兄。

他若不听,你就另外带一两个人同去,要凭两三个人的口作见证,句句都可定准。

若是不听他们,就告诉教会。若是不听教会,就看他像外邦人和税吏一样。

我实在告诉你们,凡你们在地上所捆绑的,在天上也要捆绑。凡你们在地上所释放的,在天上也要释放。

我又告诉你们,若是你们中间有两个人在地上,同心合意的求什么事,我在天上的父,必为他们成全。

因为无论在那里,有两三个人奉我的名聚会,那里有我在他们中间。

那时彼得进前来,对耶稣说,主阿,我弟兄得罪我,我当饶恕他几次呢。七次可以吗。

耶稣说,我对你说,不是到七次,乃是到七十个七次。

天国好像一个王,要和他仆人算账。

才算的时候,有人带了一个欠一千万银子的来。

因为他没有偿还之物,主人吩咐把他和他妻子儿女,并一切所有的都卖了偿还。

那仆人就俯伏拜他说,主阿,宽容我,将来我都要还清。

那仆人的主人,就动了慈心,把他释放了,并且免了他的债。

那仆人出来,遇见他的一个同伴,欠他十两银子,便揪着他,掐住他的喉咙,说,你把所欠的还我。

他的同伴就俯伏央求他,说,宽容我吧,将来我必还清。

他不肯,竟去把他下在监里,等他还了所欠的债。

众同伴看见他所作的事,就甚忧愁,去把这事都告诉了主人。

于是主人叫了他来,对他说,你这恶奴才,你央求我,我就把你所欠的都免了。

你不应当怜恤你的同伴像我怜恤你吗。

主人就大怒,把他交给掌刑的,等他还清了所欠的债。

你们各人若不从心里饶恕你的弟兄,我天父也要这样待你们了。

\chapter{马太福音第19章}
耶稣说完了这些话,就离开加利利,来到犹太的境界,约旦河外。

有许多人跟着他。他就在那里把他们的病人治好了。

有法利赛人来试探耶稣说,人无论什么缘故,都可以休妻吗。

耶稣回答说,那起初造人的,是造男造女,

并且说,因此,人要离开父母,与妻子连合,二人成为一体。这经你们没有念过吗。

既然如此,夫妻不再是两个人,乃是一体的了。所以神所配合的,人不可分开。

法利赛人说,这样,摩西为什么吩咐给妻子休书,就可以休他呢。

耶稣说,摩西因为你们的心硬,所以许你们休妻。但起初并不是这样。

我告诉你们,凡休妻另娶的,若不是为淫乱的缘故,就是犯奸淫了,有人娶那被休的妇人,也是犯奸淫了。

门徒对耶稣说,人和妻子既是这样,倒不如不娶。

耶稣说,这话不是人都能领受的。惟独赐给谁,谁才能领受。

因为有生来是阉人,也有被人阉的,并有为天国的缘故自阉的。这话谁能领受,就可以领受。

那时有人带着小孩子来见耶稣,要耶稣给他们按手祷告。门徒就责备那些人。

耶稣说,让小孩子到我这里来,不要禁止他们。因为在天国的,正是这样的人。

耶稣给他们按手,就离开那地方去了。

有一个人来见耶稣说,夫子,(有古卷作良善的夫子)我该作什么善事,才能得永生。

耶稣对他说,你为什么以善事问我呢,只有一位是善的,(有古卷作你为什么称我是良善的,除了神以外,没有一个良善的)你若要进入永生,就当遵守诫命。

他说,什么诫命。耶稣说,就是不可杀人,不可奸淫,不可偷盗,不可作假见证,

当孝敬父母。又当爱人如己。

那少年人说,这一切我都遵守了。还缺少什么呢。

耶稣说,你若愿意作完全人,可以去变卖你所有的,分给穷人,就必有财宝在天上,你还要来跟从我。

那少年人听见这话,就忧忧愁愁的走了。因为他的产业很多。

耶稣对门徒说,我实在告诉你们,财主进天国是难的。

我又告诉你们,骆驼穿过针的眼,比财主进神的国还容易呢。

门徒听见这话,就希奇得很,说,这样谁能得救呢。

耶稣看着他们说,在人这是不能的。在神凡事都能。

彼得就对他说,看哪,我们已经撇下所有的跟从你,将来我们要得什么呢。

耶稣说,我实在告诉你们,你们这跟从我的人,到复兴的时候,人子坐在他荣耀的宝座上,你们也要坐在十二宝座上,审判以色列十二个支派。

凡为我的名撇下房屋,或是弟兄,姐妹,父亲,母亲,(有古卷添妻子),儿女,田地的,必要得着百倍,并且承受永生。

然而有许多在前的将要在后,在后的将要在前。

\chapter{马太福音第20章}
因为天国好像家主,清早去雇人,进他的葡萄园作工。

和工人讲定一天一银子,就打发他们进葡萄园去。

约在巳初出去,看见市上还有闲站的人。

就对他们说,你们也进葡萄园去,所当给的,我必给你们。他们也进去了。

约在午正和申初又出去,也是这样行。

约在酉初出去,看见还有人站在那里。就问他们说,你们为什么整天在这里闲站呢。

他们说,因为没有人雇我们。他说,你们也进葡萄园去。

到了晚上,园主对管事的说,叫工人都来,给他们工钱,从后来的起,到先来的为止。

约在酉初雇来的人来了,各人得了一钱银子。

及至那先雇的来了,他们以为必要多得。谁知也是各得一钱。

他们得了,就埋怨家主说,

我们整天劳苦受热,那后来的只做了一小时,你竟叫他们和我们一样吗。

家主回答其中的一人说,朋友,我不亏负你。你与我讲定的,不是一钱银子吗。

拿你的走吧。我给那后来的和给你一样,这是我愿意的。

我的东西难道不可随我的意思用吗。因为我作好人,你就红了眼吗。

这样,那在后的将要在前,在前的将要在后了。(有古卷在此有因为被召的人多,选上的人少)

耶稣上耶路撒冷去的时候,在路上把十二个门徒带到一边对他们说,

看哪,我们上耶路撒冷去,人子要被交给祭司长和文士。他们要定他死罪。

又交给外邦人,将他戏弄,鞭打,钉在十字架上。第三日他要复活。

那时,西庇太儿子的母亲,同他两个儿子前来,拜耶稣求他一件事。

耶稣说,你要什么呢。他说,愿你叫我这两个儿子在你国里,一个坐在你右边,一个坐在你左边。

耶稣回答说,你们不知道所求的是什么。我将要喝的杯,你们能喝吗。他们说,我们能。

耶稣说,我所喝的杯,你们必要喝。只是坐在我的左右,不是我可以赐的,乃是我父为谁豫备的,就赐给谁。

那十个门徒听见,就恼怒他们弟兄二人。

耶稣叫了他们来,说,你们知道外邦人有君王为主治理他们,有大臣操权管束他们。

只是在你们中间不可这样。你们中间谁愿为大,就必作你们的用人。

谁愿为首,就必作你们的仆人。

正如人子来,不是要受人的服事,乃是要服事人。并且要舍命,作多人的赎价。

他们出耶利哥的时候,有极多的人跟随他,

有两个瞎子坐在路旁,听说是耶稣经过,就喊着说,主阿,大卫的子孙,可怜我们吧。

众人责备他们,不许他们作声。他们却越发喊着说,主阿,大卫的子孙,可怜我们吧。

耶稣就站住,叫他们来,说,要我为你们作什么。

他们说,主阿,要我们的眼睛能看见。

耶稣就动了慈心,把他们的眼睛一摸,他们立刻看见,就跟从了耶稣。

\chapter{马太福音第21章}
耶稣和门徒将近耶路撒冷,到了伯法其在橄榄山那里。

耶稣就打发两个门徒,对他们说,你们往对面村子里去,必看见一匹驴拴在那里,还有驴驹同在一处。你们解开牵到我这里来。

若有人对你们说什么,你们就说,主要用他。那人必立时让你们牵来。

这事成就,是要应验先知的话,说,

要对锡安的居民(原文作女子)说,看哪,你的王来到你这里,是温柔的,又骑着驴,就是骑着驴驹子。

门徒就照耶稣所吩咐的去行,

牵了驴和驴驹来,把自己的衣服搭在上面,耶稣就骑上。

众人多半把衣服铺在路上。还有人砍下树枝来铺在路上。

前行后随的众人,喊着说,和散那归于大卫的子孙,(和散那原有求救的意思,在此乃称颂的话)奉主名来的,是应当称颂的。高高在上和散那。

耶稣既进了耶路撒冷,合城都惊动了,说,这是谁。

众人说,这是加利利拿撒勒的先知耶稣。

耶稣进了神的殿,赶出殿里一切作买卖的人,推倒兑换银钱之人的桌子,和卖鸽子之人的凳子。

对他们说,经上记着说,我的殿必称为祷告的殿。你们倒使他成为贼窝了。

在殿里有瞎子瘸子,到耶稣跟前。他就治好了他们。

祭司长和文士,看见耶稣所行的奇事,又见小孩子在殿里喊着说,和散那归于大卫的子孙。就甚恼怒,

对他说,这些人所说的,你听见了吗。耶稣说,是的经上说,你从婴孩和吃奶的口中,完全了赞美的话。你们没有念过吗。

于是离开他们出城到伯大尼去,在那里住宿。

早晨回城的时候,他饿了。

看见路旁有一棵无花果树,就走到跟前,在树上找不着什么,不过有叶子。就对树说,从今以后,你永不结果子。那无花果树就立刻枯乾了。

门徒看见了,便希奇说,无花果树怎吗立刻枯乾了呢。

耶稣回答说,我实在告诉你们,你们若有信心,不疑惑,不但能行无花果树上所行的事,就是对这座山说,你挪开此地,投在海里,也必成就。

你们祷告,无论求什么,只要信,就必得着。

耶稣进了殿,正教训人的时候,祭司长和民间的长老来问他说,你仗着什么权柄作这些事。给你这权柄的是谁呢。

耶稣回答说,我也要问你们一句话,你们若告诉我,我就告诉你们我仗着什么权柄作这些事。

约翰的洗礼是从那里来的。是从天上来的,是从人间来的呢。他们彼此商议说,我们若说从天上来,他必对我们说,这样,你们为什么不信他呢。

若说从人间来,我们又怕百姓。因为他们都以约翰为先知。

于是回答耶稣说,我们不知道。耶稣说,我也不告诉你们我仗着什么权柄作这些事。

又说,一个人有两个儿子,他来对大儿子说,我儿,你今天到葡萄园里去作工。

他回答说,我不去。以后自己懊悔就去了。

又来对小儿子也是这样说,他回答说,父阿,我去。他却不去。

你们想这两个儿子,是那一个遵行父命呢。他们说,大儿子。耶稣说,我实在告诉你们,税吏和娼妓,倒比你们先进神的国。

因为约翰遵着义路到你们这里来,你们却不信他。税吏和娼妓倒信他。你们看见了,后来还是不懊悔去信他。

你们再听一个比喻。有个家主,栽了一个葡萄园,周围圈上篱笆,里面挖了一个压酒池,盖了一座楼,租给园户,就往外国去了。

收果子的时候近了,就打发仆人,到园户那里去收果子。

园户拿住仆人。打了一个,杀了一个,用石头打死一个。

主人又打发别的仆人去,比先前更多。园户还是照样待他们。

后来打发他的儿子到他们那里去,意思说,他们必尊敬我的儿子。

不料,园户看见他儿子,就彼此说,这是承受产业的。来吧,我们杀他,占他的产业。

他们就拿住他,推出葡萄园外杀了。

园主来的时候,要怎样处治这些园户呢。

他们说,要下毒手除灭那些恶人,将葡萄园另租给那按着时候交果子的园户。

耶稣说,经上写着,匠人所弃的石头,已作了房角的头块石头。这是主所作的,在我们眼中看为希奇。这经你们没有念过吗。

所以我告诉你们,神的国,必从你们夺去。赐给那能结果子的百姓。

谁掉在这石头上,必要跌碎。这石头掉在谁的身上,就要把谁咂得稀烂。

祭司长和法利赛人,听见他的比喻,就看出他是指着他们说的。

他们想要捉拿他,只是怕众人,因为众人以他为先知。

\chapter{马太福音第22章}
耶稣又用比喻对他们说,

天国好比一个王,为他儿子摆设娶亲的筵席。

就打发仆人去请那些被召的人来赴席。他们却不肯来。

王又打发别的仆人说,你们告诉那被召的人,我的筵席已经豫备好了,牛和肥畜已经宰了,各样都齐备。请你们来赴席。

那些人不理就走了。一个到自己田里去。一个作买卖去。

其馀的拿住仆人,凌辱他们,把他们杀了。

王就大怒,发兵除灭那些凶手,烧毁他们的城。

于是对仆人说,喜筵已经齐备,只是所召的人不配。

所以你们要往岔路口上去,凡遇见的,都召来赴席。

那些仆人就出去到大路上,凡遇见的,不论善恶都召聚了来。筵席上坐满了客。

王进来观看宾客,见那里有一个没有穿礼服的。

就对他说,朋友,你到这里来,怎吗不穿礼服呢。那人无言可答。

于是王对使唤的人说,捆起他的手脚来,把他丢在外边的黑暗里。在那里必要哀哭切齿了。

因为被召的人多,选上的人少。

当时,法利赛人出去商议,怎样就着耶稣的话陷害他。

就打发他们的门徒,同希律党的人,去见耶稣说,夫子,我们知道你是诚实人,并且诚诚实实传神的道,什么人你都不徇情面,因为你不看人的外貌。

请告诉我们,你的意见如何。纳税给凯撒,可以不可以。

耶稣看出他们的恶意,就说,假冒为善的人哪,为什么试探我。

拿一个上税的钱给我看。他们就拿一个银钱来给他。

耶稣说,这像和这号是谁的。

他们说,是凯撒的。耶稣说,这样,凯撒的物当归给凯撒,神的物当归给神。

他们听见就希奇,离开他走了。

撒都该人常说没有复活的事。那天,他们来问耶稣说,

夫子,摩西说,人若死了,没有孩子,他兄弟当娶他的妻,为哥哥生子立后。

从前在我们这里,有弟兄七人。第一个娶了妻,死了,没有孩子,撇下妻子给兄弟。

第二第三直到第七个,都是如此。

末后,妇人也死了。

这样当复活的时候,他是七个人中,那一个的妻子呢。因为他们都娶过他。

耶稣回答说,你们错了。因为不明白圣经,也不晓得神的大能。

当复活的时候,人也不娶也不嫁,乃像天上的使者一样。

论到死人复活,神在经上向你们所说的,你们没有念过吗。

他说,我是亚伯拉罕的神,以撒的神,雅各的神。神不是死人的神,乃是活人的神。

众人听见这话,就希奇他的教训。

法利赛人听见耶稣堵住了撒都该人的口,他们就聚集。

内中有一个人是律法师,要试探耶稣,就问他说,

夫子,律法上的诫命,那一条是最大的呢。

耶稣对他说,你要尽心,尽性,尽意,爱主你的神。

这是诫命中的第一,且是最大的。

其次也相仿,就是要爱人如己。

这两条诫命,是律法和先知一切道里的总纲。

法利赛人聚集的时候,耶稣问他们说,

论到基督,你们的意见如何。他是谁的子孙呢。他们回答说,是大卫的子孙。

耶稣说,这样,大卫被圣灵感动,怎吗还称他为主。说,

主对我主说,你坐在我的右边,等我把你仇敌,放在你的脚下。

大卫既称他为主,他怎吗又是大卫的子孙呢。

他们没有一个人能回答一言。从那日以后,也没有人敢再问他什么。

\chapter{马太福音第23章}
那时,耶稣对众人和门徒讲论,

说,文士和法利赛人,坐在摩西的位上。

凡他们所吩咐你们的,你们都要谨守,遵行。但不要效法他们的行为。因为他们能说不能行。

他们把难担的重担,捆起来搁在人的肩上。但自己一个指头也不肯动。

他们一切所作的事,都是要叫人看见。所以将佩戴的经文做宽了,衣裳的??子做长了。

喜爱筵席上的首座,会堂里的高位。

又喜爱人在街市上问他安,称呼他拉比。(拉比就是夫子)

但你们不要受拉比的称呼。因为只有一位是你们的夫子。你们都是弟兄。

也不要称呼地上的人为父。因为只有一位是你们的父,就是在天上的父。

也不要受师尊的称呼。因为只有一位是你们的师尊,就是基督。

你们中间谁为大,谁就要作你们的用人。

凡自高的必降为卑,自卑的必升为高。

你们这假冒为善的文士和法利赛人有祸了。因为你们正当人前把天国的门关了。自己不进去,正要进去的人,你们也不容他们进去。(有古卷在此有

你们这假冒为善的文士和法利赛人有祸了,因为你们侵吞寡妇的家产,假意作很长的祷告,你们要受更重的刑罚)

你们这假冒为善的文士法利赛人有祸了。因为你们走遍洋海陆地,勾引一个人入教。既入了教,却使他作地狱之子,比你们还加倍。

你们这瞎眼领路的有祸了。你们说,凡指着殿起誓的,这算不得什么。只是凡指着殿中金子起誓的,他就该谨守。

你们这无知瞎眼的人哪,什么是大的,是金子呢,还是叫金子成圣的殿呢。

你们又说,凡指着坛起誓的,这算不得什么。只是指着坛上礼物起誓的,他就该谨守。

你们这瞎眼的人哪,什么是大的,是礼物呢,还是叫礼物成圣的坛呢。

所以人指着坛起誓,就是指着坛和坛上一切所有的起誓。

人指着殿起誓,就是指着殿和住在殿里的起誓。

人指着天起誓,就是指着神的宝座和那坐在上面的起誓。

你们这假冒为善的文士和法利赛人有祸了。因为你们将薄荷,茴香,芹菜,太23:23)献上十分之一。那律法上更重的事,就是公义,怜悯,信实,反倒不行了。这更重的是你们当行的;那也是不可不行的。

你们这瞎眼领路的,蠓虫你们就滤出来,骆驼你们倒吞下去。

你们这假冒为善的文士和法利赛人有祸了。因为你们洗净杯盘的外面,里面却盛满了勒索和放荡。

你这瞎眼的法利赛人,先洗净杯盘的里面,好叫外面也乾净了。

你们这假冒为善的文士和法利赛人有祸了。因为你们好像粉饰的坟墓,外面好看,里面却装满了死人的骨头,和一切的污秽。

你们也是如此,在人前,外面显出公义来,里面却装满了假善和不法的事。

你们这假冒为善的文士和法利赛人有祸了。因为你们建造先知的坟,修饰义人的墓,说,

若是我们在我们祖宗的时候,必不和他们同流先知的血。

这就是你们自己证明,是杀害先知者的子孙了。

你们去充满你们祖宗的恶贯吧。

你们这些蛇类,毒蛇之种阿,怎能逃脱地狱的刑罚呢。

所以我差谴先知和智慧人并文士,到你们这里来。有的你们要杀害,要钉十字架。有的你们要在会堂里鞭打,从这城追逼到那城。

叫世上所流义人的血,都归到你们身上。从义人亚伯拉罕的血起,直到你们在殿和坛中间所杀的巴拉加的儿子撒迦利亚的血为止。

我实在告诉你们,这一切的罪,都要归到这世代了。

耶路撒冷阿,耶路撒冷阿,你常杀害先知,又用石头打死那奉差遣到你这里来的人。我多次愿意聚集你的儿女,好像母鸡把小鸡聚集在翅膀底下,只是你们不愿意。

看哪,你们的家成为荒场,留给你们。

我告诉你们,从今以后,你们不得再见我,直等到你们说,奉主名来的,是应当称颂的。

\chapter{马太福音第24章}
耶稣出了圣殿,正走的时候,门徒进前来,把殿宇指给他看。

耶稣对他们说,你们不是看见这殿宇吗。我实在告诉你们,将来在这里,没有一块石头留在石头上不被拆毁的。

耶稣在橄榄山上坐着,门徒暗暗的来说,请告诉我们,什么时候有这些事。你降临和世界的末了,有什么豫兆呢。

耶稣回答说,你们要谨慎,免得有人迷惑你们。

因为将来有好些人冒我的名来,说,我是基督,并且要迷惑许多人。

你们也要听见打仗和打仗的风声,总不要惊慌。因为这些事是必须有的。只是末期还没有到。

民要攻打民,国要攻打国。多处必有饥荒,地震。

这都是灾难的起头。(灾难原文作生产之难)。

那时,人要把你们陷在患难里,也要杀害你们。你们又要为我的名,被万民恨恶。

那时,必有许多人跌倒,也要彼此陷害,彼此恨恶。

且有好些假先知起来,迷惑多人。

只因不法的事增多,许多人的爱心,才渐渐冷淡了。

惟有忍耐到底的,必然得救。

这天国的福音,要传遍天下,对万民作见证,然后末期才来到。

你们看见先知但以理所说的,那行毁坏可憎的站在圣地。(读这经的人须要会意)。

那时,在犹太的,应当逃到山上。

在房上的,不要下来拿家里的东西。

在田里的,也不要回去取衣裳。

当那些日子,怀孕的和奶孩子的有祸了。

你们应当祈求,叫你们逃走的时候,不遇见冬天,或是安息日。

因为那时必有灾难,从世界的起头,直到如今,没有这样的灾难,后来也必没有。

若不减少那日子,凡有血气的,总没有一个得救的。只是为选民,那日子必减少了。

那时若有人对你们说,基督在这里。或说,基督在那里,你们不要信。

因为假基督,假先知,将要起来,显大神迹,大奇事。倘若能行,连选民也就迷惑了。

看哪,我豫先告诉你们了。

若有人对你们说,看哪,基督在旷野里。你们不要出去。或说。基督在内屋中。你们不要信。

闪电从东边发出,直照到西边。人子降临,也要这样。

尸首在那里,鹰也必聚在那里。

那些日子的灾难一过去,日头就变黑了,月亮也不放光,众星要从天上坠落,天势都要震动。

那时,人子的兆头要显在天上,地上的万族都要哀哭。他们要看见人子,有能力,有大荣耀,驾着天上的云降临。

他要差遣使者,用号筒的大声,将他的选民,从四方,从天这边到天那边,都招聚了来(方原文作风)。

你们可以从无花果树学个比方。当树枝发嫩长叶的时候,你们就知道夏天近了。

这样,你们看见这一切的事,也该知道人子近了,正在门口了。

我实在告诉你们,这世代还没有过去,这些事都要成就。

天地要废去,我的话却不能废去。

但那日子,那时辰,没有人知道,连天上的使者也不知道,子也不知道,惟独父知道。

挪亚的日子怎样,人子降临也要怎样。

当洪水以前的日子,人照常吃喝嫁娶,直到挪亚进方舟的那日。

不知不觉洪水来了,把他们全部冲去。人子降临也要这样。

那时,两个人在田里,取去一个,撇下一个。

两个女人推磨。取去一个,撇下一个。

所以你们要儆醒,因为不知道你们的主是那一天来到。

家主若知道几更天有贼来,就必儆醒,不容人挖透房屋。这是你们所知道的。

所以你们也要豫备。因为你们想不到的时候,人子就来了。

谁是忠心有见识的仆人,为主人所派,管理家里的人,按时分粮给他们呢。

主人来到,看见他这样行,那仆人就有福了。

我实在告诉你们,主人要派他管里一切所有的。

倘若那恶仆心里说,我的主人必来得迟,

就动手打他的同伴,又和酒醉的人一同吃喝。

在想不到的日子,不知道的时辰,那仆人的主人要来,

重重的处治他,(或作把他腰斩了)定他和假冒为善的人同罪。在那里必要哀哭切齿了。

\chapter{马太福音第25章}
那时,天国好比十个童女,拿着灯,出去迎接新郎。

其中有五个是愚拙的。五个是聪明的。

愚拙的拿着灯,却不豫备油。

聪明的拿着灯,又豫备油在器皿里。

新郎迟延的时候,他们都打盹睡着了。

半夜有人喊着说,新郎来了,你们出来迎接他。

那些童女就都起来收拾灯。

愚拙的对聪明的说,请分点油给我们。因为我们的灯要灭了。

聪明的回答说,恐怕不够你我用的。不如你们自己到卖油的那里去买吧。

他们去买的时候,新郎到了。那豫备好了的,同他进去坐席。门就关了。

其馀的童女,随后也来了,说,主阿,主阿,给我们开门。

他却回答说,我实在告诉你们,我不认识你们。

所以你们要儆醒,因为那日子,那时辰,你们不知道。

天国又好比一个人要往外国去,就叫了仆人来,把他的家业交给他们。

按着各人的才干,给他们银子。一个给了五千,一个给了二千,一个给了一千。就往外国去了。

那领五千的,随既拿去做买卖,另外赚了五千。

那领二千的,也照样另赚了二千。

但那领一千的,去掘开地,把主人的银子埋藏了。

过了许久,那些仆人的主人来了,和他们算账。

那领五千银子的,又带着那另外的五千来,说,主阿,你交给我五千银子,请看,我又赚了五千。

主人说,好,你这又良善又忠心的仆人。你在不多的事上有忠心,我把许多事派你管理。可以进来享受你主人的快乐。

那领二千的也来说,主阿,你交给我二千银子,请看,我又赚了二千。

主人说,好,你这又良善又忠心的仆人。你在不多的事上有忠心,我把许多事派你管理。可以进来享受你主人的快乐。

那领一千的,也来说,主阿,我知道你是忍心的人,没有种的地方要收割,没有散的地方要聚敛。

我就害怕,去把你的一千银子埋藏在地里。请看,你的原银在这里。

主人回答说,你这又恶又懒的仆人,你既知道我没有种的地方要收割,没有散的地方要聚敛。

就当把我的银子放给兑换银钱的人,到我来的时候,可以连本带利收回。

夺过他这一千来,给那有一万的。

因为凡有的,还要加给他,叫他有馀。没有的,连他所有的,也要夺过来。

把这无用的仆人,丢在外面黑暗里。在那里必要哀哭切齿了。

当人子在他荣耀里同着众天使降临的时候,要坐在他荣耀的宝座上。

万民都要聚集在他面前。他要把他们分别出来,好像牧羊的分别绵羊山羊一般。

把绵羊安置在右边,山羊在左边。

于是王要向那右边的说,你们这蒙我父赐福的,可来承受那创世以来为你们所预备的国。

因为我饿了,你们给我吃。渴了,你们给我喝。我作客旅,你们留我住。

我赤身露体,你们给我穿。我病了,你们看顾我。我在监里,你们来看我。

义人就回答说,主阿,我们什么时候见你饿了给你吃,渴了给你喝。

什么时候见你作客旅留你住,或是赤身露体给你穿。

又什么时候见你病了,或是在监里,来看你呢。

王要回答说,我实在告诉你们,这些事你们既作在我这弟兄中一个最小的身上,就是作在我身上了。

王又要向那左边的说,你们这被咒诅的人,离开我,进入那为魔鬼和他的使者所豫备的永火里去。

因为我饿了,你们不给我吃。渴了,你们不给我喝。

我作客旅,你们不留我住。我赤身露体,你们不给我穿。我病了,我在监里,你们不来看顾我。

他们也要回答说,主阿,我们什么时候见你饿了,或渴了,或作客旅,或赤身露体,或病了,或在监里,不伺候你呢。

王要回答说,我实在告诉你们,这些事你们既不作在我这弟兄中一个最小的身上,就是不作在我身上了。

这些人要往永刑里去。那些义人要往永生里去。

\chapter{马太福音第26章}
耶稣说完了这一切的话,就对门徒说,

你们知道过两天是逾越节,人子将要被交给人,钉在十字架上。

那时,祭司长和民间的长老-聚集在大祭司称为该亚法的院里。

大家商议,要用诡计拿住耶稣杀他。

只是说,当节的日子不可,恐怕民间生乱。

耶稣在伯大尼长大麻疯的西门家里,

有一个女人,拿着一玉瓶极贵的香膏来,趁耶稣坐席的时候,浇在他的头上。

门徒看见,就很不喜悦,说,何用这样的枉费呢。

这香膏可以卖许多钱,周济穷人。

耶稣看出他们的意思,就说,为什么难为这女人呢。他在我身上作的,是一件美事。

因为常有穷人你们同在。只是你们不常有我。

他将这香膏浇在我身上,是为我安葬作的。

我实在告诉你们,普天之下,无论在什么地方传这福音,也要述说这女人所行的,作个记念。

当下,十二门徒里,有一个称为加略人犹大的,去见祭司长说,

我把他交给你们,你们愿意给我多少钱。他们就给了他三十块钱。

从那时候,他就找机会,要把耶稣交给他们。

除酵节的第一天,门徒来问耶稣说,你吃逾越节的筵席,要我们在那里给你豫备。

耶稣说,你们进城去,到某人那里,对他说,夫子说,我的时候快到了。我与门徒要在你家守逾越节

门徒遵着耶稣所吩咐的就去豫备了逾越节的筵席

到了晚上,耶稣十二个门徒坐席。

正吃的时候,耶稣说,我实在告诉你们,你们中间有一个人要卖我了。

他们就甚忧愁,一个一个的问他说,主,是我吗。

耶稣回答说,同我蘸手在盘子里的,就是他要卖我。

人子必要去世,正如经上指着他所写的,但卖人子的人有祸了。那人不生在世上倒好。

卖耶稣的犹大问他说,拉比,是我吗。耶稣说,你说的是。

他们吃的时候,耶稣拿起饼来,祝福,就擘开,递给门徒,说,你们拿着吃,这是我的身体。

又拿起杯来,祝谢了,递给他们,说,你们都喝这个。

因为这是我立约的血,为多人流出来,使罪得赦。

但我告诉你们,从今以候,我不再喝这葡萄汁,直到我在我父的国里,同你们喝新的那日子。

他们唱了诗,就出来往橄榄山去。

那时,耶稣对他们说,今夜你们为我的缘故,都要跌倒。因为经上记着说,我要击打牧人,羊就分散了。

但我复活以后,要在你们以先往加利利去。

彼得说,众人虽然为你的缘故跌倒,我却永不跌倒。

耶稣说,我实在告诉你,今夜鸡叫以先,你要三次不认我。

彼得说,我就是必须和你同死,也总不能不认你。众门徒都是这样说。

耶稣同门徒来到一个地方,名叫客西马尼,就对他们说,你们坐在这里,太26:36)等我到那边去祷告。

于是带着彼得,和西庇太的两个儿子同去,就忧愁起来,极其难过。

便对他们说,我心里甚是忧伤,几乎要死。你们在这里等候,和我一同儆醒。

他就稍往前走,俯伏在地祷告说,我父阿,倘若可行,求你叫这杯离开我。然而不要照我的意思,只要照你的意思。

来到门徒那里,见他们睡着了,就对彼得说,怎吗样,你们不能同我儆醒片时吗。

总要儆醒祷告,免得入了迷惑。你们心灵固然愿意,肉体却软弱了。

第二次又去祷告说,我父阿,这杯若不能离开我,必要我喝,就愿你的旨意成全。

又来见他们睡着了,因为他们的眼睛困倦。

耶稣又离开他们去了。第三次祷告,说的话还是与先前一样。

于是来到门徒那里,对他们说,现在你们仍然睡觉安歇吧。(吧或作吗)时候到了,人子被卖在罪人手里了。

起来,我们走吧。看哪,卖我的人近了。

说话之间,那十二个门徒里的犹大来了,并有许多人,带着刀棒,从祭司长和民间的长老那里,与他同来。

那卖耶稣的,给了他们一个暗号,说,我与谁亲嘴,谁就是他。你们可以拿住他。

犹大随既到耶稣跟前说,请拉比安。就与他亲嘴。

耶稣对他说,朋友,你来要作的事,就作吧。于是那些人上前,下手拿住耶稣。

有跟随耶稣的一个人,伸手拔出刀来,将大祭司的仆人砍了一刀,削掉了他一个耳朵。

耶稣对他说,收刀入鞘吧。凡动刀的,必死在刀下。

你想我不能求我父,现在为我差遣十二营多天使来吗。

若是这样,经上所说,事情必须如此的话,怎吗应验呢。

当时,耶稣对众人说,你们带着刀棒,出来拿我,如同拿强盗吗。我天天坐在殿里教训人,你们并没有拿我。

但这一切的事成就了,为要应验先知书上的话。当下门徒都离开他逃走了。

拿耶稣的人,把他带到大祭司该亚法那里去。文士和长老,已经在那里聚会。

彼得远远的跟着耶稣,直到大祭司的院子,进到里面,就和差役同坐,要看这事到底怎样。

祭司长和全公会,寻梢假见证,控告耶稣,要治死他。

虽有好些人来作假见证,总得不着实据。末后有两个人前来说,

这个人曾说,我能拆毁神的殿,三日内又建造起来。

大祭司就站起来,对耶稣说,你什么都不回答吗。这些人作见证告你的是什么呢。

耶稣却不言语。大祭司对他说,我指着永生神,叫你起誓告诉我们,你是神的儿子基督不是。

耶稣对他说,你说的是。然而我告诉你们,后来你们要看见人子,坐在那权能者的右边,驾着天上的云降临。

大祭司就撕开衣服说,他说了僭妄的话,我们何必再用见证人呢。这僭妄的话,现在你们都听见了。

你们的意见如何。他们回答说,他是该死的。

他们就吐唾沫在他脸上,用拳头打他。也有用手掌打他的,说,

基督阿,你是先知,告诉我们打你的是谁。

彼得在外面院子里坐着,有一个使女前来说,你素来也是同那加利利人耶稣一夥的。

彼得在众人面前却不承认,说,我不知道你说的是什么。

既出去,到了门口,又有一个使女看见他,就对那里的人说,这个人也是同拿撒勒人耶稣一夥的。

彼得又不承认,并且起誓说,我不认得那个人。

过了不多的时候,旁边站着的人前来,对彼得说,你真是他们一党的。你的口音把你露出来了。

彼得发咒起誓的说,我不认得那个人。立时鸡就叫了。

彼得想起耶稣所说的话,鸡叫以先,你要三次不认我。他就出去痛哭。

\chapter{马太福音第27章}
到了早晨,众祭司长和民间的长老,大家商议,要治死耶稣。

就把他捆绑解去交给巡抚彼拉多。

这时候,卖耶稣的犹大,看见耶稣已经定了罪,就后悔,把那三十块钱,拿回来给祭司长和长老说,

我卖了无辜之人的血,是有罪了。他们说,他们说,那与我们有什么相干。你自己承当吧。

犹大就把那银钱丢在殿里,出去吊死了。

祭司长拾起银钱来说,这是血价,不可放在库里。

他们商议,就用那银钱买了窑户的一块田,为要埋葬外乡人。

所以那块田,直到今日还叫作血田。

这就应了先知耶利米的话,说,他们用那三十块钱,就是被估定之人的价钱,是以色列人中所估定的,

买了窑户的一块田。这是照着主所吩咐我的。

耶稣站在巡抚面前,巡抚问他说,你是犹太人的王吗。耶稣说,你说的是。

他被祭司长和长老控告的时候什么都不回答。

彼拉多就对他说,他们作见证,告你这吗多的事,你没有听见吗。

耶稣仍不回答,连一句话也不说,以致巡抚甚觉希奇。

巡抚有一个常例,每逢这节期,随众人所要的,释放一个囚犯给他们。

当时,有一个出名的囚犯叫巴拉巴。

众人聚集的时候,彼拉多就对他们说,你们要我释放那一个给你们。是巴拉巴呢。是称为基督的耶稣呢。

巡抚原知道,他们是因为嫉妒才把他解了来。

正坐堂的时候,他的夫人打发人来说,这义人的事,你一点不可管。因为我今天在梦中,为他受了许多的苦。

祭司长和长老,挑唆众人,求释放巴拉巴,除灭耶稣。

巡抚对众人说,这两个人,你们要我释放那一个给你们呢。他们说,巴拉巴。

彼拉多说,这样,那称为基督的耶稣,我怎吗办他呢。他们说,把他钉十字架。

巡抚说,为什么呢,他作了什么恶事呢。他们便极力的喊着说,把他钉十字架。

彼拉多见说也无济于事,反要生乱,就拿水在众人面前洗手,说,流这义人的血,罪不在我,你们承当吧。

众人都回答说,他的血归到我们,和我们的子孙身上。

于是彼拉多释放巴拉巴给他们,把耶稣鞭打了,交给人钉十字架。

巡抚的兵就把耶稣带进衙门,叫全营的兵都聚集在他那里。

他们给他脱了衣服,穿上一件朱红色袍子。

用荆棘编作冠冕,戴在他头上,拿一根苇子放在他右手里。跪在他面前戏弄他说,恭喜犹太人的王阿。

又吐唾沫在他脸上,拿苇子打他的头。

戏弄完了,就给他脱了袍子,仍穿上他自己的衣服,带他出去,要钉十字架。

他们出来的时候,遇见一个古利奈人,名叫西门,就勉强他同去,好背着耶稣的十字架。

到了一个地方,名叫各各他,意思就是髑髅地。

兵丁拿苦胆调和的酒,给耶稣喝。他尝了,就不肯喝。

他们既将他钉在十字架上,就拈阄分他的衣服。

又坐在那里看守他。

在他头以上,安一个牌子,写着他的罪状,说,这是犹太人的王耶稣。

当时,有两个强盗,和他同钉十字架,一个在右边,一个在左边。

从那里经过的人,讥诮他,摇着头说,

你这拆毁圣殿,三日又建造起来的,可以救自己吧。你如果是神的儿子,就从十字架上下来吧。

祭司长和文士并长老,也是这样戏弄他,说,

他救了别人,不能救自己。他是以色列的王,现在可以从十字架上下来,我们就信他。

他倚靠神,神若喜悦他,现在可以救他。因为他曾说,我是神的儿子。

那和他同钉的强盗,也是这样的讥诮他。

从午正到申初,遍地都黑暗了。

约在申初,耶稣大声喊着说,以利,以利,拉马撒巴各大尼。就是说,我的神,我的神,为什么离弃我。

站在那里的人,有的听见就说,这个人呼叫以利亚呢。

内中有一个人,赶紧跑去,拿海绒蘸满了醋,绑在苇子上,送给他喝。

其馀的人说,且等着,看以利亚来救他不来。

耶稣又大声喊叫,气就断了。

忽然殿里的幔子,从上到下裂为两半。地也震动。盘石也崩裂。

坟墓也开了。已睡圣徒的身体,多有起来的。

到耶稣复活以后,他们从坟墓里出来,进了圣城,向许多人显现。

百夫长和一同看守耶稣的人,看见地震,并所经历的事,就极其害怕,说,这真是神的儿子了。

有好些妇女在那里远远的观看。他们是从加利利跟随耶稣来服事他的。

内中有抹大拉的马利亚,又有雅各和约西的母亲马利亚,并有西庇太两个儿子的母亲。

到了晚上,有一个财主,名叫约瑟,是亚利马太来的。他也是耶稣的门徒。

这人去见彼拉多,求耶稣的身体。彼拉多就吩咐给他。

约瑟取了身体,用乾净细麻布裹好,

安放在自己的新坟墓里,就是他凿在盘石里的。他又把大石头辊到墓口,就去了。

有抹大拉的马利亚,和那个马利亚在那里,对着坟墓坐着。

次日,就是豫备日的第二天,祭司长和法利赛人聚集,来见彼拉多,说,

大人,我们记得那诱惑人的,还活着的时候,曾说,三日后我要复活。

因此,请吩咐人将坟墓把守妥当,直到第三日。恐怕他的门徒来把他偷了去,就告诉百姓说,他从死里复活了。这样,那后来的迷惑,比先前的更利害了。

彼拉多说,你们有看守的兵。去吧,尽你们所能的,把守妥当。

他们就带着看守的兵同去,封了石头,将坟墓把守妥当。

\chapter{马太福音第28章}
安息日将尽,七日的头一日,天快亮的时候,抹大拉的马利亚,和那个马利亚,来看坟墓。

忽然地大震动。因为有主的使者,从天上下来,把石头辊开,坐在上面。

他的像貌如同闪电,衣服洁白如雪。

看守的人,就因他吓得浑身乱战,甚至和死人一样。

天使对妇女说,不要害怕,我知道你们是寻梢那钉十字架的耶稣。

他不在这里,照他所说的,已经复活了。你们来看安放主的地方。

妇女们就急忙离开坟墓,又害怕,又大大的欢喜,跑去要报给他的门徒。

忽然耶稣遇见他们,说,愿你们平安。他们就上前抱住他的脚拜他。

耶稣对他们说,不要害怕,你们去告诉我的弟兄,叫他们往加利利去,在那里必见我。

他们去的时候,看守的兵,有几个进城去,将所经历的事,都报给祭司长。

祭司长和长老聚集商议,就拿许多银钱给兵丁说,

你们要这样说,夜间我们睡觉的时候,他的门徒来把他偷去了。

倘若这话被巡抚听见,有我们劝他,保你们无事。

兵丁受了银钱,就照所嘱咐他们的去行。这话就传说在犹太人中间,直到今日。

十一个门徒往加利利去,到了耶稣约定的山上。

他们见了耶稣就拜他。然而还有人疑惑。

耶稣进前来,对他们说,天上,地下所有的权柄,都赐给我了。

所以你们要去,使万民作我的门徒,奉父子圣灵的名,给他们施洗。(或作给他们施洗归于父子圣灵的名)

凡我所吩咐你们的,都教训他们遵守,我就常与你们同在,直到世界的末了。

\chapter{马可福音第1章}
神的儿子,耶稣基督福音的起头,

正如先知以赛亚书上记着说,(有古卷无以赛亚三字)看哪,我要差遣我的使者在你面前,豫备道路。

在旷野有人声喊着说,豫备主的道,修直他的路。

照这话,约翰来了,在旷野施洗,传悔改的洗礼,使罪得赦。

犹太全地,和耶路撒冷的人,都出去到约翰那里,承认他们的罪,在约旦河里受他的洗。

约翰穿骆驼毛的衣服,腰束皮带,吃的是蝗虫野蜜。

他传道说,有一位在我以后来的,能力比我更大,我就是弯腰给他解鞋带,也是不配的。

我是用水给你们施洗,他却要用圣灵给你们施洗。

那时,耶稣从加利利的拿撒勒来,在约旦河里受了约翰的洗。

他从水里一上来,就看见天裂开了,圣灵彷佛鸽子,降在他身上。

又有声音从天上来说,你是我的爱子,我喜悦你。

圣灵就把耶稣催到旷野里去。

他在旷野四十天受撒但的试探。并与野兽同在一处。且有天使来伺候他。

约翰下监以后,耶稣来到加利利,宣传神的福音,

说,日期满了,神的国近了。你们当悔改,信福音。

耶稣顺着加利利的海边走,看见西门,和西门的兄弟安得烈,在海边撒网。他们本是打鱼的。

耶稣对他们说,来跟从我,我要叫你们得人如得鱼一样。

他们就立刻舍了网,跟从了他。

耶稣稍往前走,又见西庇太的儿子雅各,和雅各的兄弟约翰,在船上补网。

耶稣随既招呼他们。他们就把父亲西庇太,和雇工人留在船上,跟从耶稣去了。

到了迦百农,耶稣就在安息日进了会堂教训人。

众人很希奇他的教训。因为他教训他们,正像有权柄的人,不像文士。

在会堂里有一个人,被污鬼附着。他喊叫说,

拿撒勒人耶稣,我们与你有什么相干,你来灭我们吗。我知道你是谁,乃是神的圣者。

耶稣责备他说,不要作声,从这人身上出来吧。

污鬼叫那人抽了一阵疯,大声喊叫,就出来了。

众人都惊讶,以致彼此对问说,这是什么事,是个新道理阿。他用权柄吩咐污鬼,连污鬼也听从了他。

耶稣的名声,就传遍了加利利的四方。

他们一出会堂,就同着雅各约翰,进了西门和安得烈的家。

西门的岳母,正害热病躺着。就有人告诉耶稣。

耶稣进前拉着他的手,扶他起来,热就退了,他就服事他们。

天晚日落的时候,有人带着一切害病的,和被鬼附的,来到耶稣跟前。

合城的人都聚集在门前。

耶稣治好了许多害各样病的人,又赶出许多鬼,不许鬼说话,因为鬼认识他。

次日早晨,天未亮的时候,耶稣起来,到旷野地方去,在那里祷告。

西门和同伴追了他去。

遇见了就对他说,众人都找你。

耶稣对他们说,我们可以往别处去,到邻近的乡村,我也好在那里传道。因为我是为这事出来的。

于是在加利利全地,进了会堂,传道赶鬼。

有一个长大麻疯的,来求耶稣,向他跪下说,你若肯,必能叫我洁净了。

耶稣动了慈心,就伸手摸他,说,我肯,你洁净了吧。

大麻疯既时离开他,他就洁净了。

耶稣严严的嘱咐他,就打发他走,

对他说,你要谨慎,什么话都不可告诉人。只要去把身体给祭司察看,又因为你洁净了,献上摩西所吩咐的礼物,对众人作证据。

那人出去,倒说许多的话,把这件事传扬开了,叫耶稣以后不得再明明的进城,只好在外边旷野地方。人从各处都就了他来。

\chapter{马可福音第2章}
过了些日子,耶稣又进了迦百农。人听见他在房子里,

就有许多人聚集,甚至连进门前都没有空地,耶稣就对他们讲道。

有人带着一个瘫子来见耶稣,是用四个人抬来的。

因为人多,不得进前,就把耶稣所在的房子,拆了房顶,既拆通了,就把瘫子连所躺卧的褥子都缒下来。

耶稣见他们的信心,就对瘫子说,小子,你的罪赦了。

有几个文士坐在那里,心里议论说,

这个人为什么这样说呢。他说僭妄的话了。除了神以外,谁能赦罪呢。

耶稣心中知道他们的心里这样议论,就说,你们心里为什么这样议论呢。

或对瘫子说,你的罪赦了,或说,起来拿你的褥子行走,那一样容易呢。

但要叫你们知道人子在地上有赦罪的权柄,就对瘫子说,

我吩咐你起来,拿你的褥子回家去吧。

那人就起来,立刻拿着褥子,当众人面前出去了。以致众人都惊奇,归荣耀与神说,我们从来没有见过这样的事。

耶稣又出到海边去,众人都就了他来,他便教训他们。

耶稣经过的时候,看见亚勒腓的儿子利未,坐在税关上,就对他说,你跟从我来。他就起来跟从了耶稣。

耶稣在利未家里坐席的时候,有好些税吏和罪人,与耶稣并门徒一同坐席。因为这样的人多,他们也跟随耶稣。

法利赛人中的文士,(有古卷作文士和法利赛人)看见耶稣和罪人并税吏一同吃饭,就对他的门徒说,他和税吏并罪人一同吃喝吗。

耶稣听见,就对他们说,健康的人用不着医生,有病的人才用得着。我来本不是召义人,乃是召罪人。

当下,约翰的门徒和法利赛人禁食。他们来问耶稣说,约翰的门徒和法利赛人的门徒禁食,你的门徒倒不禁食,这是为什么呢。

耶稣对他们说,新郎和陪伴之人同在的时候,倍伴之人岂能禁食呢。新郎还同在,他们不能禁食。

但日子将到,新郎要离开他们,那日他们就要禁食。

没有人把新布缝在旧衣服上。恐怕所补上的新布,带坏了旧衣服,破的就更大了。

也每有人把新酒装在旧皮带里。恐怕把皮袋裂开,酒和皮袋就都坏了。惟把新酒装在新皮袋里。

耶稣当安息日,从麦地经过。他的门徒行路的时候,掐了麦穗。

法利赛人对耶稣说,看哪,他们在安息日为什么作不可作的事呢。

耶稣对他们说,经上记着大卫和跟从他的人,缺乏肌饿之时所作的事,你们没有念过吗。

他当亚比亚他作大祭司的时候,怎样进了神的殿,吃了陈设饼,又给跟从他的人吃。这饼除了祭司以外,人都不可吃。

又对他们说,安息日是为人设立的,人不是为安息日设立的。

所以人子也是安息日的主

\chapter{马可福音第3章}
耶稣又进了会堂。在那里有一个人枯乾了一只手。

众人窥探耶稣,在安息日医治不医治,意思是要控告耶稣。

耶稣对那枯乾一只手的人说,起来,站在当中。

又问众人说,在安息日行善行恶,救命害命,那样是可以的呢。他们都不作声。

耶稣怒目周围看他们,忧愁他们的心刚硬,就对那人说,伸出手来。他把手一伸,手就复了原。

法利赛人出去,同希律一党的人商议,怎样可以除灭耶稣。

耶稣和门徒退到海边去。有许多人从加利利跟随他。

还有许多人听见他所作的大事,就从犹太,耶路撒冷,以土买,约旦河外,并推罗西顿的四方,来到他那里。

他因为人多,就吩咐门徒叫一只小船伺候着,免得众人拥挤他。

他治好了许多人,所以凡有灾病的,都挤进来要摸他。

污鬼无论何时看见他就俯伏在他面前喊着说,你是神的儿子。

耶稣再三的嘱咐他们,不要把他显露出来。

耶稣上了山,随自己的意思叫人来,他们便来到他那里。

他设立十二个人,要他们常和自己同在,也要差他们去传道,

并给他们权柄赶鬼。

这十二个人有西门,耶稣又给他起名叫彼得。

还有西庇太的儿子雅各,和雅各的兄弟约翰。又给这两个人起名叫半尼其,就是雷子的意思。

又有安得烈,腓力,巴多罗买,马太,多马,亚勒腓的儿子雅各,和达太,并奋锐党的西门。

还有卖耶稣的加略人犹大。

耶稣进了一个屋子,众人又聚集,甚至他连吃饭也顾不得吃。

耶稣的亲属听见,就出来要拉住他,因为他们说他癫狂了。

从耶路撒冷下来的文士说,他是被别西卜附着。又说,他是靠着鬼王赶鬼。

耶稣叫他们来,用比喻对他们说,撒但怎能赶出撒但呢。

若一国自相分争,那国就站立不住。

若一家自相分争,那家就站立不住。

若撒但自相攻打分争,他就站立不住,必要灭亡。

没有人能进壮士家里抢夺他的家具。必先捆住那壮士,才可以抢夺他的家。

我实在告诉你们,世人一切的罪,和一切亵渎的话,都可得赦免。

凡亵渎圣灵的,却永不得赦免,乃要担当永远的罪。

这话是因为他们说,他是被污鬼附着的。

当下耶稣的母亲,和弟兄,来站在外边,打发人去叫他。

有许多人在耶稣周围坐着。他们就告诉他说,看哪,你母亲,和你弟兄,在外边找你。

耶稣回答说,谁是我的母亲,谁是我的弟兄。

就四面观看那周围坐着的人,说,看哪,我的母亲,我的弟兄。

凡遵行神旨意的人,就是我的弟兄姐妹和母亲了。

\chapter{马可福音第4章}
耶稣又在海边教训人。有许多人到他那里聚集,他只得上船坐下。船在海里,众人都靠近海站在岸上。

耶稣就用比喻教训他们许多道理。在教训之间,对他们说,

你们听阿。有一个撒种的。出去撒种。

撒的时候,有落在路旁的,飞鸟来吃尽了。

有落在土浅石头地上的,土既不深,发苗最快。

日头出来一晒,因为没有根,就枯乾了。

有落在荆棘里的,荆棘长起来,把他挤住了,就不结实。

又有落在好土里的,就发生长大,结实有三十倍的,有六十倍的,有一百倍的。

又说,有耳可听的,就应当听。

无人的时候,跟随耶稣的人,和十二个门徒,问他这比喻的意思。

耶稣对他们说,神国的奥秘,只叫你们知道,若是对外人讲,凡事就用比喻。

叫他们看是看见,却不晓得。听是听见,却不明白。恐怕他们回转过来,就得赦免。

又对他们说,你们不明白这比喻吗。这样怎能明白一切的比喻呢。

撒种之人所撒的,就是道。

那撒在路旁的,就是人听了道,撒但立刻来,把撒在他心里的道夺了去。

那撒在石头地上的,就是人听了道,立刻欢喜领受。

但他心里没有根,不过是暂时的,及至为道遭了患难,或是受了逼迫,立刻就跌倒了。

还有那撒在荆棘里的,就是人听了道。

后来有世上的思虑,钱财的迷惑,和别样的私欲,进来把道挤住了,就不能结实。

那撒在好地上的,就是人听道,又领受,并且结实,有三十倍的,有六十倍的,有一百倍的。

耶稣又对他们说,人拿灯来,岂是放在斗底下,床底下,不放在灯台上吗。

因为掩藏的事,没有不显出来的。隐瞒的事,没有不露出来的。

有耳可听的,就应当听。

又说你们所听的要留心。你们用什么量器量给人,也必用什么量器量给你们,并且要多给你们。

因为有的还要给他。没有的,连他所有的也要夺去。

又说,神的国,如同人把种撒在地上,

黑夜睡觉,白日起来,这种就发芽渐长,那人却不晓得如何这样。

地生五谷,是出于自然的。先发苗,后长穗,再后穗上结成饱满的子粒。

谷既熟了,就用镰刀去割,因为收成的时候到了。

又说,神的国,我们可用什么比较呢。可用什么比喻表明呢。

好像一粒芥菜种,种在地里的时候,虽比地上的百种都小,

但种上以后,就长起来,比各样的菜都大,又长出大枝来。甚至天上的飞鸟,可以宿在他的荫下。

耶稣用许多这样的比喻,照他们所能听的,对他们讲道。

若不用比喻,就不对他们讲。没有人的时候,就把一切的道讲给门徒听。

当那天晚上,耶稣对门徒说,我们渡到那边去吧。

门徒离开众人,耶稣仍在船上,他们就把他一同带去。也有别的船和他同行。

忽然起了暴风,波浪打入船内,甚至船要满了水。

耶稣在船尾上,枕着枕头睡觉。门徒叫醒了他,说,夫子,我们丧命,你不顾吗。

耶稣醒了,斥责风,向海说,住了吧,静了吧。风就止住,大大的平静了。

耶稣对他们说,为什么胆怯。你们没有信心吗。

他们就大大的惧怕,彼此说,这到底是谁,连风和海也听从他了。

\chapter{马可福音第5章}
他们来到海边,格拉森人的地方。

耶稣一下船,就有一个被污鬼咐着的人,从坟茔里出来迎着他。

那人常住在坟茔里,没有人能捆住他,就是用铁链也不能。

因为人屡次用脚镣和铁链捆锁他,铁链竟被他挣断了,脚镣也被他弄碎了。总没有人能制伏他。

他昼夜常在坟茔里和山中喊叫,又用石头砍自己。

他远远的看见耶稣,就跑过去拜他。

大声呼叫说,至高神的儿子耶稣,我与你有什么相干。我指着神恳求你,不要叫我受苦。

是因耶稣曾吩咐他说,污鬼阿,从这人身上出来吧。

耶稣问他说,你名叫什么。回答说,我名叫群,因为我们多的缘故。

就再三的求耶稣,不要叫他们离开那地方。

在那里山坡上,有一大群猪吃食。

鬼就央求耶稣说,求你打发我们往猪群里附着猪去。

耶稣准了他们。污鬼就出来,进入猪里去。于是那群猪闯下山崖,投在海里,淹死了。猪的数目,约有二千。

放猪的就逃跑了,去告诉城里和乡下的人。众人就来要看是什么事。

他们来到耶稣那里,看见那被鬼附着的人,就是从前被群鬼所附的,坐着,穿上衣服,心里明白过来。他们就害怕。

看见这事的,便将鬼附之人所遇见的,和那群猪的事,都告诉了众人。

众人就央求耶稣离开他们的境界。

耶稣上船的时后,那从前被鬼附着的人,恳求和耶稣同在。

耶稣不许,却对他说,你回家去,到你的亲属那里,将主为你所作的,是何等大的事,是怎样怜悯你,都告诉他们。

那人就走了,在低加波利,传扬耶稣为他作了何等大的事,众人都希奇。

耶稣坐船又渡到那边去,就有许多人到他那里聚集。他正在海边上。

有一个管会堂的人,名叫睚鲁,来见耶稣,就伏在他脚前,

再三的求他说,我的小女儿快要死了,求你去按手在他身上,使他痊愈,得以活了。

耶稣就和他同去,有许多人跟随拥挤他。

有一个女人,患了十二年的血漏,

在好些医生手里,受了许多的苦。又花尽了他所有的,一点也不见好,病势反倒更重了。

他听见耶稣的事,就从后头来,杂在众人中间,摸耶稣的衣裳。

意思说,我只摸他的衣裳,就必痊愈。

于是他血漏的源头,立刻乾了。他便觉得身上的灾病好了。

耶稣顿时心里觉得有能力从自己身上出去,就在众人中间转过来说,谁摸我的衣裳。

门徒对他说,你看众人拥挤你,还说谁摸我吗。

耶稣周围观看,要见作这事的女人。

那女人知道在自己身上所成的事,就恐惧战竞,来俯伏在耶稣跟前,将实情全告诉他。

耶稣对他说,女儿,你的信救了你,平平安安的回去吧。你的灾病痊愈了。

还说话的时候,有人从管会堂的家里来说,你的女儿死了,何必还劳动先生呢。

耶稣听见所说的话,就对管会堂的说,不要怕。只要信。

于是带着彼得,雅各,和雅各的兄弟约翰同去,不许别人跟随他。

他们来到管会堂的家里,耶稣看见那里乱囔,并有人大大的哭泣哀号。

进到里面,就对他们说,为什么乱囔哭泣呢,孩子不是死了,是睡着了。

他们就嗤笑耶稣。耶稣把他们都撵出去,就带着孩子的父母,和跟随的人进入了孩子所在的地方。

就拉着孩子的手,对他说,大利大古米。翻出来,就是说,闺女,我吩咐你起来。

那闺女立时起来走。他们就大大的惊奇。闺女已经十二岁了。

耶稣切切的嘱咐他们,不要叫人知道这事。又吩咐给他东西吃。

\chapter{马可福音第6章}
耶稣离开那里,来到自己的家乡。门徒也跟从他。

他到了安息日,他在会堂里教训人。众人听见,就甚希奇,说,这人从那里有这些事呢,所赐给他的是什么智慧,他手所作的是何等的异能呢。

这不是那木匠吗。不是马利亚的儿子,雅各,约西,犹大,西门的长兄吗。他的妹妹们不也是在我们这里吗。他们就厌弃他。(厌弃他原文作因他跌倒)

耶稣对他们说,大凡先知,除了本地亲属本家之外,没有不被人尊敬的。

耶稣就在那里不得行什么异能,不过按手在几个病人身上,治好他们。

他也诧异他们不信,就往周围乡村教训人去了。

耶稣叫了十二个门徒来,差遣他们两个两个的出去。也赐给他们权柄,制伏污鬼。

并且嘱咐他们,行路的时候,不要带食物和口袋,腰袋里也不要带钱,除了拐杖以外,什么都不要带。

只要穿鞋。也不要穿两件挂子。

又对他们说,你们无论到何处,进了人的家,就住在那里,直到离开那地方。

何处的人,不接待你们,不听你们,你们离开那里的时候,就把脚上的尘土跺下去,对他们作见证。

门徒就出去,传道叫人悔改。

又赶出许多的鬼,用油抹了许多病人,治好他们。

耶稣的名声传扬出来。希律王听见了,就说,施洗的约翰从死里复活了,所以这些异能由他里面发出来。

但别人说,是以利亚。又有人说,是先知,正像先知中的一位。

希律听见,却说,是我所斩的约翰,他复活了。

先是希律为他兄弟腓力的妻子希罗底的缘故,差人去拿住约翰,锁在监里,因为希律已经娶了那妇人。

约翰曾对希律说,你娶你兄弟的妻子是不合理的。

于是希罗底怀恨他,想要杀他。只是不能。

因为希律知道约翰是义人,是圣人,所以敬畏他,保护他。听他讲论,就多照着行。并且乐意听他。(多照着行有古卷作游移不定)

有一天,恰巧是希律的生日,希律摆设筵席,请了大臣和千夫长,并加利利作首领的。

希罗底的女儿进来跳舞,使希律和同席的人都欢喜。王就对女子说,你随意向我求什么,我必给你。

又对他起誓说,随你向我求什么,就是我国的一半,我也必给你。

他就出去,对他母亲说,我可以求什么呢。他母亲说,施洗约翰的头。

他就急忙进去见王,求他说,我愿王立时把施洗约翰的头,放在盘子里给我。

王就甚忧愁。但因他所起的誓,又因同席的人,就不肯推辞。

随既差一个护卫兵,吩咐拿约翰的头来。护卫兵就去在监里斩了约翰,

把头放在盘子里,拿来给女子,女子就给他母亲。

约翰的门徒听见了,就来把他的尸首领去,葬在坟墓里。

使徒聚集到耶稣那里,将一切所作的事,所传的道,全告诉他。

他就说,你们来同我暗暗的到旷野地方去歇一歇。这是因为来往的人多,他们连吃饭也没有工夫。

他们就坐船,暗暗的往旷野地方去。

众人看见他们去,有许多认识他们的,就从各城步行,一同跑到那里,比他们先赶到了。

耶稣出来,见有许多的人,就怜悯他们。因为他们如同羊没有牧人一般。于是开口教训他们许多道理。

天已经晚了,门徒进前来说,这是野地,天已经晚了,

请叫众人散开,他们好往四面乡村里去,自己买什么吃的。

耶稣回答说,你们给他们吃吧。门徒说,我们可以去买二十两银子的饼,给他们吃吗。

耶稣说,你们有多少饼,可以去看看。他们知道了,就说,五个饼,两条鱼。

耶稣吩咐他们叫众人一帮一帮的,坐在青草地上。

众人就一排一排的坐下,有一百一排的,有五十一排的。

耶稣拿着这五个饼,两条鱼,望天祝福,擘开饼,递给门徒摆在众人面前。也把那两条鱼分给众人。

他们都吃,并且吃饱了。

门徒就把碎饼碎鱼,收拾起来,装满了十二个篮子。

吃饼的男子,共有五千。

耶稣随既催门徒上船,先渡到那边伯赛大去,等他叫众人散开。

他既辞别了他们,就往山上去祷告。

到了晚上,船在海中,耶稣独自在岸上。

看见门徒,因风不顺,摇橹甚苦。夜里约有四更天,就在海面上走往他们那里去,意思要走过他们去。

但门徒看见他在海面上走,以为是鬼怪,就喊叫起来。

因为他们都看见了他,且甚惊慌。耶稣连忙对他们说,你们放心。是我,不要怕。

于是到他们那里上了船,风就住了。他们心里十分惊奇。

这是因为他们不明白那分饼的事,心里还是愚顽。

既渡过去,来到革尼撒勒地方,就靠了岸。

一下船,众人认得是耶稣。

就跑遍那一带地方,听见他在何处,便将有病的人,用褥子抬到那里。

凡耶稣所到的地方,或村中,或城里,或乡间,他们都将病人放在街市上,求耶稣只容他们摸他的衣裳??子。凡摸着的人,就都好了。

\chapter{马可福音第7章}
有法利赛人,和几个文士,从耶路撒冷来,到耶稣那里聚集。

他们曾看见他的门徒中,有人用俗手,就是没有洗的手,吃饭。

(原来法利赛人,和犹太人,都拘守古人的遗传,若不仔细洗手,就不吃饭。

从市上来,若不洗浴,也不吃饭,还有好些别的规矩,他们历代拘守,就是洗杯,罐,铜器,等物)。

法利赛人和文士问他说,你的门徒为什么不照古人的遗传,用俗手吃饭呢。

耶稣说,以赛亚指着你们假冒为善之人所说的预言,是不错的,如经上说,这百姓用嘴唇尊敬我,心却远离我。

他们将人的吩咐,当作道理教导人,所以拜我也是枉然。

你们是离弃神的诫命,拘守人的遗传。

又说,你们诚然是废弃神的诫命,要守自己的遗传。

摩西说,当孝敬父母。又说,咒骂父母的,必治死他。

你们倒说,人若对父母说,我所当奉献给你的,已经作了各耳板,(各耳板,就是供献的意思)

以后你们就不容他再奉养父母。

这就是你们承受遗传,废了神的道。你们还作许多这样的事。

耶稣又叫众人来,对他们说,你们都要听我的话,也要明白。

从外面进去的,不能污秽人,惟有从里面出来的,乃能污秽人。(有古卷在此有,

有耳可听的就应当听)。

耶稣离开众人,进了屋子,门徒就问他这比喻的意思。

耶稣对他们说,你们也是这样不明白吗。岂不晓得凡从外面进入的,不能污秽人。

因为不是入他的心,乃是入他的肚腹,又落到茅厕里。这是说,各样的食物,都是洁净的。

又说,从人里面出来的,那才能污秽人。

因为从里面,就是从人心里,发出恶念,苟合,

偷盗,凶杀,奸淫,贪婪,邪恶,诡诈,淫荡,嫉妒,谤??,骄傲,狂妄。

这一切的恶,都是从里面出来,且能污秽人。

耶稣从那里起身,往推罗西顿的境内去。进了一家,不愿意人知道,却隐藏不住。

当下有一个妇人,他的小女儿被污鬼附着,听见耶稣的事,就来俯伏在他脚前。

这妇人是希腊人,属叙利非尼基族。他求耶稣赶出那鬼,离开他的女儿。

耶稣对他说,让儿女们先吃饱。不好拿儿女的饼丢给狗吃。

妇人回答说,主阿,不错。但是狗在桌子底下,也吃孩子们的碎渣儿。

耶稣对他说,因这句话,你回去吧。鬼已经离开你的女儿了。

他就回家去,见小孩子躺在床上,鬼已经出去了。

耶稣又离开推罗的境界,经过西顿,就从低加波利境内来到加利利海。

有人带着一个耳聋舌结的人,来见耶稣,求他按手在他身上。

耶稣领他离开众人,到一边去,就用指头探他的耳朵,吐唾沫抹他的舌头,

望天叹息,对他说,以法大,就是说,开了吧。

他的耳朵就开了,舌结也解了,说话也清楚了。

耶稣嘱咐他们,不要告诉人。但他越发嘱咐,他们越发传扬开了。

众人分外希奇,说,他所作的事都好,他连聋子也叫他们听见,哑吧也叫他们说话。

\chapter{马可福音第8章}
那时,又有许多人聚集,并没有什么吃的。耶稣叫门徒来,说,

我怜悯这众人,因为他们同我在这里已经三天,也没有吃的了。

我若打发他们饿着回家,就必在路上困乏。因为其中有从远处来的。

门徒回答说,在这野地,从那里能得饼,叫这些人吃饱呢。

耶稣问他们说,你们有多少饼。他们说,七个。

他吩咐众人坐在地上,就拿着这七个饼,祝谢了,擘开递给门徒叫他们摆开,门徒就摆在众人面前。

又有几条小鱼。耶稣祝了福,就吩咐也摆在众人面前。

众人都吃,并且吃饱了。收拾剩下的零碎,有七筐子。

人数约有四千。耶稣打发他们走了,

随既同门徒上船,来到大玛努他境内。

法利赛人出来盘问耶稣,求他从天上显个神迹给他们看,想要试探他。

耶稣心里深深的叹息说,这世代为什么求神迹呢。我实在告诉你们,没有神迹给这世代看。

他就离开他们,又上船往海那边去了。

门徒忘了带饼。在船上除了一个饼,没有别的食物。

耶稣嘱咐他们说,你们要谨慎,防备法利赛人的酵,和希律的酵。

他们彼此议论说,这是因为我们没有饼吧。

耶稣看出来,就说,你们为什么因为没有饼就议论呢。你们还不省悟,还不明白吗。你们的心还是愚顽吗。

你们有眼睛看不见吗,有耳朵,听不见吗。也不记得吗。

我擘开那五个饼分给五千人,你们收拾的零碎,装满了多少篮子呢。他们说,十二个。

又擘开那七个饼分给四千人,你们收拾的零碎,装满了多少筐子呢。他们说,七个。

耶稣说,你们还是不明白吗。

他们来到伯赛大,有人带一个瞎子来,求耶稣摸他。

耶稣拉着瞎子的手,领他到村外。就吐唾沫在他眼睛上,按手在他身上,问他说,你看见什么了。

他就抬头一看说,我看见人了。他们好像树木,并且行走。

随后又按手在他眼睛上,他定睛一看,就复了原,样样都看得清楚了。

耶稣打发他回家,说,连这村子你也不要进去。

耶稣和门徒出去,往凯撒利亚-腓立比的村庄去。在路上问门徒说,人说我是谁。

他们说,有人说,是施洗的约翰。有人说,是以利亚。又有人说,是先知里的一位。

又问他们说,你们说我是谁。彼得回答说,你是基督。

耶稣就禁戒他们,不要告诉人。

从此他教训他们说,人子必须受许多的苦,被长老祭司长和文士弃绝,并且被杀,过三天复活。

耶稣明明的说这话,彼得就拉着他,劝他。

耶稣转过来,看着门徒,就责备彼得说,撒但,退我后边去吧。因为你不体贴神的意思,只体贴人的意思。

于是叫众人和门徒来,对他们说,若有人要跟从我,就当舍己,背起他的十字架来跟从我。

因为凡救自己生命的,(生命或作灵魂下同)必丧掉生命。凡为我和福音丧掉生命的,必救了生命。

人就是赚得全世界,赔上自己的生命,有什么益处呢。

人还能拿什么换生命呢。

凡在这淫乱罪恶的世代,把我和我的道当作可耻的,人子在他父的荣耀里,同圣天使降临的时候,也要把那人当作可耻的。

\chapter{马可福音第9章}
耶稣又对他们说,我实在告诉你们,站在这里的,有人在没尝死味以前,必要看见神的国大有能力临到。

过了六天,耶稣带着彼得,雅各,约翰,暗暗的上了高山,就在他们面前变了形像。

衣服放光,极其洁白。地上漂布的,没有一个能漂得那样白。

忽然有以利亚同摩西向他们显现。并且和耶稣说话。

彼得对耶稣说,拉比,(拉比就是夫子)我们在这里真好。可以搭三座棚,一座为你,一座为摩西,一座为以利亚。

彼得不知道说什么才好。因为他们甚是惧怕,

有一朵云彩来遮盖他们。也有声音从云彩里出来说,这是我的爱子,你们要听他。

门徒忽然周围一看,不见一人,只见耶稣同他们在那里。

下山的时候,耶稣嘱咐他们说,人子还没有从死里复活,你们不要将所看见的告诉人。

门徒将这话存记在心,彼此议论从死里复活是什么意思。

他们就问耶稣说,文士为什么说,以利亚必须先来。

耶稣说,以利亚固然先来,复兴万事。经上不是指着人子说,他要受许多的苦,被人轻慢呢。

我告诉你们,以利亚已经来了,他们也任意待他,正如经上所指着他的话。

耶稣到了门徒那里,看见有许多人围着他们,又有文士和他们辩论。

众人一见耶稣,都甚希奇,就跑上去问他的安。

耶稣问他们说,你们和他们辩论的是什么。

众人中间有一个回答说,夫子,我带了我的儿子到你这里来,他被哑巴鬼附着。

无论在那里,鬼捉弄他,把他摔倒,他就口中流沫,咬牙切齿,身体枯乾,我请过你的门徒把鬼赶出去,他们却是不能。

耶稣说,嗳,不信的世代阿,我在你们这里要到几时呢。我忍耐你们要到几时呢。把他带到我这里来吧。

他们就带了他来。他一见耶稣,鬼便叫他重重的抽疯。倒在地上,翻来覆去,口中流沫。

耶稣问他父亲说,他得这病,有多少日子呢。回答说,从小的时后。

鬼屡次把他扔在火里,水里,要灭他。你若能作什么,求你怜悯我们,帮助我们。

耶稣对他说,你若能信,在信的人,凡事都能。

孩子的父亲立时喊着说,我信。但我信不足,求主帮助。(有古卷作立时流泪的喊着说)

耶稣看见众人都跑上来,就斥责那污鬼,说,你们聋哑的鬼,我吩咐你从他里头出来,再不要进去。

那鬼喊叫,使孩子大大的抽了一阵疯,就出来了。孩子好像死了一班,以致众人多半说,他是死了。

但耶稣拉他的手,扶他起来,他就站起来了。

耶稣进了屋子,门徒就暗暗的问他说,我们为什么不能赶出他去呢。

耶稣说,非用祷告,(有古卷在这有禁食二字)这一类的鬼,总不能出来。(或作不能赶他出来)

他们离开那地方,经过加利利。耶稣不愿意人知道。

于是教训门徒,说,人子将要被交在人手里,他们要杀害他。被杀以后,过三天他要复活。

门徒却不明白这话,又不敢问他。

他们来到迦百农。耶稣在屋里问门徒说,你们在路上议论的是什么。

门徒不作声,因为他们在路上彼此争论谁为大。

耶稣坐下,叫十二个门徒来,说,有人愿意作首先的,他必作众人末后的,作众人的用人。

于是领过一个小孩子来,叫他站在门徒中间。又抱起他来,对他们说,

凡为我名,接待一个像这小孩子的就是接待我。凡接待我的,不是接待我,乃是接待那差我来的。

约翰对耶稣说,夫子,我们看见一个人,奉你的名赶鬼,我们就禁止他,因为他不跟从我们。

耶稣说,不要禁止他。因为没有人奉我名行异能,反倒轻易毁谤我。

不敌挡我们的,就是帮助我们的。

凡因你们是属基督,给你们一杯水喝的,我实在告诉你们,他不能不得赏赐。

凡使这信我的一个小子跌倒的,倒不如把大磨石栓在这人的颈项上,扔在海里。

倘若你一只手叫你跌倒,就把他砍下来。

你缺了肢体进入永生,强如有两只手落到地狱,入那不灭的火里去。

倘若你一只脚叫你跌倒,就把他砍下来。

你瘸腿进入永生,强如有两只脚被丢在地狱里。

倘若你一只眼叫你跌倒,就去掉他。你只有一只眼进入神的国,强如有两只眼被丢在地狱里。

在那里虫是不死的,火是不灭的。

因为必用火当盐,腌各人。(有古卷在此有凡祭物必用盐腌)

盐本是好的,若失了味,可用什么叫他在咸呢。你们里头应当有盐,彼此和睦。

\chapter{马可福音第10章}
耶稣从那里起身,来到犹太的境界,并约旦河外。众人又聚集到他那里,他又照常教训他们。

有法利赛人来问他说,人休妻可以不可以,意思要试探他。

耶稣回答说,摩西吩咐你们的是什么。

他们说,摩西许人写了休书便可以休妻。

耶稣说,摩西因为你们的心硬,所以写这条例给你们。

但从起初创造的时候,神造人是造男造女。

因此人要离开父母,与妻子连合,二人成为一体。

既然如此,夫妻不再是两个人,乃是一体的了。

所以神配合的,人不可分开。

到了屋里,门徒就问他这事。

耶稣对他们说,凡休妻另娶的,就是犯奸淫,辜负他的妻子。

妻子若离弃丈夫另嫁,也是犯奸淫了。

有人带着小孩子来见耶稣,要耶稣摸他们,门徒便责备那些人。

耶稣看见就恼怒,对门徒说,让小孩到我这里来,不要禁止他们。因为在神国的,正是这样的人。

我实在告诉你们,凡要承受神国的,若不像小孩子,断不能进去。

于是抱着小孩子,给他们按手,为他们祝福。

耶稣出来行路的时候,有一个人跑来,跪在他面前问他说,良善的夫子,我当作什么事,才可以承受永生。

耶稣对他说,你为什么称我是良善的。除了神一位之外,再没有良善的。

诫命你是晓得的,不可杀人,不可奸淫,不可偷盗,不可作假见证,不可亏负人,当孝敬父母。

他对耶稣说,夫子,这一切我从小都遵守了。

耶稣看着他,就爱他,对他说,你还缺少一件。去变卖你所有的,分给穷人,就必有财宝在天上。你还要来跟从我。

他听见这话,脸上就变了色,忧忧愁愁的走了。因为他的产业很多。

耶稣周围一看,对门徒说,有钱财的人进神的国是何等的难哪。

门徒希奇他的话。耶稣又对他们说,小子,倚靠钱财的人进神的国,是何等的难哪。

骆驼穿过针的眼,比财主进神的国,还容易呢。

门徒就分外希奇,对他说,这样谁能得救呢。

耶稣看着他们说,在人是不能,在神却不然。因为神凡事都能。

彼得就对他说,看哪,我们已经撇下所有的跟从你了。

耶稣说,我实在告诉你们,人为我和福音,撇下房屋,或是弟兄,姐妹,父母,儿女,田地。

没有不在今世得百倍的,就是房屋,弟兄,姐妹,母亲,儿女,田地,并且要受逼迫。在来世必得永生。

然而有许多在前的将要在后,在后的将要在前。

他们行路上耶路撒冷去。耶稣在前头走,门徒就希奇,跟从的人也害怕。耶稣又叫过十二个门徒来,把自己将要遭遇的事,告诉他们说,

看哪,我们上耶路撒冷去,人子将要被交给祭司长和文士,他们要定他死罪,交给外邦人。

他们要戏弄他,吐唾抹在他脸上,鞭打他,杀害他。过了三天,他要复活。

西庇太的儿子雅各,约翰进前来,对耶稣说,夫子,我们无论求你什么,愿你给我们作。

耶稣说,要我给你们作什么。

他们说,赐我们在你的荣耀里,一个坐在你右边,一个坐在你左边。

耶稣说,你们不知道所求的是什么。我所喝的杯。你们能喝吗。我所受的洗,你们能受吗。

他们说,我们能。耶稣说,我所喝的杯,你们也要喝。我所受的洗,你们也要受。

只是坐在我的左右,不是我可以赐的。乃是为谁豫备的,就赐给谁。

那十个门徒听见,就恼怒雅各,约翰。

耶稣叫他们来,对他们说,你们知道,外邦人有尊为君王的,治理他们。有大臣操权管束他们。

只是在你们中间,不是这样。你们中间,谁愿为大,就必作你们的用人。

在你们中间,谁愿为首,就必作众人的仆人。

因为人子来,并不是要受人的服事,乃是要服事人,并且要舍命,作多人的赎价。

到了耶利哥。耶稣同门徒并许多人出耶利哥的时候,有一个讨饭的瞎子,是底买的儿子巴底买,坐在路旁。

他听见是拿撒勒的耶稣,就喊着说,大卫的子孙耶稣阿,可怜我吧。

有许多人责备他,不许他作声,他却越发大声喊着说,大卫的子孙哪,可怜我吧。

耶稣就站住,说,叫他过来。他们就叫那瞎子,对他说,放心,起来,他叫你喇。

瞎子就丢下衣服,跳起来,走到耶稣那里。

耶稣说,要我为你作什么。瞎子说,拉波尼,我要能看见。(拉波尼就是夫子)

耶稣说,你去吧。你的信救了你了。瞎子立刻看见了,就在路上跟随耶稣。

\chapter{马可福音第11章}
耶稣和门徒将近耶路撒冷,到了伯法其和伯大尼,在橄榄山那里。耶稣就打发两个门徒,

对他们说,你们往对面村子里去。一进去的时候,必看见一匹驴驹拴在那里,是从来没有人骑过的。可以解开牵来。

若有人对你们说,为什么作这事。你们就说,主要用他。那人必立时让你们牵来。

他们去了,便看见一匹驴驹,拴在门外街道上,就把他解开。

在那里站着的人,有几个说,你们解驴驹作什么。

门徒照着耶稣所说的回答,那些人就任凭他们牵去了。

他们把驴驹牵到耶稣那里,把自己的衣服搭在上面,耶稣就骑上。

有许多人,把衣服铺在路上,也有人把田间的树枝砍下来,铺在路上。

前行后随的人,都喊着说,和散那,(和散那原有求救的意思,在此乃是称颂的话)奉主名来的,是应当称颂的。

那将要来的我祖大卫之国,是应当称颂的。高高在上,和散那。

耶稣进了耶路撒冷,入了圣殿,周围看了各样物件。天色已晚,就和十二个门徒出城往伯大尼去了。

第二天,他们从伯大尼出来。耶稣饿了,

远远的看见一棵无花果树,树上有叶子,就往那里去,或者在树上可以找着什么。到了树下,竟找不着什么,不过有叶子。因为不是收无花果的时候。

耶稣就树说,从今以后,永没有人吃你的果子。他的门徒也听见了。

他们来到耶路撒冷,耶稣进入圣殿,赶出殿里作买卖的人,推倒兑换银钱之人的桌子,和卖鸽子之人的凳子。

也不许人拿着器具从殿里经过。

便教训他们说,经上不是记着说,我的殿必称为万国祷告的殿吗。你们倒使他成为贼窝了。

祭司长和文士听见这话就想法子要除灭耶稣。却又怕他,因为众人都希奇他的教训。

每天晚上,耶稣出城去。

早晨,他们从那里经过,看见无花果树连根都枯乾了。

彼得想起耶稣的话来,就对他说,拉比,请看,你所咒诅的无花果树,已经枯乾了。

耶稣回答说,你们当信服神。

我实在告诉你们,无论何人对这座山说,你挪开此地投在海里。他心里若不疑惑,只信他所说的必成,就必给他成了。

所以我告诉你们,凡你们祷告祈求的,无论是什么,只要信是得着的,就必得着。

你们站着祷告的时候,若想起有人得罪你们,就当饶恕他,好叫你们在天上的父,也饶恕你们的过犯。

你们若不饶恕人,你们在天上的父,也不饶恕你们的过犯。(有古卷无此节)

他们又来到耶路撒冷。耶稣在殿里行走的时候,祭司长和文士并长老进前来,

问他说,你仗着什么权柄作这些事,给你这权柄的是谁呢。

耶稣对他们说,我要问你们一句话,你们回答我,我就告诉你们,我仗着什么权柄作这事。

约翰的洗礼,是从天上来的是从人间来的呢。你们可以回答我。

他们彼此商论说,我们若说从天上来,他必说,这样,你们为什么不信他呢。

若说从人间来,却又怕百姓。因为众人真以约翰为先知。

于是回答耶稣说,我们不知道。耶稣说,我也不告诉你们,我仗着什么权柄作这事。

\chapter{马可福音第12章}
耶稣就用比喻对他们说,有人栽了一个葡萄园,周围圈上篱笆,挖了一个压酒池,盖了一座楼,租给园户,就往国外去了。

到了时候,打发一个仆人到园户那里,要从园户收葡萄园的果子。

园户拿住他,打了他,叫他空手回去。

再打发一个仆人到他们那里。他们打伤他的头,并且凌辱他。

又打发一个仆人去。他们就杀了他。后又打发好些仆人去。有被他们打的,有被他们杀的。

园主还有一位,是他的爱子,末后又打发他去,意思说,他们必尊敬我的儿子。

不料,那些园户彼此说,这是承受产业的。来吧,我们杀他,产业就归我们了。

于是拿住他,杀了他,把他丢在园外。

这样,葡萄园的主人要怎样办呢。他要来除灭那些园户,将葡萄园转给别人。

经上写着说匠人所弃的石头,已作了房角的头块石头。

这是主所作的,在我们眼中看为希奇。这经你们没有念过吗。

他们看出这比喻是指着他们说的,就想要捉拿他,只是惧怕百姓。于是离开他走了。

后来他们打发几个法利赛人和几个希律党的人,到耶稣那里,要就着他的话陷害他。

他们来了,就对他说,夫子,我们知道你是诚实的,什么人你都不徇情。因为你不看人的外貌,乃是诚诚实实传神的道。纳税给凯撒可以不可以。

我们该纳不该纳。耶稣知道他们的假意,就对他们说,你们为什么试探我。拿一个银钱来给我看。

他们就拿了来。耶稣说,这像和这号是谁的。他们说,是凯撒的。

耶稣说,凯撒的物当归给凯撒,神的物当归给神。他们就很希奇他。

撒都该人常说,没有复活的事。他们来问耶稣说,

夫子,摩西为我们写着说,人若死了,撇下妻子,没有孩子,他兄弟当娶他的妻,为哥哥生子立后。

有弟兄七人,第一个娶了妻,死了,没有留下孩子。

第二个娶了他,也死了,没有留下孩子。第三个也是这样。

那七个人都没有留下孩子。末了,那妇人也死了。

当复活的时候,他是那一个的妻子呢。因为他们七个人都娶过他。

耶稣说,你们所以错了,岂不是因为不明白圣经,不晓得神的大能吗。

人从死里复活,也不娶,也不嫁,乃像天上的使者一样。

论到死人复活,你们没有念过摩西的书,荆棘篇上所载的吗神对摩西说,我是亚伯拉罕的神,以撒的神,雅各的神。

神不是死人的神,乃是活人的神。你们是大错了。

有一个文士来,听见他们辩论,晓得耶稣回答的好,就问他说,诫命中那是第一要紧的呢

耶稣回答说,第一要紧的,就是说,以色列阿,你要听。主我们神,是独一的主。

你要尽心,尽性,尽意,尽力,爱主你的神。

其次,就是说,要爱人如己。再没有比这两条诫命更大的了。

那文士对耶稣说,夫子说,神是一位,实在不错。除了他以外,再没有别的神。

并且尽心,尽智,尽力,爱他,又爱人如己,就比一切燔祭,和各样祭祀,好的多。

耶稣见他回答的有智慧,就对他说,你离神的国不远了。从此以后,没有人敢再问他什么。

耶稣在殿里教训人,就问他们说,文士怎样说,基督是大卫的子孙呢。

大卫被圣灵感动说,主对我主说,你坐在我的右边,等我使你的仇敌作你的脚凳。

大卫既自己称他为主,他怎吗又是大卫的子孙呢。众人都喜欢听他。

耶稣在教训之间,说,你们要防备文士,他们好穿长衣游行,喜爱人在街市上问他们的安,

又喜爱会堂里的高位筵席上的首座。

他们侵吞寡妇的家产,假意作很长的祷告。这些人要受更重的刑罚。

耶稣对银库坐着,看众人怎样投钱入库。有好些财主,往里投了若干的钱。

有一个穷寡妇来,往里投了两个小钱,就是一个大钱。

耶稣叫门徒来,说,我实在告诉你们,这穷寡妇投入库里的,比众人所投的更多。

因为他们都是自己有馀,拿出来投在里头。但这寡妇是自己不足,把他一切养生的都投上了。

\chapter{马可福音第13章}
耶稣从殿里出来的时候,有一个门徒对他说,夫子,请看,这是何等的石头,何等的殿宇。

耶稣对他说,你看见这大殿宇吗。将来在这里没有一块石头留在石头上,不被拆毁了。

耶稣在橄榄山上对圣殿而坐。彼得,雅各,约翰,和安得烈,暗暗的问他说,

请告诉我们,什么时候有这些事呢。这一切事,将成的时候,有什么豫兆呢。

耶稣说,你们要谨慎,免得有人迷惑你们。

将来有好些人冒我的名来,说,我是基督。并且要迷惑许多人。

你们听见打仗,和打仗的风声,不要惊慌。这些事是必须有的,只是末期还没有到。

民要攻打民,国要攻打国,多处必有地震,饥荒。这都是灾难的起头。(灾难原文作生产之苦)

但你们要谨慎。因为人要把你们交给公会,并且你们在会堂里要受鞭打。又为我的缘故,站在诸侯与君王面前,对他们作见证。

然而福音必须先传给万民。

人把你们拉去交官的时候,不要豫先思虑说什么。到那时候,赐给你们什么话,你们就说什么,因为说话的不是你们,乃是圣灵。

弟兄要把弟兄,父亲要把儿子,送到死地。儿女要起来与父母为敌,害死他们。

并且你们要为我的名,被众人恨恶,惟有忍耐到底的,必然得救。

你们看见那行毁坏可憎的,站在不当站的地方。(读这经的人,须要会意)那时在犹太的,应当逃到山上。

在房上的,不要下来,也不要进去拿家里的东西。

在田里的,也不要回去取衣裳。

当那些日子,怀孕的和奶孩子的有祸了。

你们应当祈求,叫这些事不在冬天临到。

因为在那些日子必有灾难,自从)神创造万物直到如今,并没有这样的灾难。后来也必没有。

若不是主减少那日子,凡有血气的,总没有一个得救的。只是为主的选民,他将那日子减少了。

那时若有人对你们说,看哪,基督在这里。或说,基督在那里。你们不要信。

因为假基督,假先知,将要起来,显神迹奇事。倘若能行,就把选民迷惑了。

你们要谨慎。看哪,凡事我都豫先告诉你们了。

在那些日子,那灾难以后,日头要变黑了,月亮也不放光,

众星要从天上坠落,天势都要震动。

那时他们(马太二十四章三十节作地上的万族)要看见人子有大能力,大荣耀,驾云降临。

他要差遣天使,把他的选民,从四方,从地极直到天边,都招聚了来。(方原文作风)

你们可以从无花果树学个比方。当树枝发嫩长叶的时候,你们就知道夏天近了。

这样,你们几时看见这些事成就,也该知道人子近了,(人子或作神的国)正在门口了。

我实在告诉你们,这世代还没有过去,这些事都要成就。

天地要废去。我的话却不能废去。

但那日子,那时辰,没有人知道,连天上的使者也不知道,子也不知道,惟有父知道。

你们要谨慎,儆醒祈祷,因为你们不晓得那日期几时来到。

这事正如一个人离开本家,寄居外邦,把权柄交给仆人,分派各人当作的工,又吩咐看门的儆醒。

所以你们要儆醒,因为你们不知道家主什么时候来,或晚上,或半夜,或鸡叫,或早晨。

恐怕他忽然来到,看见你们睡着了。

我对你们所说的话,也是对众人说,要儆醒。

\chapter{马可福音第14章}
过两天是逾越节,又是除酵节。祭司长和文士,想法子怎吗用诡计捉拿耶稣杀他。

只是说,当节的日子不可,恐怕百姓生乱。

耶稣在伯大尼长大麻疯的西门家里坐席的时候,有一个女人,拿着一玉瓶至贵的真哪哒香膏来,打破玉瓶,把香膏浇在耶稣的头上。

有几个人心中很不喜悦,说,何用这样枉费香膏呢。

这香膏可以卖三十多两银子周济穷人。他们就向那女人生气。

耶稣说,由他吧。为什么难为他呢。他在我身上作的是一件美事。

因为常有穷人和你们同在,要向他们行善,随时都可以。只是你们不常有我。

他所作的,是尽他所能的。他是为我安葬的事,把香膏豫先浇在我身上。

我实在告诉你们,普天之下,无论在什么地方传这福音,也要述说这女人所作的以为记念。

十二门徒之中有一个加略人犹大,去见祭司长,要把耶稣交给他们。

他们听见就欢喜,又应许给他银子。他就寻思如何得便,把耶稣交给他们。

除酵节的第一天,就是宰逾越节羊羔的那一天,门徒对耶稣说,你吃逾越节的筵席,要我们往那里去豫备呢。

耶稣就打发两个门徒,对他们说,你们进城去,必有人拿着一瓶水,迎面而来。你们就跟着他。

他进那家去,你们就对那家的主人说,夫子说,客房在那里,我与门徒好在那里吃逾越节的筵席。

他必指给你们摆设整齐的一间大楼,你们就在那里为我们豫备。

门徒出去,进了城,所遇见的,正如耶稣所说的。他们就豫备了逾越节的筵席

到了晚上,耶稣和十二个门徒都来了。

他们坐席正吃的时候,耶稣说,我实在告诉你们,你们中间有一个与我同吃的人要卖我了。

他们就忧愁起来,一个一个的问他说,是我吗。

耶稣对他们说,是十二个门徒同我蘸手在盘子里的那个人。

人子必要去世,正如经上指着他所写的。但卖人子的人有祸了,那人不生在世上倒好。

他们吃的时候,耶稣拿起饼来,祝了福,就擘开递给他们说,你们拿着吃。这是我的身体。

又拿起杯来,祝谢了,递给他们。他们都喝了。

耶稣说,这是我立约的血,为多人流出来的。

我实在告诉你们,我不再喝这葡萄汁,直到我在神的国里,喝新的那日子。

他们唱了诗,就出来,往橄榄山去。

耶稣对他们说,你们都要跌倒了。因为经上记着说,我要击打牧人,羊就分散了。

但我复活以后,要在你们以先往加利利去。

彼得说,众人随然跌倒,我总不能。

耶稣对他说,我实在告诉你,就在今天夜里,鸡叫两遍以先,你要三次不认我。

彼得却极力的说,我就是必须和你同死,也总不能不认你。众门徒都是这样说。

他们来到一个地方,名叫客西马尼。耶稣对门徒说,你们坐在这里,等我祷告。

于是带着彼得,雅各,约翰同去,就惊恐起来,极其难过。

对他们说,我心里甚是忧伤,几乎要死。你们在这里,等候儆醒。

他就稍往前走,俯伏在地祷告说,倘若可行,便叫那时候过去。

他说,阿爸,父阿,在你凡事都能。求你将这杯撤去。然而不要从我的意思,只要从你的意思。

耶稣回来,见他们睡着了,就对彼得说,西门,你睡觉吗,不能儆醒片时吗。

总要儆醒祷告,免得入了迷惑。你们心灵固然愿意,肉体却软弱了。

耶稣又去祷告,说的话还是与先前一样。

又来见他们睡着了,因为他们的眼睛甚是困倦。他们也不知道怎吗回答。

第三次来,对他们说,现在你们仍然睡觉安歇吧。(吧或作吗)够了,时候到了。看哪,人子被卖在罪人手里了。

起来,我们走吧。看哪,那卖我的人近了。

说话之间,忽然那十二个门徒里的犹大来了,并有许多人带着刀棒,从祭司长和文士并长老那里与他同来。

卖耶稣的人曾给他们一个暗号,说,我与谁亲嘴,谁就是他。你们把他拿住,牢牢靠靠的带去。

犹大来了,随既到耶稣跟前说,拉比,便与他亲嘴。

他们就下手拿住他。

旁边站着的人,有一个拔出刀来,将大祭司的仆人砍了一刀,削掉了他一个耳朵。

耶稣对他们说,你们带着刀棒,出来拿我,如同拿强盗吗。

我天天教训人,同你们在殿里,你们并没有拿我。但这事成就,为要应验经上的话。

门徒都离开他逃走了。

有一个少年人,赤身披着一块麻布,跟随耶稣,众人就捉拿他。

他却丢了麻布,赤身逃走了。

他们把耶稣带到大祭司那里。又有众祭司长和长老并文士,都来和大祭司一同聚集。

彼得远远的跟着耶稣,一直进入大祭司的院里,和差役一同坐在火光里烤火。

祭司和全公会寻梢见证控告耶稣,要治死他。却寻不着。

因为有好些人作假见证告他,只是他们的见证,各不相和。

又有几个人站起来,作假见证告他说,

我们听见他说,我要拆毁这人手所造的殿,三日内就另造一座不是人手所造的。

他们就这样作见证,也是各不相合。

大祭司起来,站在中间,问耶稣说,你什么都不回答吗。这些人作见证告你的是什么呢。

耶稣却不言语,一句也不回答。大祭司又问他说,你是那当称颂者的儿子基督不是。

耶稣说,我是。你们必看见人子,坐在那权能者的右边,驾着天上的云降临。

大祭司就撕开衣服,说,我们何必再用见证人呢。

你们已经听见他这僭妄的话了。你们的意见如何。他们都定他该死的罪。

就有人吐唾沫在他脸上,又蒙着他的脸,用拳头打他,对他说,你说预言吧。差役接过他来用手掌打他。

彼得在下边,院子里,来了大祭司的一个使女。

见彼得烤火,就看着他说,你素来也是同拿撒勒人耶稣一夥的。

彼得却不承认,说,我不知道,也不明白你说的是什么。于是出来,到了前院。鸡就叫了。

那使女看见他,又对旁边站着的人说,这也是他们一党的。

彼得又不承认。过了不多的时候,旁边站着的人又对彼得说,你真是他们一党的。因为你是加利利人。

彼得就发咒起誓的说,我不认得你们说的这个人。

立时鸡叫了第二遍。彼得想起耶稣对他所说的话,鸡叫两遍以先,你要三次不认我。思想起来,就哭了。

\chapter{马可福音第15章}
一到早晨,祭司长和长老文士全公会的人大家商议,就把耶稣捆绑解去,交给彼拉多。

彼拉多问他说,你是犹太人的王吗。耶稣回答说,你说的是。

祭司长告他许多的事。

彼拉多又问他说,你看,他们告你这吗多的事,你什么都不回答吗。

耶稣仍不回答,以致彼拉多觉得希奇。

每逢这节期,巡抚照众人所求的,释放一个囚犯给他们。

有一个人名叫巴拉巴,和作乱的人一同捆绑。他们作乱的时后,曾杀过人。

众人上去求巡抚,照常例给他们办。

彼拉多说,你们要我释放犹太人的王给你们吗。

他原晓得祭司长是因为嫉妒才把耶稣解了来。

只是祭司长挑唆众人,宁可释放巴拉巴给他们。

彼拉多又说,那吗样你们所称为犹太人的王,我怎吗办他呢。

他们又喊着说,把他钉十字架。

彼拉多说,为什么呢,他作了什么恶事呢。他们便极力的喊着说,把他钉十字架。

彼拉多要叫众人喜悦,就释放巴拉巴给他们,将耶稣鞭打了,交给人钉十字架。

兵丁把耶稣带进衙门院里。叫齐了全营的兵。

他们给他穿上紫袍,又用荆棘编作冠冕给他戴上。

就厌贺他说,恭喜犹太人的王阿。

又拿一根苇子,打他的头,吐唾沫在他脸上屈膝拜他。

戏弄完了,就给他脱了紫袍,仍穿上他自己的衣服,带他出去,要钉十字架。

有一个古利奈人西门,就是亚历山大和鲁孚的父亲,从乡下来,经过那地方。他们就勉强他同去,好背着耶稣的十字架。

他们带耶稣到了各各他地方,(各各他翻出来,就是髑髅地)

拿没药调和的酒给耶稣,他却不受。

于是将他钉在十字架上,拈阄分他的衣服,看是谁得什么。

他在十字架上,是巳初的时候。

在上面有他的罪状,写的是犹太人的王。

他们又把两个强盗,和他同钉十字架。一个在右边,一个在左边。(有古卷在此有

这就应了经上的话说,他被列在罪犯之中)

从那里经过的人辱骂他,摇着头说,咳,你这拆毁圣殿,三日又建造起来的。

可以救自己从十字架上下来吧。

祭司长和文士也是这样戏弄他,彼此说,他就了别人,不能救自己。

以色列的王基督,现在可以从十字架上下来,叫我们看见,就信了。那和他同钉的人也是讥诮他。

从午正到申初遍地都黑暗了。

申初的时候,耶稣大声喊着说,以罗伊,以罗伊,拉马撒巴各大尼。翻出来,就是,我的神,我的神,为什么离弃我。

旁边站着的人,有的听见就说,看哪,他叫以利亚呢。

有一个人跑去,把海绒蘸满了醋,绑在苇子上,送给他喝,说,且等着,看以利亚来不来把他取下。

耶稣大声喊叫,气就断了。

殿里的幔子,从上到下裂为两半。

对面站着的百夫长,看见耶稣这样喊叫断气,(有古卷无喊叫二字)就说,这人真是神的儿子。

还有些妇女,远远的观看。内中有抹大拉的马利亚,又有小雅各和约西的母亲马利亚,并有撒罗米。

就是耶稣在加利利的时候,跟随他,服事他的那些人,还有同耶稣上耶路撒冷的好些妇女在那里观看。

到了晚上,因为这是豫备日,就是安息日的前一日,

有亚利马太的约瑟前来,他是尊贵的议士,也是等候神国的。他放胆进去见彼拉多,求耶稣的身礼。

彼拉多诧异耶稣已经死了。便叫百夫长来,问他耶稣死了久不久。

既从百夫长得知实情,就把耶稣的尸首赐给约瑟。

约瑟买了细麻布,把耶稣取下来,用细麻布裹好,安放在磐石凿出来的坟墓里。又辊过一个石头来挡住墓门。

抹大拉的马利亚,和约西的母亲马利亚。都看见安放他的地方。

\chapter{马可福音第16章}
过了安息日,抹大拉的马利亚,和雅各的母亲马利亚,并撒罗米,买了香膏,要去膏耶稣的身体。

七日的第一日清早,出太阳的时候,他们来到坟墓那里。

彼此说,谁给我们把石头从墓门辊开呢。

那石头原来很大,他们抬头一看,却见石头已经辊开了。

他们进了坟墓,看见一个少年人坐在右边,穿着白袍。就甚惊恐。

那少年人对他们说,不要惊恐。你们寻梢那钉十字架的拿撒勒人耶稣。他已经复活了,不在这里。请看安放他的地方。

你们可以去告诉他的门徒和彼得说,他在你们以先往加利利去。在那里你们要见他,正如他从前所告诉你们的。

他们就出来,从坟墓那里逃跑。又发抖,又惊奇,什么也不告诉人。因为他们害怕。

在七日的第一日清早,耶稣复活了,就先向抹大拉的马利亚显现。耶稣从他身上曾赶出七个鬼。

他去告诉那向来跟随耶稣的人。那时他们正哀恸哭泣。

他们听见耶稣活了,被马利亚看见,却是不信。

这事以后,门徒中间有两个人,往乡下去。走路的时候,耶稣变了形像向他们显现,

他们就去告诉其馀的门徒。其馀的门徒,也是不信。

后来,十一个门徒坐席的时候,耶稣向他们显现,责备他们不信,心里刚硬。因为他们不信那些在他复活以后看见他的人。

他又对他们说,你们往普天下去,传福音给万民听。(万民原文作凡受造的)

信而受洗的必然得救。不信的必被定罪。

信的人必有神迹随着他们,就是奉我的名赶鬼。说新方言。

手能拿蛇。若喝了什么毒物,也必不受害。手按病人,病人就必好了。

主耶稣和他们说完了话,后来被接到天上,坐在神的右边。

门徒出去,到处宣传福音,主和他们同工,用神迹随着,证实所传的道。阿们。

\chapter{路加福音第1章}
提阿非罗大人哪,有好些人提笔作书,

*述说在我们中间所成就的事,是照传道的人,从起初亲眼看见,又传给我们的

这些事我既从起头都详细考察了,就定意要按着次序写给你,

使你知道所学之道都是确实的。

当犹太王希律的时候,亚比雅班里有一个祭司,名叫撒迦利亚。他妻子是亚伦的后人,名叫伊利莎白。

他们二人,在神面前都是义人,遵行主的一切诫命礼仪,没有可指摘的。

只是没有孩子,因为伊利莎白不生育,两个人又年纪老迈了。

撒迦利亚按班次,在神前面供祭司的职分,

照祭司的规矩掣签,得进主殿烧香。

烧香的时后,众百姓在外面祷告。

有主的使者站在香坛的右边,向他显现。

撒迦利亚看见,就惊慌害怕。

天使对他说,撒迦利亚,不要害怕。因为你的祈祷已经被听见了,你的妻子伊利莎白要给你生一个儿子,你要给他起名叫约翰。

你必喜欢快乐,有许多人因他出世,也必喜乐。

他在主面前将要为大,淡酒浓酒都不喝,从母腹里就被圣灵充满了。

他要使许多以色列人回转,归于他们的神。

他必有以利亚的心志能力,行在主的前面,叫为父的心转向儿女,叫悖逆的人转从义人的智慧。又为主豫备合用的百姓。

撒迦利亚对天使说,我凭着什么可知道这事呢,我已经老了,我的妻子也年纪老迈了。

天使回答说,我是站在神面前的加百列,奉差而来,对你说话,将这好信息报给你。

到了时候,这话必然应验。只因你不信,你必哑吧不能说话,直到这事成就的日子。

百姓等候撒迦利亚,诧异他许久在殿里。

及至他出来,不能和他们说话。他们知道他在殿里见了异象。因为他直向他们打手式,竟成了哑吧。

他供职的日子已满,就回家去了。

这些日子以后,他的妻子伊利莎白怀了孕,就隐藏了五个月,

说,主在眷顾我的日子,这样看待我,要把我在人间的羞耻除掉。

到了第六个月,天使加百列奉神的差遣,往加利利的一座城去,这城名叫拿撒勒。

到一个童女那里,是已经许配大卫家的一个人,名叫约瑟,童女的名字叫马利亚。

天使进去,对他说,蒙大恩的女子,我问你安,主和你同在了。

马利亚因这话就很惊慌,又反复思想这样问安是什么意思。

天使对他说,马利亚不要怕。你在神面前已经蒙恩了。

你要怀孕生子,可以给他起名叫耶稣。

他要为大,称为至高者的儿子。主神要把他祖大卫的位给他。

他要作雅各家的王,直到永远。他的国也没有穷尽。

马利亚对天使说,我没有出嫁,怎吗有这事呢。

天使回答说,圣灵要临到你身上,至高者的能力要荫庇你。因此所要生的圣者,必称为神的儿子。(或作所要生的必称为圣称为神的儿子)

况且你的亲戚伊利莎白,在年老的时候,也怀了男胎。就是那素来称为不生育的,现在有孕六个月了。

因为出于神的话,没有一句不带能力的。

马利亚说,我是主的使女,情愿照你的话成就在我身上。天使就离开他去了。

那时候马利亚起身,急忙往山地里去,来到犹大的一座城。

进了撒迦利亚的家,问伊利莎白安。

伊利莎白一听马利亚问安,所怀的胎就在腹里跳动,伊利莎白且圣灵充满。

高声喊着说,你在妇女中是有福的,你所怀的胎也是有福的。

我主的母到我这里来,这是从那里得的呢。

因为你问安的声音,一入我耳,我腹里的胎,就欢喜跳动。

这相信的女子是有福的。因为主对他所说的话,都要应验。

马利亚说,我心尊主为大,

我灵以神我的救主为乐。

因为他顾念他使女的卑微。从今以后,万代要称我有福。

那有权能的为我成就了大事。他的名为圣。

他怜悯敬畏他的人,直到世世代代。

他用膀臂施展大能。那狂傲的人,正心里妄想,就被他赶散了。

他叫有权柄的失位,叫卑贱的升高。

叫饥饿的得饱美食,叫富足的空手回去。

他扶助了他的仆人以色列,

为要记念亚伯拉罕和他的后裔,施怜悯,直到永远,正如从前对我们列祖所说的话。

马利亚和伊利莎白同住,约有三个月,就回家了。

伊利莎白的产期到了,就生了一个儿子。

邻里亲族,听见主向他大施怜悯,就和他一同欢乐。

到了第八日,他们来要给孩子行割礼。并要照他父亲的名字,叫他撒迦利亚。

他母亲说,不可。要叫他约翰。

他们说,你亲族中没有叫这名字的。

他们就向他父亲打手式,问他要叫这孩子什么名字。

他要了一块写字的板,就写上说,他的名字是约翰。他们便都希奇。

撒迦利亚的口立时开了,舌头也舒展了,就说出话来,称颂神。

周围居住的人都惧怕,这一切的事就传遍了犹太的山地。

凡听见的人,都将这事放在心里,说,这个孩子,将来怎吗样呢。因为有主与他同在。

他父亲撒迦利亚,被圣灵充满了,就预言说,

主以色列的神,是应当称颂的。因他眷顾他的百姓,为他们施行救赎。

在他仆人大卫家中,为我们兴起了拯救的角,

(正如主藉着从创世以来,圣先知的口所说的话)

拯救我们脱离仇敌,和一切恨我们之人的手。

向我们列祖施怜悯,记念他的圣约。

就是他对我们祖宗亚伯拉罕所起的誓,

叫我们既从仇敌手中被救出来,

就可以终身在他面前,坦然无惧的用圣洁公义事奉他。

孩子阿,你要称为至高者的先知。因为你要行在主的前面,豫备他的道路。

叫他的百姓因罪得赦,就知到救恩。

因我们神怜悯的心肠,叫清晨的日光从高天临到我们,

要照亮黑暗中死荫里的人。把我们的脚引到平安的路上。

那孩子渐渐长大,心灵强健,住在旷野,直到他显明在以色列人面前的日子。

\chapter{路加福音第2章}
当那些日子,凯撒奥古斯都有旨下来,叫天下人民都报名上册。

这是居里扭作叙利亚巡抚的时候,头一次行报名上册的事。

众人各归各城,报名上册。

约瑟也从加利利的拿撒勒城上犹太去,到了大卫的城,名叫伯利恒,因他本是大卫一族一家的人。

要和他所聘之妻马利亚,一同报名上册。那时马利亚的身孕已经重了。

他们在那里的时候,马利亚的产期到了。

就生了头胎的儿子,用布包起来,放在马槽里,因为客店里没有地方。

在伯利恒之野地里有牧羊人,夜间按着更次看守羊群。

有主的使者站在他们旁边,主的荣光四面照着他们。牧羊的人就甚惧怕。

那天使对他们说,不要惧怕,我报给你们大喜的信息,是关乎万民的。

因今天在大卫的城里,为你们生了救主,就是主基督。

你们要看见一个婴孩,包着布,卧在马槽里,那就是记号了。

忽然有一大队天兵,同那天使赞美神说,

在至高之处荣耀归与神,在地上平安归与他所喜悦的人。(有古卷作喜悦归与人)。

众天使离开他们升天去了,牧羊的人彼此说,我们往伯利恒去,看看所成的事,就是主所指示我们的。

他们急忙去了,就寻见马利亚和约瑟,又有那婴孩卧在马槽里。

既然看见,就把天使论这孩子的话传开了。

凡听见的,就诧异牧羊人对他们所说的话。

马利亚却把这一切事存在心里,反复思想。

牧羊的人回去了,因所听见所看见的一切事,正如天使向他们所说的,就归荣耀与神,赞美他。

满了八天,就给孩子行割礼,与他起名叫耶酥,这就是没有成胎以前,天使所起的名。

按摩西律法满了洁净日子,他们带着孩子上耶路撒冷去,要把他献与主。

(正如主的律法上所记,凡头生的男子,必称圣归主。)

又要照主的律法上所说,或用一对班鸠,或用两只雏鸽献祭。

在耶路撒冷有一个人名叫西面,这人又公义又虔诚,素常盼望以色列的安慰者来到,又有圣灵在他身上。

他得了圣灵的启示,知道自己未死以前,必看见主所立的基督。

他受了圣灵的感动,进入圣殿。正遇见耶稣的父母抱着孩子进来,要照律法的规矩办理。

西面就用手接过他来,称颂神说,

主阿,如今可以照你的话,释放仆人安然去世。

因为我的眼睛已经看见你的救恩。

就是你在万民面前所豫备的。

是照亮外邦人的光,又是你民以色列的荣耀。

孩子的父母,因这论耶稣的话就希奇。

西面给他们祝福,

又对孩子的母亲马利亚说,这孩子被立,是要叫以色列中许多人跌倒,许多人兴起。又要作毁谤的话柄。叫许多人心里的意念显露出来。你自己的心也要被刀刺透。

又有女先知名叫亚拿,是亚设支派法内力的女儿,年纪已经老迈,从作童女出嫁的时候,同丈夫住了七年,就寡居了。

现在已经八十四岁,(或作就寡居了八十四年)并不离开圣殿,禁食祈求,昼夜事奉神。

正当那时,他进来称谢神,将孩子的事,对一切盼望耶路撒冷得救赎的人讲说。

约瑟和马利亚,照主的律法,办完了一切的事,就回加利利,到自己的城拿撒勒去了。

孩子渐渐长大,强健起来,充满智慧。又有神的恩在他身上。

每年到逾越节,他父母就上耶路撒冷去。

当他十二岁的时候,他们按着节期的规矩上去。

守满了节期,他们回去,孩童的耶稣仍旧在耶路撒冷。他的父母并不知道。

以为他在同行的人中间,走了一天的路程,就在亲族和熟识的人中找他。

既找不着,就回耶路撒冷去找他。

过了三天,就遇见他在殿里,坐在教师中间,一面听,一面问。

凡听见他的,都希奇他的聪明,和他的应对。

他父母看见就很希奇。他母亲对他说,我儿,为什么向我们这样行呢。看哪,你父亲和我伤心来找你。

耶稣说,为什么找我呢。岂不知我应当以我父的事为念吗。(或作岂不知我应当在我父的家里吗)

他所说的这话,他们不明白。

他就同他们下去,回到拿撒勒。并且顺从他们。他母亲把这一切的事都存在心里。

耶稣的智慧和身量,(身量或作年纪)并神和人喜爱他的心,都一齐增长。

\chapter{路加福音第3章}
凯撒提庇留在位第十五年,本丢彼拉多作犹太巡抚,希律作加利利分封的王,他兄弟腓力作以土利亚和特拉可尼地方分封的王,吕撒聂作亚比利尼分封的王,

亚那和该亚法作大祭司,那时,撒迦利亚的儿子约翰在旷野里,神的话临到他。

他就来到约旦河一带地方,宣讲悔改的洗礼,使罪得赦。

正如先知以赛亚书上所记的话,说,在旷野有人声喊着说,豫备主的道,修直他的路。

一切山洼都要填满,大小山冈都要削平。弯弯曲曲的地方要改为正直,高高低低的道路要改为平坦。

凡是血气的,都要见神的救恩。

约翰对那出来要受他洗的众人说,毒蛇的种类,谁指示你们逃避将来的忿怒呢。

你们要结出果子来,与悔改的心向称,不要自己心里说,有亚伯拉罕为我们的祖宗。我告诉你们,神能从这些石头中,给亚伯拉罕兴起子孙来。

现在斧子已经放在树根上,凡不结好果子的树,就砍下来丢在火里。

众人问他说,这样我们当作什么呢。

约翰回答说,有两件衣裳的,就分给那没有的。有食物的也当这样行。

又有税吏来受洗,问他说,夫子,我们当作什么呢。

约翰说,除了例定的数目,不要多取。

又有兵丁问他说,我们当作什么呢。约翰说,不要以强暴待人,也不要讹诈人,自己有钱粮就当知足。

百姓指望基督来的时候,人都心猜疑,或者约翰是基督。

约翰说,我是用水给你们施洗,但有一位能力比我更大的要来,我就是给他解鞋带也不配。他要用圣灵与火给你们施洗。

他手里拿着簸箕,要扬净他的场,把麦子收在仓里,把糠用不灭的火烧尽了。

约翰又用许多别的话劝百姓,向他们传福音。

只是分封的王希律,因他兄弟之妻希罗底的缘故,并因他所行的一切恶事,受了约翰的责备,

又另外添了一件,就是把约翰收在监里。

众百姓都受了洗,耶稣也受了洗,正祷告的时候,天就开了,

圣灵降临在他身上,形状妨佛鸽子。又有声音从天上来,说,你是我的爱子,我喜悦你。

耶稣开头传道,年纪约有三十岁,依人看来,他是约瑟的儿子,约瑟是希里的儿子,

希里是玛塔的儿子,玛塔是利未的儿子,利未是麦基的儿子,麦基是雅拿的儿子,雅拿是约瑟的儿子。

约瑟是玛他提亚的儿子,玛他提亚是亚摩斯的儿子,亚摩斯是拿鸿的儿子,拿鸿是以斯利的儿子,以斯利是拿该的儿子,

拿该是玛押的儿子,玛押是玛他提亚的儿子,玛他提亚是西美的儿子,西美是约瑟的儿子,约瑟是犹大的儿子,犹大是约亚拿的儿子,

约亚拿是利撒的儿子,利撒是所罗巴伯的儿子,所罗巴伯是撒拉铁的儿子,撒拉铁是尼利的儿子,尼利是麦基的儿子,

麦基是亚底的儿子,亚底是哥桑的儿子,哥桑是以摩当的儿子,以摩当是珥的儿子,珥是约细的儿子,

约细是以利以谢的儿子,以利以谢是约令的儿子,约令是玛塔的儿子,玛塔是利未的儿子,

利未是西缅的儿子,西缅是犹大的儿子,犹大是约瑟的儿子,约瑟是约南的儿子,约南是以利亚敬的儿子,

以利亚敬是米利亚的儿子,米利亚是买南的儿子,买南是玛达他的儿子,玛达他是拿单的儿子,拿单是大卫的儿子,

大卫是耶西的儿子,耶西是俄备得的儿子,俄备得是波阿斯的儿子,波阿斯是撒门的儿子,撒门是拿顺的儿子,

拿顺是亚米拿达的儿子,亚米拿达是亚兰的儿子,亚兰是希斯仑的儿子,希斯仑是法勒斯的儿子,法勒斯是犹大的儿子,

犹大是雅各的儿子,雅各是以撒的儿子,以撒是亚伯拉罕的儿子,亚伯拉罕是他拉的儿子,他拉是拿鹤的儿子,

拿鹤是西鹿的儿子,西鹿是拉吴的儿子,拉吴是法勒的儿子,法勒是希伯的儿子,希伯是沙拉的儿子,

沙拉是该南的儿子,该南是亚法撒的儿子,亚法撒是闪的儿子,闪是挪亚的儿子,挪亚是拉麦的儿子,

拉麦是玛土撒拉的儿子,玛土撒拉是以诺的儿子,以诺是雅列的儿子,雅列是玛勒列的儿子,玛勒列是该南的儿子,该南是以挪士的儿子,

以挪士是塞特的儿子,塞特是亚当的儿子,亚当是神的儿子。

\chapter{路加福音第4章}
耶稣被圣灵充满,从约旦河回来,圣灵将他引到旷野,四十天受魔鬼的试探。

那些日子没有吃什吗。日子满了,他就饿了。

魔鬼对他说,你若是神的儿子,可以吩咐这块石头变成食物。

耶稣回答说,经上记着说,人活着不是单靠食物,乃是靠神口里所出的一切话

魔鬼又领他上了高山,霎时间把天下的万国都指给他看。

对他说,这一切权柄荣华,我都要给你。因为这原是交付我的,我愿意给谁就给谁。

你若在我面前下拜,这都要归你。

耶稣说,经上记着说,当拜主你的神,单要事奉他。

魔鬼又领他到耶路撒冷去,叫他站在殿顶上,(顶原文作翅)对他说,你若是神的儿子,可以从这里跳下去。

因为经上记着说,主要为你吩咐他的使者保护你。

他们要用手托着你,免得你的脚碰在石头上。

耶稣对他说,经上说,不可试探主你的神。

魔鬼用完了各样的试探,就暂时离开耶稣。

耶稣满有圣灵的能力回到加利利,他的名声就传遍了四方。

他在各会堂里教训人,众人都称赞他。

耶稣来到拿撒勒,就是他长大的地方。在安息日,照他平常的规矩,进了会堂,站起来要念圣经。

有人把先知以赛亚的书交给他,他就打开,找到一处写着说,

主的灵在我身上,因为他用膏膏我,叫我传福音给贫穷的人。差遣我报告被掳的得释放,瞎眼的得看见,叫那受压制的得自由,

报告神悦纳人的禧年。

于是把书卷起来,交还执事,就坐下。会堂里的人都定睛看他。

耶稣对他们说,今天这经应验在你们耳中了。

众人都称赞他,并希奇他口中所出的恩言。又说,这不是约瑟的儿子吗。

耶稣对他们说,你们必引这俗语向我说,医生,你医治自己吧。我们听见你在迦百农所行的事,也当行在你自己家乡里。

又说,我实在告诉你们,没有先知在自己家乡被人悦纳的。

我对你们说实话,当以利亚的时候,天闭塞了三年零六个月,偏地有大饥荒,那时,以色列中有许多寡妇。

以利亚并没有奉差往他们一个人那里去,只奉差往西顿的撒勒法,一个寡妇那里去。

先知以利沙的时候,以色列中有许多长大麻疯的。但内中除了叙利亚国的乃缦,没有一个得洁净的。

会堂里的人听见这话,都怒气满胸。

就起来撵他出城,他们的城造在山上,他们带他到山崖,要把他推下去。

他却从他们中间直行,过去了。

耶稣下到迦百农,就是加利利的一座城,在安息日教训众人。

他们很希奇他的教训,因为他的话里有权柄。

在会堂里有一个人,被污鬼的精气附着,大声喊叫说,

唉,拿撒勒的耶稣,我们与你有什吗相干,你来灭我们吗,我知道你是谁,乃是神的圣者。

耶稣责备他说,不要作声,从这人身上出来吧。鬼把那人摔倒在众人中间,就出来了,却也没有害他。

众人都惊讶,彼此对问说,这是什吗道理呢。因为他用权柄能力吩咐污鬼,污鬼就出来。

于是耶稣的名声传遍了周围地方。

耶稣出了会堂,进了西门的家。西门的岳母害热病甚重。有人为他求耶稣。

耶稣站在他旁边,斥责那热病,热就退了。他立刻起来服事他们。

日落的时候,凡有病人的,不论害什吗病,都带到耶稣那里。耶稣按手在他们各人身上,医好他们。

又有鬼从好些人身上出来,喊着说,你是神的儿子。耶稣斥责他们,不许他们说话,因为他们知道他是基督。

天亮的时候,耶稣出来,走到旷野地方。众人去找他,到了他那里,要留住他,不要他离开他们。

但耶稣对他们说,我也必须在别城传神国的福音。因我奉差原是为此。

于是耶稣在加利利的各会堂传道。

\chapter{路加福音第5章}
耶稣站在革尼撒勒湖边,众人拥挤他,要听神的道。

他见有两只船湾在湖边。打鱼的人却离开船,洗网去了。

有一只船,是西门的,耶稣就上去,请他把船撑开,稍微离岸,就坐下,从船上教训众人。

讲完了,对西门说,把船开到水深之处,下网打鱼。

西门说,夫子,我们整夜劳力,并没有打着什么。但依从你的话,我就下网。

他们下了网,就圈住许多鱼,网险些裂开。

便招呼那只船上的同伴来帮助。他们就来把鱼装满了两只船,甚至船要沉下去。

西门彼得看见,就俯伏在耶稣膝前,说,主阿,离开我,我是个罪人。

他和一切同在的人,都惊讶这一网所打的鱼。

他的夥伴西庇太的儿子,雅各,约翰,也是这样。耶稣对西门说,不要怕,从今以后,你要得人了。

他们把两只船拢了岸,就撇下所有的跟从了耶稣。

有一回耶稣在一个城里,有人满身长了大麻疯,看见他就俯伏在地,求他说,主若肯,必能叫我洁净了。

耶稣伸手摸他说,我肯,你洁净了吧。大麻疯立刻就离了他的身。

耶稣嘱咐他,你切不可告诉人。只要去把身体给祭司察看,又要为你得了洁净,照摩西所吩咐的,献上礼物对众人作证据。

但耶稣的名声越发传扬出去。有极多的人聚集来听道,也指望医治他们的病。

耶稣却退到旷野去祷告。

有一天耶稣教训人,有法利赛人和教法师在旁边坐着,他们是从加利利各乡村和犹太并耶路撒冷来的。主的能力与耶稣同在,使他能医治病人。

有人用褥子抬着一个瘫子,要抬进去放在耶稣面前,

却因人多,寻不出法子抬进去,就上了房顶,从瓦间把他连褥子缒到当中,正在耶稣面前。

耶稣见他们的信心,就对瘫子说,你的罪赦了。

文士和法利赛人就议论说,这说僭妄话的是谁。除了神以外,谁能赦罪呢。

耶稣知道他们所议论的,就说,你们心里议论的是什么呢。

或说,你的罪赦了,或说,你起来行走,那一样容易呢。

但要叫你们知道人子在地上有赦罪的权柄,就对瘫子说,我吩咐你起来,拿你的褥子回家去吧。

那人当众人面前,立刻起来,拿着他躺卧的褥子回家去,归荣耀与神。

众人都惊奇,也归荣耀与神,并满心惧怕,说,我们今日看见非常的事了。

这事以后,耶稣出去,看见一个税吏,名叫利未,坐在税关上,就对他说,你跟从我来。

他就撇下所有的,起来,跟从了耶稣。

利未在自己家里,为耶稣大摆筵席。有许多税吏和别人,与他们一同坐席。

法利赛人和文士,就向耶稣的门徒发怨言,说,你们为什么和税吏,并罪人,一同吃喝呢。

耶稣对他们说,无病的人用不着医生。有病的人才用得着。

我来本不是召义人悔改。乃是召罪人悔改。

他们说,约翰的门徒屡次禁食祈祷,法利赛人的门徒也是这样。惟独你的门徒又吃又喝。

耶稣对他们说,新郎和陪伴之人同在的时候,岂能叫陪伴之人禁食呢。

但日子将到,新郎要离开他们,那日他们就要禁食了。

耶稣又设一个比喻,对他们说,没有人把新衣服撕下一块来,补在旧衣服上。若是这样,就把新的撕破了,并且所撕下来的那块新的,和旧的也不相称。

也没有人把新酒装在旧皮袋里。若是这样,新酒必将皮袋裂开,酒便漏出来,皮袋也就坏了。

但新酒必须装在新皮袋里。

没有人喝了陈酒又想喝新的,他总说陈的好。

\chapter{路加福音第6章}
有一个安息日,耶稣从麦地经过。他的门徒掐了麦穗,用手搓着吃。

有几个法利赛人说,你们为什么作安息日不可作的事呢。

耶稣对他们说,经上记着大卫和跟从他的人,饥饿之时所作的事,连这个你们也没有念过吗。

他怎吗进了神的殿,拿陈设饼吃,又给跟从的人吃。这饼除了祭司以外,别人都不可吃。

又对他们说,人子是安息日的主。

又有一个安息日,耶稣进了会堂教训人。在那里有一个人右手枯乾了。

文士和法利赛人窥探耶稣,在安息日治病不治病。要得把柄去告他。

耶稣却知道他们的意念。就对那枯乾一只手的人说,起来,站在当中。那人就起来站着。

耶稣对他们说,我问你们,在安息日行善行恶,救命害命,那样是可以的呢。

他就周围看着他们众人,对那人说,伸出手来。他把手一伸,手就复了原。

他们就满心大怒,彼此商议,怎样处治耶稣。

那时耶稣出去上山祷告。整夜祷告神。

到了天亮,叫他的门徒来。就从他们中间挑选十二个人,称他们为使徒。

这十二个人有西门,耶稣又给他起名叫彼得,还有他兄弟安得烈,又有雅各和约翰,腓力和巴多罗买,

马太和多马,亚勒腓的儿子雅各,和奋锐党的西门,

雅各的儿子犹大,(儿子或作兄弟)和卖主的加略人犹大。

耶稣和他们下了山,站在一块平地上。同站的有许多门徒,又有许多百姓,从犹太全地,和耶路撒冷,并推罗西顿的海边来。都要听他讲道,又指望医治他们的病。

还有被污鬼缠磨的,也得了医治。

众人都想要摸他。因为有能力从他身上发出来,医好了他们。

耶稣举目看着门徒说,你们贫穷的人有福了。因为神的国是你们的。

你们饥饿的人有福了。因为你们将要饱足。你们哀哭的人有福了。因为你们将要喜笑。

人为人子恨恶你们,拒绝你们,辱骂你们,弃掉你们的名,以为是恶,你们就有福了。

当那日你们要欢喜跳跃。因为你们在天上的赏赐是大的。他们的祖宗待先知也是这样。

但你们富足的人有祸了。因为你们受过你们的安慰。

你们饱足的人有祸了。因为你们将要饥饿。你们喜笑的人有祸了。因为你们将要哀恸哭泣。

人都说你们好的时候,你们就有祸了。因为他们的祖宗待假先知也是这样。

只是我告诉你们这听道的人,你们的仇敌要爱他,恨你们的要待他好。

咒诅你们的要为他祝福,凌辱你们的要为他祷告。

有人打你这边的脸,连那边的脸也由他打。有人夺你的外衣,连里衣也由他拿去。

凡求你的,就给他。有人夺你的东西去,不用再要回来。

你们愿意人怎样待你们,你们也要怎样待人。

你们若单爱那爱你们的人,有什么可酬谢的呢。就是罪人也爱那爱他们的人。

你们若善待那善待你们的人,有什么可酬谢的呢。就是罪人也是这样行。

你们若借给人,指望从他收回,有什么可酬谢的呢。就是罪人也借给罪人,要如数收回。

你们倒要爱仇敌,也要善待他们,并要借给人不指望偿还。你们的赏赐就必大了,你们也必作至高者的儿子。因为他恩待那忘恩的和作恶的。

你们要慈悲,像你们的父慈悲一样。

你们不要论断人,就不被论断。你们不要定人的罪,就不被定罪。你们要饶恕人,就必蒙饶恕。(饶恕原文作释放)

你们要给人,就必有给你们的。并且用十足的升斗,连摇带按,上尖下流的,倒在你们怀里。因为你们用什么量器量给人,也必用什么量器量给你们。

耶稣又用比喻对他们说,瞎子岂能领瞎子,两个人不是都要掉在坑里吗。

学生不能高过先生。凡学成了的不过和先生一样。

为什么看见你弟兄眼中有刺,却不想自己眼中有梁木呢。

你不见自己眼中有梁木。怎能对你弟兄说,容我去掉你眼中的刺呢。你这假冒为善的人,先去掉自己眼中的梁木,然后才能看得清楚,去掉你兄弟眼中的刺。

因为没有好树结坏果子。也没有坏树结好果子。

凡树木看果子,就可以认出他来。人不是从荆棘上摘无花果,也不是从蒺??里摘葡萄。

善人从他心里所存的善,就发出善来。恶人从他心里所存的恶,就发出恶来。因为心里所充满的,口里就说出来。

你们为什么称呼我主阿,主阿,却不遵我的话行呢。

凡到我这里来,听见我的话就去行的,我要告诉你们他像什么人。

他像一个人盖房子,深深的挖地,把根基安在磐石上。到发大水的时候,水冲那房子,房子总不能摇动。因为根基立在磐石上。有古卷作因为盖造得好

惟有听见不去行的,就像一个人在土地上盖房子,没有根基。水一冲,随即倒塌了,并且那房子坏得很大。

\chapter{路加福音第7章}
耶稣对百姓讲完了这一切的话,就进了迦百农。

有一个百夫长所宝贵的仆人,害病快要死了。

百夫长风闻耶稣的事,就托犹太人的几个长老,去求耶稣来救他的仆人。

他们到了耶稣那里,就切切的求他说,你给他行这事,是他所配得的。

因为他爱我们的百姓,给我们建造会堂。

耶稣就和他们同去。离那家不远,百夫长托几个朋友去见耶稣,对他说,主阿,不要劳动。因你到我舍下,我不敢当。

我也自以为不配去见你,只要你说一句话,我的仆人就必好了。

因为我在人的权下,也有兵在我以下,对这个说去,他就去。对那个说来,他就来。对我的仆人说,你作这事,他就去作。

耶稣听见这话,就希奇他,转身对跟随的众人说,我告诉你们,这吗大的信心,就是在以色列中我也没有遇见过。

那托来的人回到百夫长家里,看见仆人已经好了。

过不了多时,有古卷作次日耶稣往一座城去,这城名叫拿因,他的门徒和极多的人与他同行。

将近城门,有一个死人被抬出来。这人是他母亲独生的儿子,他母亲又是寡妇。有城里的许多人同着寡妇送殡。

主看见那寡妇就怜悯他,对他说,不要哭。

于是进前按着杠,抬的人就站住了。耶稣说,少年人,我吩咐你起来。

那死人就坐起,并且说话。耶稣便把他交给他母亲。

众人都惊奇,归荣耀与神说,有大先知在我们中间兴起来了。又说,神眷顾了他的百姓。

他这事的风声就传遍了犹太,和周围地方。

约翰的门徒把这些事都告诉约翰。

他便叫了两个门徒来,打发他们到主那里去,说,那将要来的是你吗,还是我们等候别人呢。

那两个人来到耶稣那里,说,施洗的约翰打发我们来问你,那将要来的是你吗,还是我们等候别人呢。

正当那时候,耶稣治好了许多有疾病的,受灾患的,被恶鬼附着的。又开恩叫好些瞎子能看见。

耶稣回答说,你们去把所看见所听见的事告诉约翰。就是瞎子看见,瘸子行走,长大麻疯的洁净,聋子听见,死人复活,穷人有福音传给他们。

凡不因我跌倒的,就有福了。

约翰所差来的人既走了,耶稣就对众人讲论约翰说,你们从前出去到旷野,是要看什么呢。要看风吹动的芦苇吗。

你们出去到底是要看什么。要看穿细软衣服的人吗。那穿华丽衣服宴乐度日的人,是在王宫里。

你们出去究竟是要看什么。要看先知吗。我告诉你们,是的,他比先知大多了。

经上记着说,我要差遣我的使者在你前面,豫备道路。所说的就是这个人。

我告诉你们,凡妇人所生的,没有一个大过约翰的。然而神国里最小的比他还大。

众百姓和税吏,既受过约翰的洗,听见这话,就以神为义。

但法利赛人和律法师,没有受过约翰的洗,竟为自己废弃了神的旨意。二十九三十两节或作众百姓和税吏听见了约翰的话就受了他的洗便以神为义但法利赛人和律法师不受约翰的洗,竟为自己废弃了神的旨意。

主又说,这样,我可用什么比这世代的人呢。他们好像什么呢。

好像孩童坐在街市上,彼此呼叫说,我们向你们吹笛,你们不跳舞,我们向你们举哀,你们不啼哭。

施洗的约翰来,不吃饼,不喝酒。你们说他是被鬼附着的。

人子来,也吃也喝。你们说他是贪食好酒的人,是税吏和罪人的朋友。

但智慧之子,都以智慧为是。

有一个法利赛人,请耶稣和他吃饭。耶稣就到法利赛人家里去坐席。

那城里有一个女人,是个罪人。知道耶稣在法利赛人家里坐席,就拿着盛香膏的玉瓶,

站在耶稣背后,挨着他的脚哭,眼泪湿了耶稣的脚,就用自己的头发擦乾,又用嘴连连亲他的脚,把香膏抹上。

请耶稣的法利赛人看见这事,心里说,这人若是先知,必知道摸他的是谁,是个怎样的女人,乃是个罪人。

耶稣对他说,西门,我有句话要对你说。西门说,夫子,请说。

耶稣说,一个债主,有两个人欠他的债。一个欠他五十两银子,一个欠五两银子。

因为他们无力偿还,债主就开恩免了他们两个人的债。这两个人那一个更爱他呢。

西门回答说,我想是那多得恩免的人。耶稣说,你断的不错。

于是转过来向着那女人,便对西门说,你看见这女人吗。我进了你的家,你没有给我水洗脚。但这女人用眼泪湿了我的脚,用头发擦乾。

你没有与我亲嘴,但这女人从我进来的时候,就不住的用嘴亲我的脚。

你没有用油抹我的头,但这女人用香膏抹我的脚。

所以我告诉你,他许多的罪都赦免了。因为他的爱多。但那赦免少的,他的爱就少。

于是对那女人说,你的罪赦免了。

同席的人心里说,这是什么人,竟赦免人的罪呢。

耶稣对那女人说,你的信救了你,平平安安的回去吧。

\chapter{路加福音第8章}
过了不多日,耶稣周游各城各乡传道,宣讲神国的福音。和他同去的有十二个门徒,

还有被恶鬼所附,被疾病所累,已经治好的几个妇女,内中有称为抹大拉的马利亚,曾有七个鬼从他身上赶出来。

又有希律的家宰苦撒的妻子约亚拿,并苏撒拿,和好些别的妇女,都是用自己的财物供给耶稣和门徒。

当许多人聚集,又有人从各城里出来见耶稣的时候,耶稣就用比喻说,

有一个撒种的出去撒种。撒的时候,有落在路旁的,被人践踏,天上的飞鸟又来吃尽了。

有落在磐石上的,一出来就枯乾了,因为得不着滋润。

有落在荆棘里的,荆棘一同生长,把他挤住了。

又有落在好土里的,生长起来,结实百倍。耶稣说了这些话,就大声说,有耳可听的,就应当听。

门徒问耶稣说,这比喻是什么意思呢。

他说,神国的奥秘,只叫你们知道。至于别人,就用比喻,叫他们看也看不见,听也听不明。

这比喻乃是这样。种子就是神的道。

那些在路旁的,就是人听了道,随后魔鬼来,从他们心里把道夺去,恐怕他们信了得救。

那些在磐石上的,就是人听道,喜欢领受,但心中没有根,不过暂时相信,及至遇见试炼就后退了。

那落在荆棘里的,就是人听了道,走开以后,被今生的思虑钱财宴乐挤住了,便结不出成熟的子粒来。

那落在好土里的,就是人听了道,持守在诚实善良的心里,并且忍耐着结实。

没有人点灯用器皿盖上,或放在床底下,乃是放在灯台上,叫进来的人看见亮光。

因为掩藏的事,没有不显出来的。隐瞒的事,没有不露出来被人知道的。

所以你们应当小心怎样听。因为凡有的,还要加给他。凡没有的,连他自以为有的,也要夺去。

耶稣的母亲和他弟兄来了,因为人多,不得到他跟前。

有人告诉他说,你母亲和你弟兄,站在外边,要见你。

耶稣回答说,听了神之道而遵行的人,就是我的母亲,我的弟兄了。

有一天耶稣和门徒上了船,对门徒说,我们可以渡到湖那边去。他们就开了船。

正行的时候,耶稣睡着了。湖上忽然起了暴风,船将满了水,甚是危险。

门徒来叫醒了他,说,夫子,我们丧命喇。耶稣醒了,斥责那狂风大浪。风浪就止住,平静了。

耶稣对他们说,你们的信心在那里呢。他们又惧怕,又希奇,彼此说,这到底是谁,他吩咐风和水,连风和水也听从他了。

他们到了格拉森有古卷作加大拉人的地方,就是加利利的对面。

耶稣上了岸,就有城里一个被鬼附着的人,迎面而来,这个人许久不穿衣服,不住房子,只住在坟茔里。

他见了耶稣,就俯伏在他面前,大声喊叫,说,至高神的儿子耶稣,我与你有什么相干。求你不要叫我受苦。

是因耶稣曾吩咐污鬼从那人身上出来。原来这鬼屡次抓住他,他常被人看守,又被铁链和脚镣捆锁,他竟把锁链挣断,被鬼赶到旷野去。

耶稣问他说,你名叫什么。他说,我名叫群。这是因为附着他的鬼多。

鬼就央求耶稣,不要吩咐他们到无底坑里去。

那里有一大群猪,在山上吃食。鬼央求耶稣,准他们进入猪里去。耶稣准了他们。

鬼就从那人出来,进入猪里去。于是那群猪闯下山崖,投在湖里淹死了。

放猪的看见这事就逃跑了,去告诉城里和乡下的人。

众人出来要看是什么事。到了耶稣那里,看见鬼所离开的那人,坐在耶稣脚前,穿着衣服,心里明白过来,他们就害怕。

看见这事的,便将被鬼附着的人怎吗得救,告诉他们。

格拉森周围的人,因为害怕得很,都求耶稣离开他们。耶稣就上船回去了。

鬼所离开的那人,恳求和耶稣同在。耶稣却打发他回去,

说,你回家去,传说神为你作了何等大的事。他就去满城传扬耶稣为他作了何等大的事。

耶稣回来的时候,众人迎接他,因为他们都等候他。

有一个管会堂的,名叫睚鲁,来俯伏在耶稣脚前,求耶稣到他家里去。

因为他有一个独生女儿,约有十二岁,快要死了。耶稣去的时候,众人拥挤他。

有一个女人,患了十二年的血漏,在医生手里花尽了他一切养生的,并没有一人能医好他。

他来到耶稣背后,摸他的衣裳??子,血漏立刻就止住了。

耶稣说,摸我的是谁。众人都不承认,彼得和同行的人都说,夫子,众人拥拥挤挤紧靠着你。有古卷在此有你还问摸我的是谁吗

耶稣说,总有人摸我。因我觉得有能力从我身上出去。

那女人知道不能隐藏,就战战兢兢的来俯伏在耶稣脚前,把摸他的缘故,和怎样立刻得好了,当着众人都说出来。

耶稣对他说,女儿,你的信救了你,平平安安的去吧。

还说话的时候,有人从管会堂的家里来说,你的女儿死了,不要劳动夫子。

耶稣听见就对他说,不要怕,只要信,你的女儿就必得救。

耶稣到了他的家,除了彼得,约翰,雅各,和女儿的父母,不许别人同他进去。

众人都为这女儿哀哭捶胸。耶稣说,不要哭,他不是死了,是睡着了。

他们晓得女儿已经死了,就嗤笑耶稣。

耶稣拉着他的手,呼叫说,女儿,起来吧。

他的灵魂便回来,他立刻起来了。耶稣吩咐给他东西吃。

他的父母惊奇得很。耶稣嘱咐他们,不要把所作的事告诉人。

\chapter{路加福音第9章}
耶稣叫齐了十二个门徒,给他们能力权柄,制伏一切的鬼,医治各样的病。

又差遣他们去宣传神国的道,医治病人。

对他们说,行路的时候,不要带拐杖,和口袋,不要带食物,和银子,也不要带两件褂子。

无论进那一家,就住在那里,也从那里起行。

凡不接待你们的,你们离开那城的时候,要把脚上的尘土跺下去,见证他们的不是。

门徒就出去,走遍各乡,宣传福音,到处治病。

分封的王希律听见耶稣所作的一切事,就游移不定。因为有人说,是约翰从死里复活。

又有人说,是以利亚显现。还有人说,是古时的一个先知又活了。

希律说,约翰我已经斩了,这却是什么人,我竟听见这样的事呢,就想要见他。

使徒回来,将所作的事告诉耶稣。耶稣就带他们暗暗的离开那里,往一座城去,那城名叫伯赛大。

但众人知道了,就跟着他去。耶稣便接待他们,对他们讲论神国得道,医治那些需医的人。

日头快要平西,十二个门徒来对他说,请叫众人散开,他们好往四面乡里去借宿找吃的。因为我们这里是野地。

耶稣说,你们给他们吃吧。门徒说,我们不过有五个饼,两条鱼。若不去为这许多人买食物就不彀。

那时,人数约有五千。耶稣对门徒说,叫他们一排一排的坐下,每排大约五十个人。

门徒就如此行,叫众人都坐下。

耶稣拿着这五个饼,两条鱼,望着天祝福,擘开,递给门徒摆在众人面前。

他们就吃,并且都吃饱了。把剩下的零碎收拾起来,装满了十二篮子。

耶稣自己祷告的时候,门徒也同他在那里。耶稣问他们说,众人说我是谁。

他们说,有人说是施洗的约翰。有人说是以利亚。还有人说,是古时的一个先知又活了。

耶稣说,你们说我是谁。彼的回答说,是神所立的基督。

耶稣切切的嘱咐他们,不可将这事告诉人。

又说,人子必须受许多的苦,被长老祭司长和文士弃绝,并且被杀,第三日复活。

耶稣又对众人说,若有人要跟从我,就当舍己,天天背起他的十字架来,跟从我。

因为凡要救自己生命的,生命或作灵魂下同必丧掉生命。凡为我丧掉生命的,必救了生命。

人若赚得全世界,却丧了自己,赔上自己,有什么益处呢。

凡把我和我的道当作可耻的,人子在自己的荣耀里,并天父与圣天使的荣耀里,降临的时候,也要把那人当作可耻的。

我实在告诉你们,站在这里的,有人在没尝死味以前,必看见神的国。

说了这话以后,约有八天,耶稣带着彼得,约翰,雅各,上山去祷告。

正祷告的时候,他的面貌就改变了,衣服洁白放光。

忽然有摩西以利亚两个人,同耶稣说话。

他们在荣光里显现,谈论耶稣去世的事,就是他在耶路撒冷将要成的事。

彼得和他的同伴都打盹,既清醒了,就看见耶稣的荣光,并同他站着的那两个人。

二人正要和耶稣分离的时候,彼得对耶稣说,夫子,我们在这里真好,可以搭三座棚,一座为你,一座为摩西,一座为以利亚。他却不知道所说的是什么。

说这话的时候,有一朵云彩来遮盖他们。他们进入云彩里就惧怕。

有声音从云彩里出来,说,这是我的儿子,我所拣选的,有古卷作这是我的爱子你们要听他。

声音住了,只见耶稣一人在那里。当那些日子,门徒不题所见的事,一样也不告诉人。

第二天,他们下了山,就有许多人迎见耶稣。

其中有一人喊叫说,夫子,求你看顾我的儿子,因为他是我的独生子。

他被鬼抓住,就忽然喊叫。鬼又叫他抽疯,口中流沫,并且重重的伤害他,难以离开他。

我求过你的门徒,把鬼赶出去,他们却是不能。

耶稣说,嗳,这又不信又悖谬的世代阿,我在你们这里,忍耐你们,要到几时呢。将你的儿子带到这里来吧。

正来的时候,鬼把他摔倒,叫他重重的抽疯。耶稣就斥责那污鬼,把孩子治好了,交给他的父亲。

众人都诧异神的大能。大能或作威荣耶稣所作的一切事,众人正希奇的时候,耶稣对门徒说,

你们要把这些话存在耳中。因为人子将要被交在人手里。

他们不明白这话,意思乃是隐藏的,叫他们不能明白,他们也不敢问这话的意思。

门徒中间起了议论,谁将为大。

耶稣看出他们心中的议论,就领一个小孩子来,叫他站在自己旁边。

对他们说,凡为我名接待这小孩子的,就是接待我。凡接待我的,就是接待那差我来的。你们中间最小的,他便为大。

约翰说,夫子,我们看见一个人奉你的名赶鬼,我们就禁止他。因为他不与我们一同跟从你。

耶稣说,不要禁止他。因为不敌挡你们的,就是帮助你们的。

耶稣被接上升的日子将到,他就定意向耶路撒冷去,

便打发使者在他前头走。他们到了撒玛利亚的一个村庄,要为他豫备。

那里的人不接待他,因他面向耶路撒冷去。

他的门徒,雅各,约翰,看见了,就说,主阿,你要我们吩咐火从天上降下来,烧灭他们,像以利亚所作的吗。有古卷无像以利亚所作的数字

耶稣转身责备两个门徒说,你们的今如何,你们并不知道。

人子来不是要灭人的性命,性命或作灵魂下同是要救人的性命。说着就往别的村庄去了。有古卷只有五十五节首句五十六节末句

他们走路的时候,有一人对耶稣说你无论往那里去,我要跟从你。

耶稣说,狐狸有洞,天空的飞鸟有窝,只是人子没有枕头的地方。

又对一个人说,跟从我来,那人说,主,容我先回去埋葬我的父亲。

耶稣说,任凭死人埋葬他们的死人。你只管去传扬神国的道。

又有一人说,主,我要跟从你。但容我先去辞别我家里的人。

耶稣说,手扶着犁向后看的,不配进神的国。

\chapter{路加福音第10章}
这事以后,主又设立七十个人,差遣他们两个两个的,在他前面往自己所要到的各城各地方去。

就对他们说,要收的庄稼多,作工的人少。所以你们当求庄稼的主,打发工人出去收他的庄稼。

你们去吧。我差你们出去,如同羊羔进入狼群。

不要带钱囊,不要带口袋,不要带鞋。在路上也不要问人的安。

无论进那一家,先要说,愿这一家平安。

那里若有当得平安的人,当得平安的人原文作平安之子你们所求的平安就必临到那家,不然,就归与你们了。

你们要住在那家,吃喝他们所供给的。因为工人得工价,是应当的。不要从这家搬到那家。

无论进那一城,人若接待你们,给你们摆上什么,你们就吃什么。

要医治那城里的病人,对他们说,神的国临近你们了。

无论进那一城,人若不接待你们,你们就到街上去,

说,就是你们城里的尘土,粘在我们的脚上,我们也当着你们擦去。虽然如此,你们该知道神的国临近了。

我告诉你们,当审判的日子,所多玛所受的,比那城还容易受呢。

哥拉汛哪,你有祸了。伯赛大阿,你有祸了。因为在你们中间所行的异能,若行在推罗西顿,他们早已披麻蒙灰坐在地上悔改了。

当审判的日子,推罗西顿所受的,比你们还容易受呢。

迦百农阿,你已经升到天上。或作你将要升到天上吗将来必推下阴间。

又对门徒说,听从你们的,就是听从我,弃绝你们的,就是弃绝我,弃绝我的,就是弃绝那差我来的。

那七十个人欢欢喜喜的回来说,主阿,因你的名,就是鬼也服了我们。

耶稣对他们说,我曾看见撒但从天上坠落,像闪电一样。

我已经给你们权柄,可以践踏蛇和蝎子,又胜过仇敌一切的能力,断没有什么能害你们。

然而不要因鬼服了你们就欢喜,要因你们的名记在天上欢喜。

正当那时,耶稣被圣灵感动就欢乐,说,父阿,天地的主,我感谢你,因为你将这些事,向聪明通达人就藏起来,向婴孩就显出来。父阿,是的,因为你的美意本是如此。

一切所有的,都是我父交付我的。除了父,没有人知道子是谁。除了子和子愿意指示的,没有人知道父是谁。

耶稣转身暗暗的对门徒说。看见你们所看见的,那眼睛就有福了。

我告诉你们,从前有许多先知和君王,要看你们所看的,却没有看见。要听你们所听的,却没有听见。

有一个律法师,起来试探耶稣说,夫子,我该作什么才可以承受永生。

耶稣对他说,律法上写的是什么。你念的是怎样呢。

他回答说,你要尽心,尽性,尽力,尽意,爱主你的神。又要爱邻舍如同自己。

耶稣说,你回答的是。你这样行,就必得永生。

那人要显明自己有理,就对耶稣说,谁是我的邻舍呢。

耶稣回答说,有一个人从耶路撒冷下耶利哥去,落在强盗手中,他们剥去他的衣裳,把他打个半死,就丢下他走了。

偶然有一个祭司,从这条路下来。看见他就从那边过去了。

又有一个利末人,来到这地方,看见他,也照样从那边过去了。

惟有一个撒玛利亚人,行路来到那里。看见他就动了慈心,

上前用油和酒倒在他的伤处,包裹好了,扶他骑上自己的牲口,带到店里去照应他。

第二天拿出二钱银子来,交给店主说,你且照应他。此外所费用的,我回来必还你。

你想这三个人,那一个是落在强盗手中的邻舍呢。

他说,是怜悯他的。耶稣说,你去照样行吧。

他们走路的时候,耶稣进了一个村庄。有一个女人名叫马大,接他到自己家里。

他有一个妹子名叫马利亚,在耶稣脚前坐着听他得道。

马大伺候的事多,心里忙乱,就进前来说,主阿,我的妹子留下我一个人伺候,你不在意吗。请吩咐他来帮助我。

耶稣回答说,马大,马大,你为许多的事,思虑烦扰。

但是不可少的只有一件。马利亚已经选择那上好的福分,是不能夺去的。

\chapter{路加福音第11章}
耶稣在一个地方祷告。祷告完了,有个门徒对他说,求主教导我们祷告,像约翰教导他的门徒。

耶稣说,你们祷告的时候,要说,我们在天上的父,有古卷只作父阿愿人都尊你的名为圣。愿你的国降临。愿你的旨意行在地上,如同行在天上。有古卷无愿你的旨意云云

我们日用的饮食,天天赐给我们。

赦免我们的罪,因为我们也赦免凡亏欠我们的人。不叫我们遇见试探。救我们脱离凶恶。有古卷无末句

耶稣又说,你们中间谁有一个朋友,半夜到他那里去说,朋友,请借给我三个饼。

因为我有一个朋友行路,来到我这里,我没有什么给他摆上。

那人在里面回答说,不要搅扰我。门已经关闭,孩子们也同我在床上了。我不能起来给你。

我告诉你们,虽不因他是朋友起来给他,但因他情词迫切的直求,就必起来照他所需用的给他。

我又告诉你们,你们祈求就给你们。寻梢就寻见。叩门就给你们开门。

因为凡祈求的就得着。寻梢的就寻见。叩门的就给他开门。

你们中间作父亲的,谁有儿子求饼,反给他石头呢。求鱼,反拿蛇当鱼给他呢。

求鸡蛋,反给他蝎子呢。

你们虽然不好,尚且知道拿好东西给儿女。何况天父,岂不更将圣灵给求他的人吗。

耶稣赶出一个叫人哑吧的鬼。鬼出去了,哑吧就说出话来众人都希奇。

内中却有人说,他是靠着鬼王别西卜赶鬼。

又有人试探耶稣,向他求从天上来的神迹。

他晓得他们的意念,便对他们说,凡一国自相分争,就成为荒场。凡一家自相分争,就必败落。

若撒但自相分争,他的国怎能站得住呢。因为你们说我是靠着别西卜赶鬼。

我若靠着别西卜赶鬼,你们的子弟赶鬼,又靠着谁呢。这样,他们就要断定你们的是非。

我若靠着神的能力赶鬼,这就是神的国临到你们了。

壮士披挂整齐,看守自己的住宅,他所有的都平安无事。

但有一个比他更壮的来,胜过他,就夺去他所依靠的盔甲兵器,又分了他的赃。

不与我相合的,就是敌我的。不同我收聚的,就是分散的。

污鬼离了人身,就在无水之地,过来过去,寻求安歇之处。既寻不着,便说,我要回到我所出来的屋里去。

到了,就看见里面打扫乾净,修饰好了。

便去另带了七个比自己更恶的鬼来,都进去住在那里。那人末后的景况,比先前更不好了。

耶稣正说这话的时候,众人中间,有一个女人大声说,怀你胎的和乳养你的有福了。

耶稣说,是却还不如听神之道而遵守的人有福。

当众人聚集的时后,耶稣开讲说,这世代是一个邪恶的世代。他们求看神迹,除了约拿的神迹以外,再没有神迹给他们看。

约拿怎样为尼尼微人成了神迹,人子也要照样为这世代的人成了神迹。

当审判的时候,南方的女王,要起来定这世代的罪。因为他从地极而来,要听所罗门的智慧话。看哪,在这里有一人比所罗门更大。

当审判的时候,尼尼微人,要起来定这世代的罪。因为尼尼微人听了约拿所传的,就悔改了。看哪,在这里有一人比约拿更大。

没有人点灯放在地窨子里,或是斗底下,总是放在灯台上,使进来的人看得见亮光。

你眼睛就是身上的灯,你的眼睛若了亮,全身就光明。眼睛若昏花,全身就黑暗。

所以你要省察,恐怕你里头的光,或者黑暗了。

若是你全身光明,毫无黑暗,就必全然光明,如同灯的明光照亮你。

说话的时候,有一个法利赛人请耶稣同他吃饭。耶稣就进去坐席。

这法利赛人看见耶稣饭前不洗手,便诧异。

主对他说,如今你们法利赛人洗净杯盘的外面。你们里面却满了勒索和邪恶。

无知的人哪,造外面的,不也造里面吗。

只要把里面的施舍给人,凡物于你们就都洁净了。

你们法利赛人有祸了。因为你们将薄荷芸香,并各样菜蔬,献上十分之一,那公义和爱神的事,反倒不行了。这原是你们当行的,那也是不可不行的。

你们法利赛人有祸了。因为你们喜爱会堂里的首位,又喜爱人在街市上问你们的安。

你们有祸了。因为你们如同不显露的坟墓,走在上面的人并不知道。

律法师中有一个回答耶稣说,夫子,你这样说,也把我们糟塌了。

耶稣说,你们律法师也有祸了。因为你们把难担的担子,放在人身上,自己一个指头却不肯动。

你们有祸了。因为你们修造先知的坟墓,那先知正是你们的祖宗所杀的。

可见你们的祖宗所作的事,你们又证明又喜欢。因为他们杀了先知,你们修造先知的坟墓。

所以神用智慧曾说,用智慧或作的智者我要差遣先知和使徒,到他们那里去。有的他们要杀害,有的他们要逼迫。

使创世以来,所流众先知血的罪,都要问在这世代的人身上。

就是从亚伯的血起,直到被杀在坛和殿中间撒迦利亚的血为止。我实在告诉你们,这都要问在这世代的人身上。

你们律法师有祸了。因为你们把知识的钥匙夺了去。自己不进去,正要进去的人,你们也阻挡他们。

耶稣从那里出来,文士和法利赛人,就极力的催逼他,引动他多说话。

私下窥听,要拿他的话柄。

\chapter{路加福音第12章}
这时,有几万人聚集,甚至彼此践踏,耶稣开讲,先对门徒说,你们要防备法利赛人的酵,就是假冒为善。

掩盖的事,没有不露出来的。隐藏的事,没有不被人知道的。

因此你们在暗中所说的,将要在明处被人听见。在室内附耳所说的,将要在房上被人宣扬。

我的朋友,我对你们说,那杀身体以后,不能再作什么的,不要怕他们。

我要指示你们当怕的是谁。当怕那杀了以后,又有权柄丢在地狱里的。我实在告诉你们,正要怕他。

五个麻雀,不是卖二分银子吗。但在神面前,一个也不忘记。

就是你们的头发也都被数过了。不要惧怕,你们比许多麻雀还贵重。

我又告诉你们,凡在人面前认我的,人子在神的使者面前也必认他。

在人面前不认我的,人子在神的使者面前也必不认他。

凡说话干犯人子的,还可得赦免,惟独亵渎圣灵的,总不得赦免。

人带你们到会堂,并官府,和有权柄的人面前,不要思虑怎吗分诉,说什么话。

因为正在那时候,圣灵要指教你们当说的话。

众人中有一个人对耶稣说,夫子,请你吩咐我的兄长和我分开家业。

耶稣说,你这个人,谁立我作你们断事的官,给你们分家业呢。

于是对众人说,你们要谨慎自守,免去一切的贪心。因为人的生命,不在乎家道丰富。

就用比喻对他们说,有一个财主,田产丰盛。

自己心里思想说,我的出产没有地方收藏,怎吗办呢。

又说,我要这吗办。要把我的仓房拆了,另盖更大的。在那里好收藏我一切的粮食和财物。

然后要对我的灵魂说,灵魂哪,你有许多财物积存,可作多年的费用。只管安安逸逸的吃喝快乐吧。

神却对他说,无知的人哪,今夜必要你的灵魂。你所豫备的,要归谁呢。

凡为自己积财,在神面前却不富足的,也是这样。

耶稣又对门徒说,所以我告诉你们,不要为生命忧虑吃什么。为身体忧虑穿什么。

因为生命胜于饮食,身体胜于衣裳。

你想乌鸦,也不种,也不收。又没有仓,又没有库,神尚且养活他。你们比飞鸟是何等的贵重呢。

你们那一个能用思虑,使寿数多加一刻呢。或作使身量多加一肘呢

这最小的事,你们尚且不能作,为什么还忧虑其馀的事呢。

你想百合花,怎吗长起来。他也不劳苦,也不纺线。然而我告诉你们,就是所罗门极荣华的时候,他所穿戴的,还不如这花一朵呢。

你们这小信的人哪,野地里的草,今天还在,明天就丢在炉里,神还给他这样的妆饰,何况你们呢。

你们不要求吃什么,喝什么,也不要挂心。

这都是外邦人所求的,你们必须用这些东西,你们的父是知道的。

你们只要求他的国,这些东西就必加给你们了。

你们这小群,不要惧怕,因为你们的父,乐意把国赐给你们。

你们要变卖所有的,周济人。为自己豫备永不坏的钱囊,用不尽的财宝在天上,就是贼不能近,虫不能蛀的地方。

因为你们的财宝在那里,你们的心也在那里。

你们腰里要束上带,灯也要点着。

自己好像仆人等候主人,从婚姻的筵席上回来。他来到叩门,就立刻给他开门。

主人来了,看见仆人儆醒,那仆人就有福了。我实在告诉你们,主人必叫他们坐席,自己束上带,进前伺候他们。

或是二更天来,或是三更天来,看见仆人这样,那仆人就有福了。

家主若知道贼什么时候来,就必儆醒,不容贼挖透房屋,这是你们所知道的。

你们也要豫备。因为你们想不到的时候,人子就来了。

彼得说,主阿,这比喻是为我们说的呢,还是为众人呢。

主说,谁是那忠心有见识的管家,主人派他管理家里的人,按时分粮给他们呢。

主人来到,看见仆人这样行,那仆人就有福了。

我实在告诉你们,主人要派他管理一切所有的。

那仆人若心里说,我的主人必来得迟。就动手打仆人和使女,并且吃喝醉酒。

在他想不到的日子,不知道的时辰,那仆人的主人要来,重重的处治他,或作把他腰斩了定他和不忠心的人同罪。

仆人知道主人的意思,却不豫备,又不顺他的意思行,那仆人必多受责打。

惟有那不知道的,作了当受责打的事,必少受责打因为多给谁,就向谁多取。多托谁,就向谁多要。

我来要把火丢在地上。倘若已经着起来,不也是我所愿意的吗。

我有当受的洗。还没有成就,我是何等的迫切呢。

你们以为我来,是叫地上太平吗。我告诉你们,不是,乃是叫人分争。

从今以后,一家五个人将要分争,三个人和两个人相争,两个人和三个人相争。

父亲和儿子相争,儿子和父亲相争。母亲和女儿相争,女儿和母亲相争。婆婆和媳妇相争,媳妇和婆婆相争。

耶稣又对众人说,你们看见西边起了云彩,就说,要下一阵雨。果然就有。

起了南风,就说,将要燥热。也就有了。

假冒为善的人哪,你们知道分辨天地的气色。怎吗不知道分辨这时候呢。

你们又为何不自己审量,什么是合理的呢。

你同告你的对头去见官,还在路上,务要尽力的和他了结。恐怕他拉你到官面前,官交付差役,差役把你下在监里

我告诉你,若有半文钱没有还清,你断不能从那里出来。

\chapter{路加福音第13章}
正当那时,有人将彼拉多使加利利人的血搀杂在他们祭物中的事,告诉耶稣。

耶稣说,你们以为这些加利利人比众加利利人更有罪,所以受这害吗。

我告诉你们,不是的。你们若不悔改,都要如此灭亡。

从前西罗亚楼倒塌了,压死十八个人,你们以为那些人比一切住在耶路撒冷的人更有罪吗。

我告诉你们,不是的。你们若不悔改,都要如此灭亡。

于是用比喻说,一个人有一棵无花果树,栽在葡萄园里。他来到树前梢果子,却找不着。

就对管园的说,看哪,我这三年,来到这无花果树前梢果子,竟找不着,把他砍了吧。何必白占地土呢。

管园的说,主阿,今年且留着,等我周围掘开土,加上粪。

以后若结果子便吧。不然再把他砍了。

安息日,耶稣在会堂里教训人。

有一个女人,被鬼附着病了十八年。腰弯得一点直不起来。

耶稣看见,便叫过他来,对他说,女人,你脱离这病了。

于是用两双手按着他。他立刻直起腰来,就归荣耀与神。

管会堂的,因为耶稣在安息日治病,就气忿忿的对众人说,有六日应当作工。那六日之内,可以来求医,在安息日却不可。

主说,假冒为善的人哪,难道你们各人在安息日不解开槽上的牛驴,牵去饮吗。

况且这女人本是亚伯拉罕的后裔,被撒但捆绑了这十八年,不当在安息日解开他的绑吗。

耶稣说这话,他的敌人都惭愧了。众人因他所行一切荣耀的事,就都欢喜了。

耶稣说,神的国,好像什么。我拿什么来比较呢。

好像一粒芥菜种,有人拿去种在园子里。长大成树,天上的飞鸟,宿在他的枝上。

又说,我拿什么来比神的国呢。

好比面酵,有妇人拿来藏在三斗面里,直等全团都发起来。

耶稣往耶路撒冷去,在所经过的各城各乡教训人。

有一个人问他说,主阿,得救的人少吗。

耶稣对众人说,你们要努力进窄门。我告诉你们,将来有许多人想要进去,却是不能。

及至家主起来关了门,你们站在外面叩门,说,主阿,给我们开门,他就回答说,我不认识你们,不晓得你们是那里来的。

那时,你们要说,我们在你面前吃过喝过,你也在我们街上教训过人。

他要说,我告诉你们,我不晓得你们是那里来的。你们这切作恶的人,离开我去吧。

你们要看见亚伯拉罕,以撒,雅各,和众先知,都在神的国里,你们却被赶到外面。在那里必要哀哭切齿了。

从东,从西,从南,从北,将有人来,在神的国里坐席。

只是有在后的将要在前,有在前的将要在后。

正当那时,有几个法利赛人来对耶稣说,离开这里去吧。因为希律想要杀你。

耶稣说,你们去告诉那个狐狸说,今天明天我赶鬼治病,第三天我的事就成全了。

虽然这样,今天明天后天我必须前行。因为先知在耶路撒冷之外丧命是不能的。

耶路撒冷阿,耶路撒冷阿,你常杀害先知,又用石头打死那奉差遣到你这里来的人。我多次愿意聚集你的儿女,好像母鸡把小鸡聚集在翅膀底下,只是你们不愿意。

看哪,你们的家成为荒场留给你们。我告诉你们,从今以后你们不得再见我,直等到你们说,奉主名来的是应当称颂的。

\chapter{路加福音第14章}
安息日,耶稣到一个法利赛人的首领家里去吃饭,他们就窥探他。

在他面前有一个患水臌的人。

耶稣对律法师和法利赛人说,安息日治病,可以不可以。

他们却不言语。耶稣就治好那人,叫他走了。

便对他们说,你们中间谁有驴或有牛,在安息日掉在井里,不立时拉他上来呢。

他们不能对答这话。

耶稣见所请的客拣择首位,就用比喻对他们说,

你被人请去赴婚姻的筵席,不要坐在首位上。恐怕有比你尊贵的客,被他请来。

那请你们的人前来对你说,让座寄这一位吧。你就羞羞惭惭的退到末位上去了。

你被请的时候,就去坐在末位上,好叫那请你的人来,对你说,朋友,请上坐,那时你在同席的人面前,就有了光彩了。

因为凡自高的必降为卑。自卑的必升为高。

耶稣又对请他的人说,你摆设午饭,或晚饭,不要请你的朋友,弟兄,亲属,和富足的邻舍。恐怕他们也请你,你就得了报答。

你摆设筵席,倒要请那贫穷的,残废的,瘸腿的,瞎眼的,你就有福了。

因为他们没有什么可报答你。到义人复活的时候,你要得着报答。

同席的友一人听见这话,就对耶稣说,在神国里吃饭的有福了。

耶稣对他说,有一人摆设大筵席,请了许多客。

到了坐席的时后,打发仆人去对所请的人说,请来吧。样样都齐备了。

众人一口同音的推辞。头一个说,我买了一块地,必须去看看。请你准我辞了。

又有一个说,我买了五对牛,要去试一试。请你准我辞了。

又有一个说,我才娶了妻,所以不能去。

那仆人回来,把这事都告诉了主人。家主就动怒,对仆人说,快出去到城里大街小巷,领那贫穷的,残废的,瞎眼的,瘸腿的来。

仆人说,主阿,你所吩咐的已经办了,还有空座。

主人对仆人说,你出去到路上和篱笆那里,勉强人进来,坐满我的屋子。

我告诉你们,先前所请的人,没有一个人得尝我的筵席。

有极多的人和耶稣同行。他转过来对他们说,

人到我这里来,若不爱我胜过爱自己的父母,妻子,儿女,弟兄,姐妹,和自己的性命,就不能作我的门徒。爱我胜过爱原文作恨

凡不背着自己十字架跟从我的,也不能作我的门徒。

你们那一个要盖一座楼,不先坐下算计花费,能盖成不能呢。

恐怕安了地基,不能成功,看见的人都笑话他,

说,这个人开了工,却不能完工。

或是一个王,出去和别的王打仗,岂不先坐下酌量,能用一万兵,去敌那领二万兵来攻打他的吗。

若是不能,就趁敌人还远的时候,派使者去求和息的条款。

这样,你们无论什么人,若不撇下一切所有的,就不能作我的门徒。

盐本是好的,盐若失了味,可用什么叫他再咸呢。

或用在田里,或堆在粪里,都不合式。只好丢在外面。有耳可听的,就应当听。

\chapter{路加福音第15章}
众税吏和罪人,都挨近耶稣要听他讲道。

法利赛人和文士,私下议论说,这个人接待罪人,又同他们吃饭。

耶稣就用比喻,说,

你们中间谁有一百只羊,失去一只,不把这九十九只撇在旷野,去找那失去的羊直到找着呢。

找着了,就欢欢喜喜的扛在肩上,回到家里。

就请朋友邻舍来,对他们说,我失去的羊已经找着了,你们和我一起欢喜吧。

我告诉你们,一个罪人悔改,在天上也要这样为他欢喜,较比为九十九个不用悔改的义人,欢喜更大。

或是一个妇人,有十块钱,若失落一块,岂不点上灯,打扫屋子,细细的找,直到找着吗。

找着了,就请朋友邻舍来,对他们说,我失落的那块钱已经找着了,你们和我一同欢喜吧。

我告诉你们,一个罪人悔改,在神的使者面前,也是这样为他欢喜。

耶稣又说,一个人有两个儿子。

小儿子对父亲说,父亲,请你把我应得的家业分给我。他父亲就把产业分给他们。

过了不多几日,小儿子就把他一切所有的,都收拾起来,往远方去了。在那里任意放荡,浪费赀财。

既耗尽了一切所有的,又遇着那地方大遭饥荒,就穷苦起来。

于是去投靠那地方的一个人,那人打发他到田里去放猪。

他恨不得拿猪所吃的豆荚充饥。也没有人给他。

他醒悟过来,就说,我父亲有多少的雇工,口粮有馀,我倒在这里饿死吗。

我要起来,到我父亲那里去,向他说,父亲,我得罪了天,又得罪了你。

从今以后,我不配称为你的儿子,把我当作一个雇工吧。

于是起来往他父亲那里去。相离还远,他父亲看见,就动了慈心,跑去抱着他的颈项,连连舆他亲嘴。

儿子说,父亲,我得罪了天,又得罪了你,从今以后,我不配称为你的儿子。

父亲却吩咐仆人说,把那上好的袍子快拿出来给他穿。把戒指戴在他指头上。把鞋穿在他脚上。

把那肥牛犊牵来宰了,我们可以吃喝快乐。

因为我这个儿子,是死而复活,失而又得的。他们就快乐起来。

那时,大儿子正在田里。他回来离家不远,听见作乐跳舞的声音。

便叫过一个仆人来,问是什么事。

仆人说,你兄弟来了。你父亲,因为得他无灾无病的回来,把牛犊宰了。

大儿子却生气,不肯进去。他父亲就出来劝他。

他对父亲说,我服事你这多年,从来没有违背过你的命。你并没有给我一只山羊羔,叫我和朋友,一同快乐。

但你这个儿子,和娼妓吞尽了你的产业,他一来了,你倒为他宰了肥牛犊。

父亲对他说,儿阿,你常和我同在,我一切所有的,都是你的。

只是你这个兄弟是死而复活,失而又得的,所以我们理当欢喜快乐。

\chapter{路加福音第16章}
耶稣又对门徒说,有一个财主的管家。别人向他主人告他浪费主人的财物。

主人叫他来,对他说,我听见你这事怎吗样呢。把你所经管的交待明白。因你不能再作我的管家。

那管家心里说,主人辞我,不用我再作管家,我将来作什么。锄地呢,无力。讨饭呢,怕羞。

我知道怎吗行,好叫人在我不作管家之后,接我到他们家里去。

于是把欠主人债的,一个一个的叫了来,问头一个说,你欠我主人多少。

他说,一百篓油。每篓约五十斤管家说,拿你的账快坐下写五十。

又问一个说,你欠多少。他说,一百石麦子。管家说,拿你的账写八十。

主人就夸奖这不义的管家作事聪明。因为今世之子,在世事之上,较比光明之子,更加聪明。

我又告诉你们,要籍着那不义的钱财,结交朋友。到了钱财无用的时候,他们可以接你们到永存的帐幕里去。

人在最小的事上忠心,在大事上也忠心。在最小的事上不义,在大事上也不义。

倘若你们在不义的钱财上不忠心,谁还把那真实的钱财托付你们呢。

倘若你们在别人的东西上不忠心,谁还把你们自己的东西给你们呢。

一个仆人不能事奉两个主。不是恶这个爱那个,就是重这个轻那个。你们不能又事奉神,又事奉玛门。

法利赛人是贪爱钱财的,他们听见这一切话,就嗤笑耶稣。

耶稣对他们说,你们是在人面前自称为义的。你们的心,神却知道。因为人所尊贵的是神看为可憎恶的。

律法和先知,到约翰为止。从此神国的福音传开了,人人努力要进去。

天地废去,较比律法的一点一画落空还容易。

凡休妻另娶的,就是犯奸淫。娶被休之妻的,也是犯奸淫。

有一个财主,穿着紫色袍和细麻布衣服,天天奢华宴乐。

又有一个讨饭的,名叫拉撒路,浑身生疮,被人放在财主门口,

要得财主桌子上掉下来的零碎充饥。并且狗来舔他的疮。

后来那讨饭的死,被天使放在亚伯拉罕的怀里。财主也死了,并且埋葬了。

他在阴间受痛苦,举目远远的望见亚伯拉罕,又望见拉撒路在他怀里。

就喊着说,我祖亚伯拉罕哪,可怜我吧,打发拉撒路来,用指头尖蘸点水,凉凉我的舌头。因为我在这火焰里,极其痛苦。

亚伯拉罕说,儿阿,你该回想你生前享过福,拉撒路也受过苦。如今他在这里得安慰,你倒受痛苦。

不但这样,并且在你我之间,有深渊限定,以致人要从这边过到你们那边,是不能的,要从那边过到我们这边,也是不能的。

财主说,我祖阿,既是这样,求你打发拉撒路到我父家去。

因为我还有五个弟兄。他可以对他们作见证,免得他们也来到这痛苦的地方。

亚伯拉罕说,他们有摩西和先知的话,可以听从。

他说,我祖亚伯拉罕哪,不是的。若有一个从死里复活的,到他们那里去的,他们必要悔改。

亚伯拉罕说,若不听从摩西和先知的话,就是有一个从死里复活的,他们也是不听劝。

\chapter{路加福音第17章}
耶稣又对门徒说,绊倒人的事是免不了的。但那绊倒人的有祸了。

就是把磨石拴在这人的颈项上,丢在海里,还强如他把这小子里的一个绊倒了。

你们要谨慎。若是你们的弟兄得罪你,就劝戒他。他若懊悔,就饶恕他。

倘若他一天七次得罪你,又七次回转说,我懊悔了,你总要饶恕他。

使徒对主说,求主加增我们的信心。

主说,你们若有信心像一粒芥菜种,就是对这棵桑树说,你要拔起根来栽在海里,他也必听从你们。

你们谁有仆人耕地,或是放羊,从田里回来,就对他说,你快来坐下吃饭呢。

岂不对他说,你给我豫备晚饭,束上带子伺候我,等我吃喝完了,你才可以吃喝吗。

仆人照所吩咐的去作,主人还谢谢他吗。

这样,你们作完了一切所吩咐的,只当说,我们是无用的仆人。所作的本是我们应分作的。

耶稣往耶路撒冷去,经过撒玛利亚和加利利。

进入一个村子,有十个长大麻疯的迎面而来,远远的站着。

高声说,耶稣,夫子,可怜我们吧。

耶稣看见,就对他们说,你们去把身体给祭司察康。他们去的时候就洁净了。

内中有一个见自己已经好了,就回来大声归荣耀与神。

又俯伏在耶稣脚前感谢他。这人是撒玛利亚人。

耶稣说,洁净了的不是十个人吗。那九个在那里呢。

除了这外族人,再没有别人回来归荣耀与神吗。

就对那人说,起来走吧。你的信救了你。

法利赛人问神的国几时来到。耶稣回答说,神的国来到,不是眼所能见的。

人也不得说,看哪,在这里。看哪,在那里。因为神的国就在你们心里。心里或作中间

他又对门徒说,日子将到,你们巴不得看见人子的一个日子,却不得看见。

人将要对你们说,看哪,在那里,看哪,在这里。你们不要出去,也不要跟随他们。

因为人子在他降临的日子,好像闪电,从天这边一闪,直照到天那边。

只是他必须先受许多苦,又被这世代弃绝。

挪亚的日子怎样,人子的日子也要怎样。

那时候的人又吃又喝,又娶又嫁,到挪亚进方舟的那日,洪水就来,把他们全都灭了。

又好像罗得的日子。人又吃又喝,又买又卖,又耕种,又盖造。

到罗得出所多玛的那日,就有火与硫磺从天上降下来,把他们全灭了。

人子显现的日子,也要这样。

当那日,人在房上,器具在屋里,不要下来拿。人在田里,也不要回家。

你们要回想罗得的妻子。

凡想保全生命的,必丧掉生命。凡丧掉生命的,必救活生命。

我对你们说,当那一夜,两个人在一个床上。要取去一个,撇下一个。

两个女人一同推磨。要取去一个,撇下一个。有古卷在此有

两个人在田里要取去一个撇下一个

门徒说,主阿,在那里有这事呢。耶稣说,尸首在那里,鹰也必聚在那里。

\chapter{路加福音第18章}
耶稣设一个比喻,是要人常常祷告,不可灰心。

说,某城里有一个官,不惧怕神,也不尊重世人。

那城里有个寡妇,常到他那里,说,我有一个对头,求你给我伸冤。

他多日不准。后来心里说,我虽不惧帕神,也不尊重世人。

只因这寡妇烦扰我,我就给他伸冤吧。免得他常来缠磨我。

主说,你们听这不义之官所说的话。

神的选民,昼夜呼吁他,他纵然为他们忍了多时,岂不终久给他们伸冤吗。

我告诉你们,要快快的给他们伸冤了,然而人子来的时候,遇得见世上有信德吗。

耶稣向那些仗着自己是义人,藐视别人的,设一个比喻,

说,有两个人上殿里去祷告。一个是法利赛人,一个是税吏。

法利赛人站着,自言自语的祷告说,神阿,我感谢你,我不像别人,勒索,不义,奸淫,也不像这个税吏。

我一个礼拜禁食两次,凡我所得的,都捐上十分之一。

那税吏远远的站着,连举目望天也不敢,只捶着胸说,神阿,开恩可怜我这个罪人。

我告诉你们,这人回家去,比那人倒算为义了,因为凡自高的,必降为卑,自卑的,必升为高。

有人抱着自己的婴孩,来见耶稣,要他摸他们,门徒看见就责备那些人。

耶稣却叫他们来,说,让小孩子倒我这里来,不要禁止他们,因为在神国的,正是这样的人

我实在告诉你们,凡要承受神国的,若不像小孩子,断不能进去。

有一个官问耶稣说,善良的夫子,我该作什吗事,才可以承受永生。

耶稣对他说,你为什吗称我是良善的,除了神一位之外,再没有良善的。

诫命你是晓得的,不可奸淫,不可杀人,不可偷盗,不可作假见证,当孝敬父母。

那人说,这一切我从小都遵守了。

耶稣听见了,就说,你还缺少一件,要变卖你一切所有的,分给穷人,就必有财宝在天上。你还要来跟从我。

他听见这话,就甚忧愁,因为他很富足。

耶稣看见他就说,有钱财的人进神的国,是何等的难哪。

骆驼穿过针的眼,比财主进神的国,还容易呢。

听见的人说,这样,谁能得救呢。

耶稣说,在人所不能的事,在神却能。

彼得说,看哪,我们已经撇下自己所有的跟从你了。

耶稣说,我实在告诉你们,人为神的国,撇下房屋,或是妻子,弟兄,父母,儿女,

没有在今世不得百倍,在来世不得永生的。

耶稣带着十二个门徒,对他们说,看哪,我们上耶路撒冷去,先知所写的一切事,都成就在人子身上。

他将要被交给外邦人,他们要戏弄他,凌辱他,吐唾沫在他脸上。

并要鞭打他,杀害他,第三日他要复活。

这些事门徒一样也不懂得,意思乃是隐藏的,他们不晓得所说的是什吗。

耶稣将近耶利哥的时候,有一个瞎子坐在路旁讨饭。

听见许多人经过,就问是什么事。

他们告诉他,是拿撒勒人耶稣经过。

他就呼叫说,大卫的子孙耶稣阿,可怜我吧。

在前头走的人,就责备他,不许他作声。他却越发喊叫说,大卫的子孙,可怜我吧。

耶稣站住,吩咐把他领过来。到了跟前,就问他说,

你要我为你作什么。他说,主阿,我要能看见。

耶稣说,你可以看见。你的信救了你了。

瞎子立刻看见了,就跟随耶稣,一路归荣耀与神。众人看见这事,也赞美神。

\chapter{路加福音第19章}
耶稣进了耶利哥,正经过的时候,

有一个人名叫撒该,作税吏长,是个财主。

他要看看耶稣是怎样的人。只因人多,他的身量又矮,所以不得看见。

就跑到前头,爬上桑树,要看耶稣,因为耶稣必从那里经过。

耶稣到了那里,抬头一看,对他说,撒该,快下来,今天我必住在你家里。

他就急忙下来,欢欢喜喜的接待耶稣。

众人看见,都私下议论说,他竟到罪人家里去住宿。

撒该站着,对主说,主阿,我把所有的一半给穷人。我若讹诈了谁,就还他四倍。

耶稣说,今天救恩到了这家,因为他也是亚伯拉罕的子孙。

人子来,为要寻梢拯救失丧的人。

众人正在听见这些话的时候,耶稣因为将近耶路撒冷,又因为他们以为神的国快要显出来,就另设一个比喻说,

有一个贵胄往远方去,要得国回来。

便叫了他的十个仆人来,交给他们十锭银子,锭原文作弥拿一弥拿约银十两说,你们去作生意,直等我回来。

他本国的人却恨他,打发使者随后去说,我们不愿意这个人作我们的王。

他既得国回来,就吩咐叫那领银子的仆人来,要知道他们作生意赚了多少。

头一个上来说,主阿,你的一锭银子,已经赚了十锭。

主人说,好良善的仆人。你既在最小的事上有忠心,可以有权柄管十座城。

第二个来说,主阿,你的一锭银子,已经赚了五锭。

主人说,你也可以管五座城。

又有一个来说,主阿,看哪,你的一锭银子在这里,我把他包在手巾里存着。

我原是怕你,因为你是严厉的人。没有放下的还要去拿,没有种下的还要去收。

主人对他说,你这恶仆,我要凭你的口,定你的罪。你既知道我是严厉的人,没有放下的还要去拿,没有种下的还要去收。

为什么不把我的银子交给银行,等我来的时候,连本带利都可以要回来呢。

就对旁边站着的人说,夺过他这一锭来,给那有十锭的。

他们说,主阿,他已经有十锭了。

主人说,我告诉你们,凡有的,还要加给他。没有的,连他所有的,也要夺过来。

至于我那些仇敌不要我作他们王的,把他们拉来,在我面前杀了吧。

耶稣说完了这话,就在前面走,上耶路撒冷去。

将近伯法其和伯大尼,在一座山名叫橄榄山那里。就打发两个门徒,说,

你们往对面村子里去。进去的时候,必看见一匹驴驹拴在那里,是从来没有人骑过的。可以解开牵来。

有人问为什么解他,你们就说,主要用他。

打发的人去了,所遇见的,正如耶稣所说的。

他们解驴驹的时候,主人问他们说,解驴驹作什么。

他们说,主要用他。

他们牵到耶稣那里,把自己的衣服搭在上面,扶着耶稣骑上。

走的时候,众人把衣服铺在路上。

将近耶路撒冷,正下橄榄山的时候,众门徒因所见过的一切异能,都欢乐起来,大声赞美神,

说,奉主名来的王,是应当称颂的。在天上有和平,在至高之处有荣光。

众人中有几个法利赛人对耶稣说,夫子,责备你的门徒吧。

耶稣说,我告诉你们,若是他们闭口不说,这些石头必要呼叫起来。

耶稣快到耶路撒冷看见城,就为他哀哭,

说,巴不得你在这个日子,知道关系你平安的事。无奈这事现在是隐藏的,叫你的眼看不出来。

因为日子将到,你的仇敌必筑起土垒,周围环绕你,四面困住你,

并要扫灭你,和你里头的儿女,连一块石头也不留在石头上。因你不知道眷顾你的时候。

耶稣进了殿,赶出里头作买卖的人,

对他们,经上说,我的殿,必作祷告的殿。你们倒使他成了贼窝了。

耶稣天天在殿里教训人。祭司长,和文士,与百姓的尊长,都想要杀他。

但寻不出法子来,因为百姓都侧耳听他。

\chapter{路加福音第20章}
有一天耶稣在殿里教训百姓,讲福音的时候,祭司长和文士并长老上前来,

问他说,你告诉我们,你仗着什么权柄作这些事,给你这权柄的是谁呢。

耶稣回答说,我也要问你们一句话。你们且告诉我。

约翰的洗礼,是从天上来的,是从人间来的呢。

他们彼此商议说,我们若说从天上来,他必说你们为什么不信他呢。

若说从人间来,百姓都要用石头打死我们。因为他们信约翰是先知。

于是回答说,不知道是从那里来的。

耶稣说,我也不告诉你们,我仗着什么权柄作这些事。

耶稣就设比喻,对百姓说,有人栽了一个葡萄园,租给园户,就往外国去住了许久。

到了时候,打发一个仆人到园户那里去,叫他们把园中当纳的果子交给他。园户竟打了他,叫他空手回去。

又打发一个仆人去。他们也打了他,并且凌辱他,叫他空手回去。

又打发第三个仆人去。他们也打伤了他,把他推出去了。

园主说,我怎吗办呢。我要打发我的爱子去。或者他们尊敬他。

他不料,园户刻见他,就彼此商量说,这是承受产业的。我们杀他吧,使产业归于我们。

于是把他推出葡萄园外杀了。这样,葡萄园的主人,要怎样处治他们呢。

他要求除灭这些园户,将葡萄园转给别人。听见的人说,这是万不可的。

耶稣看着他们说,经上记着,匠人所弃的石头,已作了房角的头块石头。这是什么意思呢。

凡掉在那石头上的,必要跌碎。那石头掉在谁的身上,就要把谁砸得稀烂。

文士和祭司长,看出这比喻是指着他们说的,当时就想要下手拿他。只是惧怕百姓。

于是窥探耶稣,打发奸细装作好人,要在他的话上得把柄,好将他交在巡抚的政权之下。

奸细就问耶稣说,夫子,我们晓得你所讲所传都是正道,也不取人的外貌,乃是诚诚实实传神的道。

我们纳税给凯撒,可以不可以。

耶稣看出他们的诡诈,就对他们说,

拿一个银钱来给我看。这像和这号是谁的。他们说,是凯撒的。

耶稣说,这样,凯撒的物当归给凯撒,神的物当归给神。

他扪当着百姓,在这话上得不着把柄。又希奇他的应对,就闭口无言了。

撒都该人常说没有复活的事。有几个来问耶稣说,

夫子,摩西为我们写着说,人若有妻无子就死了,他兄弟当娶他的妻,为哥哥生子立后。

有弟兄七人。第一个娶了妻,没有孩子死了。

第二个,第三个,也娶过他。

那七个人,都娶过他,没有留下孩子就死了。

后来妇人也死了。

这样当复活的时候,他是那一个的妻子呢。因为他们七个人都娶过他。

耶稣说,这世界的人,有娶有嫁。

惟有算为配得那世界,与从死里复活的人,也不娶也不嫁。

因为他们不能再死。和天使一样。既是复活的人,就为神的儿子。

至于死人复活,摩西在荆棘篇上,称主是亚伯拉罕的神,以撒的神,雅各的神,就指示明白了。

神原不是死人的神,乃是活人的神。因为在他那里,人都是活的。那里或作看来

有几个文士说,夫子,你说得好。

以后他们不敢再问他什么。

耶稣对他们说,人怎吗说基督是大卫的子孙呢。

诗篇上,大卫自己说,主对我主说,你坐在我的右边,

等我使你的仇敌作你的脚凳。

大卫既称他为主,他怎吗又是大卫的子孙呢。

众百姓听的时候,耶稣对门徒说,

你们要防备文士。他们好穿长衣游行,喜爱人在街市上问他们安,又喜爱会堂里的高位,筵席上的首座。

他们侵吞寡妇的家产,假意作很长的祷告。这些人要受更重的刑罚。

\chapter{路加福音第21章}
耶稣抬头观看,见财主把捐项投在库里。

又见一个穷寡妇,投了两个小钱。

就说,我实在告诉你们。这穷寡妇,所投的比众人还多。

因为众人都是自己有馀,拿出来投在捐项里。但这个寡妇是自己不足,把他一切养生的都投上了。

有人谈论圣殿,是用美石和供物妆饰的。

耶稣就说,论到你们所看见的这一切,将来日子到了,在这里没有一块石头留在石头上,不被拆毁了。

他们问他说,夫子,什么时候有这件事,这事将到的时候,有什么豫兆呢。

耶稣说,你们要谨慎,不要受迷惑,因为将来有好些人冒我的名来,说,我是基督,又说,时侯近了,你们不要跟从他们。

你们听见打仗和扰乱的事,不要惊惶,因为这些事必须先有,只是末期不能立时就到。

当时耶稣对他们说,民要攻打民,国要攻打国。

地要大大震动,多处必有饥荒瘟疫。又有可怕的异象,和大神迹,从天上显现。

但这一切的事以先,人要下手拿住你们,逼迫你们,把你们交给会堂,并且收在监里,又为我的名拉你们到君王诸侯面前。

但这些事终必为你们的见证。

所以你们当立定心意,不要豫先思想怎样分诉。

因为我必赐你们口才智慧,是你们一切敌人所敌不住,驳不倒的。

连你们父母,弟兄,亲族,朋友,也要把你们交官。你们也有被他们害死的。

你们要为我的名,被众人恨恶。

然而你们连一根头发,也必不损坏。

你们常存忍耐,就必保全灵魂。或作必得生命

你们看见耶路撒冷被兵围困,就可知道他成荒场的日子近了。

那时,在犹太的,应当逃到山上。在城里的,应当出来。在乡下的,不要进城。

因为这是报应的日子,使经上所写的都得应验。

当那些日子,怀孕的和奶孩子的有祸。因为将有大灾难降在这地方,也有震怒临到这百姓。

他们要倒在刀下,又被掳到各国去,耶路撒冷要被外邦人践踏,直到外邦人的日期满了。

日月星辰要显出异兆。地上的邦国也有困苦。因海中波浪的响声,就慌慌不定。

天势都要震动。人想起那将要临到世界的事,就都吓得魂不附体。

那时,他们要看见人子,有能力,有大荣耀,驾云降临。

一有这些事,你们就当挺身昂首。因为你们得赎的日子近了。

耶稣又设比喻对他们说,你们看无花果树,和各样的树。

他发芽的时侯,你们一看见自然晓得夏天近了。

这样,你们看见这些事渐渐的成就,也该晓得神的国近了。

我实在告诉你们,这时代还没有过去,这些事都要成就。

天地要废去,我的话却不能废去。

你们要谨慎,恐怕因贪食醉酒并今生的思虑,累住你们的心,那日子就如同纲罗忽然临到你们。

因为那日子要这样临到全地上一切居住的人。

你们要时时儆醒,常常祈求,使你们能逃避这一切要来的事,得以站立在人子面前。

耶稣每日在殿里教训人,每夜出城在一座山,名叫橄榄山住宿。

众百姓清早上圣殿,到耶稣那里,要听他讲道。

\chapter{路加福音第22章}
除酵节,又名逾越节,近了。

祭司长和文士,想法子怎吗才能杀害耶稣,是因为他们惧怕百姓。

这时,撒但入了那称为加略人犹大的心,他本是十二门徒里的一个,

他去和祭司长并守殿官商量,怎吗可以把耶稣交给他们。

他们欢喜,就约定给他银子。

他应允了,就找机会要趁众人不在跟前的时候,把耶稣交给他们。

除酵节,须宰逾越羊羔的那一天到了。

耶稣打发彼得,约翰,说,你们去为我们豫备逾越节的筵席,好叫我们吃。

他们问他说,要我们在那里豫备。

耶稣说,你们进了城,必有人拿着一瓶水迎面而来。你们就跟着他,到他所进的房子里去。

对那家的主人说,夫子说,客房在那里,我与门徒好在那里吃逾越节的筵席。

他必指给你们摆设整齐的一间大楼,你们就在那里豫备。

他们去了,所遇见的,正如耶稣所说的。他们就豫备了逾越节的筵席)。

时候到了,耶稣坐席,使徒也和他同坐。

耶稣对他们说,我很愿意在受害以先,和你们吃这逾越节的筵席。

我告诉你们,我不再吃这筵席,直到我成就在神的国里。

耶稣接过杯来,祝谢了,说,你们拿这个,大家分着喝。

我告诉你们,从今以后,我不再喝这葡萄汁,直等神的国来到。

又拿起饼来祝谢了,就擘开递给他们,说,这是我的身体,为你们舍的。你们也应当如此行,为的是记念我。

饭后也照样拿起杯来,说,这杯是用我血所立的新约,是为你们流出来的。

看哪,那卖我之人的手,与我一同在桌子上。

人子固然要照所豫定的去世。但卖人子的人有祸了。

他们就彼此对问,是那一个要作这事。

门徒起了争论,他们中间那一个可算为大。

耶稣说,外邦人有君王为主治理他们。那掌权管他们的称为恩主。

但你们不可这样。你们里头为大的,倒要像年幼的。为首领的,倒要像服事人的。

是谁为大,是坐席的呢,是服事人的呢。不是坐席的大吗。然而我在你们中间,如同服事人的。

我在磨炼之中,常和我同在的就是你们。

我将国赐给你们,正如我父赐给我一样。

叫你们在我国里,坐在我的席上吃喝。并且坐在宝座上,审判以色列十二个支派。

主又说,西门,西门,撒但想要得着你们,好筛你们,像筛麦子一样。

但我已经为你们祈求,叫你不至于失了信心。你回头以后,要坚固你的弟兄。

彼得说,主阿,我就是同你下监,同你受死,也是甘心。

耶稣说,彼得,我告诉你,今日鸡还没有叫,你要三次说不认得我。

耶稣又对他们说,我差你们出去的时候,没有钱囊,没有口袋,没有鞋,你们缺少什么没有。他们说,没有。

耶稣说,但如今有钱囊的可以带着,有口袋的也可以带着。没有刀的要卖衣服买刀。

我告诉你们,经上写着说,他被列在罪犯之中。这话必应验在我身上,因为那关系我的事,必然成就。

他们说,主阿,请看,这里有两把刀。耶稣说,壳了。

耶稣出来,照常往橄榄山去。门徒也跟随他。

到了那地方,就对他们说,你们要祷告,免得入了迷惑。

于是离开他们,约有扔一块石头那吗远,跪下祷告,

说,父阿,你若愿意,就把这杯撤去。然而不要成就我的意思,只要成就你的意思。

有一位天使,从天上显现,加添他的力量。

耶稣极其伤痛,祷告更加恳切。汗珠如大血点,滴在地上。

祷告完了,就起来,到门徒那里,见他们因为忧愁都睡着了。

就对他们说,你们为什么睡觉呢。起来祷告,免得入了迷惑。

说话之间,来了许多人,那十二个门徒里名叫犹大的,走在前头,就近耶稣,要与他亲嘴。

耶稣对他说,犹大,你用亲嘴的暗号卖人子吗。

左右的人见光景不好,就说,主阿,我们拿刀砍可以不可以。

内中有一个人,把大祭司的仆人砍了一刀,削掉了他的右耳。

耶稣说,到了这个地步,由他们吧。就摸那人的耳朵,把他治好了。

耶稣对那些来拿他的祭司长,和守殿官,并长老,说,你们带着刀棒,出来拿我,如同拿强盗吗。

我天天同你们在殿里,你们不下手拿我。现在却是你们的时候,黑暗掌权了。

他们拿住耶稣,把他带到大祭司的宅里。彼得远远的跟着。

他们在院子里生了火,一同坐着。彼得也坐在他们中间。

有一个使女,看见彼得坐在火光里,就定睛看他,说,这个人素来也是同那人一夥的。

彼得却不承认,说,女子,我不认得他。

过了不多的时候,又有一各人看见他,说,你也是他们一党的。彼得说,你这个人,我不是。

约过了一小时,又有一个人极力的说,他实在是同那人一夥的。因为他也是加利利人。

彼得说,你这个人,我不晓得你说的是什么。正说话之间鸡就叫了。

主转过身来,看彼得。彼得便想起主对他所说的话,今日鸡叫以先,你要三次不认我。

他就出去痛哭。

看守耶稣的人戏弄他,打他,

又蒙着他的眼睛问他说,你是先知,告诉我们,打你的是谁。

他们还用别的话辱骂他。

天一亮,民间的众长老连祭司长带文士都聚会。把耶稣带到他们的公会里,

说,你若是基督,就告诉我们。耶稣说,我告诉你们,你们也不信。

我若问你们,你们也不回答。

从今以后,人子要坐在神权能的右边。

他们都说,这样,你是神的儿子吗。耶稣说,你们所说的是。

他们说,何必再用见证呢。他亲口所说的,我们都亲自听见了。

\chapter{路加福音第23章}
众人都起来,把耶稣解到彼拉多面前。

就告他说,我们见这人诱惑国民,禁止纳税给凯撒,并说自己是基督,是王。

彼拉多问耶稣说,你是犹太人的王吗。耶稣回答说,你说的是。

彼拉多对祭司长和众人说,我查不出这人有什么罪来。

但他们越发极力的说,他煽惑百姓,在犹太遍地传道,从加利利起,直到这里了。

彼拉多一听见,就问这人是加利利人吗。

既晓得耶稣属希律所管,就把他送到希律那里去。那时希律正在耶路撒冷。

希律看见耶稣,就很欢喜。因为听见过他的事,久已想要见他。并且指望看他行一件神迹。

于是问他许多的话。耶稣却一言不答。

祭司长和文士,都站着极力的告他。

希律和他的兵丁就藐视耶稣,戏弄他,给他穿上华丽衣服,把他送回彼拉多那里去。

从前希律和彼拉多彼此有仇。在那一天就成了朋友。

彼拉多传齐了祭司长,和官府,并百姓,

就对他们说,你们解这人到我这里,说他是诱惑百姓的。看哪,我也曾将你们告他的事,在你们面前审问他,并没有查出他什么罪来。

就是希律也是如此,所以把他送回来。可见他没有作什么该死的事。

故此我要责打他,把他释放了。有古卷在此有

每逢这节期巡抚必须释放一个囚犯给他们

众人却一齐喊着说,除掉这个人,释放巴拉巴给我们。

这巴拉巴是因在城里作乱杀人下在监里的。

彼拉多愿意释放耶稣,就又劝解他们。

无奈他们喊着说,钉他十字架,钉他十字架

彼拉多第三次对他们说,为什么呢,这人作了什么恶事呢,我并没有查出他什么该死的罪来。所以我要责打他,把他释放了。

他们大声催逼彼拉多,求他把耶稣钉在十字架上。他们的声音就得了胜,

彼拉多这才照他们所求的定案。

把他们所求的那作乱杀人下在监里的,释放了。把耶稣交给他们,任凭他们的意思行。

带耶稣去的时候,有一个古利奈人西门,从乡下来。他们就抓住他,把十字架搁在他身上,叫他背着跟随耶稣。

有许多百姓,跟随耶稣,内中有好些妇女,妇女们为他号??痛哭。

耶稣转身对他们说,耶路撒冷的女子,不要为我哭,当为自己和自己的儿女哭。

因为日子要到,人必说,不生育的,和未会怀胎的,未曾乳养婴孩的,有福了。

那时,人要向大山说,倒在我们身上。向小山说,遮盖我们。

这些事既行在有汁水的树上,那枯乾的树,将来怎吗样呢。

又有两个犯人,和耶稣一同带来处死。

到了一个地方,名叫髑髅地,就在那里把耶稣钉在十字架上,又钉了两个犯人,一个在左边,一个在右边。

当下耶稣说,父阿,赦免他们。因为他们所作的,他们不晓得。兵丁就拈阄分他的衣服。

百姓站在那里观看。官府也嗤笑他说,他救了别人。他若是基督,神所拣选的,可以救自己吧。

兵丁也戏弄他,上前拿醋送给他喝,

说,你若是犹太人的王,可以救自己吧。

在耶稣以上有一个牌子,有古卷在此有用希腊罗马希伯来的文字写着,这是犹太人的王。

那同钉的两个犯人,有一个讥诮他说,你不是基督吗。可以救自己和我们吧。

那一个就应声责备他说,你既是一样受刑的,还不怕神吗。

我们是应该的。因为我们所受的,与我们所作的相称。但这个人没有作过一件不好的事。

就说,耶稣阿,你得国降临的时候,求你记念我。

耶稣对他说,我实在告诉你,今日你要同我在乐园里了。

那时约有午正,遍地都黑暗了,直到申初,

日头变黑了。殿里的幔子从当中裂为两半。

耶稣大声喊着说,父阿,我将我的灵魂交在你手里。说了这话,气就断了。

百夫长看见所成的事,就归荣耀与神说,这真是个义人。

聚集观看的众人,见了这所成的事,都捶着胸回去了。

还有一切与耶稣熟识的人,和从加利利跟着他来的妇女们,都远远的站着,看着这些事。

有一个人名叫约瑟,是个议士,为人善良公义。

众人所谋所为,他并没有附从。他本是犹太亚利马太城里素常盼望神国的人。

这人去见彼拉多,求耶稣的身体。

就取下来用细麻布裹好,安放在石头凿成的坟墓里,那里头从来没有葬过人。

那日是豫备日,安息日也快到了。

那些从加利利和耶稣同来的妇女,跟在后面,看见了坟墓,和他的身体怎样安放。

他们就回去,豫备了香料香膏。他们在安息日,便遵着诫命安息了。

\chapter{路加福音第24章}
七日的头一日,黎明的时候,那些妇女带着所豫备的香料,来到坟墓前。

看见石头已经从坟墓辊开了。

他们就进去,只是不见主耶稣的身体。

正在猜疑之间,忽然有两个人站在旁边。衣服放光。

妇女们惊怕,将脸伏地。那两个人就对他们说,为什么在死人中找活人呢。

他不在这里,已经复活了。当记念他还在加利利的时候,怎样告诉你们,

说,人子必须被交在罪人手里,钉在十字架上,第三日复活。

他们就想起耶稣的话来,

便从坟墓那里回去,把这一切事告诉十一个使徒和其馀的人。

那告诉使徒的,就是抹大拉的马利亚,和约亚拿,并雅各的母亲马利亚,还有与他们在一处的妇女。

他们这些话,使徒以为是胡言,就不相信。

彼得起来,跑到坟墓前,低头往里看,见细麻布独在一处,就回去了,心里希奇所成的事。

正当那日,门徒中有两个人往一个村子去,这村子名叫以马忤斯,离耶路撒冷约有二十五里。

他们彼此谈论所遇见的这一切事。

正谈论相问的时候,耶稣亲自就近他们,和他们同行。

只是他们的眼睛迷糊了,不认识他。

耶稣对他们说,你们走路彼此谈论的是什么事呢。他们就站住,脸上带着愁容。

二人中有一个名叫革流巴的,回答说,你在耶路撒冷作客,还不知道这几天在那里所出的事吗。

耶稣说,什么事呢。他们说,就是拿撒勒人耶稣的事。他是个先知,在神和众百姓面前,说话行事都有大能。

祭司长和我们的官府,竟把他解去定了死罪,钉在十字架上。

但我们素来所盼望要赎以色列民的就是他。不但如此,而且这事成就,现在已经三天了。

再者,我们中间有几个妇女使我们惊奇,他们清早到了坟墓那里。

不见他的身体,就回来告诉我们说,看见了天使显现,说他活了。

又有我们的几个人,往坟墓那里去,所遇见的,正如妇女们所说的,只是没有看见他。

耶稣对他们说,无知的人哪,先知所说的一切话,你们的心,信得太迟钝了。

基督这样受害,又进入他的荣耀,岂不是应当的吗。

于是从摩西和众先知起,凡经上所指着自己的话,都给他们讲解明白了。

将近他们所去的村子,耶稣好像还要往前行

他们却强留他说,时候晚了,日头已经平西了,请你同我住下吧。耶稣就进去,要同他们住下。

到了坐席的时候,耶稣拿起饼来,祝谢了,擘开,递给他们。

他们的眼睛明亮了,这才认出他来。忽然耶稣不见了。

他们彼此说,在路上,他和我们说话,给我们讲解圣经的时候,我们的心岂不是火热的吗。

他们就立时起身,回耶路撒冷去,正遇见十一个使徒,和他们的同人,聚集在一处。

说,主果然复活,已经现给西门看了。

两个人就把路上所遇见,和擘饼的时候怎样被他们认出来的事,都述说了一遍。

正说这话的时候,耶稣亲自站在他们当中,说,愿你们平安。

他们却惊慌害怕,以为所看见的是魂。

耶稣说,你们为什么愁烦。为什么心里起疑念呢。

你们看我的手,我的脚,就知道实在是我了。摸我看看。魂无骨无肉,你们看我是有的。

说了这话,就把手和脚给他们看。

他们正喜得不敢信,并且希奇,耶稣就说,你们这里有什么吃的没有。

他们便给他一片烧鱼。有古卷在此有和一块蜜房

他接过来,在他们面前吃了。

耶稣对他们说,这就是我从前与你们同在之时,所告诉你们的话,说,摩西的律法,先知的书,和诗篇上所记的,凡指着我的话,都必须应验。

于是耶稣开他们的心窍,使他们能明白圣经。

又对他们说,照经上所写的,基督必受害,第三日从死里复活。

并且人要奉他的名传悔改赦罪的道,从耶路撒冷起直传到万邦。

你们就是这些事的见证。

我要将我父所应许的降在你们身上。你们要在城里等候,直到你们领受从上头来的能力。

耶稣领他们到伯大尼的对面,就举手给他们祝福。

正祝福的时候,他就离开他们,被带到天上去了。

他们就拜他,大大的欢喜,回耶路撒冷去。

常在殿里称颂神。

\chapter{约翰福音第1章}
太初有道,道与神同在,道就是神。

这道太初与神同在。

万物是藉着他造的。凡被造的,没有一样不是藉着他造的。

生命在他里头。这生命就是人的光。

光照在黑暗里,黑暗却不接受光。

有一个人,是从神那里差来的,名叫约翰。

这人来,为要作见证,就是为光作见证,叫众人因他可以信。

他不是那光,乃是要为光作见证。

那光是真光,照亮一切生在世上的人。

他在世界,世界也是藉着他造的,世界却不认识他。

他到自己的地方来,自己的人倒不接待他。

凡接待他的,就是信他名的人,他就赐他们权柄,作神的儿女。

这等人不是从血气生的,不是从情欲生的,也不是从人意生的,乃是从神生的。

道成了肉身住在我们中间,充充满满的有恩典有真理。我们也见过他的荣光,正是父独生子的荣光。

约翰为他作见证,喊着说,这就是我曾说,那在我以后来的,反成了在我以前的。因他本来在我以前。

从他丰满的恩典里我们都领受了,而且恩上加恩。

律法本是藉着摩西传的,恩典和真理,都是由耶稣基督来的。

从来没有人看见神。只有在父怀里的独生子将他表明出来。

约翰所作的见证,记在下面。犹太人从耶路撒冷差祭司和利未人到约翰那里,问他说,你是谁。

他就明说,并不隐瞒。明说,我不是基督。

他们又问他说,这样你是谁呢,是以利亚吗。他说,我不是。是那先知吗,他回答说,不是。

于是他们说,你到底是谁,叫我们好回覆差我们来的人。你自己说,你是谁。

他说,我就是那在旷野有人声喊着说,修直主的道路,正如以赛亚所说的。

那些人是法利赛人差来的。(或作那差来的是法利赛人)

他们就问他说,你既不是基督,不是以利亚,也不是那先知,为什么施洗呢。

约翰回答说,我是用水施洗,但有一位站在你们中间,是你们不认识的,

就是那在我以后来的,我给他解鞋带,也不配。

这是在约旦河外,伯大尼,(有古卷作伯大巴喇)约翰施洗的地方作的见证。

次日,约翰看见耶稣来到他那里,就说,看哪,神的羔羊,除去(或作背负)世人罪孽的。

这就是我曾说,有一位在我以后来,反成了在我以前的。因他本来在我以前。

我先前不认识他,如今我用水施洗,为要叫他显明给以色列人。

约翰又作见证说,我曾看见圣灵,彷佛鸽子从天降下,住在他的身上。

我先前不认识他。只是那差我来用水施洗的,对我说,你看见圣灵降下来,住在谁的身上,谁就是用圣灵施洗的。

我看见了,就证明这是神的儿子。

再次日,约翰同两个门徒站在那里。

他见约稣行走,就说,看哪,这是神的羔羊。

两个门徒听见他的话,就跟从了耶稣。

耶稣转过身来,看见见他们跟着,就问他们说,你们要什么。他们说,拉比,在那里住。(拉比翻出来,就是夫子

耶稣说,你们来看。他们就去看他在那里住,这一天便与他同住,那时约有申正了。

听见约翰的话,跟从耶稣的那两个人,一个是西门彼得的兄弟安得烈。

他先找着自己的哥哥西门,对他说,我们遇见弥赛亚了,(弥赛亚翻出来,就是基督)

于是领他去见耶稣。耶稣看见他说,你是约翰的儿子西门,(约翰马太十六十七节称约拿)你要称为矶法。(矶法翻出来,就是彼得)

又次日,耶稣想要往加利利去,遇见腓力,就对他说,来跟从我吧。

这腓力是伯赛大人,和安得烈同城。

腓力找着拿但业,对他说,摩西在律法上所写的,和众先知所记的那一位,我们遇见了,就是约瑟的儿子拿撒勒人耶稣。

拿但业对他说,拿撒勒还能出什么好的吗。腓力说,你来看。

耶稣看见拿但业来,就指着他说,看哪,这是个真以色列人,他心里是没有诡诈的。

拿但业对耶稣说,你从那里知道我呢。耶稣回答说,腓力还没有招呼你,你在无花果树底下,我就看见你了。

拿但业说,拉比,你是神的儿子,你是以色列的王。

耶稣对他说,因为我说在无花果树底下看见你,你就信吗。你将要看见比这更大的事。

又说,我实实在在的告诉你们,你们将要看见天开了,神的使者上去下来在人子身上。

\chapter{约翰福音第2章}
第三日,在加利利的迦拿有娶亲的筵席。耶稣的母亲在那里。

耶稣和他的门徒也被请去赴席。

酒用尽了,耶稣的母亲对他说,他们没有酒了。

耶稣说,母亲,(原文作妇人)我与你有什么相干。我的时候还没有到。

他母亲对用人说,他告诉你们什么,你们就作什么。

照犹太人洁净的规矩,有六口缸摆在那里,每口可以盛两三桶水。

耶稣对用人说,把缸倒满了水。他们就倒满了,直到缸口。

耶稣又说,现在可以舀出来,送给管筵席的。他们就送了去。

管筵席尝了那水变的酒,并不知道是那里来的,只有舀水的用人知道。管筵席的便叫新郎来。

对他说,人都是先摆上好酒。等客喝足了,才摆上次的。你倒把好酒留到如今。

这是耶稣所行的头一件神迹,是在加利利的迦拿行的,显出他的荣耀来。他的门徒就信他了。

这事以后,耶稣与他的母亲弟兄和门徒,都下迦百农去。在那里住了不多几日。

犹太人的逾越节近了,耶稣就上耶路撒冷去。

看见殿里有卖牛羊鸽子的,并有兑换银钱的人,坐在那里。

耶稣就拿绳子作成鞭子,把牛羊都赶出殿去。倒出兑换银钱之人的银钱,推翻他们的桌子。

又对卖鸽子的说,把这些东西拿去。不要将我父的殿,当作买卖的地方。

他的门徒就想起经上记着说,我为你的殿,心里焦急,如同火烧。

因此,犹太人问他说,你既作这些事,还显什么神迹给我们看呢。

耶稣回答说,你们拆毁这殿,我三日内要再建立起来。

犹太人便说,这殿是四十六年才造成的,你三日内就再建立起来吗。

但耶稣这话,是以他的身体为殿。

所以到他从死里复活以后,门徒就想起他说过这话,便信了圣经和耶稣所说的。

当耶稣在耶路撒冷过逾越节的时候,有许多人看见他所行的神迹,就信了他的名。

耶稣却不将自己交托他们,因为他知道万人。

也用不着谁见证人怎样。因他知道人心里所存的。

\chapter{约翰福音第3章}
有一个法利赛人,名叫尼哥底母,是犹太人的官。

这人夜里来见耶稣,说,拉比,我们知道你是由神那里来作师傅的。因为你所行的神迹,若没有神同在,无人能行。

耶稣回答说,我实实在在的告诉你,人若不重生,就不能见神的国。

尼哥底母说,人已经老了,如何能重生呢。岂能再进母腹生出来吗。

耶稣说,我实实在在的告诉你,人若不是从水和圣灵生的,就不能进神的国。

从肉身生的,就是肉身。从灵生的,就是灵。

我说,你们必须重生,你不要以为希奇。

风随着意思吹,你听见风的响声,却不晓得从那里来,往那里去。凡从圣灵生的,也是如此。

尼哥底母问他说,怎能有这事呢。

耶稣回答说,你是以色列人的先生,还不明白这事吗。

我实实在在的告诉你,我们所说的,是我们知道的,我们所见证的,是我们见过的。你们却不领受我们的见证。

我对你们说地上的事,你门尚且不信,若说天上的事,如何能信呢。

除了从天降下仍旧在天的人子,没有人升过天。

摩西在旷野怎样举蛇,人子也必照样被举起来。

叫一切信他的都得永生。(或作叫一切信的人在他里面得永生)

神爱世人,甚至将他的独生子赐给他们,叫一切信他的,不至灭亡,反得永生。

因为神差他的儿子降世,不是要定世人的罪,(或作审判世人下同)乃是要叫世人因他得救。

信他的人,不被定罪。不信的人,罪已经定了,因为他不信神独生子的名。

光来到世间,世人因自己的行为是恶的,不爱光倒爱黑暗,定他们的罪就是在此。

凡作恶的便恨光,并不来就光,恐怕他的行为受责备。

但行真理的必来就光,要显明他所行的是靠神而行。

这事以后,耶稣和门徒到了犹太地,在那里居住施洗。

约翰在靠近撒冷的哀嫩也施洗,因为那里水多。众人都去受洗。

那时约翰还没有下在监里。

约翰的门徒,和一个犹太人辩论洁净的礼。

就来见约翰说,拉比,从前同你在约旦河外,你所见证的那位,现在施洗,众人都往他那里去。

约翰说,若不是从天上赐的,人就不能得什么。

我曾说,我不是基督,是奉差遣在他前面的,你们可以给我作见证。

娶新妇的,就是新郎。新郎的朋友站着听见新郎的声音就甚喜乐。故此我这喜乐满足了。

他必兴旺,我必衰微。

从天上来的,是在万有之上。从地上来的,是属乎地,他所说的,也是属乎地从天上来的,是在万有之上。

他将所见所闻的见证出来,只是没有人领受他的见证。

那领受他见证的,就印上印,证明神是真的。

神所差来的,就说神的话。因为神赐圣灵给他,是没有限量的。

父爱子,已将万有交在他手里。

信子的人有永生。不信子的人得不着永生,(原文作不得见永生)神的震怒常在他身上。

\chapter{约翰福音第4章}
主知道法利赛人听见他收门徒施洗比约翰还多,

(其实不是耶稣亲自施洗,乃是他的门徒施洗)

他就离了犹太,又往加利利去。

必须经过撒玛利亚。

于是到了撒玛利亚的一座城,名叫叙加,靠近雅各给他儿子约瑟的那块地。

在那里有雅各井。耶稣因走路困乏,就坐在井旁。那时约有午正。

有一个撒玛利亚的妇人来打水。耶稣对他说,请你给我水喝。

那时门徒进城买食物去了。

撒玛利亚的妇人对他说,你既是犹太人,怎吗向我一个撒玛列亚妇人要水喝呢。原来犹太人和撒玛利亚人没有来往。

耶稣回答说,你若知道神的恩赐,和对你说给我水喝的是谁,你必早求他,他也必早给了你活水。

妇人说,先生没有打水的器具,井又深,你从那里得活水呢。

我们的祖宗雅各,将这井留给我们。他自己和儿子并牲畜,也都喝这井里的水,难道你比他还大吗。

耶稣回答说,凡喝这水的,还要再渴。

人若喝我所赐的水就永远不渴。我所赐的水,要在他里头成为泉源,直涌到永生。

妇人说,先生,请把这水赐给我,叫我不渴,也不用来这吗远打水。

耶稣说,你去叫你丈夫也到这里来。

妇人说,我没有丈夫。耶稣说,你说没有丈夫,是不错的。

你已经有五个丈夫。你现在有的,并不是你的丈夫。你这话是真的。

妇人说,先生,我看出你是先知。

我们的祖宗在这山上礼拜。你们倒说,应当礼拜的地方是在耶路撒冷。

耶稣说,妇人,你当信我,时候将到,你们拜父,也不在这山上,也不在耶路撒冷。

你们所拜的,你们不知道。我们所拜的,我们知道。因为救恩是从犹太人出来的。

时候将到,如今就是了,那真正拜父的,

神是个灵(或无个字)所以拜他的,必须用心灵和诚实拜他。

妇人说,我知道弥赛亚,(就是那称为基督的)要来。他来了,必将一切的事都告诉我们。

耶稣说,这和你说话的就是他。

当下门徒回来,就希奇耶稣和一个妇人说话。只是没有人说,你是要什么。或说,你为什么和他说话。

那妇人就留下水罐子,往城里去,对众人说,

你们来看,有一个人将我素来所行的一切事,都给我说出来了,莫非这就是基督吗。

众人就出城往耶稣那里去。

这其间,门徒对耶稣说,拉比请吃。

耶稣说,我有食物吃,是你们不知道的。

门徒就彼此对问说,莫非有人拿什么给他吃吗。

耶稣说,我的食物,就是遵行差我来者的旨意,作成他的工。

你们岂不说,到收割的时候,还有四个月吗。我告诉你们,举目向田观看,庄稼已经熟了,(原文作发白)可以收割了。

收割的人得工价,积畜五谷到永生。叫撒种的和收割的一同快乐。

俗语说,那人撒种,这人收割,这话可见是真的。

我差你们去收你们所没有劳苦的。别人劳苦,你们享受他们所劳苦的。

那城里有好些撒玛利亚人信了耶稣,因为那妇人作见证说,他将我素来所行的一切事,都给我说出来了。

于是撒玛利亚人来见耶稣,求他在他们那里住下。他便在那里住了两天。

因耶稣的话,信的人就更多了。

便对妇人说,现在我们信,不是因为你的话,是我们亲自见了,知道这真是救世主。

过了那两天,耶稣离了那地方,往加利利去。

因为耶稣自己作见证说,先知在本地是没有人尊敬的。

到了加利利,加利利人既然看见他在耶路撒冷过节所行的一切事,就接待他。因为他们也是上去过节。

耶稣又到了加利利的迦拿,就是他从前变水为酒的地方。有一个大臣,他的儿子在迦百农患病。

他听见耶稣从犹太到了加利利,就来见他,求他下去医治他的儿子。因为他儿子快要死了。

耶稣就对他说,若不看见神迹奇事,你们总是不信。

那大臣说,先生,求你趁着我的孩子还没有死,就下去。

耶稣对他说,回去吧。你的儿子活了。那人信耶稣所说的话,就回去了。

正下去的时候,他的仆人迎见他,说,他的儿子活了。

他就问什么时候见好的。他们说,昨日未时热就退了。

他便知道这正是耶稣对他说,你的儿子活了的时候,他自己全家就都信了。

这是耶稣在加利利行的第二件神迹,是他从犹太回去以后行的。

\chapter{约翰福音第5章}
这事以后,到了犹太的一个节期。耶稣就上耶路撒冷去。

在耶路撒冷,靠近羊门,有一个池子,希伯来话叫作毕士大,旁边有五个廊子。

里面躺着瞎眼的,瘸腿的,血气枯乾的,许多病人。(有古卷在此有等候水动

因为有天使按时下池搅动那水,水动之后,谁先下去,无论什么病,就痊愈了)。

在那里有一个人,病了三十八年。

耶稣看见他躺着,知道他病了许久,就问他说,你要痊愈吗。

病人回答说,先生,水动的时候,没有人把我放在池子里。我正去的时候,就有别人比我先下去。

耶稣对他说,起来,拿你的褥子走吧。

那人立刻痊愈,就拿起褥子走了。

那天是安息日,所以犹太人对那医好的人说,今天是安息日,你拿褥子是不可以的。

他却回答说,那使我痊愈的,对我说,拿你的褥子走吧。

他们问他说,对你说拿褥子走的,是什么人。

那医好的人不知道是谁。因为那里的人多,耶稣已经躲开了。

后来耶稣在殿里遇见他,对他说,你已经痊愈了。不要再犯罪。恐怕你遭遇的更加利害。

那人就去告诉犹太人,使他痊愈的是耶稣。

所以犹太人逼迫耶稣,因为他在安息日作了这事。

耶稣就对他们说,我父作事到如今,我也作事。

所以犹太人越发想要杀他。因他不但犯了安息日,并且称神为他的父,将自己和神当作平等。

耶稣对他们说,我实实在在的告诉你们,子凭着自己不能作什么,惟有看见父所作的,子才能作。父所作的事,子也照样作。

爱子,将自己所作的一切事指给他看。还要将比这更大的事指给他看,叫你们希奇。

父怎样叫死人起来,使他们活着,子也照样随自己的意思使人活着。

父不审判什么人,乃将审判的事全交与子。

叫人都尊敬子,如同尊敬父一样。不尊敬子的,就是不尊敬差子来的父。

我实实在在的告诉你们,那听我话,又信差我来者的,就有永生,不至于定罪,是已经出死入生了。

我实实在在的告诉你们,时候将到,现在就是了,死人要听见神儿子的声音。听见的人就要活了。

因为父怎样在自己有生命,就赐给他儿子也照样在自己有生命。

并且因为他是人子,就赐给他行审判的权柄。

你们不要把这事看作希奇。时候要到,凡在坟墓里的,都要听见他的声音,就出来。

行善的复活得生,作恶的复活定罪。

我凭着自己不能作什么。我怎吗听见,就怎吗审判。我的审判也是公平的。因为我不求自己的意思,只求那差我来者的意思。

我若为自己作见证,我的见证就不真。

另有一位给我作见证。我也知道他给我作的见证是真的。

你们曾差人到约翰那里,他为真理作过见证。

其实我所受的见证,不是从人来的。然而我说这些话,为要叫你们得救。

约翰是点着的明灯。你们情愿暂时喜欢他的光。

但我有比约翰更大的见证。因为父交给我要我成就的事,就是我所作的事,这便见证我是父所差来的。

差我来的父,也为我作过见证。你们从来没有听见他的声音,也没有看见他的形像。

你们并没有他的道存在心里。因为他所差来的,你们不信。

你们查考圣经。(或作应当查考圣经)因你们以为内中有永生。给我作见证的就是这经。

然而你们不肯到我这里来得生命。

我不受从人来的荣耀。

但我知道你们心里,没有神的爱。

我奉我父的名来,你们并不接待我。若有别人奉自己的名来,你们倒要接待他。

你们要互相受荣耀,却不求从独一之神来的荣耀,怎能信我呢。

不要想我在父面前要告你们。有一位告你们的,就是你们所仰赖的摩西。

你们如果信摩西,也必信我。因为他书上有指着我写的话。

你们若不信他的书,怎能信我的话呢。

\chapter{约翰福音第6章}
这事以后,耶稣渡过加利利海,就是提比哩亚海。

有许多人,因为看见他在病人身上所行的神迹,就跟随他。

耶稣上了山,和门徒一同坐在那里。

那时犹太人的逾越节近了。

耶稣举目看见许多人来,就对腓力说,我们从那里买饼叫这些人吃呢。

他说这话,是要试验腓力。他自己原知道要怎样行。

腓力回答说,就是二十两银子的饼,叫他们各人吃一点,也是不够的。

有一个门徒,就是西门彼得的兄弟安得烈,对耶稣说,

在这里有一个孩童,带着五个大麦饼,两条鱼。只是分给这许多人,还算什么呢。

耶稣说,你们叫众人坐下。原来那地方的草多,众人就坐下。数目约有五千。

耶稣拿起饼来,祝谢了,就分给那坐着的人。分鱼也是这样,都随着他们所要的。

他们吃饱了,耶稣对门徒说,把剩下的零碎,收拾起来,免得有糟蹋的。

他们便将那五个大麦饼的零碎,就是众人剩下的,收拾起来,装满了十二个篮子。

众人看见耶稣所行的神迹。就说,这真是那要到世间来的先知。

耶稣既知道众人要来强逼他作王,就独自又退到山上去了。

到了晚上,他的门徒下海边去,

上了船,要过海往迦百农去。天已经黑了,耶稣还没有来到他们那里。

忽然狂风大作,海就翻腾起来。

门徒摇橹约行了十里多路,看见耶稣在海面上走,渐渐近了船,他们就害怕。

耶稣对他们说,是我。不要怕。

门徒就喜欢接他上船,船立时到了他们所要去的地方。

第二日,站在海那边的众人,知道那里没有别的船,只有一只小船,又知道耶稣没有同他的门徒上船,乃是门徒自己去的。

然而有几只小船从提比哩亚来,靠近主祝谢后分饼给人吃的地方。

众人见耶稣和门徒,都不在那里,就上了船,往迦百农去找耶稣。

既在海那边找着了,就对他说,拉比,是几时到这里来的。

耶稣回答说,我实实在在的告诉你们,你们找我,并不是因见了神迹,乃是因吃饼得饱。

不要为那必坏的食物劳力,要为那存到永生的食物劳力,就是人子要赐给你们的。因为人子是父神所印证的。

众人问他说,我们当行什么,才算作神的工呢。

耶稣回答说,信神所差来的,这就是作神的工。

他们又说,你行什么神迹,叫我们看见就信你。你到底作什么事呢。

我们的祖宗在旷野吃过吗哪,如经上写着说,他从天上赐下粮来给他们吃。

耶稣说,我实实在在的告诉你们,那从天上来的粮,不是摩西赐给你们的,乃是我父将天上的真粮赐给你们。

因为神的粮,就是那从天上降下来赐生命给世界的。

他们说,主阿,常将这粮赐给我们。

耶稣说,我就是生命的粮。到我这里来的,必定不饿。信我的,永远不渴。

只是我对你们说过,你们已经看见我,还是不信。

凡父所赐给我的人,必到我这里来。到我这里来的,我总不丢弃他。

因为我从天上!降下来,不是要按自己的意思行,乃是要按那差我来者的意思行。

差我来者的意思,就是他所赐给我的,叫一个也不失落,在末日却叫他复活。

因为我父的意思,是叫一切见子而信的人得永生。并且在末日我要叫他复活。

犹太人因为耶稣说,我是从天上降下来的粮,就私下议论他。

说,这不是约瑟的儿子耶稣吗。他的父母我们岂不认得吗。他如今怎吗说,我是从天上降下来的呢。

耶稣回答说,你们不要大家议论。

若不是差我来的父吸引人,就没有能到我这里来的。到我这里来的,在末日我要叫他复活。

在先知书上写着说,他们都要蒙神的教训。凡听见父之教训又学习的,就到我这里来。

这不是说,有人看过父,惟独从神来的,他看见过父。

我实实在在的告诉你们,信的人有永生。

我就是生命的粮。

你们的祖宗在旷野吃过吗哪,还是死了。

这是从天上降下来的粮,叫人吃了就不死。

我是从天上降下来生命的粮。人若吃这粮,就必永远活着。我所要赐的粮,就是我的肉,为世人之生命所赐的。

因此,犹太人彼此争论说,这个人怎能把他的肉,给我们吃呢。

耶稣说,我实实在在的告诉你们,你们若不吃人子的肉,不喝人子的血,就没有生命在你们里面。

吃我肉,喝我血的人就有永生。在末日我要叫他复活。

我的肉真是可吃的,我的血真是可喝的。

吃我肉喝我血的人,常在我里面,我也常在他里面。

永活的父怎样差我来,我又因父活着,照样,吃我肉的人,也要因我活着。

这就是从天上降下来的粮。吃这个粮的人,就永远活着,不像你们的祖宗吃过吗哪,还是死了。

这些话是耶稣在迦百农会堂里教训人说的。

他的门徒中有好些人听见了,就说,这话甚难,谁能听呢。

耶稣心里知道门徒为这话议论,就对他们说,这话是叫你们厌弃吗。(厌弃原文作跌倒)

倘或你们看见人子升到他原来所在之处,怎吗样呢。

叫人活着的乃是灵,肉体是无益的。我对你们所说的话,就是灵,就是生命。

只是你们中间有不信的人。耶稣从起头就知道,谁不信他,谁要卖他。

耶稣又说,所以我对你们说过,若不是蒙我父的恩赐,没有人能到我这里来。

从此他门徒中多有退去的,不再和他同行。

耶稣就对那十二个门徒说,你们也要去吗。

西门彼得回答说,主阿,你有永生之道,我们还归从谁呢。

我们已经信了,又知道你是神的圣者。

耶稣说,我不是拣选了你们十二个门徒吗,但你们中间有一个是魔鬼。

耶稣这话是指着加略人西门的儿子犹大说的。他本是十二个门徒里的一个,后来要卖耶稣的。

\chapter{约翰福音第7章}
这事以后,耶稣在加利利游行,不愿在犹太游行。因为犹太人想要杀他。

当时犹太人的住棚节近了。

耶稣的弟兄就对他说,你离开这里上犹太去吧,叫你的门徒也看见你所行的事。

人要显扬名声,没有在暗处行事的。你如果行这事,就当将自己显明给世人看。

因为连他的弟兄说这话,是因为不信他。

耶稣就对他们说,我的时候还没有到。你们的时候常是方便的。

世人不能恨你们,却是恨我。因为我指证他们所作的事是恶的。

你们上去过节吧。我现在不上去过这节。因为我的时候还没有满。

耶稣说了这话,仍旧住在加利利。

但他兄弟上去以后,他也上去过节,不是明去,似乎是暗去的。

正在节期,犹太人寻梢耶稣说,他在那里。

众人为他纷纷议论。有的说,他是好人。有的说,不然,他是迷惑众人的。

只是没有人明明的讲论他,因为怕犹太人。

到了节期,耶稣上殿里去教训人。

犹太人就希奇说,这个人没有学过,怎吗明白书呢。

耶稣说,我的教训不是我自己的,乃是那差我来者的。

人若立志遵着他的旨意行,就必晓得这教训或是出于神,或是我凭着自己说的。

人凭着自己说,是求自己的荣耀。惟有求那差他来者的荣耀,这人是真的,在他心里没有不义。

摩西岂不是传律法给你们吗。你们却没有一个人守律法。为什么想要杀我呢。

众人回答说,你是被鬼附着了。谁想要杀你。

耶稣说,我作了一件事,你们都以为希奇。

摩西传割礼给你们,(其实不是从摩西起的,乃是从祖先起的)因此你们也要在安息日给人行割礼。

人若在安息日受割礼,免得违背摩西的律法。我在安息日叫一个人全然好了,你们就向我生气吗。

不可按外貌断定是非,总要按公平断定是非。

耶路撒冷人中有的说,这不是他们想要杀的人吗。

你看他还明明的讲道,他们也不向他说什么。难道官长真知道这是基督吗。

然而我们知道这个人从那里来。只是基督来的时候,没有人知道他是从那里来的。

那时耶稣在殿里教训人,大声说,你们也知道我,也知道我从那里来。我来并不是由于自己,但那差我来的是真的。你们不认识他。

我却认识他。因为我是从他来的,他也差了我来。

他们就想要捉拿耶稣。只是没有人下手,因为他的时候还没有到。

但众人中间有些信他的,说,基督来的时候,他所行的神迹,岂能比这人所行的更多吗。

法利赛人听见众人为耶稣这样纷纷议论,祭司长和法利赛人,就打发差役去捉拿他。

于是耶稣说,我还有不多的时候和你们同在,以后就回到差我来的那里去。

你们要找我,却找不着。我所在的地方你们不能到。

犹太人就彼此对问说,这人要往那里去,叫我们找不着呢。难道他要往散住希腊中的犹太人那里去教训希腊人吗。

他说,你们要找我,却找不着,我所在的地方,你们不能到。这话是什么意思呢。

节期的末日,就是最大之日,耶稣站着高声说,人若渴了,可以到我这里来喝。

信我的人,就如经上所说,从他复中要流出活水的江河来。

耶稣这话是指着信他之人,要受圣灵说的,那时还没有赐下圣灵来,因为耶稣尚未得着荣耀。

众人听见这话,有的说,这真是那先知。

有的说,这是基督。但也有的说,基督岂是从加利利出来的吗。

经上岂不是说,基督是大卫的后裔,从大卫本乡伯利恒出来的吗。

于是众人因着耶稣起了分争。

其中有人要捉拿他。只是无人下手。

差役回到祭司长和法利赛人那里。他们对差役说,你们为什么没有带他来呢。

差役回答说,从来没有像他这样说话的。

法利赛人说,你们也受了迷惑吗。

官长或是法利赛人,岂有信他的呢。

但这些不明白律法的百姓,是被咒诅的。

内中有尼哥底母,就是从前去见耶稣的,对他们说,

不先听本人的口供,不知道他所作的事,难道我们的律法还定他的罪吗。

他们回答说,你也是出于加利利吗。你且去查考,就可知道,加利利没有出过先知。

\chapter{约翰福音第8章}
于是各人都回家去了。耶稣却往橄榄山去。

清早又回到殿里。众百姓都到他那里去,他就坐下教训他们。

文士和法利赛人,带着一个行淫时被拿的妇人来,叫他站在当中。

就对耶稣说,夫子,这妇人是正行淫之时被拿的。

摩西在律法上吩咐我们,把这样的妇人用石头打死。你说该把他怎吗样呢。

他们说这话,乃试探耶稣,要得着告他的把柄。耶稣却弯着腰用指头在地上画字。

他们还是不住的问他,耶稣直起腰来,对他们说,你们中间谁是没有罪的,谁就可以先拿石头打他。

于是又弯着腰用指头在地上画字。

他们听见这话,就从老到少一个一个的都出去了。只剩下耶稣一人。还有那妇人仍然站在当中。

耶稣就直起腰来,对他说,妇人,那些人在那里呢。没有人定你的罪吗。

他说,主阿,没有。耶稣说,我也不定你的罪。去吧。从此不要再犯罪了。

耶稣又对众人说,我是世上的光。跟从我的,就不在黑暗里走,必要得着生命的光。

法利赛人对他说,你是为自己作见证。你的见证不真。

耶稣说,我虽然为自己作见证,我的见证还是真的。因我知道我是从那里来,往那里去。你们却不知道我是从那里来,往那里去。

你们是以外貌(原文作凭肉身)判断人。我却不判断人。

就是判断人,我的判断也是真的。因为不是我独自在这里,还有差我来的父与我同在。

你们的律法上也记着说,两个人的见证是真的。

我是为自己作见证,还有差我来的父,也是为我作见证。

他们就问他说,你的父在那里。耶稣回答说,你们不认识我,也不认识我的父。若认识我,也就认识我的父。

这些话是耶稣在殿里的库房,教训人时所说的。也没有人拿他。因为他的时候还没有到。

耶稣又对他们说,我要去了,你们要找我,并且你们要死在罪中。我所去的地方,你们不能到。

犹太人说,他说我所去的地方,你们不能到,难道他要自尽吗。

耶稣对他们说,你们是从下头来的,我是从上头来的。你们是属这世界的,我不是属这世界的。

所以我对你们说,你们要死在罪中,你们若不信我是基督,必要死在罪中。

他们就问他说,你是谁。耶稣对他们说,就是我从起初所告诉你们的。

我有许多事讲论你们,判断你们,但那差我来的是真的。我在他那里所听见的,我就传给世人。

他们不明白耶稣是指着父说的。

所以耶稣说,你们举起人子以后,必知道我是基督,并且知道我没有一件事,是凭着自己作的。我说这话,乃是照着父所教训我的。

那差我来的,是与我同在。他没有撇下我独自在这里,因为我常作他所喜欢的事。

耶稣说这话的时候,就有许多人信他。

耶稣对信他的犹太人说,你们若常常遵守我的道,就真是我的门徒。

你们必晓得真理,真理必叫你们得以自由。

他们回答说,我们是亚伯拉罕的后裔,从来没有作过谁的奴仆。你怎吗说,你们必得以自由呢。

耶稣回答说,我实实在在的告诉你们。所有犯罪的,就是罪的奴仆。

奴仆不能永远住在家里,儿子是永远住在家里。

所以天父的儿子若叫你们自由,你们就真自由了。

我知道你们是亚伯拉罕的子孙,你们却想要杀我。因为你们心里容不下我的道。

我所说的,是在我父那里看见的。你们所行的,是在你们的父那里听见的。

他们说,我们的父就是亚伯拉罕。耶稣说,你们若是亚伯拉罕的儿子,就必行亚伯拉罕所行的事。

我将在神那里所听见的真理,告诉了你们,现在你们却想要杀我。这不是亚伯拉罕所行的事。

你们是行你们父所行的事。他们说,我们不是从淫乱生的。我们只有一位父就是神。

耶稣说,倘若神是你们的父,你们就必爱我。因为我本是出于神,也是从神而来,并不是由着自己来,乃是他差我来。

你们为什么不明白我的话呢,无非是因你们不能听我的道。

你们是出于你们的父魔鬼,你们父的私欲,你们偏要行,他从起初是杀人的,不守真里。因他心里没有真里,他说谎是出于自己,因他本来是说谎的,也是说谎之人的父。

我将真理告诉你们,你们就因此不信我。

你们中间谁能指证我有罪呢。我既然将真理告诉你们,为什么不信我呢。

出于神的,必听神的话。你们不听,因为你们不是出于神。

犹太人回答说,我们说,你是撒玛利亚人,并且是鬼附着的,这话岂不正对吗。

耶稣说,我不是鬼附着的。我尊敬我的父,你们倒轻慢我。

我不求自己的荣耀。有一位为我求荣耀定是非的。

我实实在在的告诉你们,人若遵守我的道,就永远不见死。

犹太人对他说,现在我们知道你是鬼附着的。亚伯拉罕死了,众先知也死了。你还说,人若遵守我的道,就永远不尝死味。

难道你比我们的祖宗亚伯拉罕还大吗。他死了,众先知也死了。你将自己当作什么人呢。

耶稣回答说,我若荣耀自己,我的荣耀就算不得什么。荣耀我的乃是我的父,就是你们所说是你们的神。

你们未曾认识他。我却认识他。我若说不认识他,我就是说谎的,像你们一样,但我认识他,也遵守他的道。

你们的祖宗亚伯拉罕欢欢喜喜的仰望我的日子。既看见了,就快乐。

犹太人说,你还没有五十岁,岂能见过亚伯拉罕呢。

耶稣说,我实实在在的告诉你们,还没有亚伯拉罕,就有了我。

于是他们拿石头要打他。耶稣却躲藏,从殿里出去了。

\chapter{约翰福音第9章}
耶稣过去的时候,看见一个人生来是瞎眼的。

门徒问耶稣说,拉比,这人生来是瞎眼的,是谁犯了罪,是这人呢,是他父母呢。

耶稣回答说,也不是这人犯了罪,也不是他父母犯了罪,是要在他身上显出神的作为来。

趁着白日,我们必须作那差我来者的工。黑夜将到,就没有人能作工了。

我在世上的时候,是世上的光。

耶稣说了这话,就吐唾沫在地上,用唾沫和泥抹在瞎子的眼睛上,

对他说,你往西罗亚池子里去洗,(西罗亚翻出来,就是奉差遣)他去一洗,回头就看见了。

他的邻舍,和那素常见他是讨饭的,就说,这不是那从前坐着讨饭的人吗。

有人说,是他。又有人说,不是,却是像他。他自己说,是我。

他们对他说,你的眼睛是怎吗开的呢。

他回答说,有一个人名叫耶稣。他和泥抹我的眼睛,对我说,你往西罗亚池子去洗。我去一洗,就看见了。

他们说,那个人在那里。他说,我不知道。

他们把从前瞎眼的人,带到法利赛人那里。

耶稣和泥开他眼睛的日子是安息日。

法利赛人也问他是怎吗得看见的。瞎子对他们说,他把泥抹在我的眼睛上,我去一洗,就看见了。

法利赛人中有的说,这个人不是从神来的,因为他不守安息日。又有人说,一个罪人怎能行这样的神迹呢。他们就起了分争。

他们又对瞎子说,他既然开了你的眼睛,你说他是怎样的人呢。他说,是个先知。

犹太人不信他从前是瞎眼,后来能看见的,等到叫了他的父母来,

问他们说,这是你们的儿子吗。你们说他生来是瞎眼的,如今怎吗能看见了呢。

他父母回答说,他是我们的儿子,生来就瞎眼,这是我们知道的。

至于他如今怎吗能看见,我们却不知道。是谁开了他的眼睛,我们也不知道。他已经成了人,你们问他吧。他自己必能说。

他父母说这话,是怕犹太人,因为犹太人已经商议定了,若有认耶稣是基督的,要把他赶出会堂。

因此他父母说,他已经成了人,你们问他吧。

所以法利赛人第二次叫了那从前瞎眼的人来,对他说,你该将荣耀归给神。我们知道这人是个罪人。

他说,他是个罪人不是,我不知道。有一件事我知道,从前我是眼瞎的,如今能看见了。

他们就问他说,他向你作什么,是怎吗开了你的眼睛呢。

他回答说,我方才告诉你们,你们不听。为什么又要听呢。莫非你们也要作他的门徒吗。

他们就骂他说,你是他的门徒。我们是摩西的门徒。

神对摩西说话,是我们知道的。只是这个人,我们不知道他从那里来。

那人回答说,他开了我的眼睛,你们竟不知道他从那里来,这真是奇怪。

我们知道神不听罪人。惟有敬奉神旨意的,神才听他。

从创世以来,未曾听见有人把生来是瞎子的眼睛开了。

这人若不是从神来的,什么也不能作。

他们回答说,你全然生在罪孽中,还要教训我们吗。于是把他赶出去了。

耶稣听说他们把他赶出去。后来遇见他,就说,你信神的儿子吗。

他回答说,主阿,谁是神的儿子,叫我信他呢。

耶稣说,你已经看见他,现在和你说话的就是他。

他说,主阿,我信。就拜耶稣。

耶稣说,我为审判到这世上来,叫不能看见的,可以看见。能看见的,反瞎了眼。

同他在那里的法利赛人,听见这话,就说,难道我们也瞎了眼吗。

耶稣对他们说,你们若瞎了眼,就没有罪了。但如今你们说,我们能看见,所以你们的罪还在。

\chapter{约翰福音第10章}
我实实在在的告诉你们,人进羊圈,不从门进去,倒从别处爬进去,那人就是强盗。

从门进去的,才是羊的牧人。

看门的就给他开门。羊也听他的声音。他按着名叫自己的羊,把羊领出来。

既放出自己的羊来,就在前头走,羊也跟着他,因为认得他的声音。

羊不跟着生人,因为不认得他的声音。必要逃跑。

耶稣将这比喻告诉他们。但他们不明白所说的是什么意思。

所以耶稣又对他们说,我实实在在的告诉你们,我就是羊的门,

凡在我以先来的,都是贼,是强盗。羊却不听他们。

我就是门。凡从我进来的,必然得救,并且出入得草吃。

盗贼来,无非要偷窃,杀害,毁坏。我来了,是要叫羊(或作人)得生命,并且得的更丰盛。

我是好牧人,好牧人为羊舍命。

若是雇工,不是牧人,羊也不是他自己的,他看见狼来,就撇下羊逃走。狼抓住羊,赶散了羊群。

雇工逃走,因为他是雇工,并不顾念羊。

我是好牧人。我认识我的羊,我的羊也认识我。

正如父认识我,我也认识父一样。并且我为羊舍命。

我另外有羊,不是这个圈里的。我必须领他们来,他们也要听我的声音。并且要合成一群,归一个牧人了。

我父爱我,因我将命舍去,好再取回来。

没有人夺我的命去,是我自己舍的。我有权柄舍了,也有权柄取回来。这是我从父所受的命令。

犹太人为这些话,又起了分争。

内中有好些人说,他是被鬼附着,而且疯了。为什么听他呢。

又有人说,这不是鬼附之人所说的话。鬼岂能叫瞎子的眼睛开了呢。

在耶路撒冷有修殿节。是冬天的时候。

耶稣在殿里所罗门的廊下行走。

犹太人围着他,说,你叫我们犹疑不定到几时呢。你若是基督,就明明的告诉我们。

耶稣回答说,我已经告诉你们,你们不信。我奉我父之名所行的事,可以为我作见证。

只是你们不信,因为你们不是我的羊。

我的羊听我的声音,我也认识他们,他们也跟着我。

我又赐给他们永生。他们永不灭亡,谁也不能从我手里把他们夺去。

我父把羊赐给我,他比万有都大。谁也不能从我父手里把他夺去。

我与父原为一。

犹太人又拿起石头要打他。

耶稣对他们说,我从父显出许多善事给你们看,你们是为那一件事拿石头打我呢。

犹太人回答说,我们不是为善事拿石头打你,是你说僭妄的话。又为你是个人,反将自己当作神。

耶稣说,你们的律法上岂不是写着,我曾说你们是神吗。

经上得话是不能废的。若那些承受神道的人,尚且称为神,

父所分别为圣,又差到世间来的,他自称是神的儿子,你们还向他说,你说僭妄的话吗。

我若不行我父的事,你们就不必信我。

我若行了,你们纵然不信我,也当信这些事。叫你们又知道,又明白,父在我里面,我也在父里面。

他们又要拿他。他却逃出他们的手走了。

耶稣又往约旦河外去,到了约翰起初施洗的地方,就住在那里。

有许多人来到他那里。他们说约翰一件神迹没有行过。但约翰指着这人所说的一切话都是真的。

在那里信耶稣的人就多了。

\chapter{约翰福音第11章}
有一个患病的人,名叫拉撒路,住在伯大尼,就是马利亚和他姐姐马大的村庄。

这马利亚就是那用香膏抹主,又用头发擦他脚的。患病的拉撒路是他的兄弟。

他姊妹两个就打发人去见耶稣说,主阿,你所爱的人病了。

耶稣听见就说,这病不至于死,乃是为神的荣耀,叫神的儿子因此得荣耀。

耶稣素来爱马大,和他妹子,并拉撒路。

听见拉撒路病了,就在所居之地,仍住了两天。

然后对门徒说,我们再往犹太去吧。

门徒说,拉比,犹太人近来要拿石头打你,你还往那里去吗。

耶稣回答说,白日不是有十二小时吗。人在白日走路,就不至跌倒,因为看见这世上的光。

若在黑夜走路,就必跌倒,因为他没有光。

耶稣说了这话,随后对他们说,我们的朋友拉撒路睡了,我去叫醒他。

门徒说,主阿,他若睡了,就必好了。

耶稣这话是指着他死说的。他们却以为是说照常睡了。

耶稣就明明的告诉他们说,拉撒路死了。

我没有在那里就欢喜,这是为你们的缘故,好叫你们相信。如今我们可以往他那里去吧。

多马,又称为低土马,就对那同作门徒的说,我们也去和他同死吧。

耶稣到了,就知道拉撒路在坟墓里,已经四天了。

伯大尼离耶路撒冷不远,约有六里路。

有些犹太人来看马大和马利亚,要为他们的兄弟安慰他们。

马大听见耶稣来了,就出去迎接他。马利亚却仍然坐在家里。

马大对耶稣说,主阿,你若早在这里,我兄弟必不死。

就是现在,我也知道,你无论向神求什么,神也必赐给你。

耶稣说,你兄弟必然复活。

马大说,我知道在末世复活的时候,他必复活。

耶稣对他说,复活在我,生命也在我。信我的人,虽然死了,也必复活。

凡活着信我的人,必永远不死。你信这话吗。

马大说,主阿,是的,我信你是基督,是神的儿子,就是那要临到世界的。

马大说了这话,就回去暗暗的叫他妹子,马利亚说,夫子来了,叫你。

马利亚听见了就急忙起来,到耶稣那里去。

那时,耶稣还没有进村子,仍在马大迎接他的地方。

那些同马利亚在家里安慰他的犹太人,见他急忙起来出去,就跟着他,以为他要往坟墓那里胎哭。

马利亚到了耶稣那里,看见他,就俯伏在他脚前,说,主阿,你若早在这里,我兄弟必不死。

耶稣看见他哭,并看见与他同来的犹太人也哭,就心里悲叹,又甚忧愁。

便说,你们把他安放在那里。他们回答说,请主来看。

耶稣哭了。

犹太人就说,你看他爱这人是何等恳切。

其中有人说,他既然开了瞎子的眼睛,岂不能叫这人不死吗。

耶稣又心里悲叹,来到坟墓前。那坟墓是个洞,有一块石头挡着。

耶稣说,你们把石头挪开。那死人的姐姐马大对他说,主阿,他现在必是发臭了,因为他死了已经四天了。

耶稣说,我不是对你说过,你若信,就必看见神的荣耀吗。

他们就把石头挪开。耶稣举目望天说,父阿,我感谢你,因为你已经听我。

我也知道你常听我,但我说这话,是为周围站着的众人,叫他们信是你差了我来。

说了这话,就大声呼叫说,拉撒路出来。

那死人就出来了,手脚裹着布,脸上包着手巾。耶稣对他们说,解开,叫他走。

那些来看马利亚的犹太人,见了耶稣所作的事,就多有信他的。

但其中也有去见法利赛人的,将耶稣所作的事告诉他们。

祭司长和法利赛人聚集公会,说,这人行好些神迹,我们怎吗办呢。

若这样由着他,人人都要信他。罗马人也要来夺我们的地土,和我们的百姓。

内中有一个人,名叫该亚法,本年作大祭司,对他们说,你们不知道什么。

独不想一个人替百姓死,免得通国灭亡,就是你们的益处。

他这话不是出于自己,是因他本年作大祭司,所以预言耶稣将要替这一国死。

也不但替这一国死,并要将神四散的子民,都聚集归一。

从那日起,他们就商议要杀耶稣。

所以耶稣不再显然行在犹太人中间,就离开那里往靠近矿野的地方。到了一座城,名叫以法莲,就在那里和门徒同住。

犹太人的逾越节近了。有许多人从乡下上耶路撒冷去,要在节前洁净自己。

他们就寻梢耶稣,站在殿里彼此说,你们的意思如何,他不来过节吗。

那时,祭司长和法利赛人早已吩咐说,若有人知道耶稣在那里,就要报明,好去拿他。

\chapter{约翰福音第12章}
逾越节前六日,耶稣来到伯大尼,就是他叫拉撒路从死里复活之处。

有人在那里给耶稣豫备筵席。马大伺候,拉撒路也在那同耶稣坐席的人中。

马利亚就拿着一斤极贵的真哪哒香膏,抹耶稣的脚,又用自己的头发去擦。屋里就满了膏的香气。

有一个门徒,就是那将要卖耶稣的加略人犹大,

说,这香膏为什么不卖三十两银子周济穷人呢。

他说这话,并不是挂念穷人,乃因他是个贼,又带着钱曩,常取其中所存的。

耶稣说,由他吧,他是为我安葬之日存留的。

因为常有穷人和你们同在。只是你们不常有我。

有许多犹太人知道耶稣在那里,就来了,不但是为耶稣的缘故,也是要看他从死里所复活的拉撒路。

但祭司长商议连拉撒路也要杀了。

因有好些犹太人,为拉撒路的缘故,回去信了耶稣。

第二天,有许多上来过节的人,听见耶稣将到耶路撒冷,

就拿着棕树枝,出去迎接他,喊着说,和撒那,奉主名来的以色列王,是应当称颂的。

耶稣得了一个驴驹,就骑上。如经上所记的说,

锡安的民哪,(民原文作女子)不要惧怕,你的王骑着驴驹来了。

这些事门徒起先不明白。等到耶稣得了荣耀以后,才想起这话是指着他写的,并且众人果然向他这样行了。

当耶稣呼唤拉撒路叫他从死复活出坟墓的时候,同耶稣在那里的众人,就作见证。

众人因听见耶稣行了这神迹,就去迎接他。

法利赛人彼此说,看哪,你们是徒劳无益,世人都随从他去了。

那时,上来过节礼拜的人中,有几个希腊人。

他们来见加利利伯赛大的腓力,求他说,先生,我们愿意见耶稣。

腓力去告诉安得烈,安得烈同腓力去告诉耶稣。

耶稣说,人子得荣耀的时候到了。

我实实在在的告诉你们,一粒麦子不落在地里死了,仍旧是一粒。若死了,就结出许多子粒来。

爱惜自己生命的,就失丧生命。在这世上恨恶自己生命的,就要保守生命到永生。

若有人服事我,就当跟从我。我在那里,服事我的人,也要在那里。若有人服事我,我父必尊重他。

我现在心里忧愁,我说什么才好呢。父阿,救我脱离这时候。但我原是为这时候来的。

父阿,愿你荣耀你的名。当时就有声音从天上来说,我已经荣耀了我的名,还要再荣耀。

站在旁边的众人听见,就说,打雷了。还有人说,有天使对他说话。

耶稣说,这声音不是为我,是为你们来的。

现在这世界受审判。这世界的王要被赶出去。

我若从地上被举起来,就要吸引万人来归我。

耶稣这话原是指着自己将要怎样死说的。

众人回答说,我们听见律法上有话说,基督是永存的。你怎吗说,人子必须被举起来呢。这人子是谁呢。

耶稣对他们说,光在你们中间,还有不多的时候,应当趁着有光行走,免得黑暗临到你们。那在黑暗里行走的,不知道往何处去。

你们应当趁着有光,信从这光,使你们成为光明之子。

耶稣说了这话,就离开他们,隐藏了。

他虽然在他们面前行了许多神迹,他们还是不信他。

这是要应验先知以赛亚的话说,主阿,我们所传的,有谁信呢。

他们所以不能信,因为以赛亚又说,主叫他们瞎了眼,硬了心,免得他们眼睛看见,心里明白,回转过来,我就医治他们。

以赛亚因为看见他的荣耀,就指着他说这话。

虽然如此,官长中却有好些信他的。只因法利赛人的缘故,就不承认,恐怕被赶出会堂。

这是因他们爱人的荣耀,过于爱神的荣耀。

耶稣大声说,信我的,不是信我,乃是信那差我来的。

人看见我,就是看见那差我来的。

我到世上来,乃是光,叫凡信我的,不住在黑暗里。

若有人听见我的话不遵守,我不审判他。我来本不是要审判世界,乃是要拯救世界。

弃绝我不领受我话的人,有审判他的。就是我所讲的道,在末日要审判他。

因为我没有凭着自己讲。惟有差我来的父,已经给我命令,叫我说什么,讲什么。

我也知道他的命令就是永生。故此我所讲的话,正是照着父对我所说的。

\chapter{约翰福音第13章}
逾越节以前,耶稣知道自己离世归父的时候到了。他既爱世间属自己的人,就爱他们到底。

吃晚饭的时候,(魔鬼已将卖耶稣的意思,放在西门的儿子加略人犹大心里)。

耶稣知道父已将万有交在他手里,且知道自己是从神出来的,又要归到神那里去,

就离席站起来脱了衣服,拿一条手巾束腰。

随后把水倒在盆里,就洗门徒的脚,并用自己所束的手巾擦乾。

挨到西门彼得,彼得对他说,主阿,你洗我的脚吗。

耶稣回答说,我所作的,你如今不知道,后来必明白。

彼得说,你永不可洗我的脚。耶稣说,我若不洗你,你就与我无分了。

西门彼得说,主阿,不但我的脚,连手和头也要洗。

耶稣说,凡洗过澡的人,只要把脚一洗,全身就乾净了。你们是乾净的,然而不都是乾净的。

耶稣原知道要卖他的是谁,所以说,你们不都是乾净的。

耶稣洗完了他们的脚,就穿上衣服,又坐下,对他们说,我向你们所作的,你们明白吗。

你们称呼我夫子,称呼我主,你们说的不错。我本来是。

我是你们的主,你们的夫子,尚且洗你们的脚,你们也当彼此洗脚。

我给你们作了榜样,叫你们照着我向你们所作的去作。

我实实在在的告诉你们,仆人不能大于主人。差人也不能大于差他的人。

你们既知道这事,若是去行就有福了。

我这话不是指着你们众人说的。我知道我所拣选的是谁。现在要应验经上的话,说,同我吃饭的人,用脚踢我。

如今事情还没有成就,我要先告诉你们,叫你们到事情成就的时候,可以信我是基督

我实实在在的告诉你们,有人接待我所差遣的,就是接待我。接待我,就是接待那差遣我的。

耶稣说了这话,心里忧愁,就明说,我实实在在的告诉你们,你们中间有一个人要卖我了。

门徒彼此对看,猜不透所说的是谁。

有一个门徒,是耶稣所爱的,侧身挨近耶稣的怀里。

西门彼得点头对他说,你告诉我们,主是指着谁说的。

那门徒便就势靠着耶稣的胸膛,问他说,主阿,是谁呢。

耶稣回答说,我蘸一点饼给谁,就是谁。耶稣就蘸了一点饼,递给加略人西门的儿子犹大。

他吃了以后,撒但就入了他的心。耶稣对他说,你所作的快作吧。

同席的人,没有一个知道是为什么对他们说这话。

有人因犹大带着钱曩,以为耶稣是对他说,你去买我们过节所应用的东西。或是叫他拿什么周济穷人。

犹大受了那点饼,立刻就出去。那时候是夜间了。

他既出去,耶稣就说,如今人子得了荣耀,神在人子身上也得了荣耀。

神要因自己荣耀人子,并且要快快的荣耀他。

小子们,我还有不多的时候,与你们同在。后来你们要找我,但我所去的地方,你们不能到。这话我曾对犹太人说过,如今也照样对你们说。

我赐给你们一条新命令,乃是叫你们彼此相爱。我怎样爱你们,你们也要怎样相爱。

你们若有彼此相爱的心,众人因此就认出你们是我的门徒了。

西门彼得问耶稣说,主往那里去。耶稣回答说,我所去的地方,你现在不能跟我去。后来却要跟我去。

彼得说,主阿,我为什么现在不能跟你去。我愿意为你舍命。

耶稣说,你愿意为我舍命吗。我实实在在的告诉你,鸡叫以先,你要三次不认我。

\chapter{约翰福音第14章}
你们心里不要忧愁。你们信神,也当信我。

在我父的家里,有许多住处。若是没有,我就早已告诉你们了。我去原是为你们预备地方去。

我若去为你们预备了地方,就必再来接你们到我那里去,我在那里,叫你们也在那里。

我往那里去,你们知道。那条路,你们也知道。(有古卷作我往那里去你们知道那条路)

多马对他说,主阿,我们不知道你往那里去,怎吗知道那条路呢。

耶稣说,我就是道路,真理,生命。若不藉着我。没有人能到父那里去。

你们若认识我,也就认识我的父。从今以后,你们认识他,并且已经看见他。

腓力对他说,求主将父显给我们看,我们就知足了。

耶稣对他说,腓力,我与你们同在这样长久,你还不认识我吗。人看见了我,就是看见了父。你怎吗说,将父显给我们看呢。

我在父里面,父在我里面,你不信吗。我对你们所说的话,不是凭着自己说的,乃是住在我里面的父作他自己的事。

你们当信我,我在父里面,父在我里面。既或不信,也当因我所作的事信我。

我实实在在的告诉你们,我所作的事,信我的人也要作。并且要作比这更大的事。因为我往父那里去。

你们奉我的名,无论求什么,我必成就,叫父因子得荣耀。

你们若奉我的名求什么,我必成就。

你们若爱我,就必遵守我的命令。

我要求父,父就另外赐给你们一位保惠师,(或作训慰师下同)叫他永远与你们同在。

就是真理的圣灵,乃世人不能接受的。因为不见他,也不认识他。你们却认识他。因他常与你们同在,也要在你们里面。

我不撇下你们为孤儿,我必到你们这里来。

还有不多的时候,世人不再看见我。你们却看见我。因为我活着,你们也活着。

到那日,你们就知道我在父里面,你们在我里面,我也在你们里面。

有了我的命令又遵守的,这人就是爱我的。爱我的必蒙我父爱他,我也要爱他,并且要向他显现。

犹大(不是加略人犹大)问耶稣说,主阿,为什么要向我们显现,不向世人显现呢。

耶稣回答说,人若爱我,就必遵守我的道。我父也必爱他,并且我们要到他那里去,与他同住。

不爱我的人就不遵守我的道。你们所听见的道不是我的,乃是差我来之父的道。

我还与你们同住的时候,已将这些话对你们说了。

但保惠师,就是父因我的名所要差来的圣灵,他要将一切的事,指教你们,并且要叫你们想起我对你们所说的一切话。

我留下平安给你们,我将我的平安赐给你们。我所赐的,不像世人所赐的。你们心里不要忧愁,也不要胆怯。

你们听见我对你们说了,我去还要到你们这里来。你们若爱我,因我到父那里去,就必喜乐,因为父是比我大的。

现在事情还没有成就,我豫先告诉你们,叫你们到事情成就的时候,就可以信。

以后我不再和你们多说话,因为这世界的王将到。他在我里面是毫无所有。

但要叫世人知道我爱父,并且父怎样吩咐我,我就怎样行。起来我们走吧。

\chapter{约翰福音第15章}
我是真葡萄树,我父是栽培的人。

凡属我不结果子的枝子,他就剪去。凡结果子的,

现在你们因我讲给你们的道,已经乾净了。

你们要常在我里面,我也常在你们里面。枝子若不常在葡萄树上,自己就不能结果子。你们若不常在我里面,也是这样。

我是葡萄树,你们是枝子。常在我里面的,我也常在他里面,这人就多结果子。因为离了我,你们就不能作什么。

人若不常在我里面,就像枝子丢在外面枯乾,人拾起来,扔在火里烧了。

你们若常在我里面,我的话也常在你们里面,凡你们所愿意的,祈求就给你们成就。

你们多结果子,我父就因此得荣耀,你们也就是我的门徒了。

我爱你们,正如父爱我一样。你们要常在我的爱里。

你们若遵守我的命令,就常在我的爱里。正如我遵守了我父的命令,常在他的爱里。

这些事我已经对你们说了,是要叫我的喜乐,存在你们心里,并叫你们的喜乐可以满足。

你们要彼此相爱,像我爱你们一样,这就是我的命令。

人为朋友舍命,人的爱心没有比这个更大的。

你们若遵行我所吩付的,就是我的朋友了。

以后我不再称你们为仆人。因仆人不知道主人所作的事。我乃称你们为朋友。因我从我父所听见的。已经都告诉你们了。

不是你们拣选了我,是我拣选了你们,并且分派你们去结果子,叫你们的果子长存。使你们奉我的名,无论向父求什么,他就赐给你们。

我这样吩咐你们,是要叫你们彼此相爱。

世人若恨你们,你们知道(或作该知道)恨你们以先,已经恨我了。

你们若属世界,世界必爱属自己的。只因你们不属世界。乃是我从世界拣选了你们,所以世界就恨你们。

你们要记念我从前对你们所说的话,仆人不能大于主人。他们若逼迫了我,也要逼迫你们。若遵守了我的话,也要遵守你们的话。

但他们因我的名,要向你们行这一切的事,因为他们不认识那差我来的。

我若没有来教训他们,他们就没有罪。但如今他们的罪无可推诿了。

恨我的,也恨我的父。

我若没有在他们中间行过别人未曾行的事,他们就没有罪。但如今连我与我的父,他们也看见也恨恶了。

这要应验他们律法上所写的话说,他们无故恨我。

但我要从父那里差保惠师来,就是从父出来真里的圣灵。他来了,就要为我作见证。

你们也要作见证,因为你们从起头就与我同在。

\chapter{约翰福音第16章}
我已将这些事告诉你们,使你们不至于跌倒。

人要把你们赶出会堂。并且时候将到,凡杀你们的,就以为是事奉神。

他们这样行,是因未曾认识父,也未曾认识我。

我将这些事告诉你们,是叫你们到了时候,可以想起我对你们说过了。我起先没有将这事告诉你们,因为我与你们同在。

现今我往差我来的父那里去。你们中间并没有人问我,你往那里去。

只因我将这事情告诉你们,你们就满心忧愁。

然而我将真情告诉你们。我去是与你们有益的。我若不去,保惠师就不到你们这里来。我若去,就差他来。

他既来了,就要叫世人为罪,为义,为审判,自己责备自己。

为罪,是因他们不信我。

为义,是因我往父那里去,你们就不再见我。

为审判,是因这世界的王受了审判。

我还有好些事要告诉你们,但你们现在担当不了(或作不能领会)。

只等真理的圣灵来了,他要引导你们明白(原文作进入)一切的真理。因为他不是凭自己说的,乃是把他所听见的都说出来。并要把将来的事告诉你们。

他要荣耀我。因为他要将受于我的,告诉你们。

凡父所有的,都是我的,所以我说,他要将受于我的,告诉你们。

等不多时,你们就不得见我。再等不多时,你们还要见我。

有几个门徒就彼此说,他对我们说,等不多时,你们就不得见我。再等不多时,你们还要见我。又说,因我往父那里去。这是什么意思呢。

门徒彼此说,他说等不多时,到底是什么意思呢。我们不明白他所说的话。

耶稣看出他们要问他,就说,我说等不多时,你们就不得见我,再等不多时,你们还要见我。你们为这话彼此相问吗。

我实实在在的告诉你们,你们将要痛哭,哀号,世人倒要喜乐。你们将要忧愁,然而你们的忧愁,要变为喜乐。

妇人生产的时候,就忧愁,因为他的时候到了。既生了孩子,就不再记念那苦楚,因为欢喜世上生了一个人。

你们现在也是忧愁。但我要再见你们,你们的心就喜乐了。这喜乐,也没有人能夺去。

到那日,你们什么也就不问我了。我实实在在的告诉你们,你们若向父求什么,他必因我的名,赐给你们。

向来你们没有奉我的名求什么,如今你们求就必得着,叫你们的喜乐可以满足。

这些事,我是用比喻对你们说的。时候将到,我不再用比喻对你们说,乃要将父明明的告诉你们。

到那日,你们要奉我的名祈求。我并不对你们说,我要为你们求父。

父自己爱你们,因为你们已经爱我,又信我是从父出来的。

我从父出来,到了世界。我又离开世界,往父那里去。

门徒说,如今你是明说,并不用比喻了。

现在我们晓得你凡事都知道,也不用人问你。因此我们信你是从神出来的。

耶稣说,现在你们信吗。

看哪,时候将到,且是已经到了,你们要分散,各归自己的地方去,留下我独自一人。其实我不是独自一人,因为有父与我同在。

我将这事告诉你们,是叫你们在我里面有平安。在世上你们有苦难。但你们可以放心,我已经胜了世界。

\chapter{约翰福音第17章}
耶稣说了这话,就举目望天说,父阿,时候到了。愿你荣耀你的儿子,使儿子也荣耀你。

正如你曾赐给他权柄,管理凡有血气的,叫他将永生赐给你所赐给他的人。

认识你独一的真神,并且认识你所差来的耶稣基督,这就是永生。

我在地上已经荣耀你,你所托付我的事,我已成全了。

父阿,现在求你使我同你享荣耀,就是未有世界以先,我同你所有的荣耀。

你从世上赐给我的人,我已将你的名显明与他们。他们本是你的,你将他们赐给我,他们也遵守了你的道。

如今他们知道,凡你所赐给我的,都是从你那里来的。

因为你所赐给我的道,我已经赐给他们。他们也领受了,又确实知道,我是从你出来的,并且信你差了我来。

我为他们祈求。不为世人祈求,却为你所赐给我的人祈求,因他们本是你的。

凡是我的都是你的,你的也是我的。并且我因他们得了荣耀。

从今以后,我不在世上,他们却在世上,我往你那里去。圣父阿,求你因你所赐给我的名保守他们,叫他们合而为一,像我们一样。

我与他们同在的时候,因你所赐给我的名,保守了他们,我也护卫了他们,其中除了那灭亡之子,没有一个灭亡的。好叫经上的话得应验。

现在我往你们那里去。我还在世上说这话,是叫他们心里充满我的喜乐。

我已将你的道赐给他们。世界又恨他们,因为他们不属世界,正如我不属世界一样。

我不求你叫他们离开世界,只求你保守他们脱离

他们不属世界,正如我不属世界一样。

求你用真理使他们成圣。你的道就是真理。

你怎样差我到世上,我也照样差他们到世上。

我为他们的缘故,自己分别为圣,叫他们也因真理成圣。

我不但为这些人祈求,也为那些因他们的话信我的人祈求。

使他们都合而为一。正如你父在我里面,我在你里面,使他们也在我们里面。叫世人可以信你差了我来。

你赐给我的荣耀,我已经赐给他们,使他们合而为一,像我们合而为一。

我在他们里面,你在我里面,使他们完完全全的合而为一。叫世人知道你差了我来,也知道你爱他们如同爱我一样。

父阿,我在那里,愿你所赐给我的人也同我在那里,叫他们看见你所赐给我的荣耀。因为创立世界以前,你已经爱我了。

公义的父阿,世人未曾认识你,我却认识你。这些人也知道你差了我来。

我已将你的名指示他们,还要指示他们,使你所爱我的爱在他们里面,我也在他们里面。

\chapter{约翰福音第18章}
耶稣说了这话,就同门徒出去,过了汲沦溪,在那里有一个园子,他和门徒进去了。

卖耶稣的犹大也知道那地方。因为耶稣和门徒屡次上那里去聚集。

犹大领了一队兵,和祭司长并法利赛人的差役,拿着灯笼,火把,兵器,就来到园里。

耶稣知道将要临到自己的一切事,就出来,对他们说,你们找谁。

他们回答说,找拿撒勒人耶稣。耶稣说,我就是。卖他的犹大也同他们站在那里。

耶稣一说我就是,他们就退后倒在地上。

他又问他们说,你们找谁。他们说,找拿撒勒人耶稣。

耶稣说,我已经告诉你们,我就是。你们若找我,就让这些人去吧。

这要应验耶稣从前的话,说,你所赐给我的人,我没有失落一个。

西门彼得带着一把刀,就拔出来,将大祭司的仆人砍了一刀,削掉他的右耳。那仆人名叫马勒古。

耶稣就对彼得说,收刀入鞘吧。我父所给我的那杯,我岂可不喝呢。

那队兵和千夫长并犹太人的差役,就拿住耶稣,把他捆绑了。

先带到亚那面前。因为亚那是本年作大祭司该亚法的岳父。

这该亚法,就是从前向犹太人发议论说,一个人替百姓死是有益的那位。

西门彼得跟着耶稣,还有一个门徒跟着。那门徒是大祭司所认识的。他就同耶稣进了大祭司的院子。

彼得却站在门外。大祭司所认识的那门徒出来,和看门使女说了一声,就领彼得进去。

那看门的使女对彼得说,你不也是这人的门徒吗。他说,我不是。

仆人和差役,因为天冷,就生了炭火,站在那里烤火。彼得也同他们站着烤火。

大祭司就以耶稣的门徒和他的教训盘问他。

耶稣回答说,我从来是明明的对世人说话。我常在会堂和殿里,就是犹太人聚集的地方,教训人。我在暗地里,并没有说什么。

你为什么问我呢。可以问那听见的人,我对他们说的是什么。我所说的,他们都知道。

耶稣说了这话,旁边站着的一个差役,用手掌打他说,你这样回答大祭司吗。

耶稣说,我若说的不是,你可以指证那不是。我若说的是,你为什么打我呢。

亚那就把耶稣解到大祭司该亚法那里,仍是捆着解去的。

西门彼得正站着烤火,有人对他说,你不也是他的门徒吗。彼得不承认,说,我不是。

有大祭司的一个仆人,是彼得削掉耳朵那人的亲属,说,我不是看见你同他在园子里吗。

彼得又不承认。立时鸡就叫了。

众人将耶稣从该亚法那里往衙门内解去。那时天还早。他们自己却不进衙门,恐怕染了污秽,不能吃逾越节的筵席

彼拉多就出来,到他们那里,说,你们告这人是为什么事呢。

他们回答说,这人若不是作恶的,我们就不把他交给你。

彼拉多说,你们带他去,按着你们的律法审问他吧。犹太人说,我们没有杀人的权柄。

这要应验耶稣所说,自己将要怎样死的话了。

彼拉多又进了衙门,叫耶稣来,对他说,你是犹太人的王吗。

耶稣回答说,这话是你自己说的,还是别人论我对你说的呢。

彼拉多说,我岂是犹太人呢。你本国的人和祭司长,把你交给我。你作了什么事呢。

耶稣回答说,我的国不属这世界。我的国若属这世界,我的臣仆必要争战,使我不至于被交给犹太人。只是我的国不属这世界。

彼拉多就对他说,这样,你是王吗。耶稣回答说,你说我是王。我为此而生,也为此来到世间,特为给真理作见证。凡属真理的人,就听我的话。

彼拉多说,真理是什么呢。说了这话,又出来到犹太人那里,对他们说,我查不出他有什么罪来。

但你们有个规矩,在逾越节要我给你们释放一个人,你们要我给你们释放犹太人的王吗。

他们又喊着说,不要这人,要巴拉巴。这巴拉巴是个强盗。

\chapter{约翰福音第19章}
当下彼拉多将耶稣鞭打了。

兵丁用荆棘编作冠冕,戴在他头上,给他穿上紫袍。

又挨近他说,恭喜犹太人的王阿。他们就用手掌打他。

彼拉多又出来对众人说,我带他出来见你们,叫你们知道我查不出他有什么罪来。

耶稣出来,戴着荆棘冠冕,穿着紫袍。彼拉多对他们说,你们看这个人。

祭司长和差役看见他,就喊着说,钉他十字架,钉他十字架。彼拉多说,你们自己把他钉十字架吧。我查不出他有什么罪来。

犹太人回答说,我们有律法,按那律法,他是该死的,因他以自己为神的儿子。

彼拉多听见这话,越发害怕。

又进衙门,对耶稣说,你是那里来的。耶稣却不回答。

彼拉多说,你不对我说话吗。你岂不知我有权柄释放你,也有权柄把你钉十字架吗。

耶稣回答说,若不是从上头赐给你的,你就毫无权柄辨我。所以把我交给你的那人,罪更重了。

从此彼拉多想要释放耶稣。无奈犹太人喊着说,你若释放这个人,就不是凯撒的忠臣。(原文作朋友)凡以自己为王的,就是背叛凯撒了。

彼拉多听见这话,就带耶稣出来,到了一个地方,名叫铺华石处,希伯来话叫厄巴大,就在那里坐堂。

那日是逾越节的日子,约有午正。彼拉多对犹太人说,看哪,这是你们的王。

他们喊着说,除掉他,除掉他,钉他在十字架上。彼拉多说,我可以把你们的王钉十字架吗。祭司长回答说,除了凯撒,我们没有王。

于是彼拉多将耶稣交给他们去钉十字架。

他们就把耶稣带了去。耶稣背着自己的十字架出来,到了一个地方,名叫髑髅地,希伯来话叫各各他。

他们就在那里钉他在十字架上,还有两个人和他一同钉着,一边一个,耶稣在中间。

彼拉多又用牌子写了一个名号,安在十字架上。写的是犹太人的王,拿撒勒人耶稣。

有许多犹太人念这名号。因为耶稣被钉十字架的地方,与城相近,并且是用希伯来,罗马,希腊,三样文字写的。

犹太人的祭司长,就对彼拉多说,不要写犹太人的王。要写他自己说我是犹太人的王。

彼拉多说,我所写的,我已经写上了。

兵丁既然将耶稣钉在十字架上,就拿他的衣服分为四分,每兵一分。又拿他的里衣。这件里衣,原是没有缝儿,是上下一片织成的。

他们就彼此说,我们不要撕开,只要拈阄,看谁得着。这要应验经上的话说,他们分了我的外衣,为我的里衣拈阄。兵丁果然作了这事。

站在耶稣十字架旁边的,有他母亲,与他母亲的姊妹,并革罗吧的妻子马利亚,和抹大拉的马利亚。

耶稣见母亲和他所爱的那个门徒站在旁边,就对他说,母亲,(原文作妇人)看你的儿子。

又对那门徒说,看你的母亲。从此那门徒就接他到自己家里去了。

这事以后,耶稣知道各样的事已经成了,为要使经上的话应验,就说,我渴了。

有一个器皿盛满了醋,放在那里。他们就拿海绒蘸满了醋,绑在牛膝草上,送到他口。

耶稣尝(原文作受)了那醋,就说,成了。便低下头,将灵魂交付神了。

犹太人因这日是豫备日,又因那安息日是个大日,就求彼拉多叫人打断他们的腿,把他们拿去,免得尸首当安息日留在十字架上。

于是兵丁来,把头一个的腿,并与耶稣同钉第二个人的腿,都打断了。

只是来到耶稣那里,见他已经死了,就不打断他的腿。

惟有一个兵拿枪扎他的肋旁,随既有血和水流出来。

看见这事的那人就作见证,他的见证也是真的,并且他知道自己所说的是真的,叫你们也可以信。

这些事成了,为要应验经上的话说,他的骨头,一根也不可折断。

经上又有一句说,他们要仰望自己所扎的人。

这些事以后,有亚利马太人约瑟,是耶稣的门徒,只因怕犹太人,就暗暗的作门徒,他来求彼拉多,要把耶稣的身体领去。彼拉多允准,他就把耶稣的身体领去了。

又有尼哥底母,就是先前夜里去见耶稣的,带着没药,和沈香,约有一百斤前来。

他们就照犹太人殡葬的规矩,把耶稣的身体,用细麻布加上香料裹好了。

在耶稣钉十字架的地方,有一个园子。园子里有一座新坟墓,是从来没有葬过人的。

只因是犹太人的豫备日,又因那坟墓近,他们就把耶稣安放在那里。

\chapter{约翰福音第20章}
七日的第一日清早,天还黑的时候,抹大拉的马利亚来到坟墓那里,看见石头从坟墓挪开了。

就跑来见西门彼得,和耶稣所爱的那个门徒,对他们说,有人把主从坟墓里挪了去,我们不知道放在那里。

彼得和那门徒就出来,往坟墓那里去。

两个人同跑,那门徒比彼得跑的更快,先到了坟墓。

低头往里看,就见细麻布还放在那里。只是没有进去。

西门彼得随后也到了,进坟墓里去,就看见细麻布还放在那里。

又看见耶稣的裹头巾,没有和细麻布放在一处,是另一处卷着。

先到坟墓的那门徒也进去,看见就信了。

因为他们还不明白圣经的意思,就是耶稣必要从死里复活。

于是两个门徒回自己的住处去了。

马利亚却站在坟墓外面哭。哭的时候,低头往坟墓里看,

就见两个天使,穿着白衣,在安放耶稣身体的地方坐着,一个在头,一个在脚。

天使对他说,妇人,你为什么哭。他说,因为有人把我主挪了去,我不知道放在那里。

说了这话,就转过身来,看见耶稣站在那里,却不知道是耶稣。

耶稣问他说,妇人,为什么哭,你找谁呢。马利亚以为是看园的,就对他说,先生,若是你把他移了去,请告诉我,你把他放在那里,我便去取他。

耶稣说,马利亚。马利亚就转过来,用希伯来话对他说,拉波尼。(拉波尼就是夫子的意思

耶稣说,不要摸我。因为我还没有升上去见我的父。你往我弟兄那里去,告诉他们说,我要升上去,见我的父,也是你们的父。见我的神,也是你们的神。

抹大拉的马利亚就去告诉门徒说,我已经看见了主。他又将主对他说的这话告诉他们。

那日,(就是七日的头一日)晚上,门徒所在的地方,因怕犹太人,门都关了。耶稣来站在当中,对他们说,愿你们平安。

说了这话,就把手和肋旁,指给他们看。门徒看见主,就喜乐了。

耶稣又对他们说,愿你们平安。父怎样差遣了我,我也照样差遣你们。

说了这话,就向他们吹一口气,说,你们受圣灵。

你们赦免谁的罪,谁的罪就赦免了。你们留下谁的罪,谁的罪就留下了。

那十二个门徒中,有称为抵土马的多马。耶稣来的时候,他没有和他们同在。

那些门徒对他说,我们已经看见主了。多马却说,我非看见他手上的钉痕,用指头探入那钉痕,又用手探入他的肋旁,我总不信。

过了八日,门徒又在屋里,多马也和他们同在,门都关了。耶稣来站在当中说,愿你们平安。

就对多马说,伸过你的指头来,摸(摸原文作看)我的手。伸出你的手来,探入我的肋旁。不要疑惑,总要信。

多马说,我的主,我的神。

耶稣对他说,你因看见了我才信。那没有看见就信的,有福了。

耶稣在门徒面前,另外行了许多神迹,没有记在这书上。

但记这些事,要叫你们信耶稣是基督,是神的儿子。并且叫你们信了他,就可以因他的名得生命。

\chapter{约翰福音第21章}
这些事以后,耶稣在提比哩亚海边,又向门徒显现。他怎样显现记在下面。

有西门彼得,和称为抵土马的多马,并加利利的迦拿人拿但业,还有西庇太的两个儿子,又有两个门徒,都在一处。

西门彼得对他们说,我打鱼去。他们说,我们也和你同去。他们就出去,上了船,那一夜并没有打着什么。

天将亮的时候,耶稣站在岸上。门徒却不知道是耶稣。

耶稣就对他们说,小子,你们有吃的没有。他们回答说,没有。

耶稣说,你们把网撒在船的右边,就必得着。他们便撒下网去,竟拉不上来了,因为鱼甚多。

耶稣所爱的那门徒对彼得说,是主。那时西门彼得赤着身子,一听见是主,就束上一件外衣,跳在海里。

其馀的门徒(离岸不远,约有二百肘,(古时以肘为尺,一肘约有今时尺半)就在小船把那网鱼拉过来。

他们上了岸,就看见那里有炭火,上面有鱼,又有饼。

耶稣对他们说,把刚才打的鱼,拿几条来。

西门彼得就去,(或作上船)把网拉到岸上,那网满了大鱼,共一百五十三条。鱼虽这样多,网却没有破。

耶稣说,你们来吃早饭。门徒中没有一个敢问他,你是谁,因为知道是主。

耶稣就来拿饼和鱼给他们。

耶稣从死里复活以后,向门徒显现,这是第三次。

他们吃完了早饭,耶稣对西门彼得说,约翰的儿子西门,(约翰马太十六章十七节称约拿)你爱我比这些更深吗。彼得说,主阿,是的。你知道我爱你。耶稣对他说,你喂养我的小羊。

耶稣第二次又对他说,约翰的儿子西门,你爱我吗。彼得说,主阿,是的。你知道我爱你。耶稣说,你牧养我的羊。

第三次对他说,约翰的儿子西门,你爱我吗。彼得因为耶稣第三次对他说,你爱我吗,就忧愁,对耶稣说,主阿,你是无所不知的,你知道我爱你。耶稣说,你喂养我的羊。

我实实在在的告诉你,你年少的时候,自己束上带子,随意往来,但年老的时候,你要伸手来,别人要把你束上,带你到不愿意去的地方。

耶稣说这话,是指着彼得要怎样死荣耀神。说了这话,就对他说,你跟从我吧。

彼得转过来,看见耶稣所爱的那门徒跟着,就是在晚饭的时候,靠着耶稣胸膛,说,主阿,卖你的是谁的那门徒。

彼得看见他,就问耶稣说,主阿,这人将来如何。

耶稣对他说,我若要他等到我来的时候,与你何干。你跟从我吧。

于是这话传在弟兄中间,说那门徒不死。其实耶稣不是说他不死。乃是说我若要他等到我来的时候,与你何干。

为这些事作见证,并且记载这些事的,就是这门徒。我们也知道他的见证是真的。

耶稣所行的事,还有许多,若是一一的都写出来,我想所写的书,就是世界也容不下去了。

\chapter{使徒行传第1章}
提阿非罗阿,我已经作了前书,论到耶稣开头一切所行所教训的,

直到他藉着圣灵吩咐所拣选的使徒,以后被接上升的日子为止。

他受害之后,用许多的凭据,将自己活活的显给使徒看,四十天之久向他们显现,讲说神国的事。

耶稣和他们聚集的时候,嘱咐他们说,不要离开耶路撒冷,要等候父所应许的,就是你们听我说过的。

约翰是用水施洗。但不多几日,你们要受圣灵的洗。

他们聚集的时候,问耶稣说,主阿,你复兴以色列国,就在这时候吗。

耶稣对他们说,父凭着自己的权柄,所定的时候日期,不是你们可以知道的。

但圣灵降临在你们身上,你们就必得着能力。并要在耶路撒冷,犹太全地,和撒玛利亚,直到地极,作我的见证。

说了这话,他们正看的时候,他就被取上升,有一朵云彩把他接去,便看不见他了。

当他往上去,他们定睛望天的时候,忽然有两个人,身穿白衣,站在旁边,说,

加利利人哪,你们为什么站着望天呢。这离开你们被接升天的耶稣,你们见他怎样往天上去,他还要怎样来。

有一座山名叫橄榄山,离耶路撒冷不远,约有安息日可走的路程。当下门徒从那里回耶路撒冷去。

进了城,就上了所住的一间楼房。在那里有彼得,约翰,雅各,安得烈,腓力,多马,巴多罗买,马太,亚勒腓的儿子雅各,奋锐党的西门,和雅各的儿子(或作兄弟)犹大。

这些人,同着几个妇人,和耶稣的母亲马利亚,并耶稣的弟兄,都同心合意的恒切祷告。

那时,有许多人聚会,约有一百二十名,彼得就在弟兄中间站起来,说,

弟兄们,圣灵藉大卫的口,在圣经上,预言领人捉拿耶稣的犹大。这话是必须应验的。

他本来列在我们数中,并且在使徒的职任上得了一分。

这人用他作恶的工价,买了一块田,以后身子仆倒,肚腹崩裂,肠子都流出来。

住在耶路撒冷的众人都知道这事,所以按着他们那里的话,给那块田起名叫亚革大马,就是血田的意思。

因为诗篇上写着说,愿他的住处,变为荒场,无人在内居住。又说,愿别人得他的职分。

所以主耶稣在我们中间始终出入的时候,

就是从约翰施洗起,直到主离开我们被接上升的日子为止,必须从那常与我们作伴的人中,立一位与我们同作耶稣复活的见证。

于是选举两个人,就是那叫作巴撒巴又称呼犹士都的约瑟,和马提亚。

众人就祷告说,主阿,你知道万人的心,求你从这两个人中,指明你所拣选的是谁,叫他得这使徒的位分。

这位分犹大已经丢弃,往自己的地方去了。

于是众人为他们摇签,摇出马提亚来。他就和十一个使徒同列。

\chapter{使徒行传第2章}
五旬节到了,门徒都聚集在一处。

忽然从天上有响声下来,好像一阵大风吹过,充满了他们所坐的屋子。

又有舌头如火焰显现出来,分开落在他们各人头上。

他们就都被圣灵充满,按着圣灵所赐的口才,说起别国的话来。

那时,有虔诚的犹太人,从天下各国来,住在耶路撒冷。

这声音一响,众人都来聚集,各人听见门徒用众人的乡谈说话,就甚纳闷。

都惊讶希奇说,看哪,这说话的不都是加利利人吗。

我们各人,怎样听见他们说我们生来所用的乡谈呢。

我们帕提亚人,玛代人,以拦人,和住在美索不达米亚,犹太,加帕多家,本都,亚细亚,

弗吕家,旁非利亚,埃及的人,并靠近古利奈的利比亚一带地方的人,从罗马来的客旅中,或是犹太人,或是进犹太教的人,

克里特和阿拉伯人,都听见他们用我们的乡谈,讲说神的大能作为。

众人就都惊讶猜疑,彼此说,这是什么意思呢。

还有人讥诮说,他们无非是新酒灌满了。

彼得和十一个使徒,站起,高声说,犹太人,和一切住在耶路撒冷的人哪,这些事你们当知道,也当侧耳听我的话。

你们想这些人是醉了,其实不是醉了,因为时候刚到巳初。

这正是先知约珥所说的。神说,在末后的日子,我要将我的灵浇灌凡有血气的。

神说,在末后的日子,我要将我的灵浇灌凡有血气的。你们的儿女要说预言。你们的少年人要见异象。老年人要作异梦。

在那些日子,我要将我的灵浇灌我的仆人和使女,他们就要说预言。

在天上我要显出奇事,在地下我要显出神迹,有血,有火,有烟雾。

日头要变为黑暗,月亮要变为血,这都在主大而明显的日子未到以前。徒02:20)日头要变为黑暗,月亮要变为血,这都在主大而明显的日子未到以前。

到那时候,凡求告主名的,就必得救。

以色列人哪,请听我的话。神藉着拿撒勒人耶稣,在你们中间施行异能,奇事,神迹,将他证明出来,这是你们自己知道的。

他既按着神的定旨先见,被交与人,你们就藉着无法之人的手,把他钉在十字架上杀了。

神却将死的痛苦解释了,叫他复活。因为他原不能被死拘禁。

大卫指着他说,我看见主常在我眼前,他在我右边,叫我不至于摇动。

所以我心里欢喜,我的灵(原文作舌)快乐。并且我的肉身要安居在指望中。

因你必不将我的灵魂撇在阴间,也不叫你的圣者见朽坏。

你已将生命的道路指示我,必叫我因见你的面,(或作叫我在你面前)得着满足的快乐。

弟兄们,先祖大卫的事,我可以明明的对你们说,他死了,也葬埋了,并且他的坟墓,直到今日还在我们这里。

大卫既是先知,又晓得神曾向他起誓,要从他的后裔中,立一位坐在他宝座上。

就豫先看明这事,讲论基督复活说,他的灵魂,不撇在阴间,他的肉身,也不见朽坏。

这耶稣,神已经叫他复活了,我们都为这事作见证。

他既被神的右手高举,(或作他既高举在神的右边)又从父受了所应许的圣灵,就把你们所看见所听见的,浇灌下来。

大卫并没有升到天上,但自己说,主对我说,你坐在我的右边,

等我使你仇敌作你的脚凳。

故此,以色列全家当确实的知道,你们钉在十字架上的这位耶稣,神已经立他为主为基督了。

众人听见这话,觉得扎心,就对彼得和其馀的使徒说,弟兄们,我们当怎样行。

彼得说,你们各人要悔改,奉耶稣基督的名受洗,叫你们的罪得赦,就必领受所赐的圣灵。

因为这应许是给你们,和你们的儿女,并一切在远方的人,就是主我们神所召来的。

彼得还用许多话作见证,劝勉他们说,你们当救自己脱离这弯曲的世代。

于是领受他们的人,就受了洗,那一天,门徒约添了三千人。

都恒心遵守使徒的教训,彼此交接,擘饼,祈祷。

众人都惧怕。使徒又行了许多奇事神迹。

信的人都在一处,凡物公用。

并且卖了田产家业,照各人所需用的分给各人。

他们天天同心合意,恒切的在殿里且在家中擘饼,存着欢喜诚实的心用饭,

赞美神,得众民的喜爱。主将得救的人,天天加给他们。

\chapter{使徒行传第3章}
申初祷告的时候,彼得,约翰,上圣殿去。

有一个人,生来是瘸腿的,天天被人抬来,放在殿的一个门口,那门名叫美门,要求进殿的人周济。

他看见彼得约翰将要进殿,就求他们周济。

彼得约翰定睛看他。彼得说,你看我们。

那人就留意看他们,指望得着什么。

彼得说,金银我都没有,只把我所有的给你,我奉拿撒勒人耶稣基督的名,叫你起来行走。

于是拉着他的右手,扶他起来,他的脚和踝子骨,立刻健壮了。

就跳起来,站着,又行走。同他们进了殿,走着,跳着,赞美神。

百姓都看见他行走,赞美神。

认得他是那素常坐在殿的美门口求周济的,就因他所遇着的事,满心希奇惊讶。

那人正在称为所罗门的廊下,拉着彼得,约翰。众百姓一齐跑到他们那里,很觉希奇。

彼得看见,就对百姓说,以色列人哪,为什么把这事当作希奇呢。为什么定睛看我们,以为我们凭自己的能力和虔诚,使这人行走呢。

亚伯拉罕,以撒,雅各的神,就是我们列祖的神,已经荣耀了他的仆人耶稣。(仆人或作儿子)你们却把他交付彼拉多。彼拉多定意要释放他,你们竟在彼拉多面前弃绝了他。

你们弃绝了那圣洁公义者,反求着释放一个凶手给你们。

你们杀了那生命的主,神却叫他从死里复活了。我们都是为这事作见证。

我们因他的名,他的名便叫你们所看见所认识的这人,健壮了。正是他所赐的信心,叫这人在你们众人面前全然好了。

弟兄们,我晓得你们作这事,是出于不知,你们的官长也是如此。

但神曾藉众先知的口,预言基督将要受害,就这样应验了。

所以你们当悔改归正,使你们的罪得以涂抹,这样,那安舒的日子,就必从主面前来到。

主也必差遣所豫定给你们的基督耶稣降临。

天必留他,等到万物复兴的时候,就是神从创世以来,藉着圣先知的口所说的。

摩西曾说,主神要从你们弟兄中间,给你们兴起一位先知像我,凡他向你们所说的,你们都要听从。

凡不听从那先知的,必要从民中全然灭绝。

从撒母耳以来的众先知,凡说预言的,也都说到这些日子。

你们是先知的子孙,也承受神与你们祖宗所立的约,就是对亚伯拉罕说,地上万族,都要因你的后裔得福。

神既兴起他的仆人,(或作儿子)就先差他到你们这里来,赐福给你们,叫你们各人回转,离开罪恶。

\chapter{使徒行传第4章}
使徒对百姓说话的时候,祭司们和守殿官,并撒都该人,忽然来了。

因他们教训百姓,本着耶稣,传说死人复活,就很烦恼。

于是下手拿住他们。因为天已经晚了,就把他们押到第二天。

但听道之人,有许多信的,男丁数目,约有五千。

第二天,官府,长老,和文士,在耶路撒冷聚会。

又有大祭司亚那,和该亚法,约翰,亚历山大,并大祭司的亲族都在那里。

叫使徒站在当中,就问他们说,你们用什么能力,奉谁的名,作这事呢。

那时,彼得被圣灵充满,对他们说,

治民的官府,和长老阿,倘若今日,因为在残疾人身上所行的善事,查问我们他是怎样得了痊愈。

你们众人,和以色列百姓,都当知道,站在你们面前的这人得痊愈,是因你们所钉十字架,神叫他从死里复活的,拿撒勒人耶稣基督的名。

他是你们匠人所弃的石头,已经成了房角的头块石头。

除他以外,别无拯救。因为在天下人间,没有赐下别的名,我们可以靠着得救。

他们见彼得约翰的胆量,又看出他们原是没有学问的小民,就希奇,认明他们是跟过耶稣的。

又看见那治好了的人,和他们一同站着,就无话可驳。

于是吩咐他们从公会出去,就彼此商议说,

我们当怎样办这两个人呢,因为他们诚然行了一件明显的神迹,凡住耶路撒冷的人都知道,我们也不能说没有。

惟恐这事越传杨在民间,我们必须恐吓他们,叫他们不再奉这名对人讲论。

于是叫了他们,禁止他们,总不可奉耶稣的名讲论教训人。

彼得约翰说,听从你们,不听从神,这在神面前合理不合理,你们自己酌量吧。

我们所看见所听见的,不能不说。

官长为百姓的缘故,想不出法子刑罚他们,又恐吓一番,把他们释放了。这是因众人为所行的奇事,都归荣耀与神。

原来藉着神迹医好的那人,有四十多岁了。

二人既被释放,就到会友那里去,把祭司长和长老所说的话,都告诉他们。

他们听见了,就同心合意的,高声向神说,主阿,你是造天,地,海,和其中万物的。

你曾藉着圣灵,托你仆人我们祖宗大卫的口,说,外邦为什么争闹,万民为什么谋算虚妄的事。

世上的君王一齐起来,臣宰也聚集,要敌挡主,并主的受膏者。(或作基督)

希律和本丢彼拉多,外邦人和以色列民,果然在这城里聚集,要攻打你所膏的圣仆耶稣,(仆或作子)。

成就你手和你意旨所豫定必有的事。

他们恐吓我们,现在求主鉴察。

一面叫你仆人大放胆量,讲你的道,一面伸出你的手来,医治疾病,并且使神迹奇事,因着你圣仆耶稣的名行出来。仆或作子

祷告完了,聚会的地方震动,他们就都被圣灵充满,放胆讲论神的道。

那许多信的人,都是一心一意的,没有一人说,他的东西有一样是自己的,都是大家公用。

使徒大有能力,见证主耶稣复活。众人也都蒙大恩。

内中也没有一个缺乏的,因为人人将田产房屋都卖了,把所卖的价银拿来,放在使徒脚前。

照各人所需用的,分给各人。

有一个利未人,生在塞浦路斯,名叫约瑟,使徒称他为巴拿巴。(巴拿巴翻出来,就是劝慰子)。

他有田地,也卖了,把价银拿来,放在使徒脚前。

\chapter{使徒行传第5章}
有一个人,名叫亚拿尼亚,同他的妻子撒非喇,卖了田产。

把价银私自留下几分,他的妻子也知道,其馀的几分,拿来放在门徒脚前。

彼得说,亚拿尼亚为什么撒但充满了你的心,叫你欺哄圣灵,把田地的价银私自留下几分呢。

田地还没有卖,不是你自己的吗。既卖了,价银不是你作主吗。你怎吗心起这意念呢。你不是欺哄人,是欺哄神了。

亚拿尼亚听见这话,就仆倒断了气。听见的人都甚惧怕。

有些少年人起来,把他包裹抬出去埋葬了。

约过了三小时,他的妻子进来,还不知道这事。

彼得对他说,你告诉我,你们卖田地的价银,就是这些吗。他说,就是这些。

彼得说,你们为什么同心试探主的灵呢。埋葬你丈夫之人的脚,已到门口,他们也要把你抬出去。

妇人立刻仆倒在彼得脚前,断了气。那些人进来,见他已经死了,就抬出去,埋在他丈夫旁边。

全教会,和听见这事的人,都甚惧怕。

主藉使徒的手,在民间行了许多神迹奇事,(他们(或作信的人)同心合意的在所罗门的廊下。

其馀的人,没有一个敢贴近他们。百姓却尊重他们。

信而归主的人越发增添,连男带女很多)。

甚至有人将病人抬到街上,放在床上,或褥子上,指望彼得过来的时候,或者得他的影儿照在什么人身上。

还有许多人,带着病人,和被污鬼缠磨的,从耶路撒冷四围的城邑来,全都得了医治。

大祭司和他的一切同人,就是撒都该教门的人,都起来,满心忌恨。

就下手拿住使徒,收在外监。

但主的使者,夜间开了监门,领他们出来,

说,你们去站在殿里,把这生命的道,都讲给百姓听。

使徒听了这话,天将亮的时候,就进殿里去教训人。大祭司和他的同人来了,叫齐公会的人,和以色列族的众长老,就差人到监里去,要把使徒提出来。

但差役到了,不见他们在监里,就回来禀报说,

我们看见监牢关得极妥当,看守的人也站在门外,及至开了门,里面一个人都不见。

守殿官和祭司长听见这话,心里犯难,不知这事将来如何。

有一个人来禀报说,你们收在监里的人,现在站在殿里教训百姓。

于是守殿官和差役去带使徒来,并没有用强暴。因为怕百姓用石头打他们。

带到了,便叫使徒站在公会前,大祭司问他们说,

我们不是严严的禁止你们,不可奉这名教训人吗。你们倒把你们的道理充满了耶路撒冷,想要叫这人的血归到我们身上。

彼得和众使徒回答说,顺从神,不顺从人,是应当的。

你们挂在木头上杀害的耶稣,我们祖宗的神已经叫他复活。

神且用右手将他高举,(或作他就是神高举在自己的右边)叫他作君王,作救主,将悔改的心,和赦罪的恩,赐给以色列人。

我们为这事作见证。神赐给顺从之人的圣灵,也为这事作见证。

公会的人听见就极其恼怒,想要杀他们。

但有一个法利赛人,名叫迦玛列,是众百性所敬重的教法师,在公会中站起来,吩咐人把使徒暂且带到外面去。

就对众人说,以色列人哪,论到这些人,你们应当小心怎样办理。

从前丢大起来,自夸为大。附从他的人约有四百。他被杀后,附从他的全部散了,归于无有。

此后报名上册的时候,又有加利利的犹大起来,引诱些百姓跟从他,他也灭亡,附从他的人也都四散了。

现在我劝你们不要管这些人,任凭他们吧,他们所谋的,所行的,若是出于人,必要败坏。

若是出于神,你们就不能败坏他们。恐怕你们倒是攻击神了。

公会的人听从了他,便叫使徒来,把他们打了,又吩咐他们不可奉耶稣的名讲道,就把他们释放了。

他们离开公会,心里欢喜。因为被算是配为这名受辱。

他们就每日在殿里,在家里,不住的教训人,传耶稣是基督。

\chapter{使徒行传第6章}
那时,门徒增多,有说希腊话的犹太人,向希伯来人发怨言。因为在天天的供给上忽略了他们的寡妇。

十二使徒叫众门徒来,对他们说,我们撇下神的道,去管理饭食,原是不合宜的。

所以弟兄们,当从你们中间选出七个有好名声,被圣灵充满,智慧充足的人,我们就派他们管理这事。

但我们要专心以祈祷传道为事。

大家都喜悦这话,就拣选了司提反,乃是大有信心,圣灵充满的人,又拣选腓利,伯罗哥罗,尼迦挪,提门,巴米拿,并进犹太教的安提阿人尼哥拉。

叫他们站在使徒面前。使徒祷告了,就按手在他们头上。

神的道兴旺起来。在耶路撒冷门徒数目加增的甚多。也有许多祭司信从了这道。

司提反满得恩惠能力,在民间行了大奇事和神迹。

当时有称利百地拿会堂的几个人,并有古利奈,亚力山太,基利家,亚细亚,各处会堂的几个人,都起来,和司提反辩论。

司提反是以智慧和圣灵说话,众人敌挡不住。

就买出人来说,我们听见他说谤??摩西,和神的话。

他们又耸动了百姓,长老,并文士,就忽然来捉拿他,把他带到公会去,

设下假见证说,这个人说话,不住的糟践圣所和律法。

我们曾听见他说,这拿撒勒人耶稣,要毁坏此地,也要改变摩西所交给我们的规条。

在公会里坐着的人,都是定睛看他,见他的面貌,好像天使的面貌。

\chapter{使徒行传第7章}
大祭司就说,这些事果然有吗。

司提反说,诸位父兄请听。当日我们的祖宗亚伯拉罕在美索不达米亚还未住哈兰的时候,荣耀的神向他显现,

对他说,你要离开本地和亲族,往我要指示你的地方去。

他就离开迦勒底人之地住在哈兰。他父亲死了以后,神使他从那里搬到你们现在所住之地。

在这地方神并没有给他产业,连立足之地也没有给他。但应许要将这地赐给他和他的后裔为业。那时他还没有儿子。

神说,他的后裔,必寄居外邦,那里的人,要叫他们作奴仆,苦待他们四百年。

神又说,使他们作奴仆的那国,我要惩罚,以后他们要出来,在这地方事奉我。

神又赐他割礼的约。于是亚伯拉罕生了以撒,第八日给他行了割礼。以撒生雅各,雅各生十二位先祖。

先祖嫉妒约瑟,把他卖到埃及去。神却与他同在,

救他脱离一切苦难,又使他在埃及王法老面前,得恩典有智慧。法老就派他作埃及国的宰相兼管全家。

后来埃及和迦南全地遭遇饥荒,大受艰难,我们的祖宗,就绝了粮。

雅各听见在埃及有粮,就打发我们的祖宗,初次往那里去。

第二次约瑟与弟兄相认,他的亲族也被法老知道了。

约瑟就打发弟兄请父亲雅各,和全家七十五个人都来。

于是雅各下了埃及,后来他和我们的祖宗都死在那里。

又被带到示剑,葬于亚伯拉罕在示剑用银子从哈抹子孙买来的坟墓里。

及至神应许亚伯拉罕的日期将到,以色列民在埃及兴盛众多,

直到有不晓得约瑟的新王兴起。

他用诡计待我们的宗族,苦害我们的祖宗,叫他们丢弃婴孩,使婴孩不能存活。

那时,摩西生下来,俊美非凡,在他父亲家里抚养了三个月。

他被丢弃的时候,法老的女儿拾了去,养为自己的儿子。

摩西学了埃及人一切的学问,说话行事,都有才能。

他将到四十岁,心中起意,去看望他的弟兄以色列人。

到了那里,见他们一个人受冤屈,就护庇他,为那受欺压的人报仇,打死了埃及人。

他以为弟兄必明白神是藉他的手搭救他们。他们却不明白。

第二天,遇见两个以色列人争斗,就劝他们和睦,说,你们二位是弟兄,为什么彼此欺负呢。

那欺负邻舍的,把他推开说,谁立你作我们的首领,和审判官呢。

难道你要杀我,像昨天杀那埃及人吗。

摩西听见这话就逃走了,寄居于米甸。在那里生了两个儿子。

过了四十年,在西奈山的旷野,有一位天使,从荆棘火焰中,向摩西显现。

摩西见了那异象,便觉希奇。正进前观看的时候,有主的声音说,

我是你列祖的神,就是亚伯拉罕的神,以撒的神,雅各的神。徒07:32)摩西战战竞竞,不敢观看。

主对他说,把你脚上的鞋脱下来。因为你所站之地是圣地。

我的百姓在埃及所受的困苦,我实在看见了。他们悲叹的声音,我也听见了。我下来要救他们,你来,我要差你往埃及去。

这摩西,就是百姓弃绝说,谁立你作我们的首领,和审判官的。神却藉那在荆棘中显现之使者的手,差派他作首领,作救赎的。

这人领百姓出来,在埃及,在红海,在旷野,四十年间行了奇事神迹。

那曾对以色列人说,神要从你们弟兄中间,给你们兴起一位先知像我的,就是这位摩西。

这人曾在旷野会中,和西奈山上与那对他说话的天使同在,又与我们的祖宗同在,并且领受活泼的圣言传给我们。

我们的祖宗不肯听从,反弃绝他,心里归向埃及,

对亚伦说,你且为我们造些神像,在我们前面引路。因为领我们出埃及地的那个摩西,我们不知道他遭了什么事。

那时,他们造了一个牛犊,又拿祭物献给那像,欢喜自己手中的工作。

神就转脸不顾,任凭他们事奉天上的日月星辰,正如先知书上所写的说,以色列家阿,你们四十年间在旷野,岂是将牺牲和祭物献给我吗。

你们抬着摩洛的帐幕,和理番神的星。就是你们所造为要敬拜的像。因此,我要把你们迁到巴比伦外去。

我们的祖宗在旷野,有法柜的帐幕,是神吩咐摩西叫他照所看见的样式作的。

这帐幕,我们的祖宗相继承,当神在他们面前赶出外邦人去的时候,他们同约书亚把张幕搬进承受为业之地,直存到大卫的日子。

大卫在神面前蒙恩,祈求为雅各的神豫备居所。

却是所罗门为神造成殿宇。

其实至高者并不住人手所造的。就如先知所言,

主说,天是我的座位,地是我的脚凳。你们要为我造何等的殿宇,那里是我安息的地方呢。

这一切都是我手中所造的吗。

你们这硬着颈项,心与耳未受割礼的人,常时抗拒圣灵。你们的祖宗怎样,你们也怎样。

那一个先知,不是你们祖宗逼迫呢。他们也把豫先传说那义者要来的人杀了。如今你们又把那义者卖了,杀了。

你们受了天使所传的律法,竟不遵守。

众人听见这话,就极其恼怒,向司提反咬牙切齿。

但司提反被圣灵充满,定睛望天,看见神的荣耀,又看见耶稣站在神的右边。

就说,我看见天开了,人子站在神的右边。

众人大声喊叫,捂着耳朵,齐心拥上前去。

把他推到城外,用石头打他。作见证的人,把衣裳放在一个少年人名叫扫罗的脚前。

他们正用石头打的时候,司提反呼吁主说,求耶稣接收我的灵魂。

又跪下大声喊着说,主阿,不要将这罪归于他们。说了这话就睡了。扫罗也喜悦他被害。

\chapter{使徒行传第8章}
从这日起,耶路撒冷的教会,大遭逼迫。除了使徒以外,门徒都分散在犹太和撒玛列亚各处。

有虔诚的人,把司提反埋葬了,为他捶胸大哭。

扫罗却残害教会,进各人的家,拉着男女下在监里。

那些分散的人,往各处去传道。

腓利下撒玛利亚城去,宣讲基督。

众人听见了,又看见腓利所行的神迹,就同心合意的听从他的话。

因为有许多人被污鬼附着,那些鬼大声呼叫,从他们身上出来。还有许多瘫痪的,瘸腿的,都得了医治。

在那城里,就大有欢喜。

有一个人,名叫西门,向来在那城里行邪术,妄自尊大,使撒玛利亚的百姓惊奇。

无论大小,都听从他,说,这人就是称为神的大能者。

他们听从他,因他久用邪术,使他们惊奇。

及至他们信了腓利所传神国的福音,和耶稣基督的名,连男带女就受了洗。

西门自己也信了。既受了洗,就常与腓利在一处。看见他所行的神迹和大异能,就甚惊奇。

使徒在耶路撒冷,听见撒玛利亚人领受了神的道,就打发彼得约翰往他们那里去。

两个人到了,就为他们祷告,要叫他们受圣灵。

因为圣灵还没有降在他们一人身上。他们只奉主耶稣的名受了洗。

于是使徒按手在他们头上,他们就受了圣灵。

西门看见使徒按手,便有圣灵赐下。就拿钱给使徒说,

把这权柄也给我,叫我手按着谁。谁就可以受圣灵。

彼得说,你的银子,和你一同灭亡吧。因你想神的恩赐,是可以用钱买的。

你在这道上,无分无关。因为在神面前,你的心不正。

你当懊悔你这罪恶,祈求主。或者你心里的意念可得赦免。

我看出你正在苦胆之中,被罪恶捆绑。

西门说,愿你们为我求主,叫你们所说的,没有一样临到我身上。

使徒既证明主道,而且传讲,就回耶路撒冷去,一路在撒玛利亚好些村庄传扬福音。

有主的一个使者对腓利说,起来,向南走,往那从耶路撒冷下迦萨的路上去。那路是旷野。

腓利就起身去了。不料,有一个埃塞俄比亚人(既古实见以赛亚十八章一节)是个有大权的太监,在埃塞俄比亚女王干大基的手下总管银库,他上耶路撒冷礼拜去了。

现在回来,在车上坐着,念先知以赛亚的书。

圣灵对腓利说,你去贴近那车走。

腓利就跑到太监那里,听见他念先知以赛亚的书,便问他说,你所念的,你明白吗。

他说,没有人指教我,怎能明白呢。于是请腓利上车,与他同坐。

他所念的那段经,说,他像羊被牵到宰杀之地,又像羊羔在剪毛的手下无声,他也是这样不开口。

他卑微的时候,人不按公义审判他。(原文作他的审判被夺去)谁能述说他的世代,因为他的生命从地上夺去。

太监对腓利说,请问先知说这话,是指着谁,是指着自己呢,是指着别人呢。

腓利就开口从这经上起,对他传讲耶稣

二人正往前走,到了有水的地方,太监说,看哪,这里有水,我受洗有什么妨碍呢。(有古卷在此有腓利说,

你若是一心相信就可以,他回答说,我信耶稣基督是神的儿子)

于是吩咐车站住,腓利和太监二人同下水里去,腓利就给他施洗。

从水里上来,主的灵把腓利提了去,太监也不再见他了,就欢欢喜喜的走路。

后来有人在亚锁都遇见腓利,他走遍那地方,在各城宣传福音,直到凯撒利亚。

\chapter{使徒行传第9章}
扫罗仍然向主的门徒,口吐威吓凶杀的话,去见大祭司,

求文书给大马士革的各会堂,若是找着信奉这道的人,无论男女,都准他捆绑带到耶路撒冷。

扫罗行路,将到大马士革,忽然从天上发光,四面照着他。

他就仆倒在地,听见有声音对他说,扫罗,扫罗,你为什么逼迫我。

他说,主阿,你是谁。主说,我就是你所逼迫的耶稣。

起来,进城去,你当作的事,必有人告诉你。

同行的人,站在那里,说不出话来,听见声音,却看不见人。

扫罗从地上起来,睁开眼睛,竟不能看见什么。有人拉他的手,领他进了大马士革。

三日不能看见,也不吃,也不喝。

当下在大马士革,有一个门徒,名叫亚拿尼亚。主在异象中对他说,亚拿尼亚。他说,主,我在这里。

主对他说,起来,往直街去,在犹大的家里,访问一个大数人名叫扫罗。他正祷告。

又看见了一个人,名叫亚拿尼亚,进来按手在他身上,叫他能看见。

亚拿尼亚回答说,主阿,我听见许多人说,这人怎样在耶路撒冷多多苦害你的圣徒。

并且他在这里有从祭司长得来的权柄捆绑一切求告你名的人。

主对亚拿尼亚说,你只管去。他是我所拣选的器皿,要在外邦人和君王并以色列人面前,宣扬我的名。

我也要指示他,为我的名必须受许多的苦难。

亚拿尼亚就去了,进入那家,把手按在扫罗身上说,兄弟扫罗,在你来的路上,向你显现的主,就是耶稣,打发我来,叫你能看见,又被圣灵充满。

扫罗的眼睛上,好像有鳞立刻掉下来,他就能看见,于是起来受了洗。

吃过饭就健壮了。

扫罗和大马士革的门徒同住了些日子。就在各会堂里宣传耶稣,说他是神的儿子。

凡听见的人,都惊奇说,在耶路撒冷残害求告这名的,不是这人吗。并且他到这里来,特要捆绑他们带到祭司长那里。

但扫罗越发有能力,驳倒住大马士革的犹太人,证明耶稣是基督。

过了好些日子,犹太人商议要杀扫罗。

但他们的计谋,被扫罗知道了。他们又昼夜在城门守候要杀他。

他的门徒就在夜间,用筐子把他从城墙上缒下去。

扫罗到了耶路撒冷,想与门徒结交。他们却都怕他,不信他是门徒。

惟有巴拿巴接待他,领他去见使徒,把他在路上怎吗看见主,主怎样向他说话,他在大马士革,怎样奉耶稣的名放胆传道,都述说出来。

于是扫罗在耶路撒冷,和门徒出入来往,

奉主的名,放胆传道。并与说希腊话的犹太人,讲论辩驳。他们却想法子要杀他。

弟兄们知道了,就送他下凯撒利亚,打发他往大数去。

那时犹太,加利利,撒玛利亚,各处的教会都得平安,被建立。凡事敬畏主,蒙圣灵的安慰,人数就增多了。

彼得周流四方的时候,也到了居住吕大的圣徒那里。

遇见一个人,名叫以尼雅,得了瘫痪,在褥子上躺卧八年。

彼得对他说,以尼雅,耶稣基督医好你了。起来收拾你的褥子。他就立刻起来了。

凡住吕大和沙仑的人都看见了他,就归服主。

在约帕有一个女徒,名叫大比大,翻希腊话,就是多加。(多加就是羚羊的意思)他广行善事,多施周济。

当时,他患病而死。有人把他洗了。停在楼上。

吕大原与约帕相近。门徒听见彼得在那里,就打发两个人去见他,央求他说,快到我们那里去,不要耽延。

彼得就起身和他们同去。到了,便有人领他上楼。众寡妇都站在彼得旁边哭,拿多加与他们同在时,所做的里衣外衣给他看。

彼得叫他们都出去,就跪下祷告,转身对着死人说,大比大,起来,他就睁开眼睛,见了彼得,便坐起来。

彼得伸手扶他起来,叫圣徒和寡妇进去,把多加活活的交给他们。

这事传遍了约帕,就有许多人信了主。

此后彼得在约帕一个硝皮匠西门的家里,住了多日。

\chapter{使徒行传第10章}
在凯撒利亚有一个人,名叫哥尼流,是意大利营的百夫长。

他是个虔诚人,他和全家都敬畏)神,多多周济百姓,常常祷告)神。

有一天,约在申初,他在异象中,明明看见神的一个使者进去,到他那里,说,哥尼流。

哥尼流定睛看他,惊怕说,主阿,什么事呢。天使说,你的祷告,和你的周济,达到神面前已蒙记念了。

现在你打发人往约帕去,请那称呼彼得的西门来。

他住在海边一个硝皮匠西门的家里。房子在海边上。

向他说话的天使去后,哥尼流叫了两个家人,和常伺候他的一个虔诚兵来。

把这事都述说给他们听,就打发他们往约帕去。

第二天,他们行路将近那城,彼得约在午正,上房顶去祷告。

觉得饿了,想要吃。那家的人正豫备饭的时候,彼得魂游象外。

看见天开了,有一物降下,好像一块大布。系着四角,槌在地上。

里面有地上各样四足的走兽,和昆虫,并天上的飞鸟。

又有声音向他说,彼得起来,宰了吃。

彼得却说,主阿,这是不可的,凡俗物,和不洁净的物,我从来没有吃过。

第二次有声音向他说,神所洁净的,你不可当作俗物。

这样一连三次,那物随既收回天上去了。

彼得心里正在猜疑之间,不知所看见的异象是什么意思,哥尼流所差来的人,已经访问到西门的家,站在门外,

喊着问,有称呼彼得的西门住在这里没有。

彼得还思想那异象的时候,圣灵向他说,有三个人来找你。

起来,下去,和他们同往,不要疑惑。因为是我差他们来的。

于是彼得下去见那些人,说,我就是你们所找的人。你们来是为什么缘故。

他们说,百夫长哥尼流是个义人,敬畏神,为犹太通国所称赞,他蒙一位圣天使指示,叫他请你到他家里去,听你的话。

彼得请他们进去,住了一宿。

次日起身和他们同去,还有约帕的几个弟兄同着他去。又次日,他们进入凯撒利亚。哥尼流已经请了他的亲属密友,等候他们。

彼得一进去,哥尼流就迎接他,俯伏在他脚前拜他。

彼得却拉他说,我也是人。

彼得和他说着话进去,见有好些人在那里聚集,

就对他们说,你们知道犹太人,和别国的人亲近往来,本是不合例的。但神已经指示我,无论什么人,都不能看作俗而不洁净的。

所已我被请的时候,就不推辞而来。现在请问,你们叫我来有什么意思呢。

哥尼流说,前四天这个时候,我在家中守着申初的祷告,忽然有一个人,穿着光明的衣裳,站在我面前,

说,哥尼流,你的祷告,已蒙垂听,你的周济,达到神面前已蒙记念了。

你当打发人往约帕去,请那称呼彼得的西门来,他住在海边一个硝皮匠西门的家里。

所以我立时打发人去请你,你来了很好。现在我们都在神面前,要听主所吩咐你的一切话。

彼得就开口说,我真看出神是不偏待人。

原来各国中,那敬畏主行义的人,都为主所悦纳。

神藉着耶稣基督(他是万有的主)传和平的福音,将这道赐给以色列人。

这话在约翰宣传洗礼以后,所加利利起,传遍了犹太。

神怎样以圣灵和能力,膏拿撒勒人耶稣,这都是你们知道的。他周流四方行善事,医好凡被魔鬼压制的人。因为神与他同在。

他在犹太人之地,并耶路撒冷,所行的一切事,有我们作见证。他们把他挂在木头上杀了。

第三日神叫他复活,显现出来,

不是显给众人看,乃是显给神豫先所拣选为他作见证的人看,就是我们这些在他从死里复活以后,和他同吃同喝的人。

他吩咐我们传道给众人,证明他是神所立定的,要作审判活人死人的主。

众先知也为他作见证,说,凡信他的人,必因他的名,得蒙赦罪。

彼得还说这话的时候,圣灵降在一切听道的人身上。

那些奉割礼和彼得同来的信徒,见圣灵的恩赐也浇在外邦人身上,就都希奇。

因听见他们说方言,称赞神为大。

于是彼得说,这些人既受了圣灵,与我们一样,谁能禁止用水给他们施洗呢。

就吩咐奉耶稣基督的名给他们施洗。他们又请彼得住了几天。

\chapter{使徒行传第11章}
使徒和在犹太的众弟兄,听说外邦人也领受了神的道。

及至彼得上了耶路撒冷,那些奉割礼的门徒和他争辩说,

你进入未受割礼之人的家,和他们一同吃饭了。

彼得就开口,把这事挨次给他们讲解说,

我在约帕城里祷告的时候,魂游象外,看见异象,有一物降下,好像一块大布,系着四角,从天上缒下,直来到我跟前。

我定睛观看,见内中有地上四足的牲畜,和野兽,昆虫,并天上的飞鸟。

我且听见有声音向我说,彼得,起来,宰了吃。

我说,主阿,这是不可的。凡俗而不洁净的物,从来没有入过我的口。

第二次,有声音从天上说,神所洁净的,你不可当作俗物。

这样一连三次,就都收回天上去了。

正当那时,有三个人站在我们所住的房门前,是从凯撒利亚差来见我的。

圣灵吩咐我和他们同去,不要疑惑。(或作不要分别等类)同着我去的,还有这六位弟兄。我们都进了那人的家。

那人就告诉我们,他如何看见一位天使,站在他屋里,说,你打发人往约帕去,请那称呼彼得的西门来。

他有话告诉你,可以叫你和你的全家得救。

我一开讲,圣灵便降在他们身上,正像当初降在我们身上一样。

我就想起主的话说,约翰是用水施洗,但你们要受圣灵的洗。

神既然给他们恩赐,像在我们信主耶稣基督的时候,给了我们一样,我是谁,能拦阻神呢。

众人听见这话,就不言语了。只归荣耀与神,说,这样看来,神也赐恩给外邦人,叫他们悔改得生命了。

那些因司提反的事遭患难四散的门徒,直走到腓尼基,和塞浦路斯,并安提阿。他们不向别人讲道,只向犹太人讲。

但内中有塞浦路斯,和古利奈人,他们到了安提阿,也向希腊人传讲主耶稣。(有古卷作也向说希腊话的犹太人传讲主耶稣)

主与他们同在,信而归主的人就很多了。

这风声传到耶路撒冷教会人的耳中,他们就打发巴拿巴出去,走到安提阿为止。

他到了那里,看见神所赐的恩,就欢喜,劝勉众人,立定心志,恒久靠主。

巴拿巴原是个好人,被圣灵充满,大有信心。于是有许多人归服了主。

他又往大数去找扫罗,

找着了,就带他到安提阿去。他们足有一年的工夫,和教会一同聚集,教训了许多人。门徒称为基督徒,是从安提阿起首。

当那些日子,有几位先知从耶路撒冷下到安提阿。

内中有一位,名叫亚迦布站起来,藉着圣灵,指明天下将有大饥荒。这事到革老丢年间果然有了。

于是门徒定意,照各人的力量捐钱,送去供给住在犹太的弟兄。

他们就这样行,把捐项托巴拿巴和扫罗,送到众长老那里。

\chapter{使徒行传第12章}
那时,希律王手下苦害教会中几个人。

用刀杀了约翰的哥哥雅各。

他见犹太人喜欢这事,又去捉拿彼得。那时正是除酵的日子。

希律拿了彼得收在监里,交付四班兵丁看守,每班四个人,意思要在逾越节后,把他提出来,当着百姓办他

于是彼得被囚在监里。教会却为他切切的祷告神。

希律将要提他出来的前一夜,彼得被两条铁链锁着,睡在两个兵丁当中。看守的人也在门外看守。

忽然有主的一个使者,站在旁边,屋里有光照耀。天使拍彼得的肋旁,拍醒了他,说,快快起来。那铁链就从他手上脱落下来。

天使对他说,束上带子。穿上鞋。他就那样作。天使又说,披上外衣跟着我来。

彼得就出来跟着他,不知道天使所作是真的,只当见了异象。

过了第一层,第二层监牢,就来到临街的铁门。那门自己开了。他们出来,走过一条街,天使便离开他去了。

彼得醒悟过来,说,我现在真知道主差遣他的使者,救我脱离希律的手,和犹太百姓一切所盼望的。

想了一想,就往那称呼马可的约翰他母亲马利亚家去。在那里有好些人聚集祷告。

彼得敲外门,有一个使女,名叫罗大出来探听。

听见是彼得的声音,就欢喜的顾不得开门,跑进去告诉众人说,彼得站在门外。

他们说,你是疯了。使女极力的说,真是他。他们说,必是他的天使。

彼得不住的敲门。他们开了门,看见他,就甚惊奇。

彼得摆手,不要他们作声,就告诉他们主怎样领他出监。又说,你们把这事告诉雅各,和众弟兄。于是出去往别处去了。

到了天亮,兵丁扰乱得很,不知道彼得往那里去了。

希律找他,找不着,就审问看守的人,吩咐把他们拉去杀了。后来希律离开犹太,下凯撒利亚去,住在那里。

希律恼怒推罗西顿的人。他们那一块地方,是从王的地土得粮,因此就托了王的内侍臣伯拉斯都的情,一心来求和。

希律在所定的日子,穿上朝服,坐在位上,对他们讲论一番。

百姓喊着说,这是神的声音,不是人的声音。

希律不归荣耀给神,所以主的使者立刻罚他。他被虫所咬,气就绝了。

神的道日见兴旺,越发广传。

巴拿巴和扫罗,办完了他们供给的事,就从耶路撒冷回来,带着称呼马可的约翰同去。

\chapter{使徒行传第13章}
在安提阿的教会中,有几位先知和教师,就是巴拿巴,和称呼尼结的西面,古利奈人路求,与分封之王希律同养的马念,并扫罗。

他们事奉主,禁食的时候,圣灵说,要为我分派巴拿巴和扫罗,去作我召他们所作的工。

于是禁食祷告,按手在他们头上,就打发他们去了。

他们既被圣灵差遣,就下到西流基,从那里坐船往塞浦路斯去。

到了撒拉米,就在犹太人各会堂里传讲神的道。也有约翰作他们的帮手。

经过全岛,直到帕弗,在那里遇见一个有法术假充先知的犹太人,名叫巴耶稣。

这人常和方伯士求保罗同在,士求保罗是个通达人。他请了巴拿巴和扫罗来,要听神的道。

只是那行法术的吕马,(这名翻出来就是行法术的意思)敌挡使徒,要叫方伯不信真道。

扫罗又名保罗,被圣灵充满定睛看他,

说,你这充满各样诡诈奸恶,魔鬼的儿子,众善得仇敌,你混乱主的正道还不止住吗。

现在主的手加在你身上。你要瞎眼,暂且不见日光。他的眼睛立刻昏蒙黑暗,四下里求人拉着手领他。

方伯看见所作的事,很希奇主的道,就信了。

保罗和他的同人,从帕弗开船,来到旁非利亚的别加。约翰就离开他们回耶路撒冷去。

他们离了别加往前行,来到彼西底的安提阿。在安息日进会堂作下。

读完了律法先知的书,管会堂的叫人过去,对他们说,二位兄台,若有什么劝勉众人的话,请说。

保罗就起来,举手说,以色列人,和一切敬畏神的人,请听。

这以色列民的神,拣选了我们的祖宗,当民寄居埃及的时候,抬举他们,用大能的手领他们出来。

又在旷野容忍他们约有四十年。(容忍或作抚养)

既灭了迦南地七族的人,就把那地分给他们为业。

此后,给他们设立士师,约有四百五十年,直到先知撒母耳的时候。

后来他们求一个王,神就将便雅悯支派中,基士的儿子扫罗,给他们作王四十年。

既废了扫罗,就选立大卫作他们的王。又为他作见证说,我寻得耶西的儿子大卫,他是合我心意的人,凡事要遵行我的旨意。

从这人的后裔中,神已经照着所应许的,为以色列人立了一位救主,就是耶稣。

在他没有出来以先,约翰向以色列众民宣讲悔改的洗礼。

约翰将行尽他的程途说,你们以为我是谁,我不是基督。只是有一位在我以后来的,我解他脚上的鞋带,也是不配的。

弟兄们,亚伯拉罕的子孙,和你们中间敬畏神的人哪,这救世的道,是传给我们的。

耶路撒冷居住的人,和他们的官长,因为不认识基督,也不明白每日安息日所读众先知的书,就把基督定了死罪,正应了先知的预言。

虽然查不出他有当死的罪来,还是求彼拉多杀他。

既成就了经上指着他所记的一切话,就把他从木头上取下来,放在坟墓里。

神却叫他从死里复活。

那从加利利同上耶路撒冷的人多日看见他,这些人如今在民间是他的见证。

我们也报好信息给你们,就是那应许祖宗的话,

神已经向我们这作儿女的应验,叫耶稣复活了。正如诗篇第二篇上记着说,你是我的儿子,我今日生你。

论到神叫他从死里复活,不再归于朽坏,就这样说,我必将所应许大卫那圣洁可靠的恩典,赐给你们。

又有一篇上说,你必不叫你的圣者见朽坏。

大卫在世的时候,遵行了神的旨意,就睡了,(或作大卫按神的旨意服事了他那一世的人就睡了)。归到他祖宗那里,已经见朽坏。

惟独神所复活的,他并未见朽坏。

所以弟兄们,你们当晓得,赦罪的道是由这人传给你们的。

你们靠摩西的律法,在一切不得称义的事上,信靠这人,就都称义了。

所以你们务要小心,免得先知书上所说的临到你们。

主说,你们这轻慢的人要观看,要惊奇,要灭亡。因为在你们的时候,我行一件事,虽有人告诉你们,你们总是不信。

他们出会堂的时候,众人请他们到下安息日,再讲这话给他们听。

散会以后,犹太人和敬虔进犹太教的人,多有跟从保罗,巴拿巴的,二人对他们讲道,劝他们务要恒久在神的恩中。

到下安息日,合城的人,几乎都来聚集,要听神的道。

但犹太人看见人这样多,就满心嫉妒,硬驳保罗所说的话,并毁谤。

保罗和巴拿巴放胆说,神的道先讲给你们,原是应当的,只因你们弃绝这道,断定自己不配得永生,我们就转向外邦人去。

因为主曾这样吩咐我们说,我已经立你作外邦人的光,叫你施行救恩直到地极。

外邦人听见这话,就欢喜了,赞美神的道,凡豫定得永生的人都信了。

于是主的道,传遍了那一带地方。

但犹太人挑唆虔敬尊贵的妇女,和城内有名望的人,逼迫保罗,巴拿巴,将他们赶出境外。

二人对着众人跺下脚上的尘土,就往以哥念去了。

门徒满心喜乐,又被圣灵充满。

\chapter{使徒行传第14章}
二人在以哥念同进犹太人的会堂,在那里讲的叫犹太人,和希腊人,信的人很多。

但那不顺从的犹太人耸动外邦人,叫他们心里恼恨弟兄。

二人在那里住了多日,倚靠主放胆讲道。主藉他们的手,施行神迹奇事,证明他的恩道。

城里的众人就分了党。有附从犹太人的,有附从使徒的。

那时,外邦人和犹太人,并他们的官长,一齐拥上来,要凌辱使徒,用石头打他们。

使徒知道了,就逃往吕高尼的路司得,特庇,两个城,和周围地方去。

在那里传福音。

路司得城里,坐着一个两脚无力的人,生来是瘸腿的,从来没有走过。

他听保罗讲道。保罗定睛看他,见他有信心,可得痊愈,

就大声说,你起来,两脚站直。那人就跳起来而且行走。

众人看见保罗所作的事,就用吕高尼的话,大声说,有神藉着人形,降临在我们中间了。

于是称巴拿巴为宙斯,称保罗为希耳米,因为他说话领首。

有城外宙斯庙的祭司,牵着牛,拿着花圈,来到门前,要同众人(向使徒献祭。

巴拿巴,保罗,二使徒听见,就撕开衣裳,跳进众人中间,喊着说,

诸君,为什么作这事呢。我们也是人,性情和你们一样。我们传福音给你们,是叫你们离弃这些虚妄,归向那创造天,地,海,和其中万物的永生神。

他在从前的世代,任凭万国各行其道。

然而为自己未尝不显出证据来,就如常施恩惠,从天降雨,赏赐丰年,叫你们饮食饱足,满心喜乐。

二人说了这些话,仅仅的拦住众人不献祭与他们。

但有些犹太人,从安提阿和以哥念来,挑唆众人,就用石头打保罗,以为他是死了,便拖到城外。

门徒正围着他,他就起来,走进城去。第二天,同巴拿巴往特庇去,

对那城里的人传了福音,使好些人作门徒。就回路司得,以哥念,安提阿去,

坚固门徒的心,劝他们恒守所信的道。又说,我们进入神的国,必须经历许多艰难。

二人在各教会中选立了长老,又禁食祷告,就把他们交托所信的主。

二人经过彼西底,来到旁非利亚。

在别加讲了道,就下亚大利去。

从那里坐船,往安提阿去。当初他们被众人所托蒙神之恩,要辨现在所作之工,就是在这地方。

到了那里,聚集了会众,就述说神藉他们所行的一切事,并神怎样为外邦人开了信道的门。

二人就在那里同门徒住了多日。

\chapter{使徒行传第15章}
有几个人,从犹太下来,教训弟兄们说,你们若不摩西的规条受割礼,不能得救。

保罗巴拿巴与他们大大的分争辩论,众门徒就定规,叫保罗,巴拿巴和本会中几个人,为所辩论的,上耶路撒冷去,见使徒和长老。

于是教会送他们起行,他们经过腓尼基,撒玛利亚,随处传说外邦人归主的事,叫众弟兄都甚欢喜。

到了耶路撒冷,教会和使徒并长老,都接待他们,他们就述说神同他们所行的一切事。

惟有几个信徒是法利赛教门的人,起来说,必须给外邦人行割礼,吩咐他们遵守摩西的律法。

使徒和长老,聚会商议这事。

辩论已经多了,彼得就起来,说,诸位弟兄,你们知道神早已在你们中间拣选了我,叫外邦人从我口中得听福音之道,而且相信。

知道人心的神,也为他们作了见证。赐圣灵给他们,正如给我们一样。

又藉着信,洁净了他们的心,并不分他们我们。

现在为什么试探神,要把我们祖宗和我们所不能负的轭,放在门徒的颈项上呢。

我们得救,乃是因主耶稣的恩,和他们一样,这是我们所信的。

众人都默默无声,听巴拿巴和保罗,述说神藉他们在外邦人所行的神迹奇事。

他们住了声,雅各就说,诸位弟兄,请听我的话。

方才西门述说神当初怎样眷顾外邦人,从他们中间选取百姓归于自己的名下

众先知的话,也与这意思相合。

正如经上所写的,此后我要回来,重新修造大卫倒塌的帐幕,把那破坏的,重新修造建立起来。

叫馀剩的人,就是凡称为我名下的外邦人,都寻求主。

这话是从创世以来,显明这事的主说的。

所以据我的意见,不可难为那归服神的外邦人。

只要写信,吩咐他们禁戒偶像的污秽和奸淫,并勒死的牲畜,和血。

因为从古以来,摩西的书在各城有人传讲,每逢安息日,在会堂里诵读。

那时,使徒和长老并全教会,定意从他们中间拣选人,差他们和保罗,巴拿巴,同往安提阿去。所拣选的,就是称呼巴撒巴的犹大,和西拉,这两个人在弟兄中是作首领的。

于是写信交付他们,内中说,使徒和作长老的弟兄们,问安提阿,叙利亚,基利家外邦众弟兄的安。

我们听说有几个人,从我们这里出去,用言语搅扰你们,惑乱你们的心。(有古卷在此有你们必须受割礼守摩西的律法)。其实我们并没有吩咐他们。

所以我们同心定意,拣选几个人,差他们同我们所亲爱的巴拿巴,和保罗,住你们那里去。

这二人是为我主耶稣基督的名,不顾性命的。

我们就差了犹大和西拉,他们也要亲口诉说这些事。

因为圣灵和我们,定意不将别的重担放在你们身上。惟有几件事是不可少的,

就是禁戒祭偶像的物,和血,并勒死的牲畜,和奸淫。这几件你们若能自己禁戒不犯,就好了。愿你们平安。

他们既奉了差遣,就下安提阿去,聚集众人,交付书信。

众人念了,因为信上安慰的话,就欢喜了。

犹大和西拉也是先知,就用许多话劝勉弟兄,坚固他们。

住了些日子,弟兄们打发他们平平安安的回到差遣他们的人那里去。(有古卷在此有徒15:34节,惟有西拉定意仍住在那里)

但保罗和巴拿巴,仍住在安提阿,和许多别人一同教训人,传主的道。

过了些日子,保罗对巴拿巴说,我们可以回到从前宣传主道的各城,看望弟兄们景况如何。

巴拿巴有意,要带称呼马可的约翰同去。

但保罗,因为马可从前在旁非利亚离开他们,不和他们同去作工,就以为不可带他去。

于是二人起了争论,甚至彼此分开。巴拿巴带着马可,坐船往塞浦路斯去。

保罗拣选了西拉,也出去,蒙弟兄们把他交于主的恩中。

他就走遍叙利亚,基利家,坚固众教会。

\chapter{使徒行传第16章}
保罗来到特庇,又到路司得。在那里有一个门徒,名叫提摩太,是信主之犹太妇人的儿子,他父亲却是希腊人。

路司得和以哥念的弟兄,都称赞他。

保罗要带他同去,只因那些地方的犹太人,都知道他父亲是希腊人,就给他行了割礼。

他们经过各城,把耶路撒冷使徒和长老所定的条规,交给门徒遵守。

于是众教会信心越发坚固,人数天天加增。

圣灵既然禁止他们在亚细亚讲道,他们就经过弗吕家,加拉太一带地方。

到了每西亚的边界,他们想要往庇推尼去,耶稣的灵却不许。

他们就越过每西亚,下到特罗亚去。

在夜间有异象现与保罗。有一个马其顿人,站着求他说,请你过到马其顿来帮助我们。

保罗既看见这异象,我们随既想要往马其顿去,以为神召我们传福音给那里的人听。

于是从特罗亚开船,一直行到撒摩特喇,第二天到了尼亚波利。

从那里来到腓立比,就是马其顿这一方的头一个城。也是罗马的驻防城。我们在这城里住了几天。

当安息日,我们出城门,到了河边,知道那里有一个祷告的地方,我们就坐下对那聚会的妇女讲道。

有一个卖紫色布疋的妇人,名叫吕底亚,是推雅推喇城的人,素来敬拜神。他听见了,主就开导他的心,叫他留心听保罗所讲的话。

他和他一家,既领了洗,便求我们说,你们若以为我是真信主的,(或作你们若以为我是忠心事奉主的)请到我家里来住。于是强留我们。

后来,我们往那祷告的地方去。有一个使女迎着面来,他被巫鬼所附,用法术,叫他主人得大财利。

他跟随保罗和我们,喊着说,这些人是至高神的仆人,对你们传说救人的道。

他一连多日这样喊叫,保罗就心中厌烦,转身对那鬼说,我奉耶稣基督的名,吩咐你从他身上出来。那鬼当时就出来了。

使女的主人们,见得利的指望没有了,便揪住保罗和西拉,拉他们到市上去见首领。

又带到官长面前说,这些人原是犹太人,竟骚扰我们的城,

传我们罗马人所不可受,不可行的规矩。

众人就一同起来攻击他们。官长吩咐剥了他们的衣裳,用棍打。

打了许多棍,便将他们下在监里,嘱咐禁卒严紧看守。

禁卒领了这样的命,就把他们下在监里,两脚上了木狗。

约在半夜,保罗和西拉,祷告唱诗赞美神,众囚犯也侧耳而听。

忽然地大震动,甚至监牢的地基都摇动了。监门立刻全开,众囚犯的锁链也都松开了。

禁卒一醒,看见监门全开,以为囚犯已经逃走,就拔刀要自杀。

保罗大声呼叫说,不要伤害自己,我们都在这里。

禁卒叫人拿灯来,就跳进去,战战竞竞的,俯伏在保罗西拉面前。

又领他们出来说,二位先生,我当怎样行才能得救。

他们说,当信主耶稣,你和你一家都必得救。

他们就把主的道,讲给他和他全家的人听。

当夜就在那时候,禁卒把他们带去,洗他们的伤。他和属乎他的人,立时都受了洗。

于是禁卒领他们上自己家里去,给他们摆上饭,他和全家,因为信了神,都很喜乐。

到了天亮,官长打发差役来说,释放那两个人吧。

禁卒就把这话告诉保罗说,官长打发人来叫释放你们。如今可以出监,平平安安的去吧。

保罗却说,我们是罗马人,并没有定罪,他们就在众人面前打了我们,又把我们下在监里。现在要私下撵我们出去吗,这是不行的。叫他们自己来领我们出去吧。

差役把这话回禀官长。官长听见他们是罗马人,就害怕了。

于是来劝他们,领他们出来,请他们离开那城。

二人出了监,往吕底亚家里去。见了弟兄们,劝慰他们一番,就走了。

\chapter{使徒行传第17章}
保罗和西拉,经过暗妃波里,亚波罗尼亚,来到帖撒罗尼迦,在那里有犹太人的会堂。

保罗照他素常的规矩进去,一连三个安息日,本着圣经与他们辩论,

讲解陈明基督必须受害,从死里复活。又说,我所传与你们的这位耶稣,就是基督。

他们中间有些人听了劝,就附从保罗和西拉。并有许多虔敬的希腊人,尊贵的妇女也不少。

但那不信的犹太人心里嫉妒,招聚了些市井匪类,搭夥成群,耸动合城的人,闯进耶孙的家,要将保罗西拉带到百姓那里。

找不着他们,就把耶孙和几个弟兄,拉到地方官那里,喊叫说,那搅乱天下的,也到这里来了。

耶孙收留他们。这些人都违背凯撒的命令,说另有一个王耶稣。

众人和地方官,听见这话,就惊慌了。

于是取了耶孙和其馀之人的保状,就释放了他们。

弟兄们,随既在夜间打发保罗和西拉往庇哩亚去。二人到了,就进入犹太人的会堂。

这地方的人,贤于帖撒罗尼迦的人,甘心领受这道,天天考查圣经,要晓得这道,是与不是。

所以他们中间多有相信的。又有希腊尊贵的妇女,男子也不少。

但帖撒罗尼迦的犹太人,知道保罗又在庇哩亚传神的道,也就往那里去,耸动搅扰众人。

当时弟兄们便打发保罗往海边去。西拉和提摩太仍住在庇哩亚。

送保罗的人带他到了雅典。既领了保罗的命令,叫西拉和提摩太速速到他这里来,就回去了。

保罗在雅典等候他们的时后,看见满城都是偶像,就心里着急。

于是在会堂里,与犹太人,和虔敬的人,并每日在市上所遇见的人辩论。

还有以彼古罗和斯多亚两门的学士,与他争论。有的说,这胡言乱语的要说什么。有的说,他似乎是传说外邦鬼神的。这话是保罗传讲耶稣,与复活的道。

他们就把他带到亚略巴古说,你所讲的这新道,我们也可以知道吗。

因为你有些奇怪的事,传到我们耳中。我们愿意知道这些事是什么意思。

雅典人,和住在那里的客人,都不顾别的事,只将新闻说说听听

保罗站在亚略巴古当中,说,众位雅典人哪,我看你们凡事很敬畏鬼神。

我游行的时候,观看你们所敬拜的,遇见一座坛,上面写着未识之神。你们所不认识而敬拜的,我现在告诉你们。

创造宇宙和其中万物的神,既是天地的主,就不住人手所造的殿。

也不用人手服事,好像缺少什么,自己倒将生命气息,万物,赐给万人。

他从一本造出万族的人,(本有古卷作血脉),住在全地上,并且豫先定准他们的年限,和所住的疆界。

要叫他们寻求神,或者可以揣摩而得,其实他离我们各人不远。

我们生活,动作,存留,都在乎他,就如你们作诗的,有人说,我们也是他所生的。

我们既是神所生的,就不当以为神的神性像人用手艺,心思,所雕刻的金,银,石。

世人蒙昧无知的时后,神并不监察,如今却吩咐各处的人都要悔改。

因为他已经定了日子,要藉着他所设立的人,按公义审判天下。并且叫他从死里复活,给万人作可信的凭据。

众人听见从死里复活的话,就有讥诮他的,又有人说,我们再听你讲这个吧。

于是保罗从他们中出去了。

但有几个人贴近他,信了主,其中有亚略巴古的官丢尼修,并一个妇人,名叫大马哩,还有别人一同信从。

\chapter{使徒行传第18章}
这事以后,保罗离了雅典,来到哥林多。

遇见一个犹太人,名叫亚居拉,他生在本都。因为革老丢犹太人都离开罗马,新近带着妻百基拉,从意大利来。保罗就投奔了他们。

他们本是制造帐棚为业。保罗因与他们同业,就和他们同住作工。

每逢安息日,保罗在会堂里辩论,劝化犹太人和希腊人。

西拉和提摩太从马其顿来的时候,保罗为道迫切,向犹太人证明耶稣是基督。

他们既抗拒,毁谤,保罗就抖着衣裳说,你们的罪归到你们自己头上,(罪原文作血)与我无干,(原文作我却乾净)从今以后,我要往外邦人那里去,

于是离开那里,到了一个人的家中,这人名叫提多犹士都,是敬拜神的,他的家靠近会堂。

管会堂的基利司布和全家都信了主。还有许多哥林多人听了,就相信受洗。

夜间主在异象中对保罗说,不要怕,只管讲,不要闭口。

有我与你同在,必没有人下手害你。因为在这城里我有许多的百姓。

保罗在那里住了一年零六个月,将神的道教训他们。

到迦流作亚该亚方伯的时候,犹太人同心起来攻击保罗,拉他到公堂,

说,这个人劝人不按着律法敬拜神。

保罗刚要开口,迦流就对犹太人说,你们这些犹太人,如果是为冤枉,或奸恶的事,我理当耐性听你们。

但所争论的,若是关乎言语,名目,和你们的律法,你们自己去辨吧。这样的事我不愿意审问。

就把他们撵出公堂。

众人便揪住管会堂的所提尼,在堂前打他。这些事迦流都不管。

保罗又住了多日,就辞别了弟兄,坐船往叙利亚去,百基拉,亚居拉和他同去。他因为许过愿,就在坚革哩剪了头发。

到了以弗所,保罗就把他们留在那里,自己进了会堂,和犹太人辩论。

众人请他多住些日子,他却不允。

就辞别他们说,神若许我,我还要回到你们这里。于是开船离了以弗所。

在凯撒利亚下了船,就上耶路撒冷去问教会安,随后下安提阿去。

住了些日子,又离开那里,挨次经过加拉太和弗吕家地方,坚固众门徒。

有一个犹太人,名叫亚波罗,来到以弗所。他生在亚力山太,是有学问的,最能讲解圣经。(学问或作口才)。

这人已经在主的道上受了教训,心里火热,将耶稣的事,详细讲论教训人。只是他单晓得约翰的洗礼。

他在会堂放胆讲道,百基拉,亚居拉听见,就接他来,将神的道给他讲解更加详细。

他想要往亚该亚去。弟兄们就勉励他,并写信请门徒接待他。(或作弟兄们就写信劝门徒接待他)他到了那里,多帮助那蒙恩信主的人。

在众人面前极有能力,驳倒犹太人,引圣经证明耶稣是基督。

\chapter{使徒行传第19章}
亚波罗在哥林多的时候,保罗经过了上边一带地方,就来到以弗所。在那里遇见几个门徒。

问他们说,你们信的时候,受了圣灵没有。他们回答说,没有,也未曾听见有圣灵赐下来

保罗说,这样,你们受的是什么洗呢。他们说,是约翰的洗。

保罗说,约翰所行的是悔改的洗,告诉百姓,当信那在他以后要来的,就是耶稣。

他们听见这话,就奉主的名受洗。

保罗按手在他们头上,圣灵便降在他们身上。他们就说方言,又说预言。(或作又讲道)。

一共约有十二个人。

保罗进会堂,放胆讲道,一连三个月,辩论神国的事,劝化众人。

后来有些人,心里刚硬不信,在众人面前毁谤这道,保罗就离开他们,也叫门徒与他们分离,便在推喇奴的学房,天天辩论。

这样有两年之久,叫一切住在亚细亚的,无论是游太人,是希腊人,都听见主的道。

神藉保罗的手,行了些非常的奇事。

甚至有人从保罗身上拿手巾,或围裙,放在病人身上,病就退了,恶鬼也出去了。

那时,有几个游行各处,念咒赶鬼的犹太人,向那被恶鬼附的人,擅自称主耶稣的名,说,我奉保罗所传的耶稣,敕令你们出来。

作这事的,有犹太祭司长士基瓦的七个儿子。

恶鬼回答他们说,耶稣我认识,保罗我也知道。你们却是谁呢。

恶鬼所附的人,就跳在他们身上,胜了其中二人,制伏他们,叫他们赤着身子受了伤,从那房子里逃出去了。

凡住在以弗所的,无论是犹太人,是希腊人,都知道这事,也都惧怕,主耶稣的名从此就尊大了。

那已经信的,多有人来承认诉说自己所行的事。

平素行邪术的,也有许多人把书拿来,堆积在众人面前焚烧。他们算计书价,便知道共合五万块钱。

主的道大大兴旺而且得胜,就是这样。

这些事完了,保罗心里定意,经过了马其顿,亚该亚就往耶路撒冷去,又说,我到了那里以后,也必须往罗马去看看。

于是从帮助他的人中,打发提摩太,以拉都二人,往马其顿去。自己暂时等在亚细亚。

那时,因为这道起的扰乱不小。

有一个银匠,名叫底米丢,是制造亚底米神银龛的,他使这样手艺人生意发达。

他聚集他们和同行的工人,说,众位,你们知道我们是倚靠这生意发财。

这保罗不但在以弗所,也几乎在亚细亚全地,引诱迷惑许多人,说,人手所作的不是神,这是你们所看见所听见的。

这样,不独我们这事业,被人藐视,就是大女神亚底米的庙,也要被人轻忽,连亚细亚全地,和普天下,所敬拜的大女神之威荣,也要销灭了。

众人听见,就怒气填胸,喊着说,大哉以弗所人的亚底米阿。

满城都轰动起来。众人拿住保罗同行的马其顿人,该犹,和亚里达古,齐心拥进戏园里去。

保罗想要进去,到百姓那里,门徒却不许他去。

还有亚细亚几位首领,是保罗的朋友,打发人来劝他,不要冒险到戏园里去。

聚集的人,纷纷乱乱,有喊叫这个的,有喊叫那个的,大半不知道为什么聚集。

有人把亚历山大从众人中带出来,犹太人推他往前,亚历山大就摆手,要向百姓分诉。

只因他们认出他是犹太人,就大家同声喊着说,大哉以弗所人的亚底米阿。如此约有两小时

那城里的书记,安抚了众人,就说,以弗所人哪,谁不知道以弗所人的城,是看守大亚底米的庙,和从宙斯那里落下来的像呢。

这事既是驳不倒的,你们就当安静,不可造次。

你们把这些人带来,他们并没有偷窃庙中之物,也没有谤??我们的女神。

若是底米丢和他同行的人,有控告人的事,自有放告的日子。(或作自有公堂)也有方伯,可以彼此对告。

你们若问别的事,就可以照常例聚集断定。

今日的搅扰,本是无缘无故,我们难免被查问。论到这样聚众,我们也说不出所以然来。

说了这话,便叫众人散去。

\chapter{使徒行传第20章}
乱定之后,保罗请门徒来,劝勉他们,就辞别起行,往马其顿去。

走遍了那一带地方,用许多话劝勉门徒,(或作众人)然后来到希腊。

在那里住了三个月,将要坐船往叙利亚去。犹太人设计要害他,他就定意从马其顿回去。

同他到亚细亚去的,有庇哩亚人毕罗斯的儿子所巴特,帖撒罗尼迦人亚里达古,和西公都,还有特庇人该犹,并提摩太,又有亚细亚人推基古,和特罗非摩。

这些人先走在特罗亚等候我们。

过了除酵的日子,我们从腓立比开船,五天到了特罗亚,和他们相会,在那里住了七天。

七日的第一日,我们聚会擘饼的时候,保罗因为要次日起行,就与他们讲论,直到半夜。

我们聚会的那座楼上,有些灯烛。

有一个少年人,名叫犹推古,坐在窗台上,困倦沉睡。保罗讲了多时,少年人睡熟了,就从三层楼上掉下去。扶起他来,已经死了。

保罗下去,伏在他身上,抱着他,说,你们不要发慌,他的灵魂还在身上。

保罗又上去,擘饼,吃了,谈论许久,直到天亮,这才走了。

有人把那童子活活的领来,得的安慰不小。

我们先上船开往亚朔去,意思要在那里接保罗。因为他是这样安排的,他自己打算要步行。

他既在亚朔与我们相会,我们就接他上船,来到米推利尼。

从那里开船,次日到了基阿的对面。又次日,在撒摩靠岸。又次日,来到米利都。

乃因保罗早已定意越过以弗所,免得在亚细亚耽延。他急忙前走,巴不得赶五旬节能到耶路撒冷。

保罗从米利都打发人往以弗所去,请教会的长老来。

他们来了,保罗就说,你们知道,自从我到亚细亚的日子以来,在你们中间始终为人如何,

服事主,凡事谦卑,眼中流泪,又因犹太人的谋害,经历试炼。

你们也知道,凡与你们有益的,我没有一样避讳不说的。或在众人面前,或在各人家里,我都教导你们。

又对犹太人,和希腊人,证明当向神悔改,信靠我主耶稣基督。

现在我往耶路撒冷去,心甚迫切,(原文作心被捆绑)不知道在那里要遇见什么事。

但知道圣灵在各城里向我指证,说,有捆锁与患难等待我。

我却不以性命为念,也不看为宝贵,只要行完我的路程,成就我从主耶稣所领受的职事,证明神恩惠的福音。

我素常在你们中间来往,传讲神国的道,如今我晓得你们以后都不得再见我的面了。

所以我今日向你们证明,你们中间无论何人死亡,罪不在我身上。(原文作我于众人的血是洁净的)。

因为神的旨意,我并没有一样避讳不传给你们的。

圣灵立你们作全群的监督,你们就当为自己谨慎,也为全群谨慎,牧养神的教会,就是他用自己血所买来的。(或作救赎)

我知道我去之后,必有凶暴的豺狼,进入你们中间,不爱惜羊群。

就是你们中间,也必有人起来,说悖谬的话,要引诱门徒跟从他们。

所以你们应当儆醒,记念我三年之久,昼夜不住的流泪,劝戒你们各人。

如今我把你们交托神,和他恩惠的道。这道能建立你们,叫你们和一切成圣的人同得基业。

我未曾贪图一个人的金,银,衣服。

我这两只手,常供给我和同人的需用,这是你们自己知道的。

我凡事给你们作榜样,叫你们知道,应当这样劳苦,扶助软弱的人,又当记念主耶稣的话,说,施比受更为有福。

保罗说完了这话,就跪下同众人祷告。

众人痛哭,抱着保罗的颈项,和他亲嘴。

叫他们最伤心的,就是他说,以后不能再见我的面那句话。于是送他上船去了。

\chapter{使徒行传第21章}
我们离别了众人,就开船一直行到哥士。第二天到了罗底,从那里到帕大喇。

遇见一只船,要往腓尼基去,就上船起行。

望见塞浦路斯,就从南边行过,往叙利亚去,我们就在推罗上岸。因为船要在那里卸货。

找着了门徒,就在那里住了七天。他们被圣灵感动,对保罗说,不要上耶路撒冷去。

过了这几天,我们就起身前行。他们众人同妻子儿女,送我们到城外,我们都跪在岸上祷告,彼此辞别。

我们上了船,他们就回家去了。

我们从推罗行尽了水路,来到多利买,就问那里的弟兄安,和他们同住了一天。

第二天,我们离开那里,来到凯撒利亚。就进了传福音的腓利家,和他同住。他是那七个执事里的一个。

他有四个女儿,都是处女,是说预言的。

我们在那里多住了几天,有一个先知,名叫亚迦布,从犹太下来。

到了我们这里,就拿保罗的腰带,捆上自己的手脚,说,圣灵说,犹太人在耶路撒冷,要如此捆绑这腰带的主人,把他交给外邦人手里。

我们和那本地的人,听见这话,都苦劝保罗不要上耶路撒冷去。

保罗说,你们为什么这样痛哭,使我心碎呢。我为主耶稣的名,不但被人捆绑,就是死在耶路撒冷,也是愿意的。

保罗既不听劝,我们便住口了,只说,愿主的旨意成就便是了。

过了几日,我们收拾行李上耶路撒冷去。

有凯撒利亚的几个门徒和我们同去,带我们到一个久为(久为或作老)门徒的家里,叫我们与他同住,他名叫拿孙,是塞浦路斯人。

到了耶路撒冷,弟兄们欢欢喜喜的接待我们。

第二天,保罗同我们去见雅各。长老们也都在那里。

保罗问了他们安,便将神用他传教,在外邦人中间所行之事,一一的述说了。

他们听见,就归荣耀与神,对保罗说,兄台,你看犹太人中信主的有多少万,并且都为律法热心。

他们听见人说,你教训一切在外邦的犹太人,离弃摩西,对他们说,不要给孩子行割礼,也不要遵行条规。

众人必听见你来了,这可怎吗办呢。

你就照着我们的话行吧,我们这里有四个人,都有愿在身。

你带他们去,与他们一同行洁净的礼,替他们拿出规费,叫他们得以剃头。这样,众人就可知道,先前所听见你的事都是虚的。并可知道,你自己为人,循规蹈矩,遵行律法。

至于信主的外邦人,我们已经写信拟定,叫他们谨忌那祭偶像之物,和血,并勒死的牲畜,与奸淫。

于是保罗带着那四个人,第二天与他们一同行了洁净的礼,进了殿,报明洁净的日期满足。只等祭司为他们各人献祭。

那七日将完,从亚细亚来的犹太人,看见保罗在殿里,就耸动了众人,下手拿他,

喊叫说,以色列人来帮助,这就是在各处教训众人糟践我们百姓,和律法,并这地方的。他又带着希腊人进殿,污秽了这圣地。

这话是因他们曾看见以弗所人特罗非摩,同保罗在城里,以为保罗带他进了殿。

合城都震动,百姓一齐跑来,拿住保罗,拉他出殿,殿门立刻都关了。

他们正想要杀他,有人报信给营里的千夫长说,耶路撒冷合城都乱了。

千夫长立时带着兵丁,和几个百夫长,跑下去到他们那里。他们见了千夫长和兵丁,就止住不打保罗。

于是千夫长上前拿住他,吩咐用两条铁链捆锁。又问他是什么人,作的是什么事。

众人有喊叫这个的,有喊叫那个的。千夫长因为这样乱囔,得不着实情,就吩咐人把保罗带进营楼去。

到了台阶,众人挤得兄猛,兵丁只得将保罗抬起来。

众人跟在后面,喊着说,除掉他。

将要带他进营楼,保罗对千夫长说,我对你说句话,可以不可以。他说,你懂得希腊话吗。

你莫非是从前作乱,带领四千凶徒,往旷野去的那埃及人吗。

保罗说,我本是犹太人,生在基利家的大数,并不是无名小城的人,求你准我对百姓说话。

千夫长准了,保罗就站在台阶上,向百姓摆手,他们都静默无声,保罗使用希伯来话对他们说。

\chapter{使徒行传第22章}
诸位父兄请听,我现在对你们分诉。

众人听他说的是希伯来话,就更加安静了。

保罗说,我原是犹太人,生在基利家的大数,长在这城里,在迦玛列门下,按着我们祖宗严紧的律法受教,热心事奉神,像你们众人今日一样。

我也曾逼迫奉这道的人,直到死地,无论男女都锁拿下。

这是大祭司和众长老都可以给我作见证的。我又领了他们达与兄弟的书信,往大马士革去,要把在那里奉这道的人锁拿,带到耶路撒冷受刑。

我将到大马士革,正走的时后,约在晌午,忽然从天上发大光,四面照着我。

我就仆倒在地,听见有声音对我说,扫罗,扫罗,你为什么逼迫我。

我回答说,主阿,你是谁。他说,我就是你所逼迫的拿撒勒人耶稣。

与我同行的人,看见了那光,却没有听明那位对我说话的声音。

我说,主阿,我当作什么。主说,起来,进大马士革去,在那里要将所派你作的一切事,告诉你。

我因那光的荣耀,不能看见,同行的人,就拉着我手进了大马士革。

那里有一个人,名叫亚拿尼亚,按着律法是虔诚人,为一切住在那里的犹太人所称赞。

他来见我,站在旁边,对我说,兄弟扫罗,你可以看见。我当时往上一看,就看见了他。

他又说,我们祖宗的神,拣选了你,叫你明白他的旨意,又得见那义者,听他口中所出的声音。

因为你要将所看见的,所听见的,对着万人为他作见证。

现在你为什么耽延呢,起来,求告他的名受洗,洗去你的罪。

后来我回到耶路撒冷,在殿里祷告的时候,魂游象外,

看见主向我说,你赶紧的离开耶路撒冷,不可迟延,因你为我作的见证,这里的人,必不领受。

我就说,主阿,他们知道我从前把你的人,收在监里,又在各会堂里鞭打他们。

并且你的见证人司提反,被害流血的时候,我也站在旁边欢喜。又看守害死他之人的衣裳。

主向我说,你去吧。我要差你远远的往外邦人那里去。

众人听他说到这句话,就高声说,这样的人,从世上除掉他吧。他是不当活着的。

众人喧囔,摔掉衣裳,把尘土向空中扬起来。

千夫长就吩咐将保罗带进营楼去,叫人用鞭子拷问他,要知道他们向他这样喧囔,是为什么缘故。

刚用皮条捆上,保罗对旁边站着的百夫长说,人是罗马人,又没有定罪,你们就鞭打他,有这个例吗。

百夫长听见这话,就去见千夫长,告诉他说,你要作什么。这人是罗马人。

千夫长就来问保罗说,你告诉我,你是罗马人吗。保罗说,是。

千夫长说,我用许多银子,才入了罗马的民藉。保罗说,我生来就是。

于是那些要拷问保罗的人,就离开他去了。千夫长继知道他是罗马人,又因为捆绑了他,也害怕了。

第二天,千夫长为要知道犹太人控告保罗的实情,便解开他,吩咐祭司长和全公会的人,都聚集,将保罗带下来,叫他站在他们面前。

\chapter{使徒行传第23章}
保罗定睛看着公的人,说,弟兄们,我在神面前行事为人,都是凭着良心,直到今日。

大祭司亚拿尼亚,就吩咐旁边站着的人打他的嘴。

保罗对他说,你这粉饰的墙。神要打你。你坐堂为的是按律法审问我,你竟违背律法,吩咐人打我吗。

站在旁边的人说,你辱骂神的大祭司吗。

保罗说,弟兄们,我不晓得他是大祭司。经上记着说,不可毁谤你百姓的官长。

保罗看出大众,一半是撒都该人,一半是法利赛人,就在公会中大声说,弟兄们,我是法利赛人,也是法利赛人的子孙。我现在受审问,是为盼望死人复活。

说了这话,法利赛人和撒都该人,就争论起来,会众分为两党。

因为撒都该人说,没有复活,也没有天使,和鬼魂,法利赛人却说,两样都有。

于是大大的喧囔起来。有几个法利赛党的文士站起来,争辩说,我们看不出这人有什么恶处,倘若有鬼魂,或是天使,对他说过话,怎吗样呢。

那时大起争吵,千夫长恐怕保罗被他们扯碎了,就吩咐兵丁下去,把他从众人当中抢出来,带进营楼去。

当夜,主站在保罗旁边说,放心吧,你怎样在耶路撒冷为我作见证,也必怎样在罗马为我作见证。

到了天亮,犹太人同谋起誓,说,若不先杀保罗,就不吃不喝。

这样同心起誓的,有四十多人。

他们来见祭司长和长老说,我们已经起了一个大誓,若不先杀保罗,就不吃什么。

现在你们和公会要知会千夫长,叫他带下保罗到你们这里来,假作要详细察考他的事。我们已经豫备好了,不等他来到跟前就杀他。

保罗的外甥,听见他们设下埋伏,就来到营楼里告诉保罗。

保罗请一个百夫长来,说,你领这少年人去见千夫长,他有事告诉他。

于是把他领去见千夫长说,被囚的保罗请我到他那里,求我领这少年人来见你。他有事告诉你。

千夫长就拉着他的手,走到一旁,私下问他说,你有什么事告诉我呢。

他说,犹太人已经约定,要求你明天带下保罗到公会里去,假作要详细查问他的事。

你切不要随从他们,因为他们有四十多人埋伏,已经起誓,说,若不先杀保罗,就不吃不喝。现在豫备好了,只等你应允。

于是千夫长打发少年人走,嘱咐他说,不要告诉人你将这事报告我了。

千夫长便叫了两个百夫长来,说,豫备步兵二百,马兵七十,长枪手二百,今夜亥初往凯撒利亚去。

也要豫备牲口叫保罗骑上,护送到巡抚腓力斯那里去。

千夫长又写了文书,

大略说,革老丢吕西亚,请巡抚腓力斯大人安。

这人被犹太人拿住,将要杀害,我得知他是罗马人,就带兵丁下去救他出来。

因要知道他们告他的缘故,我就带他下到他们的公会去。

便查知他被告,是因他们律法的辩论,并没有什么该死该绑的罪名。

后来有人把要害他的计谋告诉我,我就立时解他到你那里去,又吩咐告他的人,在你面前告他。(有古卷在此有愿你平安)

于是兵丁照所吩咐他们的,将保罗夜里带到安提帕底。

第二天,让马兵护送,他们就回营楼去。

马兵来到凯撒利亚,把文书呈给巡抚,便叫保罗站在他面前。

巡抚看了文书,问保罗是那省的人,既晓得他是基利家人,

就说,等告你的人来到,我要细听你的事,便吩咐人把他看守在希律的衙门里。

\chapter{使徒行传第24章}
过了五天,大祭司亚拿尼亚,同几个长老,和一个辩士帖土罗,下来,向巡抚控告保罗。

保罗被提了来,帖土罗就告他说,

腓利斯大人,我们因你得以大享太平,并且这一国的弊病,因着你的先见,得以更正了。我们随时随地,满心感谢不尽。

惟恐多说,你嫌烦絮,只求你宽容听我们说几句话。

我们看这个人,如同瘟疫一般,是鼓动普天下众犹太人生乱的,又是拿撒勒教党里的一个头目。

连圣殿他也想要污秽。我们把他捉住了。(有古卷在此有要按我们的律法审问。

不料千夫长吕西亚前来甚是强横从我们手中把他夺去,吩咐告他的人到你这里来)。

你自己究问他,就可以知道我们告他的一切事了。

众犹太人也随着告他说,事情诚然是这样。

巡抚点头叫保罗说话,他就说,我知道你在这国里断事多年,所以我乐意为自己分诉。

你查问就可以知道,从我上耶路撒冷礼拜,到今日,不过有十二天。

他们并没有看见我在殿里,或是在会堂里,或是在城里,和人辩论,耸动众人。

他们现在所告我的事,并不能对你证实了。

但有一件事,我向你承认,就是他们所称为异端的道,我正按着那道事奉我祖宗的神,又信合乎律法的,和先知书上一切所记载的。

并且靠着神,盼望死人,无论善恶,都要复活,就是他们自己也有这个盼望。

我因此自己勉励,对神,对人,常存无亏的良心。

过了几年,我带着周济本国的捐项和供献的物上去。

正献的时候,他们看见我在殿里已经洁净了,并没有聚众,也没有吵囔。惟有几个从亚细亚来的犹太人。

他们若有告我的事,就应当到你面前来告我。

既或不然,这些人,若看出我站在公会前,有妄为的地方,他们自己也可以说明。

纵然有,也不过一句话,就是我站在他们中间大声说,我今日在你们面前受审,是为死人复活的道理

腓力斯本是详细晓得这道,就支吾他们说,且等千夫长吕西亚下来,我要审断你们的事。

于是吩咐百夫长看守保罗并且宽待他,也不拦阻他的亲友来供给他。

过了几天,腓力斯和他夫人犹太的女子土西拉,一同来到,就叫了保罗来,听他讲论信基督耶稣的道。

保罗讲论公义,节制,和将来的审判,腓力斯甚觉恐惧,说,你暂且去吧,等我得便再叫你来。

腓力斯又指望保罗送他银钱,所以屡次叫他来,和他谈论。

过了两年,波求非斯都接了腓力斯的任,腓力斯要讨犹太人的喜欢,就留保罗在监里。

\chapter{使徒行传第25章}
非斯都到了任,过了三天,就从凯撒利亚上耶路撒冷去。

祭司长,和犹太人的首领,向他控告保罗,

又央告他,求他的情,将保罗提到耶路撒冷来。他们要在路上埋伏杀害他。

非斯都却回答说,保罗押在凯撒利亚,我自己快要往那里去。

又说,你们中间有权势的人,与我一同下去,那人若有什么不是,就可以告他。

非斯都在他们那里,住了不过十天八天,就下凯撒利亚去。第二天坐堂,吩咐将保罗提上来。

保罗来了,那些从耶路撒冷下来的犹太人,周围站着,将许多重大的事控告他,都是不能证实的。

保罗分诉说,无论犹太人的律法,或是圣殿,或是凯撒,我都没有干犯。

但非斯都要讨犹太人的喜欢,就问保罗说,你愿意上耶路撒冷去,在那里听我审断这事吗。

保罗说,我站在凯撒的堂前,这就是我应当受审的地方。我向犹太人并没有行过什么不义的事,这也是你明明知道的。

我若行了不义的事,犯了什么该死的罪,就是死,我也不辞。他们所告我的事若都不实,就没有人可以把我交给他们。我要上告于凯撒。

非斯都和议会商量了,就说,你既上告于凯撒,可以往凯撒那里去。

过了些日子,亚基帕王,和百尼基氏,来到凯撒利亚,问非斯都安。

在那里住了多日,非斯都将保罗的事告诉王,说,这里有一个人,是腓力斯留在监里的。

我在耶路撒冷的时候,祭司长和犹太的长老,将他的事禀报了我,求我定他的罪。

我对他们说,无论什么人,被告还没有和原告对质,未得机会分诉所告他的事,就先定他的罪,这不是罗马人的条例。

及至他们都来到这里,我就不耽延,第二天便坐堂,吩咐把那人提上来。

告他的人站着告他。所告的,并没有我所逆料的那等恶事。

不过是有几样辩论,为他们自己敬鬼神的事,又为一个人名叫耶稣,是已经死了,保罗却说他是活着的。

这些事当怎样究问,我心里作难。我以问他说,你愿意上耶路撒冷去,在那里为这些事听审吗。

但保罗求我留下,他要听皇上审断,我就吩咐把他留下,等我解他到凯撒那里去。

亚基帕对非斯都说,我自己也愿听这人辩论。非斯都说,明天你可以听。

第二天,亚基帕和百尼基大张威势而来,同着众千夫长,和城里的尊贵人,进了公厅。非斯都吩咐一声,就有人将保罗带进来。

非斯都说,亚基帕王,和在这里的诸位阿,你们看这人,就是一切犹太人在耶路撒冷,和这里,曾向我恳求,呼叫说,不可容他再活着。

但我查明他没有犯什么该死的罪。并且他自己上告于皇帝,所以我定意把他解去

论到这人,我没有确实的事,可以奏明主上。因此我带他到你们面前,也特意带他到你亚基帕王面前,为要在查问之后,有所陈奏。

据我看来,解送囚犯,不指明他的罪案,是不合理的。

\chapter{使徒行传第26章}
亚基帕对抱罗说,准你为自己辩明。

于是保罗挽手分诉说,亚基帕王阿,犹太人所告我的一切事,今日得在你面前分诉,实在万幸。

更可幸的,是你熟悉犹太人的规矩,和他们的辩论。所以求你耐心听我。

我从起初在本国的民中,并在耶路撒冷,自幼为人如何,犹太人都知道。

他们若肯作见证,就晓得我从起初,是按着我们教中最严紧的教门,作了法利赛人。

现在我站在这里受审,是因为指望神向我们祖宗所应许的。

这应许,我们十二个支派昼夜切切的事奉神,都指望得着。王阿,我被犹太人控告,就是因这指望。

神叫死人复活,你们为什么看作不可信的呢。

从前我自己以为应当多方攻击拿撒勒人耶稣的名。

我在耶路撒冷也曾这样行了。既从祭司长得了权柄,我就把许多圣徒囚在监里。他们被杀,我也出名定案。

在各会堂,我屡次用刑,强逼他们说亵渎的话。又分外恼恨他们,甚至追逼他们直到外邦的城邑。

那时,我领了祭司长的权柄和命令,往大马士革去。

王阿,我在路上,晌午的时候,看见从天发光,比日头还亮,四面照着我,并与我同行的人。

我们都仆倒在地,我就听见有声音,用希伯来话,向我说,扫罗,扫罗,为什么逼迫我。你用脚踢刺是难的。

我说,主阿,你是谁。主说,我就是你所逼迫的耶稣。

你起来站着,我特意向你显现,要派你作执事作见证,将你所看见的事,和我将要指示你的事,证明出来。

我也要救你脱离百姓和外邦人的手。

我差你到他们那里去,要叫他们的眼睛得开,从黑暗中归向光明,从撒但权下归向神。又因信我,得蒙赦罪,和一切成圣的人同得基业。

亚基帕王阿,我故此没有违背那从天上来的异象。

先在大马士革,后在耶路撒冷,和犹太全地,以及外邦,劝勉他们应当悔改归向神,行事与悔改的心相称。

因此,犹太人在殿里拿住我,想要杀我。

然而我蒙神的帮助,直到今日还站得住,对着尊贵卑贱老幼作见证。所讲的,并不外乎众先知和摩西所说,将来必成的事。

就是基督必须受害,并且因从死里复活,要首先把光明的道,传给百姓和外邦人。

保罗这样分诉,非斯都大声说,保罗,你癫狂了吧。你的学问太大,反叫你癫狂了。

保罗说,非斯都大人,我不是癫狂,我说的乃是真实明白话。

王也晓得这些事,所以我向王放胆直言,我深信这些事没有一件向王隐藏的。因都不是在背地里作的。

亚基帕王阿,你信先知吗,我知道你是信的。

亚基帕对保罗说,你想少微一劝,便叫我作基督徒阿。(或作你这样劝我几乎叫我作基督徒了)。

保罗说,无论是少劝,是多劝,我向神所求的,不但你一个人,就是今天一切听我的,都要像我一样,只是不要像我有这些锁链。

于是王,和巡抚,并百尼基,与同坐的人,都起来,

退到里面,彼此谈论说,这人并没有犯什么该死该绑的罪。

亚基帕又对非斯都说,这人若没有上告于凯撒,就可以释放了。

\chapter{使徒行传第27章}
非斯都既然定规了,叫我们坐船往意大利去,便将保罗,和别的囚犯,交给卿营里的一个百夫长,名叫犹流。

有一只亚大米田的船,要沿着亚细亚一带地方的海边走,我们就上了船开行,有马其顿的帖撒罗尼迦人,亚里达古,和我们同去。

第二天,到了西顿。犹流宽待保罗,准他往朋友那里去,受他们的照应。

从那里又开船,因为风不顺,就贴着塞浦路斯背风岸行去。

过了基利家旁非利亚前面的海,就到吕家的每拉。

在那里百夫长遇见一只亚力山太的船,要往意大利去,便叫我们上了那船。

一连多日,船行得慢,仅仅来到革尼土的对面。因为被风拦阻,就贴着克里特背风岸,从撒摩尼对面行过。

我们沿岸行走,仅仅来到一个地方,名叫佳澳。离那里不远,有拉西亚城。

走的日子多了,已经过了禁食的节期,行船又危险,保罗就劝众人说,

众位,我看这次行船,不但货物和船要受伤损,大遭破坏,连我们的性命也难保。

但百夫长信从掌船的和船主,不信从保罗所说的。

且因在这海口过冬不便,船上的人,就多半说,不如开船离开这地方,或者能到非尼基过冬。非尼基是克里特的一个海口,一面朝东北,一面朝东南。

这时微微起了南风,他们以为得意,就起了锚,贴近克里特行去。

不多几时,狂风从岛上扑下来,那风名叫友拉革罗。

船被风抓住,敌不住风,我们就任风刮去。

贴着一个小岛的背风岸奔行,那岛名叫高大,在那里仅仅收住了小船。

既然把小船拉上来,就用缆索捆绑船底。又恐怕在赛耳底沙滩上搁了浅,就落下篷来,任船飘去。

我们被风浪逼得甚急,第二天众人就把货物抛在海里。

到第三天,他们又亲手把船上的器具抛弃了。

太阳和星辰多日不显露,又有狂风大浪催逼,我们得救的指望就都绝了。

众人多日没有吃什么,保罗就出来站在他们中间说,众位,你们本该听我的话,不离开克里特,免得遭这样的伤损破坏。

现在我还劝你们放心。你们的性命,一个也不失丧,惟独失丧这船。

因我所属所事奉的神,他的使者昨夜站在我旁边说,

保罗,不要害怕,你必定站在凯撒面前。并且与你同船的人,神都赐给你了。

所以众位可以放心,我信)神,他怎样对我说,事情也要怎样成就。

只是我们必要撞在一个岛上。

到了第十四天夜间,船在亚得里亚海,飘来飘去,约到半夜,水手以为渐近旱地,

就探深浅,探得有十二丈,稍往前行,又探深浅,探得有九丈。

恐怕撞在石头上,就从船尾抛下四个锚,盼望天亮。

水手想要逃出船去,把小船放在海里,假作要从船头抛锚的样子。

保罗对百夫长和兵丁说,这些人若不等在船上,你们必不能得救。

于是兵丁砍断小船的绳子,由他飘去。

天渐亮的时候保罗劝众人都吃饭,说,你们悬望忍饿不吃什么,已经十四天了。

所以我劝你们吃饭,这是关乎你们救命的事。因为你们各人连一根头发,也不至损坏。

保罗说了这话,就拿着饼,在众人面前祝谢了神,擘开吃。

于是他们都放下心,也就吃了。

我们在船上的,共有二百七十六个人

他们吃饱了,就把船上的麦子,抛在海里,为要叫船轻一点。

到了天亮,他们不认识那地方,但见一个海湾,有岸可登,就商议能把船拢进去不能。

于是砍断缆索,弃锚在海里,同时也松开舵绳,拉起头篷,顺着风向岸行去。

但遇着两水夹流的地方,就把船搁了浅。船头胶住不动,船尾被浪的猛力冲坏。

兵丁的意思,要把囚犯杀了,恐怕有??水脱逃的。

但百夫长要救保罗,不准他们任意而行,就吩咐会??水的,跳下水先上岸。

其馀的人,可以用板子,或船上的零碎东西上岸。这样众人都得了救上了岸。

\chapter{使徒行传第28章}
我们既已得救,才知道那岛名叫马耳他。

土人看待我们,有非常的情分,因为当时下雨,天气又冷,就生火,接待我们众人。

那时保罗拾起一捆柴,放在火上,有一条毒蛇,因为热了出来,咬住他的手。

土人看见那毒蛇,悬在他手上,就彼此说,这人必是个凶手,虽然从海里救上来,天理还不容他活着。

保罗竟把那蛇,甩在火里,并没有受伤。

土人想他必要肿起来,或是忽然仆倒死了。看了多时,见他无害,就转念说,他是个神。

离那地方不远,有田产是岛长部百流的。他接纳我们,尽情款待三日。

当时,部百流的父亲,患热病和痢疾躺着。保罗进去,为他祷告,按手在他身上,治好了他。

从此,岛上其馀的病人,也来得了医治。

他们又多方的尊敬我们。到了开船的时候,也把我们所需用的送到船上。

过了三个月,我们上了亚力山太的船,往前行。这船以宙斯双子为记,是在海岛过了冬的。

到了叙拉古,我们停泊三日。

又从那里绕行,来到利基翁。过了一天,起了南风,第二天就来到部丢利。

在那里遇见弟兄们,请我们与他们同住七天。这样我们来到罗马。

那里的弟兄们,一听见我们的信息,就出来到亚比乌市,和三馆地方迎接我们。保罗见了他们,就感谢神,放心壮胆。

进了罗马城,(有古卷在此有百夫长把众囚犯交给御营的统领惟有)保罗蒙准,和一个看守他的兵,另住在一处。

过了三天,保罗请犹太人的首领来。他们来了,就对他们说,弟兄们,我虽没有作什么事干犯本国的百姓,和我们祖宗的规条,却被锁绑,从耶路撒冷解在罗马人的手里。

他们审问了我,就愿意释放我。因为在我身上,并没有该死的罪。

无奈犹太人不服,我不得己,只好上告于凯撒。并没有什么事,要控告我本国的百姓。

因此,我请你们来见面说话。我原为以色列人所指望的,被这链子捆锁。

他们说,我们并没有接着从犹太来论你的信,也没有弟兄到这里来,报信给我们说,你有什么不好处。

但我们愿意听你的意见如何。因为这教门,我们晓得是到处被毁谤的。

他们和保罗约定了日子,就有许多人到他的寓处来,保罗从早到晚,对他们讲论这事,证明神国的道,引摩西的律法和先知的书,以耶稣的事,劝勉他们。

他所说的话,有信的,有不信的。

他们彼此不合,就散了。未散以先,保罗说了一句话,说,圣灵藉先知以赛亚,向你们祖宗所说的话,是不错的。

他说,你去告诉这百姓说,你们听是要听见,却不明白。看是要看见,却不晓得。

因为这百姓,油蒙了心,耳朵发沉,眼睛闭着。恐怕眼睛看见,耳朵听见,心里明白,回转过来,我就医治他们。

所以你们当知道,神这救恩,如今传给外邦人,他们也必听受。(有古卷在此有,

保罗说了这话犹太人议论纷纷的就走了)。

保罗在自己所租的房子里,住了足足两年。凡来见他的人,他全都接待,

放胆传讲神国的道,将耶稣基督的事教导人,并没有人禁止。

\chapter{罗马书第1章}
耶稣基督的仆人保罗,奉召为使徒,特派传神的福音。

这福音是神从前藉众先知,在圣经上所应许的。

论到他儿子,我主耶稣基督。按肉体说,是从大卫后裔生的。

按圣善的灵说,因从死里复活,以大能显明是神的儿子。

我们从他受了恩惠,并使徒的职分,在万国之中叫人为他的名信服真道。

其中也有你们这蒙召属耶稣基督的人。

我写信给你们在罗马为神所爱,奉召作圣徒的众人。愿恩惠平安,从我们的父神,并主耶稣基督,归与你们。

第一,我靠着耶稣基督,为你们众人感谢我的神。因你们的信德传遍了天下。

我在他儿子福音上,用心灵所事奉的神,可以见证我怎样不住的题到你们,

在祷告之间,常常恳求,或者照神的旨意,终能得平坦的道路往你们那里去。

因为我切切的想见你们,要把些属灵的恩赐分给你们,使你们可以坚固。

这样我在你们中间,因你与我彼此的信心,就可以同得安慰。

弟兄们,我不愿意你们不知道,我屡次定意往你们那里去,要在你们中间得些果子,如同在其馀的外邦人中一样。只是到如今仍有阻隔。

无论是希腊人,化外人,聪明人,愚拙人,我都欠他们的债。

所以情愿尽我的力量,将福音也传给你们在罗马的人。

我不以福音为耻。这福音本是神的大能,要救一切相信的,先是犹太人,后是希腊人。

因为神的义,正在这福音上显明出来。这义是本于信以致于信。如经上所记,义人必因信得生。

原来神的忿怒,从天上显明在一切不虔不义的人身上,就是那些行不义阻挡真理的人。

神的事情,人所能知道的,原显明在人心里。因为神已经给他们显明。

自从造天地以来,神的永能和神性是明明可知的,虽是眼不能见,但藉着所造之物,就可以晓得,叫人无可推诿。

因为他们虽然知道神,却不当作神荣耀他,也不感谢他。他们的思念变为虚妄,无知的心就昏暗了。

自称为聪明,反成了愚拙,

将不能朽坏之神的荣耀变为偶像,彷佛必朽坏的人,和飞禽走兽昆虫的样式。

所以神任凭他们,逞着心里的情欲行污秽的事,以致彼此玷辱自己的身体。

他们将神的真实变为虚谎,去敬拜事奉受造之物,不敬奉那造物的主。主乃是可称颂的,直到永远。阿们。

因此神任凭他们放纵可羞耻的情欲。他们的女人,把顺性的用处,变为逆性的用处。

男人也是如此,弃了女人顺性的用处,欲火攻心,彼此贪恋,男和男行可羞耻的事,就在自己身上受这妄为当得的报应。

他们既然故意不认识神,神就任凭他们存邪僻的心行那些不合理的事,

装满了各样不义,邪恶,贪婪,恶毒(或作阴毒)。满心是嫉妒,凶杀,争竞,诡诈,毒恨。

又是谗毁的,背后说人的,怨恨神的(或作被神所憎恶的),侮慢人的,狂傲的,自夸的,捏造恶事的,违背父母的,

无知的,背约的,无亲情的,不怜悯人的。

他们虽知道神判定,行这样事的人是当死的,然而他们不但自己去行,还喜欢别人去行。

\chapter{罗马书第2章}
你这论断人的,无论你是谁,也无可推诿,你在什么事上论断人,就在什么事上定自己的罪。因你这论断人的,自己所行却和别人一样。

我们知道这样行的人,神必照真理审判他。

你这人哪,你论断行这事的人,自己所行的却和别人一样,你以为能逃脱神的审判吗。

还是你藐视他丰富的恩慈,宽容,忍耐,不晓得他的恩慈是领你悔改呢。

你竟任着你刚硬不悔改的心,为自己积蓄忿怒,以致神震怒,显出他公义审判的日子来到。

他必照各人的行为报应各人。

凡恒心行善寻求荣耀尊贵,和不能朽坏之福的,就以永生报应他们。

惟有结党不顺从真理,反顺从不义的,就以忿怒恼恨报应他们。

将患难,困苦,加给一切作恶的人,先是犹太人,后是希腊人。

却将荣耀,尊贵,平安,加给一切行善的人,先是犹太人,后是希腊人。

因为神不偏待人。

凡没有律法犯了罪的,也必不按律法灭亡。凡在律法以下犯了罪的,也必按律法受审判,

(原来在神面前,不是听律法的为义,乃是行律法的称义。

没有律法的外邦人,若顺着本性行律法上的事,他们虽然没有律法,自己就是自己的律法。

这是显出律法的功用刻在他们心里,他们是非之心同作见证,并且他们的思念互相较量,或以为是,或以为非

就在神藉耶稣基督审判人隐秘事的日子,照着我的福音所言。

你称为犹太人,又倚靠律法,且指着神夸口。

既从律法中受了教训,就晓得神的旨意,也能分别是非(或作也喜爱那美好的事)

又深信自己是给瞎子领路的,是黑暗中人的光,

是蠢笨人的师傅,是小孩子的先生,在律法上有知识和真理的模范。

你既是教导别人,还不教导自己吗。你讲说人不可偷窃,自己还偷窃吗。

你说人不可奸淫,自己还奸淫吗。你厌恶偶像,自己还偷窃庙中之物吗。

你指着律法夸口,自己倒犯律法,玷辱神吗。

神的名在外邦人中,因你们受了亵渎,正如经上所记的。

你若是行律法的割礼固然于你有益。若是犯律法的,你的割礼就算不得割礼。

所以那未受割礼的,若遵守律法的条例,他虽然未受割礼,岂不算是有割礼吗。

然而那本来未受割礼的,若能全守律法,岂不是要审判你这有仪文和割礼竟犯律法的人吗。

因为外面作犹太人的,不是真犹太人,外面肉身的割礼,也不是真割礼。

惟有里面作的,才是真犹太人。真割礼也是心里的,在乎灵,不在乎仪文。这人的称赞不是从人来的,乃是从神来的。

\chapter{罗马书第3章}
这样说来,犹太人有什么长处,割礼有什么益处呢。

凡事大有好处。第一是神的圣言交托他们。

既便有不信的,这有何妨呢。难道他们的不信,就废掉神的信吗。

断乎不能。不如说,神是真实的,人都是虚谎的。如经上所记,你责备人的时候,显为公义。被人议论的时候,可以得胜。

我且照着人的常话说,我们的不义,若显出神的义来,我们可以怎吗说呢。神降怒,是他不义吗。

断乎不是。若是这样,神怎能审判世界呢。

若神的真实,因我的虚谎,越发显出他的荣耀,为什么我还受审判,好像罪人呢。

为什么不说,我们可以作恶以成善呢,这是毁谤我们的人,说我们有这话。这等人定罪,是该当的。

这却怎吗样呢。我们比他们强吗。决不是的。因为我们已经证明,犹太人和希腊人都在罪恶之下。

就如经上所记,没有义人,连一个也没有。

没有明白的,没有寻求神的。

都是偏离正路,一同变为无用。没有行善的,连一个也没有。

他们的喉咙是敞开的坟墓。他们用舌头弄诡诈。嘴唇里有虺蛇的毒气。

满口是咒骂苦毒。

杀人流血他们的脚飞跑。

所经过的路,便行残害暴虐的事。

平安的路,他们未曾知道。

他们眼中不怕神。

我们晓得律法上的话,都是对律法以下之人说的,好塞住各人的口,叫普世的人都伏在神的审判之下。

所以凡有血气的没有一个,因行律法,能在神面前称义。因为律法本是叫人知罪。

但如今神的义在律法以外已经显明出来,有律法和先知为证。

就是神的义,因信耶稣基督,加给一切相信的人,并没有分别。

因为世人都犯了罪,亏缺了神的荣耀。

如今却蒙神的恩典,因基督耶稣的救赎,就白白的称义。

神设立耶稣作挽回祭,是凭着耶稣的血,藉着人的信,要显明神的义。因为他用忍耐的心,宽容人先时所犯的罪。

好在今时显明他的义,使人知道他自己为义,也称信耶稣的人为义。

既是这样,那里能夸口呢。没有可夸的了。用何法没有的呢,是用立功之法吗。不是,乃用信主之法。

所以(有古卷作因为)我们看定了,人称义是因着信,不在乎遵行律法。

难道神只作犹太人的神吗。也不是作外邦人的神吗。是的,也作外邦人的神。

神既是一位他就要因信称那受割礼的为义,也要因信称那未受割礼的为义。

这样,我们因信废了律法吗。断乎不是,更是坚固律法。

\chapter{罗马书第4章}
如此说来,我们的祖宗亚伯拉罕,凭着肉体得了什么呢。

倘若亚伯拉罕是因行为称义,就有可夸的。只是在神面前并无可夸的。

经上说什么呢。说,亚伯拉罕信神,这就算为他的义。

作工的得工价,不算恩典,乃是该得的,

惟有不作工的,只信称罪人为义的神,他的信就算为义。

正如大卫称那在行为以外,蒙神算为义的人是有福的。

他说,得赦免其过,遮盖其罪的,这人是有福的。

主不算为有罪的,这人是有福的。

如此看来,这福音是单加给那受割礼的人吗。不也是加给那未受割礼的人吗。因我们所说,亚伯拉罕的信,就算为他的义。

是怎吗算的呢。是在他受割礼的时候呢。是在他未受割礼的时候呢。不是在受割礼的时候,乃是在未受割礼的时候。

并且他受了割礼的记号,作他未受割礼的时候因信称义的印证,叫他作一切未受割礼而信之人的父,使他们也算为义。

又作受割礼之人的父,就是那些不但受割礼,并且按我们的祖宗亚伯拉罕,未受割礼而信之踪迹去行的人。

因为神应许亚伯拉罕和他后裔,必得承受世界,不是因律法,乃是因信而得的义。

若是属乎律法的人,才得为后嗣,信就归于虚空,应许也就废弃了。

因为律法是惹动忿怒的。(或作叫人受刑的)那里没有律法,那里就没有过犯。

所以人得为后嗣是本乎信。因此就属乎恩。叫应许定然归给一切后裔。不但归给那属乎律法的,也归给那效法亚伯拉罕之信的。

亚伯拉罕所信的,是那叫死人复活使无变为有的神,他在主面前作我们世人的父。如经上所记,我已经立你作多国的父。

他在无可指望的时候,因信仍有指望,就得以作多国的父,正如先前所说,你的后裔将要如此。

他将近百岁的时候,虽然想到自己的身体如同已死,撒拉的生育已经断绝,他的信心还是不软弱。

并且仰望神的应许,总没有因不信,心里起疑惑。反倒因信,心里得坚固,将荣耀归给神。

且满心相信,神所应许的必能作成。

所以这就算为他的义。

算为他义的这句话,不是单为他写的,

也是为我们将来得算为义之人写的。就是我们这信神使我们的主耶稣从死里复活的人。

耶稣被交给人,是为我们的过犯,复活是为叫我们称义。(或作耶稣是为我们的过犯交付了是为我们称义复活了)

\chapter{罗马书第5章}
我们既因信称义,就藉着我们的主耶稣基督,得与神相和。

我们又藉着他,因信得进入现在所站的恩典中,并且欢欢喜喜盼望神的荣耀。

不但如此,就是在患难中,也是欢欢喜喜的。因为知道患难生忍耐。

忍耐生老练。老练生盼望。

盼望不至于羞耻,因为所赐给我们的圣灵将神的爱浇灌在我们心里。

因我们还软弱的时候,基督就按所定的日期为罪人死。

为义人死,是少有的,为仁人死,或者有敢作的。

惟有基督在我们还作罪人的时候为我们死,神的爱就在此向我们显明了。

现在我们既靠着他的血称义,就更要藉着他免去神的忿怒。

因为我们作仇敌的时候,且藉着神儿子的死,得与神和好,既已和好,就更要因他的生得救了。

不但如此,我们既藉着我主耶稣基督,得与神和好,也就藉着他,以神为乐。

这就如罪是从一人入了世界,死又是从罪来的,于是死就临到众人,因为众人都犯了罪。

没有律法之先,罪已经在世上。但没有律法,罪也不算罪。

然而从亚当到摩西,死就作了王,连那些不与亚当犯一样罪过的,也在他的权下。亚当乃是那以后要来之人的豫像。

只是过犯不如恩赐。若因一人的过犯,众人都死了,何况神的恩典,与那因耶稣基督一人恩典中的赏赐,岂不更加倍的临到众人吗。

因一人犯罪就定罪,也不如恩赐。原来审判是由一人而定罪,恩赐乃是由许多过犯而称义。

若因一人的过犯,死就因这一人作了王,何况那些受洪恩又蒙所赐之义的,岂不更要因耶稣基督一人在生命中作王吗。

如此说来,因一次的过犯,众人都被定罪,照样,因一次的义行,众人也就被称义得生命了。

因一人的悖逆,众人成为罪人,照样,因一人的顺从,众人也成为义了。

律法本是外添的,叫过犯显多。只是罪在那里显多,恩典就更显多了。

就如罪作王叫人死,照样恩典也藉着义作王,叫人因我们的主耶稣基督得永生。

\chapter{罗马书第6章}
这样,怎吗说呢。我们可以仍在罪中,叫恩典显多吗。

断乎不可。我们在罪上死了的人,岂可仍在罪中活着呢。

岂不知我们这受洗归入基督耶稣的人,是受洗归入他的死吗。

所已,我们藉着洗礼归入死,和他一同埋葬,原是叫我们一举一动有新生的样式,像基督藉着父的荣耀从死里复活一样。

我们若在他死的形状上与他联合,也要在他复活的形状上与他联合。

因为知道我们的旧人和他同定十字架,使罪身灭绝,叫我们不再作罪的奴仆。

因为已死的人,是脱离了罪。

我们若是与基督同死,就信必与他同活。

因为知道基督既从死里复活,就不再死,死也不再作他的主了。

他死是向罪死了,只有一次。他活是向神活着。

这样,你们向罪也当看自己是死的。向神在基督耶稣里却当看自己是活的。

所以不要容罪在你们必死的身上作王,使你们顺从身子的私欲。

也不要将你们的肢体献给罪作不义的器具。倒要像从死里复活的人,将自己献给神。并将肢体作义的器具献给神。

罪必不能作你们的主。因为你们不在律法之下,乃在恩典之下。

这却怎吗样呢。我们在恩典之下,不在律法之下,就可以犯罪吗。断乎不可。

岂不晓得你们献上自己作奴仆,顺从谁,就作谁的奴仆吗。或作罪的奴仆,以至于死。或作顺命的奴仆,以至成义。

感谢神,因为你们从前虽然作罪的奴仆,现今却从心里顺服了所传给你们道理的模范。

你们既从罪里得了释放,就作了义的奴仆。

我因你们肉体的软弱,就照人的常话对你们说,你们从前怎样将肢体献给不洁不法作奴仆,以至于不法。现今也要照样将肢体献给义作奴仆,以至于成圣。

因为你们作罪的奴仆的时候,就不被义约束了。

你们现今所看为羞耻的事,当日有什么果子呢。那些事的结局就是死。

但现今你们既从罪里得了释放,作了神的奴仆,就有成圣的果子,那结局就是永生。

因为罪的工价乃是死。惟有神的恩赐,在我们的主基督耶稣里乃是永生。

\chapter{罗马书第7章}
弟兄们,我现在对明白律法的人说,你们岂不晓得律法管人是在活着的时候吗。

就如女人有了丈夫,丈夫还活着,就被律法约束。丈夫若死了,就脱离了丈夫的律法。

所以丈夫活着,她若归于别人,便叫淫妇。丈夫若死了,她就脱离了丈夫的律法,虽然归于别人,也不是淫妇。

我的弟兄们,这样说来,你们藉着基督的身体,在律法上也是死了。叫你们归于别人,就是归于那从死里复活的,叫我们结果子给神。

因为我们属肉体的时候,那因律法而生的恶欲,就在我们肢体中发动,以致结成死亡的果子。

但我们既然在捆我们的律法上死了,现今就脱离了律法,叫我们服事主,要按着心灵(心灵或作圣灵)的新样,不按着仪文的旧样。

这样,我们可说什么呢。律法是罪吗。断乎不是。只是非因律法,我就不知何为罪。非律法说,不可起贪心。我就不知何为贪心。

然而罪趁着机会,就藉着诫命叫诸般的贪心在我里头发动。因为没有律法罪是死的。

我以前没有律法是活着的,但是诫命来到,罪又活了,我就死了。

那本来叫人活的诫命,反倒叫我死。

因为罪趁着机会,就藉着诫命引诱我,并且杀了我。

这样看来,律法是圣洁的,诫命也是圣洁,公义,良善的。

既然如此,那良善的是叫我死吗。断乎不是。叫我死的乃是罪。但罪藉着那良善的叫我死,就显出真是罪。叫罪因着诫命更显出是恶极了。

我们原晓得律法是属乎灵的,但我是属乎肉体的,是已经卖给罪了。

因为我所作的,我自己不明白。我所愿意的,我并不作。我所恨恶的,我倒去作。

若我所作的,是我所不愿意的,我就应承律法是善的。

既是这样,就不是我作的,乃是住在我里头的罪作的。

我也知道,在我里头,就是我肉体之中,没有良善。因为立志为善由得我,只是行出来由不得我。

故此,我所愿意的善,我反不作。我所不愿意的恶,我倒去作。

若我去作所不愿意作的,就不是我作的,乃是住在我里头的罪作的。

我觉得有个律,就是我愿意为善的时候,便有恶与我同在。

因为按着我里面的意思。(原文作人)我是喜欢神的律。

但我觉得肢体中另有个律,和我心中的律交战,把我掳去叫我附从那肢体中犯罪的律。

我真是苦阿,谁能救我脱离这取死的身体呢。

感谢神,靠着我们的主耶稣基督就能脱离了这样看来,我以内心顺服神的律。我肉身却顺服罪的律了。

\chapter{罗马书第8章}
如今那些在基督耶稣里的,就不定罪了。

因为赐生命的圣灵的律,在基督耶稣里释放了我,使我脱离罪和死的律了。

律法既因肉体软弱,有所不能行的,神就差遣自己的儿子,成为罪身的形状,罗08:03)作了赎罪祭,在肉体中定了罪案,

使律法的义成就在我们这不随从肉体,只随从圣灵的人身上。

因为随从肉体的人体贴肉体的事,随从圣灵的人体贴圣灵的事。

体贴肉体的,就是死,体贴圣灵的,乃是生命,平安。

原来体贴肉体的,就是与神为仇。因为不服神的律法,也是不能服。

而且属肉体的人,不能得神的喜欢。

如果神的灵住在你们心里,你们就不属肉体,乃属圣灵了。人若没有基督的灵,就不是属基督的。

基督若在你们心里,身体就因罪而死,心灵却因义而活。

然而叫耶稣从死里复活者的灵,若住在你们心里,那叫基督耶稣从死里复活的,也必藉着住在你们心里的圣灵,使你们必死的身体又活过来。

弟兄们,这样看来,我们并不是欠肉体的债,去顺从肉体活着。

你们若顺从肉体活着必要死。若靠着圣灵治死身体的恶行必要活着。

因为凡被神的灵引导的,都是神的儿子。

你们所受的不是奴仆的心,仍旧害怕。所受的乃是儿子的心,因此我们呼叫阿爸,父。

圣灵与我们的心同证我们是神的儿女。

既是儿女,便是后嗣,就是神的后嗣,和基督同作后嗣。如果我们和他一同受苦,也必和他一同得荣耀。

我想现在的苦楚,若比将来要显于我们的荣耀,就不足介意了。

受造之物,切望等候神的众子显出来。

因为受造之物服在虚空之下,不是自己愿意,乃是因那叫他如此的。

但受造之物仍然指望脱离败坏的辖制,得享神儿女自由的荣耀。(享原文作入)

我们知道一切受造之物,一同叹息劳苦,直到如今。

不但如此,就是我们这有圣灵初结果子的,也是自己心里叹息,等候得着儿子的名分,乃是我们的身体得赎。

我们得救是在乎盼望。只是所见的盼望不是盼望。谁还盼望他所见的呢。(有古卷作人所看见的何必再盼望呢)

但我们若盼望那所不见的,就必忍耐等候。

况且我们的软弱有圣灵帮助,我们本不晓得当怎样祷告,只是圣灵亲自用说不出来的叹息,替我们祷告。

鉴查人心的,晓得圣灵的意思因为圣灵照着神的旨意替圣徒祈求。

我们晓得万事都互相效力,叫爱神的人得益处,就是按他旨意被召的人。

因为他豫先所知道的人,就豫先定下效法他儿子的模样,使他儿子在许多弟兄中作长子。

豫先所定下的人又召他们来。所召来的人,又称他们为义。所称为义的人,又叫他们得荣耀。

既是这样,还有什么说的呢。神若帮助我们,谁能抵挡我们呢。

神既不爱惜自己的儿子为我们众人舍了,岂不也把万物和他一同白白的赐给我们吗。

谁能控告神所拣选的人呢。有神称他们为义了。(或作是称他们为义的神吗)

谁能定他们的罪呢。有基督耶稣已经死了,而且从死里复活,现今在神的右边,也替我们祈求。(有基督云云或作是已经死了而且从死里复活现今在神的右边也替我们祈求的基督耶稣吗)

谁能使我们与基督的爱隔绝呢。难道是患难吗,是困苦吗,是逼迫吗,是肌饿吗,是赤身露体吗,是危险吗,是刀剑吗。

如经上所记,我们为你的缘故,终日被杀。人看我们如将宰的羊

然而靠着我们的主,在这一切的事上,已经得胜有馀了。

因为我深信无论是生,是天使,是掌权的,是有能的,是现在的事,是将来的事,

是高处的,是低处的,是别的受造之物,都不能叫我们与神的爱隔绝。这爱是在我们的主基督里的。

\chapter{罗马书第9章}
我在基督里说真话,并不谎言,有我良心被圣灵感动,给我作见证。

我是大有忧愁,心里时常伤痛。

为我弟兄,我骨肉之亲,就是自己被咒诅,与基督分离,我也愿意。

他们是以色列人。那儿子的名分,荣耀,诸约,律法,礼仪,应许,都是他们的。

列祖就是他们的祖宗,按肉体说,基督也是从他们出来的,他是在万有之上,永远可称颂的神。阿们。

这不是说神的话落了空。因为从以色列生的,不都是以色列人。

也不因为是亚伯拉罕的后裔,就都作他的儿女。惟独从以撒生的,才要称为你的后裔。

这就是说,肉身所生的儿女,不是神的儿女。惟独那应许的儿女,才算是后裔。

因为所应许的话是这样说,到明年这时候我要来,撒拉必生一个儿子。

不但如此,还有利百加,既从一个人,就是从我们的祖宗以撒怀了孕。

(双子还没生下来,善恶还没有作出来,只因要显明神拣选人的旨意,不在乎人的行为,乃在乎召人的主)。

神就对利百加说,将来大的要服事小的。

正如经上所记,雅各是我所爱的,以扫是我所恶的。

这样,我们可说什么呢。难道神有什么不公平吗。断乎没有。

因他对摩西说,我要怜悯谁,就怜悯谁,要恩待谁,就恩待谁。

据此看来,这不在乎那定意的,也不在乎那奔跑的,只在乎发怜悯的神。

因为经上有话向法老说,我将你兴起来,特要在你身上彰显我的权能,并要使我的名传遍天下。

如此看来,神要怜悯谁,就怜悯谁,要叫谁刚硬,就叫谁刚硬。

这样,你必对我说,他为什么还指责人呢。有谁抗拒他的旨意呢。

你这个人哪,你是谁,竟敢向神强嘴呢。受造之物岂能对造他的说,你为什么这样造我呢。

陶匠难道没有权柄,从一团泥里拿一块作成贵重的器皿,又拿一块作成卑贱的器皿吗。

倘若神要显明他的忿怒,彰显他的权能,就多多忍耐宽容那可怒预备遭毁灭的器皿。

又要将他丰盛的荣耀,彰显在那蒙怜悯早豫备得荣耀的器皿上。

这器皿就是我们被神所召的,不但是从犹太人中,也是从外邦人中,这有什么不可呢。

就像神在何西阿书上说,那本来不是我子民的,我要称为我的子民。本来不是蒙爱的,我要称为蒙爱的。

从前在什么地方对他们说,你们不是我的子民,将来就在那里称他们为永生神的儿子。

以赛亚指着以色列人喊着说,以色列人虽多如海沙,得救的不过是剩下的馀数。

因为主要在世上施行他的话,叫他的话都成全,速速的完结。

又如以赛亚先前说过,若不是万军之主给我们存留馀种,我们早已像所多玛,蛾摩拉的样子了。

这样,我们可说什么呢。那本来不追求义的外邦人,反得了义,就是因信而得的义。

但以色列人追求律法的义,反得不着律法的义。

这是什么缘故呢。是因为他们不凭着信心求,只凭着行为求。他们正跌在那绊脚石上。

就如经上所记,我在锡安放一块绊脚的石头,跌人的磐石。信靠他的人必不至于羞愧。

\chapter{罗马书第10章}
弟兄们,我心里所愿的,向神所求的,是要以色列人得救。

我们可以证明他们向神有热心,但不是按着真知识。

因为不知道神的义,想要立自己的义,就不服神的义了。

律法的总结就是基督,使凡信他的都得着义。

摩西写着说,人若行那出于律法的义,就必因此活着。

惟有出于信心的义如此说,你不要心里说,谁要升到天上去呢。就是要领下基督来。

谁要下到阴间去呢。就是要领基督从死里上来。

他到底怎吗说呢。他说,这道离你不远,正在你口里,在你心里。就是我们所传信主的道。

你若口里认耶稣为主,心里信神叫他从死里复活,就必得救。

因为人心里相信,就可以称义。口里承认,就可以得救。

经上说,凡信他的人,必不至于羞愧。

犹太人和希腊人,并没有分别。因为众人同有一位主,他也厚待一切求告他的人。

因为凡求告主名的,就必得救。

然而人未曾信他,怎能求他呢。未曾听见他,怎能信他呢。没有传道的,怎能听见呢。

若没有奉差遣,怎能传道呢。如经上所记,报福音传喜信的人,他们的脚踪何等佳美,

只是人没有都听从福音。因为以赛亚说,主阿,我们所传的有谁信呢。

可见信道是从听道来的,听道是从基督的话来的。

但我说,人没有听见吗。诚然听见了。他们的声音传遍天下,他们的言语传到地极。

我再说,以色列人不知道吗。先有摩西说,我要用那不成子民的,惹动你们的愤恨。我要用那无知的民,触动你们的怒气。

又有以赛亚放胆说,没有寻梢我的,我叫他们遇见。没有访问我的,我向他们显现。

至于以色列人,他说,我整天伸手招呼那悖逆顶嘴的百姓。

\chapter{罗马书第11章}
我且说,神弃绝了他的百姓吗。断乎没有。因为我也是以色列人,亚伯拉罕的后裔,属便雅悯支派的。

神并每有弃绝他豫先所知道的百姓。你们岂不晓得经上论到以利亚是怎吗说的呢。他在神面前怎样控告以色列人,说,

主阿,他们杀了你的先知,拆了你的祭坛,只剩下我一个人,他们还要寻索我的命。

神的回话是怎吗说的呢。他说,我为自己留下七千人,是未曾向巴力屈膝的。

如今也是这样,照着拣选的恩典还有所留的馀数。

既是出于恩典,就不在乎行为。不然,恩典就不是恩典了。

这是怎吗样呢。以色列人所求的,他们没有得着。惟有蒙拣选的人得着了,其馀的就成了顽梗不化的。

如经上所记,神给他们昏迷的心,眼睛不能看见,耳朵不能听见,直到今日。

大卫也说,愿他们的筵席变为网罗,变为机槛,变为绊脚石,作他们的报应。

愿他们的眼睛昏蒙,不得看见。愿你时常弯下他们的腰。

我且说,他们失脚是要他们跌倒吗。断乎不是。反倒因他们的过失,救恩便临到外邦人,要激动他们发愤。

若他们的过失,为天下的富足,他们的缺乏,为外邦人的富足。何况他们的丰满呢。

我对你们外邦人说这话。因我是外邦人的使徒,所以敬重我的职分。(敬重原文作荣耀)

或者可已激动我骨肉之亲发愤,好救他们一些人。

若他们被丢弃,天下就得与神和好,他们被收纳,岂不是死而复生吗。

所献的新面,若是圣洁,全团也就圣洁了。树根若是圣洁,树枝也就圣洁了。

若有几根枝子被折下来,你这野橄榄得接在其中,一同得着橄榄根的肥汁。

你就不可向旧枝子夸口,若是夸口,当知道不是你托着根,乃是根托着你。

你若说,那枝子被折下来,是特为叫我接上。

不错。他们因为不信,所以被折下来。你因为信,所以立得住。你不可自高,反要惧怕。

神既不爱惜原来的枝子,也必不爱惜你。

可见神的恩慈,和严厉。向那跌倒的人,是严厉的。向你是有恩慈的,只要你长久在他的恩慈里。不然,你也要被砍下来。

而且他们若不是长久不信,仍要被接上。因为神能够把他们从新接上。

你是从那天生的野橄榄上砍下来的,尚且逆着性得接在好橄榄上,何况这本树的枝子,要接在本树上呢。

弟兄们,我不愿意你们不知道这奥秘,(恐怕你们自以为聪明)就是以色列人有几分是硬心的,等到外邦人的数目添满了。

于是以色列全家都要得救,如经上所记,必有一位救主,从锡安出来,要消除雅各家的一切罪恶。

又说,我除去他们罪的时候,这就是我与他们所立的约。

就着福音说,他们为你们的缘故是仇敌。就着拣选说,他们为列祖的缘故是蒙爱的。

因为神的恩赐和选召,是没有后悔的。

你们从前不顺服神,如今因他们的不顺服,你们倒蒙了怜恤。

这样,他们也是不顺服,叫他们因着施给你们的怜恤,现在也就蒙怜恤。

因为神将众人都圈在不顺服之中,特意要怜恤众人。

深哉,神丰富的智慧和知识。他的判断,何其难测,他的踪迹,何其难寻,

谁知道主的心,谁作过他的谋士呢,

谁是先给了他,使他后来偿还呢。

因为万有都是本于他,倚靠他,归于他。愿荣耀归给他,直到永远。阿们。

\chapter{罗马书第12章}
所以弟兄们,我以神的慈悲劝你们,将身体献上,当作活祭,是圣洁的,是神所喜悦的。你们如此事奉,乃是理所当然的。

不要效法这世界。只要心意更新而变化,叫你们察验何为神的善良,纯全可喜悦的旨意。

我凭着所赐我的恩,对你们各人说,不要看自己过于所当看的。要照着神所分给各人信心的大小,看得合乎中道。

正如我们一个身子上有好些肢体,肢体也不都是一样的用处。

我们这许多人,在基督里成为一身,互相联络作肢体,也是如此。

按我们所得的恩赐,各有不同。或说预言,就当照着信心的程度说预言

或作执事,就当专一执事。或作教导的,就当专一教导。

或作劝化的,就当专一劝化。施舍的就当诚实。治理的,就当殷勤。怜悯人的,就当甘心。

爱人不可虚假,恶要厌恶,善要亲近。

爱弟兄,要彼此亲热。恭敬人,要彼此推让。

殷勤不可懒惰。要心里火热。常常服事主。

在指望中要喜乐。在患难中要忍耐。祷告要恒切。

圣徒缺乏要帮补。客要一味的款待。

逼迫你们的,要给他们祝福。只要祝福,不可咒诅。

与喜乐的人要同乐。与哀哭的人要同哭。

要彼此同心。不要志气高大,倒要俯就卑微的人。(人或作事)不要自以为聪明。

不要以恶报恶,众人以为美的事,要留心去作。

若行,总要尽力与众人和睦。

亲串的弟兄,不要自己伸冤,宁可让步,听凭主怒。(或作让人发怒)因为经上记着,主说,伸冤在我。我必报应。

所以,你的仇敌若饿了,就给他吃。若渴了,就给他喝。因为陪这样行,就是把炭火堆在他的头上。

你不可为恶所胜,反要以善胜恶。

\chapter{罗马书第13章}
在上有权柄的,人人当顺服他。因为没有权柄不是出于神的。凡掌权的都是神所命的。

所以抗拒掌权的,就是抗拒神的命。抗拒的必自取刑罚

作官的原不是叫行善的惧怕,乃是叫作恶的惧怕。你愿意不惧怕掌权的吗。你只要行善,就可得他的称赞。

因为他是神的用人,是与你有益的。你若作恶,却当惧怕。因为他不是空空的佩剑。他是神的用人,是伸冤的,刑罚那作恶的。

所以你们必须顺服,不但是因为刑罚,也是因为良心。

你们纳粮,也为这缘故。因他们是神的差役,常常特管这事。

凡人所当得的,就给他。当得粮的,给他纳粮。当得税的,给他上税。当惧怕的,惧怕他。当恭敬的,恭敬他。

凡事都不可亏欠人,惟有彼此相爱,要常以为亏欠。因为爱人的就完全了律法。

像那不可奸淫,不可杀人,不可偷盗,不可贪婪,或有别的诫命,都包在爱人如己这一句话之内了。

爱是不加害与人的,所以爱就完全了律法。

再者,你们晓得现今就是该趁早睡醒的时候,因为我们得救,现今比初信的时候更近了。

黑夜已深,白昼将近。我们就当脱去暗昧的行为,带上光明的兵器。

行事为人要端正,好像行在白昼。不可荒宴醉酒。不可好色邪荡。不可争竞嫉妒。

总要披戴主耶稣基督,不要为肉体安排,去放纵私欲。

\chapter{罗马书第14章}
信心软弱的,你们要接纳,但不要辩论所疑惑的事。

有人信百物都可吃。但那软弱的,只吃蔬菜。

吃的人不可轻看不吃的人。不吃的人不可论断吃的人。因为神已经收纳他了。

你是谁,竟论断别人的仆人呢。他或站住,或跌倒,自有他的主人在。而且他也必站住。因为主能使他站住。

有人看这日比那日强,有人看日日都是一样。只是各人心里要意见坚定。

守日的人,是为主守的。吃的人,是为主吃的,因他感谢神。不吃的人,是为主不吃的,也感谢神。

我们没有一个人为自己活,也没有一个人为自己死。

我们若活着,是为主而活。若死了,是为主而死。所以我们或活或死,总是主的人。

因此基督死了,又活了,为要作死人并活人的主。

你这个人,为什么论断弟兄呢。又为什么轻看弟兄呢。因为我们都要站在神的台前。

经上写着,主说,我凭着我的永生起誓,万膝必向我跪拜,万口必向我承认。

这样看来,我们各人必要将自己的事,在神面前说明。

所以我们不可再彼此论断。宁可定意谁也不给弟兄放下绊脚人之物。

我凭着主耶稣确知深信,凡物本来没有不洁净的。惟独人以为不洁净的,在他就不洁净了。

你若因食物叫弟兄忧愁,就不是按着爱人的道理行。基督已经替他死,你不可因你的食物叫他败坏。

不可叫你的善被人毁谤。

因为神的国,不在乎吃喝,只在乎公义,和平,并圣灵中的喜乐。

在这几样上服事基督的,就为神所喜悦,又为人所称许。

所以我们务要追求和睦的事,与彼此建立德行的事。

不可因食物毁坏神的工程。凡物固然洁净,但有人因食物叫人跌倒,就是他的罪了。

无论是吃肉,是喝酒,是什么别的事,叫弟兄跌倒,一概不作才好。

你有信心,就当在神面前守着。人在自己以为可行的事上,能不自责,就有福了。

若有疑心而吃的,就必有罪。因为他吃,不是出于信心。凡不出于信心的都是罪。

\chapter{罗马书第15章}
我们坚固的人,应当担代不坚固人的软弱,不求自己的喜悦。

我们各人务要叫邻舍喜悦,使他得益处,建立德行

因为基督也不求自己的喜悦,如经上所记,辱骂你人的辱骂,都落在我身上。

从前所写的圣经都是为教训我们写的,叫我们因圣经所生的忍耐和安慰,可以得着盼望。

但愿赐忍耐安慰的神,叫你们彼此同心,效法基督耶稣。

一心一口,荣耀神,我们主耶稣基督的父。

所以你们要彼此接纳,如同基督接纳你们一样,使荣耀归与神。

我说,基督是为神真理作了受割礼人的执事,要证实所应许列祖的话。

并叫外邦人,因他的怜悯,荣耀神。如经上所记,因此我要在外邦中称赞你,歌颂你的名。

又说,你们外邦人,当与主的百姓一同欢乐。

又说,外邦阿,你们当赞美主。万民哪,你们都当颂赞他。

又有以赛亚说,将来有耶西的根,就是那兴起来要治理外邦的。外邦人要仰望他。

但愿使人有盼望的神,因信将诸般的喜乐平安,充满你们的心,使你们藉着圣灵的能力,大有盼望。

弟兄们,我自己也深信你们是满有良善,充足了诸般的知识,也能彼此劝戒。

但我稍微放胆写给你们,是要题醒你们的记性,特因神所给我的恩典,

使我为外邦人作基督耶稣的仆役,作神福音的祭司,叫所献上的外邦人,因着圣灵,成为圣洁,可蒙悦纳。

所以论到神的事我在基督耶稣里有可夸的。

除了基督藉我作的那些事,我什么都不敢题。只题他藉我言语作为,用神迹奇事的能力,并圣灵的能力,使外邦人顺服。

甚至我从耶路撒冷,直转到以利哩古,到处传了基督的福音。

我立了志向,不在基督的名被称过的地方传福音,免得建造在别人的根基上。

就如经上所记,未曾闻知他信息的,将要看见。未曾听过的,将要明白。

我因多次被拦阻,总不得到你们那里去。

但如今在这里再没有可传的地方,而且这好几年,我切心想望到西班牙去的时候,可以到你们那里。

盼望从你们那里经过,得见你们,先与你们彼此交往,心里稍微满足,然后蒙你们送行。

但现在我往耶路撒冷去,供给圣徒。

因为马其顿,和亚该亚人乐意凑出捐项,给耶路撒冷圣徒中的穷人。

这固然是他们乐意的。其实也算是所欠的债。因外邦人,既然在他们属灵的好处上有分,就当把养身之物供给他们

等我辨完了这事,把这善果向他们交付明白,我就要路过你们那里,往西班牙去。

我也晓得去的时候,必带着基督丰盛的恩典而去。

弟兄们,我藉着我们主耶稣基督,又藉着圣灵的爱,劝你们与我一同竭力,为我祈求神。

叫我脱离在犹太不顺从的人,也叫我为耶路撒冷所辨的捐项,可蒙圣徒悦纳。

并叫我顺着神的旨意,欢欢喜喜的到你们那里,与你们同得安息。

愿赐平安的神,常和你们众人同在,阿们。

\chapter{罗马书第16章}
我对你们举荐我们的姊妹非比,她是坚革哩教会中的女执事。

请你们为主接待她,合乎圣徒的体统。她在何事上,要你们帮助,你们就帮助她。因她素来帮助许多人,也帮助了我。

问百基拉和亚居拉安。他们在基督耶稣里与我同工,

也为我的命,将自己的颈项,置之度外。不但我感谢他们,就是外邦的众教会,也感谢他们。

又问在他们家中的教会安。问我所亲爱的以拜尼土安。他在亚细亚是归基督初结的果子。

又问马利亚安。她为你们多受劳苦。

又问我亲属与我一同坐监的安多尼古和犹尼亚安。他们在使徒中是有名望的,也是比我先在基督里。

又问我在主里面所亲爱的暗伯利安。

又问在基督里与我们同工的耳巴奴,并我所亲爱的士大古安。

又问在基督里经过试验的亚比利安。问亚利多布家里的人安。

又问我亲属希罗天安。问拿其数家在主里的人安。

又问为主劳苦的土非拿氏和土富撒氏安。问可亲爱为主多受劳苦的彼息氏安。

又问在主蒙拣选的鲁孚和他母亲安。他的母亲就是我的母亲。

又问亚逊其土,弗勒干,黑米,八罗巴,黑马,并与他们在一处的弟兄们安。

又问非罗罗古,和犹利亚,尼利亚,和他姊妹,同阿林巴,并与他们在一处的众圣徒安。

你们亲嘴问安,彼此务要圣洁。基督的众教会都问你们安。

弟兄们,那些离开你们,叫你们跌倒,背乎所学之道的人,我劝你们要留意躲避他们。

因为这样的人不服事我们的主基督,只服事自己的肚腹。用花言巧语,诱惑那些老实人的心。

你们的顺服,已经传于众人,所以我为你们欢喜。但我愿意你们在善事上聪明,在恶上愚拙。

赐平安的神,快要将撒但践踏在你们脚下。愿我主耶稣基督的恩,常和你们同在。

与我同工的提摩太,和我的亲属路求,耶孙,所西巴德,问你们安。

我这代笔写信的德丢,在主里面问你们安。

那接待我,也接待全教会的该犹,问你们安。

城内管银库的以拉都,和兄弟括土,问你们安。

惟有神能照我传的福音,和所讲的耶稣基督,并照永古隐藏不言的奥秘,坚固你们的心。

这奥秘如今显明出来,而且按着永生神的命,藉众先知的书指示万国的民,使他们信服真道。

愿荣耀因耶稣基督归与独一全智的神,直到永远。阿们。

\chapter{哥林多前书第1章}
奉神旨意,蒙召作耶稣基督使徒的保罗,同兄弟所提尼,

写信给在哥林多神的教会,就是在基督耶稣里成圣,蒙召作圣徒的,以及所有在各处求告我主耶稣基督之名的人。基督是他们的主,也是我们的主。

愿恩惠平安,从神我们的父,并主耶稣基督,归与你们。

我常为你们感谢我的神,因神在基督耶稣里所赐给你们的恩惠。

又因你们在他里面凡事富足,口才知识都全备。

正如我为基督作的见证,在你们心里得以坚固。

以致你们在恩赐上没有一样不及人的。等候我们的主耶稣基督显现。

他也必坚固你们到底,叫你们在我们主耶稣基督的日子,无可责备。

神是信实的,你们原是被他所召,好与他儿子,我们的主耶稣基督,一同得分。

弟兄们,我藉我们主耶稣基督的名,劝你们都说一样的话。你们中间也不可分党。只要一心一意彼此相合。

因为革来氏家里的人,曾对我题起弟兄们来,说你们中间有分争。

我的意思就是你们各人说,我是属保罗的。我是属亚波罗的。我是属矶法的。我是属基督的。

基督是分开的吗。保罗为你们钉了十字架吗。你们是奉保罗的名受了洗吗。

我感谢神,除了基列司布并该犹以外,我没有给你们一个人施洗。

免得有人说,你们是奉我的名受洗。

我也给司提反家施过洗。此外给别人施洗没有,我却记不清。

基督差遣我,原不是为施洗,乃是为传福音。并不用智慧的言语,免得基督的十字架落了空。

因为十字架的道理,在那灭亡的人为愚拙。在我们得救的人却为神的大能。

就如经上所记,我要灭绝智慧人的智慧,废弃聪明人的聪明。

智慧人在那里文士在那里。这世上的辩士在那里。神岂不是叫这世上的智慧变成愚拙吗。

世人凭自己的智慧,既不认识神,神就乐意用人所当作愚拙的道理,拯救那些信的人。这就是神的智慧了。

犹太人是要神迹,希腊人是求智慧。

我们却是传钉十字架的基督,在犹太人为绊脚石,在外邦人为愚拙。

但在那蒙召的无论是犹太人,希腊人,基督总为神的能力,神的智慧。

因神的愚拙总比人智慧。神的软弱总比人强壮。

弟兄们哪,可见你们蒙召的,按着肉体有智慧的不多,有能力的不多,有尊贵的也不多。

神却拣选了世上愚拙的,叫有智慧的羞愧。又拣选了世上软弱的,叫那强壮的羞愧。

神也拣选了世上卑贱的,被人厌恶的,以及那无有的,为要废掉那有的。

使一切有血气的,在神面前一个也不能自夸。

但你们得在基督耶稣里,是本乎神,神又使他成为我们的智慧,公义,圣洁,救赎。

如经上所记,夸口的当指着主夸口。

\chapter{哥林多前书第2章}
弟兄们,从前我到你们那里去,并没有用高言大智对你们宣传神的奥秘。

因为我曾定了主意,在你们中间不知道别的,只知道耶稣基督,并他钉十字架。

我在你们那里,又软弱,又惧怕,又甚战竞。

我说的话,讲的道,不是用智慧委婉的言语,乃是用圣灵和大能的明证。

叫你们的信不在乎人的智慧,只在乎神的大能。

然而在完全的人中,我们也讲智慧。但不是这世上的智慧,也不是这世上有权有位将要败亡之人的智慧。

我们讲的,乃是从前所隐藏,神奥秘的智慧,就是神在万世以前豫定使我们得荣耀的。

这智慧世上有权有位的人没有一个知道的。他们若知道,就不把荣耀的主钉在十字架上了。

如经上所记,神为爱他的人所豫备的,是眼睛未曾看见,耳朵未曾听见,人心也未曾想到的。

只有神藉着圣灵向我们显明了。因为圣灵渗透万事,就是神深奥的事也渗透了。

除了在人里头的灵,谁知道人的事。像这样,除了神的灵,也没有人知道神的事。

我们所领受的,并不是世上的灵,乃是从神来的灵,叫我们能知道神开恩赐给我们的事。

并且我们讲说这些事,不是用人智慧所指教的言语,乃是用圣灵所指教的言语,将属灵的话,解释属灵的事。(或作将属灵的事讲与属灵的人)

然而属血气的人不领会神圣灵的事,反倒以为愚拙。并且不能知道,因为这些事惟有属灵的人才能看透。

属灵的人能看透万事,却没有一人能看透了他。

谁曾知道主的心去教导他呢。但我们是有基督的心了。

\chapter{哥林多前书第3章}
弟兄们,我从前对你们说话,不能把你们当作属灵的,只得把你们当作属肉体,在基督里为婴孩的。

我是用奶喂你们,没有用饭喂你们。那时你们不能吃,就是如今还是不能。

你们仍是属肉体的。因为在你们中间有嫉妒分争,这岂不是属乎肉体,照着世人的样子行吗。

有说,我是属保罗的。有说,我是属亚波罗的。这岂不是你们和世人一样吗。

亚波罗算什么。保罗算什么。无非是执事,照主所赐给他们各人的,引导你们相信。

我栽种了,亚波罗浇灌了。惟有神叫他生长。

可见栽种的算不得什么,浇灌的也算不得什么。只在那叫他生长的神。

栽种的和浇灌的都是一样。但将来各人要照自己的工夫,得自己的赏赐。

因为我们是与神同工的。你们是神所耕种的田地,所建造的房屋。

我照神所给我的恩,好像一个聪明的工头,立好了根基,有别人在上面建造。只是各人要谨慎怎样在上面建造。

因为那已经立好的根基,就是耶稣基督,此外没有人能立别的根基。

若有人用金,银,宝石,草木,禾楷,在这根基上建造。

各人的工程必然显露。因为那日子要将他表明出来,有火发现。这火要试验各人的工程怎样。

人在那根基上所建造的工程,若存得住,他就要得赏赐。

人的工程若被烧了,他就要受亏损。自己却要得救。虽然得救乃像从火里经过的一样。

岂不知你们是神的殿,神的灵住在你们里头吗。

若有人毁坏神的殿,神必要毁坏那人。因为神的殿是圣的,这殿就是你们。

人不可自欺。你们中间若有人,在这世界自以为有智慧,倒不如变作愚拙,好成为有智慧的。

因这世界的智慧,在神看是愚拙。如经上记着说,主叫有智慧的中了自己的诡计。

又说,主知道智慧人的意念是虚妄的。

所以无论谁,都不可拿人夸口。因为万有全是你们的。

或保罗,或亚波罗,或矶法,或世界,或生,或死,或现今的事,或将来的事,全是你们的。

并且你们是属基督的。基督又是属神的。

\chapter{哥林多前书第4章}
人应当以我们为基督的执事,为神奥秘事的管家。

所求于管家的,是要他有忠心。

我被你们论断,或被别人论断,我都以为极小的事。连我自己也不论断自己。

我虽不觉得自己有错,却也不能因此得以称义。但判断我的乃是主。

所以时候未到,什么都不要论断,只等主来,他要照出暗中的隐情,显明人心的意念。那时各人要从神那里得着称赞。

弟兄们,我为你们的缘故,拿这些事转比自己和亚波罗。叫你们效法我们不可过于圣经所记。免得你们自高自大,贵重这个,轻看那个。

使你与人不同的是谁呢。你有什么不是领受的呢。若是领受的,为何自夸,彷佛不是领受的呢。

你们已经饱足了,已经丰富了,不用我们,自己就作王了。我愿意你们果真作王,叫我们也得与你们一同作王。

我想神把我们使徒明明列在末后,好像定死罪的囚犯。因为我们成了一台戏,给世人和天使观看。

我们为基督的缘故算是愚拙的,你们在基督里倒是聪明的,我们软弱,你们倒强壮。你们有荣耀,我们倒被藐视。

直到如今,我们还是又饥又渴,又赤身露体,又挨打,又没有一定的住处。

并且劳苦,亲手作工,被人咒骂,我们就祝福。被人逼迫,我们就忍受。

被人毁谤,我们就善劝。直到如今,人还把我们看作世界上的污秽,万物中的渣滓。

我写这话,不是叫你们羞愧,乃是警戒你们,好像我所亲爱的儿女一样。

你们学基督的,师傅虽有一万,为父的却是不多,因我在基督耶稣里用福音生了你们。

所以我求你们效法我。

因此我已打发提摩太到你们那里去。他在主里面,是我所亲爱有忠心的儿子。他必题醒你们,记念我在基督里怎样行事,在各处各教会中怎样教导人。

有些人自高自大,以为我不到你们那里去。

然而主若许我,我必快到你们那里去。并且我所要知道的,不是那些自高自大之人的言语,乃是他们的权能。

因为神的国不在乎言语,乃在乎权能。

你们愿意怎吗样呢。是愿意我带着刑杖到你们那里去呢,还是要我存慈爱温柔的心呢。

\chapter{哥林多前书第5章}
风闻在你们中间有淫乱的事。这样的淫乱,连外邦人中也没有,就是有人收了他的继母。

你们还是自高自大,并不哀痛,把行这事的人从你们中间赶出去。

我身子虽不在你们那里,心却在你们那里,好像我亲自与你们同在,已经判断了行这事的人,

就是你们聚会的时候,我的心也同在,奉我们主耶稣的名,并用我们主耶稣的权能,

要把这样的人交给撒但,败坏他的肉体,使他的灵魂在主耶稣的日子可以得救。

你们这自夸是不好的。岂不知一点面酵能使全团发起来吗。

你们既是无酵的面,应当把旧酵除净,好使你们成为新团。因为我们逾越节的羔羊基督已经被杀献祭了。

所以,我们守这节不可用旧酵,也不可用恶毒(或作阴毒),邪恶的酵,只用诚实真正的无酵饼。

我先前写信给你们说,不可与淫乱的人相交。

此话不是指这世上一概行淫乱的,或贪婪的,勒索的,或拜偶像的,若是这样,你们除非离开世界方可。

但如今我写信给你们说,若有称弟兄,是行淫乱的,或贪婪的,或拜偶像的,或辱骂的,或醉酒的,或勒索的。这样的人不可与他相交。就是与他吃饭都不可。

因为审判教外的人与我何干。教内的人岂不是你们审判的吗。

至于外人有神审判他们。你们应当把那恶人从你们中间赶出去。

\chapter{哥林多前书第6章}
你们中间有彼此相争的事,怎敢在不义的人面前求审,不在圣徒面前求审呢。

岂不知圣徒要审判世界吗。若世界为你们所审,难道你们不配审判这最小的事吗。

岂不知我们要审判天使吗。何况今生的事呢。

既是这样,你们若今生的事当审判,是派教会所轻看的人审判吗。

我说这话,是要叫你们羞耻。难道你们中间没有一个智慧人,能审断弟兄们的事吗。

你们竟是弟兄与弟兄告状,而且告在不信主的人面前。

你们彼此告状,这已经是你们的大错了。为什么不情愿受欺呢。为什么不情愿吃亏呢。

你们倒是欺压人,亏负人,况且所欺压,所亏负的就是弟兄。

你们岂不知不义的人不能承受神的国吗。不要自欺。无论是淫乱的,拜偶像的,奸淫的,作娈童的,亲男色的,

偷窃的,贪婪的,醉酒的,辱骂的,勒索的,都不能承受神的国。

你们中间也有人从前是这样。但如今你们奉主耶稣基督的名,并藉着神的灵,已经洗净,成圣称义了。

凡事我都可行。但不都有益处。凡事我都可行,但无论那一件,我总不受他的辖制。

食物是为肚腹,肚腹是为食物。但神要叫这两样都废坏。身子不是为淫乱,乃是为主。主也是为身子。

并且神已经叫主复活,也要用自己的能力叫我们复活。

岂不知你们的身子是基督的肢体吗。我可以将基督的肢体作为娼妓的肢体吗。断乎不可。

岂不知与娼妓联合的,便是与他成为一体吗。因为主说,二人要成为一体。

但与主联合的,便是与主成为一灵。

你们要逃避淫行。人所犯的,无论什么罪,都在身子以外。惟有行淫的,是得罪自己的身子。

岂不知你们的身子就是圣灵的殿吗。这圣灵是从神而来,住在你们里头的。并且你们不是自己的人。

因为你们是重价买来的。所以要在你们的身子荣耀神。

\chapter{哥林多前书第7章}
论到你们信上所题的事,我说男不近女倒好。

但要免淫乱的事,男人当各有自己的妻子,女子也当各有自己的丈夫。

丈夫当用合宜之分待妻子,妻子待丈夫也要如此。

妻子没有权柄主张自己的身子,乃在丈夫。丈夫也没有权柄主张自己的身子,乃在妻子。

夫妻不可彼此亏负,除非两相情愿,暂时分房,为要专心祷告方可,以后仍要同房,免得撒但趁着你们情不自禁,引诱你们。

我说这话,原是准你们的,不是命你们的。

我愿意众人像我一样。只是各人领受神的恩赐,一个是这样,一个是那样。

我对着没有嫁娶的和寡妇说,若他们常像我就好。

倘若自己禁止不住,就可以嫁娶。与其欲火攻心,倒不如嫁娶为妙。

至于那已经嫁娶的,我吩咐他们,其实不是我吩咐,乃是主吩咐,说,妻子不可离开丈夫。

若是离开了,不可再嫁。或是同丈夫和好。丈夫也不可离弃妻子。

我对其馀的人说,不是主说,倘若某弟兄有不信的妻子,妻子也情愿和他住,他就不要离弃妻子。

妻子有不信的丈夫,丈夫也情愿和他同住,他就不要离弃丈夫。

因为不信的丈夫,就因着妻子成了圣洁。并且不信的妻子,就因着丈夫成了圣洁。(丈夫原文作弟兄)不然,你们的儿女就不洁净。但如今他们是圣洁的了。

倘若那不信的人要离去,就由他离去吧。无论是弟兄,是姐妹,遇着这样的事,都不必拘束。神召我们原是要我们和睦。

你这作妻子的,怎吗知道不能救你的丈夫呢。你这作丈夫的,怎吗知道不能救你的妻子呢。

只要照主所分给各人的,和神所召各人的而行。我吩咐各教会都是这样。

有人已受割礼蒙召呢,就不要废割礼。有人未受割礼蒙召呢,就不要受割礼。

受割礼算不得什么,不受割礼也算不得什么。只要守神的诫命就是了。

各人蒙召的时候是什么身分,仍要守住这身分。

你是作奴隶蒙召的吗,不要因此忧虑。若能以自由,就求自由更好。

因为作奴仆蒙召于主的,就是主所释放的人。作自由之人蒙召的,就是基督的奴仆。

你们是重价买来的。不要作人的奴仆。

弟兄们,你们各人蒙召的时候是什么身分,仍要在神面前守住这身分。

论到童身的人,我没有主的命令,但我既蒙主怜恤,能作忠心的人,就把自己的意见告诉你们

因现今的艰难,据我看来,人不如守素安常才好。

你有妻子缠着呢,就不要求脱离。你没有妻子缠着呢,就不要求妻子。

你若娶妻,并不是犯罪。处女若出嫁,也不是犯罪。然而这等人肉身必受苦难。我却愿意你们免这苦难。

弟兄们,我对你们说,时候减少了。从此以后,那有妻子的,要像没有妻子。

哀哭的,要像不哀哭。快乐的,要像不快乐。置买的,要像无有所得。

用世物的,要像不用世物。因为这世界的样子将要过去了。

我愿你们无所挂虑。没有娶妻的,是为主的事挂虑,想怎样叫主喜悦。

娶了妻的,是为世上的事挂虑,想怎样叫妻子喜悦。

妇人和处女也有分别。没有出嫁的,是为主的事挂虑,要身体灵魂都圣洁。已经出嫁的,是为世上的事挂虑,想要怎样叫丈夫喜悦。

我说这话,是为你们的益处。不是要牢笼你们,乃是要叫你们行合宜的事,得以殷勤服事主,没有分心的事。

若有人以为自己待他的女儿不合宜,女儿也过了年岁,事又当行,他就可随意办理,不算有罪,叫二人成亲就是了。

倘若人心里坚定,没有不得己的事,并且由得自己作主,心里又决定了留下女儿不出嫁,如此行也好。

这样看来,叫自己的女儿出嫁是好。不叫他出嫁更好。

丈夫活着的时候,妻子是被约束的。丈夫若死了,妻子就可以自由,随意再嫁。只是要嫁这在主里面的人。

然而按我的意见,若常守节更有福气。我也想自己是被神的灵感动了。

\chapter{哥林多前书第8章}
论到祭偶像之物,我们晓得我们都有知识。但知识是叫人自高自大,惟有爱心能造就人。

若有人以为自己知道什么,按他所当知道的,他仍是不知道。

若有人爱神,这人乃是神所知道的。

论到吃祭偶像之物,我们知道偶像在世上算不得什么。也知道神只有一位,再没有别的神。

虽有称为神的,或在天,或在地。就如那许多的神,许多的主。

然而我们只有一位神,就是父,万物都本于他,我们也归于他。并有一位主,就是耶稣基督,万物都是藉着他有的,我们也是藉着他有的。

但人不都有这等知识。有人到如今因拜惯了偶像,就以为所吃的是祭偶像之物。他们的良心既然软弱,也就污秽了。

其实食物不能叫神看中我们。因为我们不吃也无损,吃也无益。

只是你们要谨慎,恐怕你们这自由,竟成了那软弱人的绊脚石。

若有人见你这有知识的,在偶像的庙里坐席,这人的良心若是软弱,岂不放胆去吃那祭偶像之物吗。

因此,基督为他死的那软弱弟兄,也就因你的知识沉伦了。

你们这样得罪弟兄们,伤了他们软弱的良心,就是得罪基督。

所以食物若叫我弟兄跌倒,我就永远不吃肉,免得叫我弟兄跌倒了。

\chapter{哥林多前书第9章}
我不是自由的吗。我不是使徒吗。我不是见过我们的主耶稣吗。你们不是我在主里面所作之工吗。

假若在别人我不是使徒,在你们我总是使徒。因为你们在主里正是我作使徒的印证

我对那磐问我的人,就是这样分诉。

难道我们没有权柄靠福音吃喝吗。

难道我们没有权柄娶信主的姊妹为妻,带着一同往来,彷佛其馀的使徒,和主的弟兄,并矶法一样吗。

独有我与巴拿巴没有权柄不作工吗。

有谁当兵,自备粮饷呢。有谁栽葡萄园,不吃园里的果子呢。有谁牧养牛羊,不吃牛羊的奶呢。

我说这话,岂是照人的意见。律法不也是这样说吗。

就如摩西的律法记着说,牛在场上踹谷的时候,不可笼住它的嘴。难道神所挂念的是牛吗。

不全是为我们说的吗。分明是为我们说的。因为耕种的当存着指望去耕种。打场的也当存得粮的指望去打场。

我们若把属灵的种子撒在你们中间,就是从你们收割奉养肉身之物,这还算大事吗。

若别人在你们身上有这权柄,何况我们呢。然而我们没有用过这权柄,倒凡事忍受,免得基督的福音被阻隔。

你们岂不知为圣事劳碌的,就吃殿中的物吗。伺候祭坛的,就分领坛上的物吗。

主也这样命定,叫传福音的靠福音养生。

但这权柄我全没有用过。我写这话,并非要你们这样待我。因为我宁可死,也不叫人使我所夸的落了空。

我传福音原没有可夸的。因为我是不得已的。若不传福音,我便有祸了。

我若甘心作这事,就有赏赐。若不甘心,责任却已经托付我了。

既是这样,我的赏赐是什么呢。就是我传福音的时候,叫人不花钱得福音,免得用尽我传福音的权柄。

我虽是自由的,无人辖管,然而我甘心作了众人的仆人,为要多得人。

向犹太人,我就作犹太人,为要得犹太人。向律法以下的人,我虽不在律法以下,还是作律法以下的人,为要得律法以下的人。

向没有律法的人,我就作没有律法的人,为要得没有律法的人。其实我在神面前,不是没有律法,在基督面前,正在律法之下。

向软弱的人,我就作软弱的人,为要得软弱的人。向什么样的人,我就作什么样的人。无论如何,总要救些人。

凡我所行的,都是为福音的缘故,为要与人同的这福音的好处。

岂不知在场上赛跑的都跑,但得奖赏的只有一人。你们也当这样跑,好叫你们得着奖赏。

凡较力争胜的,诸事都有节制。他们不过是要得能坏的冠冕。我们却是要得不能坏的冠冕。

所以我奔跑,不像无定向的。我斗拳,不像打空气的。

我是攻克己身,叫身服我。恐怕我传福音给别人,自己反被弃绝了。

\chapter{哥林多前书第10章}
弟兄们,我不愿意你们不晓得,我们的祖宗从前都在云下,都从海中经过。

都在云里海里受洗归了摩西。

并且都吃了一样的灵食。

也都喝了一样的灵水。所喝的是出于随着他们的灵磐石。那磐石就是基督。

但他们中间,多半是神不喜欢的人。所以在旷野倒毙。

这些事都是我们的监戒,叫我们不要贪恋恶事,像他们那样贪恋的。

也不要拜偶像,像他们有人拜的。如经上所记,百姓坐下吃喝,起来玩耍。

我们也不要行奸淫,像他们有人行的,一天就倒毙了二万三千人。

也不要试探主,(主有古卷作基督)像他们有人试探的,就被蛇所灭。

你们也不要发怨言,像他们有发怨言的,就被灭命的所灭。

他们遭遇这些事,都要作为监戒。并且写在经上,正是警戒我们这末世的人。

所以自己以为站得稳的,须要谨慎,免得跌倒。

你们所遇见的试探,无非是人所能受的。神是信实的,必不叫你们受试探过于所能受的。在受试探的时候,总要给你们开一条出路,叫你们能忍受得住。

我所亲爱的弟兄阿,你们要逃避拜偶像的事。

我好像对明白人说的,你们要审察我的话。

我们所祝福的杯,岂不是同领基督的血吗。我们所擘开的饼,岂不是同领基督的身体吗。

我们虽多,仍是一个饼,一个身体。因为我们都是分受这一个饼。

你们看属肉体的以色列人。那吃祭物的,岂不是在祭坛上有分吗。

我是怎样说呢。岂是说祭偶像之物算得什么呢。或说偶像算得什么呢。

我乃是说,外邦人所献的祭,是祭鬼,不是祭神。我不愿意你们与鬼相交。

你们不能喝主的杯,又喝鬼的杯。不能吃主的筵席,又吃鬼的筵席。

我们可惹主的愤恨吗。我们比他还有能力吗。

凡事都可行。但不都有益处。凡事都可行。但不都造就人

无论何人,不要求自己的益处,乃要求别人的益处。

凡市上所卖的,你们只管吃,不要为良心的缘故问什么话。

因为地和其中所充满的,都属乎主。

倘有一个不信的请你们赴席,你们若愿意去,凡摆在你们面前的,只管吃,不要为良心的缘故问什么话。

若有人对你说,这是献过祭的物,就要为那告诉你们的人,并为良心的缘故,不吃。

我说的良心,不是你的,乃是他的。我这自由,为什么被别人的良心论断呢。

我若谢恩而吃,为什么因我谢恩的物被人毁谤呢。

所以你们或吃或喝,无论作什么,都要荣耀神而行。

不拘是犹太人,是希腊人,是神的教会,你们都不要使他跌倒。

就好像我凡事都叫众人喜欢,不求自己的益处,只求众人的益处,叫他们得救。

\chapter{哥林多前书第11章}
你们该效法我,像我效法基督一样。

我称赞你们,因你们凡事记念我,又坚守我所传给你们的。

我愿意你们知道,基督是各人的头。男人是女人的头,神是基督的头。

凡男人祷告或是讲道(讲道或作说预言下同),若蒙着头,就是羞辱自己的头。

凡女人祷告或讲道,若不蒙着头,就羞辱自己的头。因为这就如同剃了头发一样。

女人若不蒙着头,就该剪了头发。女人若以剪发剃发为羞愧,就该蒙着头。

男人本不该蒙着头,因为他是神的形像和荣耀,但女人是男人的荣耀。

起初,男人不是由女人而出。女人乃是由男人而出。

并且男人不是为女人造的。女人乃是为男人造的。

因此,女人为天使的缘故,应当在头上有服权柄的记号。

然而照主的安排,女也不是无男,男也不是无女。

因为女人原是由男人而出,男人也是由女人而出。但万有都是出乎神。

你们自己审察,女人祷告神,不蒙着头,是合宜的吗。

你们的本性不也指示你们,男人若有长头发,便是他的羞辱吗。

但女人有长头发,乃是他的荣耀。因为这头发是给他作盖头的。

若有人想要辩驳,我们却没有这样的规矩К神的众教会也是没有的。

我现今吩咐你们的话,不是称赞你们。因为你们聚会不是受益,乃是招损。

第一,我听说你们聚会的时候,彼此分门别类。我也稍微的信这话。

在你们中间不免有分门结党的事,好叫那些有经验的人,显明出来。

你们聚会的时候,算不得吃主的晚餐。

因为吃的时候,各人先吃自己的饭,甚至这个肌饿,那个酒醉。

你们要吃喝,难道没有家吗。还是藐视神的教会,叫那没有的羞愧呢。我向你们可怎吗说呢。可因此称赞你们吗。我不称赞。

我当日传给你们的,原是从主领受的,就是主耶稣被卖的那一夜,拿起饼来,

祝谢了,就擘开,说,这是我的身体,为你们舍的(舍有古卷作擘开)。你们应当如此行,为的是记念我。

饭后,也照样拿起杯来,说,这杯是用我的血所立的新约。你们每逢喝的时候,要如此行,为的是记念我。

你们每逢吃这饼,喝这杯,是表明主的死,直等到他来。

所以无论何人,不按理吃主的饼,喝主的杯,就是干犯主的身主的血了。

人应当自己省察,然后吃这饼,喝这杯。

因为人吃喝,若不分辨是主的身体,就是吃喝自己的罪了。

因此,在你们中间有好些软弱的,与患病的,死的也不少(死原文作睡)。

我们若是先分辨自己,就不至于受审。

我们受审的时候,乃是被主惩治。免得我们和世人一同定罪。

所以我弟兄们,你们聚会吃的时候,要彼此等待。

若有人肌饿,可以在家里先吃。免得你们聚会自己取罪。其馀的事,我来的时候再安排。

\chapter{哥林多前书第12章}
弟兄们,论到属灵的恩赐,我不愿意你们不明白。

你们作外邦人的时候,随事被牵引受迷惑,去服事那哑吧偶像。这是你们知道的。

所以我告诉你们,被神的灵感动的,没有说耶稣是可咒诅的。若不是被圣灵感动的,也没有能说耶稣是主的。

恩赐原有分别,圣灵却是一位。

职事也有分别,主却是一位。

功用也有分别,神却是一位,在众人里面运行一切的事。

圣灵显在各人身上,是叫人得益处。

这人蒙圣灵赐他智慧的言语。那人也蒙这位圣灵赐他知识的言语。

又有一人蒙这位圣灵赐他信心。还有一人蒙这位圣灵赐他医病的恩赐。

又有一人能行异能。又叫一人能作先知。又叫一人能辨别诸灵。又叫一人能说方言。又叫一人能翻方言。

这一切都是这位圣灵所运行,随己意分给各人的。

就如身子是一个,却有许多肢体。而且肢体虽多,仍是一个身子。基督也是这样。

我们不拘是犹太人,是希腊人,是为奴的,是自主的,都从一位圣灵受洗,成了一个身体。饮于一位圣灵。

身子原不是一个肢体,乃是许多肢体。

设若脚说,我不是手,所以不属乎身子。他不能因此就不属乎身子。

设若耳说,我不是眼,所以不属乎身子。他也不能因此就不属乎身子。

若全身是眼,从那里听声呢。若全身是耳,从那里闻味呢。

但如今神随自己的意思,把肢体俱各安排在身上。

若都是一个肢体,身子在那里呢。

但如今肢体是多的,身子却是一个。

眼不能对手说,我用不着你。头也不能对脚说,我用不着你。

不但如此,身上肢体人以为软弱的,更是不可少的。

身上肢体,我们看为不体面的,越发给他加上体面。不俊美的,越发得着俊美。

我们俊美的肢体,自然用不着装饰。但神配搭这身子,把加倍的体面给那有缺欠的肢体。

免得身上分门别类。总要肢体彼此相顾。

若一个肢体受苦,所有的肢体就一同受苦。若一个肢体得荣耀,所有的肢体就一同快乐。

你们就是基督的身子,并且各作肢体。

神在教会所设立的,第一是使徒。第二是先知,第三是教师。其次是行异能的。再次是得恩赐医病的。帮助人的。治理事的。说方言的。

岂都是使徒吗。岂都是先知吗。岂都是教师吗。岂都是行异能的吗。

岂都是得恩赐医病的吗。岂都是说方言的吗。岂都是翻方言的吗。

你们要切切的求那更大的恩赐,我现今把最妙的道指示你们。

\chapter{哥林多前书第13章}
我若能说万人的方言,并天使的话语却没有爱,我就成了呜的锣,响的钹一般。

我若有先知讲道之能,也明白各样的奥秘,各样的知识。而且有全备的信,叫我能够移山,却没有爱,我就算不得什么。

我若将所有的周济穷人,又舍己身叫人焚烧,却没有爱,仍然与我无益。

爱是恒久忍耐,又有恩慈。爱是不嫉妒。爱是不自夸。不张狂。

不作害羞的事。不求自己的益处。不轻易发怒。不计算人的恶。

不喜欢不义。只喜欢真理。

凡事包容。凡事相信。凡事盼望。凡事忍耐。

爱是永不止息。先知讲道之能,终必归于无有。说方言之能,终必停止,知识也终必归于无有。

我们现在所知道的有限,先知所讲的也有限。

等那完全的来到,这有限的必归于无有了。

我作孩子的时候,话语像孩子,心思像孩子,意念像孩子。既成了人,就把孩子的事丢弃了。

我们如今彷佛对着镜子观看,馍糊不清。(馍糊不清原文作如同谜)到那时,到那时,就要面对面了。我如今所知道的有限。到那时就全知道,如同主知道我一样。

如今常存的有信,有望,有爱,这三样,其中最大的是爱。

\chapter{哥林多前书第14章}
你们要追求爱,也要切慕属灵的恩赐,其中更要羡慕的,是作先知讲道。(原文作是说预言下同)

那说方言的,原不是对人说,乃是对神说。因为没有人听出来。然而他在心灵里,却是讲说各样的奥秘。

但作先知讲道的,是对人说,要造就,安慰,劝勉人。

说方言的,是造就自己。作先知讲道的,乃是造就教会。

我愿意你们都说方言。更愿意你们作先知讲道。因为说方言的,若不翻出来,使教会被造就,那作先知讲道的,就比他强了。

弟兄们,我到你们那里去,若只说方言,不用启示,或知识,或预言,或教训,给你们讲解,我与你们有什么益处呢。

就是那有声无气的物,或箫,或琴,若发出来的声音,没有分别,怎能知道所吹所弹的是什么呢。

若吹无定的号声,谁能豫备打仗呢。

你们也是如此,舌头若不说容易明白的话,怎能知道所说的是什么呢。这就是向空说话了。

世上的声音,或者甚多,却没有一样是无意思的。

我若不明白那声音的意思,这说话的人必以我为化外之人,我也以他为化外之人。

你们也是如此。既是切慕属灵的恩赐,就当求多得造就教会的恩赐。

所以那说方言的,就当求着能翻出来。

我若用方言祷告,是我的灵祷告。但我的悟性没有果效。

这却怎吗说呢。我要用灵祷告,也要用悟性祷告。我要用灵歌唱,也要用悟性歌唱。

不然,你用灵祝谢,那在座不通方言的人,既然不明白你的话,怎能在你感谢的时候说阿们呢。

你感谢的固然是好,无奈不能造就别人。

我感谢神,我说方言比你们众人还多。

但在教会中,宁可用悟性说五句教导人的话,强如说万句方言。

弟兄们,在心志上不要作小孩子。然而在恶事上要作婴孩。在心志上总要作大人。

律法上记着,主说,我要用外邦人的舌头,和外邦人的嘴唇,向这百姓说话。虽然如此,他们还是不听从我。

这样看来,说方言,不是为信的人作证据,乃是为不信的人。作先知讲道,不是为不信的人作证据,乃是为信的人。

所以全教会聚在一处的时候,若都说方言,偶然有不通方言的,或是不信的人进来,岂不说你们癫狂了吗。

若都作先知讲道,偶然有不信的,或是不通方言的人进来,就被众人劝醒,被众人审明。

他心里的隐情显露出来,就必将脸伏地,敬拜神,说神真是在你们中间了。

弟兄们,这却怎吗样呢。你们聚会的时候,各人或有诗歌,或有教训,或有启示,或有方言,或有翻出来的话。凡事都当造就人。

若有说方言的,只好两个人,至多三个人,且要轮流说,也要一个人翻出来。

若没有人翻,就当在会中闭口。只对自己和神说,就是了。

至于作先知讲道的,只好两个人,或是三个人,其馀的就当慎思明辨。

若旁边坐着的得了启示,那先说话的就当闭口不言。

因为你们都可以一个一个的作先知讲道,叫众人学道理,叫众人得劝勉。

先知的灵,原是顺服先知的。

因为神不是叫人混乱,乃是叫人安静。

妇女在会中要闭口不言,像在圣徒的众教会一样。因为不准他们说话。他们总要顺服,正如律法所说的。

他们若要学什么,可以在家里问自己的丈夫。因为妇女在会中说话原是可耻的。

神的道理,岂是从你们出来吗。岂是单临到你们吗。

若有人以为自己是先知或是属灵的,就该知道,我所写给你们的是主的命令。

若有不知道的,就由他不知道吧。

所以我弟兄们,你们要切慕作先知讲道,也不要禁止说方言。

凡事都要规规矩矩的按着次序行。

\chapter{哥林多前书第15章}
弟兄们,我如今把先前所传给你们的福音,告诉你们知道,这福音你们也领受了,又靠着站立得住。

并且你们若不是徒然相信,能以持守我所传给你们的,就必因这福音得救。

我当日所领受又传给你们的,第一,就是基督照圣经所说,为我们的罪死了。

而且埋葬了。又照圣经所说,第三天复活了。

并且显给矶法看。然后显给十二使徒看。

后来一时显给五百多弟兄看,其中一大半到如今还在,却也有已经睡了的。

以后显给雅各看。再显给众使徒看。

末了也显给我看。我如同未到产期而生的人一般。

我原是使徒中最小的,不配称为使徒,因为我从前逼迫神的教会。

然而我今日成了何等人,是蒙神的恩才成的。并且他所赐我的恩,不是徒然的。我比众使徒格外劳苦。这原不是我,乃是神的恩与我同在。

不拘是我是众使徒,我们如此传,你们也如此信了。

既传基督是从死里复活了,怎吗在你们中间,有人说没有死人复活的事呢。

若没有死人复活的事,基督也就没有复活了。

若基督没有复活,我们所传的便是枉然,你们所信的也是枉然。

并且明显我们是为神妄作见证的。因我们见证神是叫基督复活了。若死人真不复活,神也没有叫基督复活了。

因为死人若不复活,基督也没有复活了。

基督若没有复活,你们的信便是徒然。你们仍在罪里。

就是在基督里睡了的人也灭亡了。

我们若靠基督,只在今生有指望,就算比众人更可怜。

但基督已经从死里复活,成为睡了之人初熟的果子。

死既是因一人而来,死人复活也是因一人而来。

在亚当里众人都死了。照样,在基督里众人也都要复活。

但各人是按着自己的次序复活。初熟的果子是基督。以后在他来的时候,是那些属基督的。

再后末期到了,那时,基督既将一切执政的,掌权的,有能的,都毁灭了,就把国交与父神。

因为基督必要作王,等神把一切仇敌,都放在他的脚下。

尽末了所毁灭的仇敌,就是死。

因为经上说,神叫万物都服在他的脚下。既说万物都服了他,明显那叫万物服他的不在其内了。

万物既服了他,那时,子也要自己服那叫万物服他的,叫神在万物之上,为万物之主。

不然那些为死人受洗的,将来怎样呢。若死人总不复活,因何为他们受洗呢。

我们又因何时刻冒险呢。

弟兄们,我在主基督耶稣里,指着你们所夸的口,极力的说,我是天天冒死。

我若当日像寻常人,在以弗所同野兽战斗,那于我有什么益处呢。若死人不复活,我们就吃吃喝喝吧。因为明天要死了。

你们不要自欺。滥交是败坏善行。

你们要醒悟为善,不要犯罪。因为有人不认识神。我说这话,是要叫你们羞愧。

或有人问,死人怎样复活。带着什么身体来呢。

无知的人哪,你所种的,若不死就不能生。

并且你所种的,不是那将来的形体,不过是子粒,既如麦子,或是别样的谷。

但神随自己的意思,给他一个形体,并叫各等子粒,各有自己的形体。

凡肉体各有不同。人是一样,兽又是一样,鸟又是一样,鱼又是一样。

有天上的形体,也有地上的形体。但天上形体的荣光是一样,地上形体的荣光又是一样。

日有日的荣光,月有月的荣光,星有星的荣光。这星和那星的荣光,也有分别。

死人复活也是这样。所种的是必朽坏的,复活的是不朽坏的。

所种的是羞辱的,复活的是荣耀的。所种的是软弱的,复活的是强壮的。

所种的是血气的身体,复活的是灵性的身体。若有血气的身体,也必有灵性的身体。

经上也是这样记着说,首先的人亚当,成了有灵的活人。(灵或作血气)末后的亚当,成了叫人活的灵。

但属灵的不在先,属血气的在先。以后才有属灵的。

头一个人是出于地,乃属土。第二个人是出于天。

那属土的怎样,凡属土的也就怎样。属天的怎样,凡属天的也就怎样。

我们既有属土的形状,将来也必有属天的形状。

弟兄们,我告诉你们说,血肉之体,不能承受神的国。必朽坏的,不能承受不朽坏的。

我如今把一件奥秘的事告诉你们。我们不是都要睡觉,乃是都要改变,

就在一霎时,眨眼之间,号筒末次吹响的时候。因号筒要响,死人要复活成为不朽坏的,我们也要改变。

这必朽坏的,总要变成不朽坏的。(变成原文作穿下同)这必死的,总要变成不死的。

这必朽坏的既变成不朽坏的。这必死的既变成不死的。那时经上所记,死被得胜吞灭的话就应验了。

死阿,你得胜的权势在那里。死阿,你的毒钩在那里。

死的毒钩就是罪。罪的权势就是律法。

感谢神,使我们藉着我们的主耶稣基督得胜。

所以我亲爱的弟兄们,你们务要坚固不可摇动,常常竭力多作主工,因为知道你们的劳苦,在主里面不是徒然的。

\chapter{哥林多前书第16章}
论到为圣徒捐钱,我从前怎样吩咐加拉太的众教会,你们也当怎样行。

每逢七日的第一日,各人要照自己的进项抽出来留着。免得我来的时候现凑。

及至我来到了,你们写信举荐谁,我就打发他们,把你们的捐资送到耶路撒冷去。

若我也该去,他们可以和我同去。

我要从马其顿经过。既经过了,就要到你们那里去。

或者和你们同住几时,或者也过冬。无论我往那里去,你们就可以给我送行。

我如今不愿意路过见你们。主若许我,我就指望和你们同住几时。

但我要仍旧住在以弗所,直等到五旬节。

因为有功效的门,为我开了,并且反对的人也多。

若是提摩太来到,你们要留心,叫他在你们那里无所惧怕。因为他劳力作主的工,像我一样。

所以无论谁,都不可藐视他。只要送他平安前行,叫他到我这里来。因我指望他和弟兄们同来。

至于兄弟亚波罗,我再三的劝他,同弟兄们到你们那里去。但这时他决不愿意去。几时有了机会他必去。

你们务要儆醒,在真道上站立得稳,要作大丈夫,要刚强。

凡你们所作的,都要凭爱心而作。

弟兄们,你们晓得司提反一家,是亚该亚初结的果子。并且他们专以服事圣徒为念。

我劝你们顺服这样的人,并一切同工同劳的人。

司提反,和福徒拿都,并亚该古,到这里来,我很喜欢。因为你们待我有不及之处,他们补上了。

他们叫我和你们心里都快活。这样的人,你们务要敬重。

亚细亚的众教会问你们安。亚居拉和百基拉,并在他们家里的教会,因主多多的问你们安。

众弟兄都问你们安。你们要亲嘴问安,彼此务要圣洁。

我保罗亲笔问安。

若有人不爱主,这人可诅可咒。主必要来。

愿主耶稣基督的恩,常与你们众人同在。

我在基督耶稣里的爱与你们众人同在。阿们。

\chapter{哥林多后书第1章}
奉神旨意,作基督耶稣使徒的保罗,和兄弟提摩太,写信给在哥林多神的教会,并亚该亚遍处的众圣徒。

愿恩惠平安,从神我们的父和主耶稣基督归与你们。

愿颂赞归与我们的主耶稣基督的父神,就是发慈悲的父,赐各样安慰的神。

我们在一切患难中,他就安慰我们,叫我们能用神所赐的安慰,去安慰那遭各样患难的人。

我们既多受基督的苦楚,就靠基督多得安慰。

我们受患难呢,是为叫你们得安慰得拯救。我们得安慰呢,也是为叫你们得安慰。这安慰能叫你们忍受我们所受的那样苦楚。

我们为你们所存的盼望是确定的。因为知道你们既是同受苦楚,也必同得安慰。

弟兄们,我们不要你们不晓得,我们从前在亚细亚遭遇苦难,被压太重,力不能胜,甚至连活命的指望都绝了。

自己心里也断定是必死的,叫我们不靠自己,只靠叫死人复活的神。

他曾救我们脱离那极大的死亡,现在仍要救我们,并且我们指望他将来还要救我们。

你们以祈祷帮助我们,好叫许多人为我们谢恩,就是为我们因许多人所得的恩。

我们所夸的,是自己的良心,见证我们凭着神的圣洁和诚实,在世为人,不靠人的聪明,乃靠神的恩惠,向你们更是这样。

我们现在写给你们的话,并不外乎你们所念的,所认识的,我也盼望你们到底还是要认识。

正如你们已经有几分认识我们。以我们夸口,好像我们在我们主耶稣的日子,以你们夸口一样。

我既然这样深信,就早有意到你们那里去,叫你们再得益处。

也要从你们那里经过,往马其顿去,再从马其顿回到你们那里,叫你们给我送行往犹太去。

我有此意,岂是反复不定吗。我所起的意,岂是从情欲起的,叫我忽是忽非吗。

我指着信实的神说,我们向你们所传的道,并没有是而又非的。

因为我和西拉,并提摩太,在你们中间所传神的儿子耶稣基督,总没有是而又非的,在他只有一是。

神的应许,不论有多少,在基督都是是的,所以藉着他也都是实在的,(实在原文作阿们)叫神因我们得荣耀。

那在基督里坚固我们和你们,并且膏我们的,就是神。

他又用印印了我们,并赐圣灵在我们心里作凭据。(原文作质)。

我呼吁神给我的心作见证,我没有往哥林多去,是为要宽容你们。

我们并不是辖管你们的信心,乃是帮助你们的快乐。因为你们凭信才站立得住。

\chapter{哥林多后书第2章}
我自己定了主意,再到你们那里去,必须大家没有忧愁。

倘若我叫你们忧愁,除了我叫那忧愁的人以外,谁能叫我快乐呢。

我曾把这事写给你们,恐怕我到的时候,应该叫我快乐的那些人,反倒叫我忧愁。我也深信,你们众人都以我的快乐为自己的快乐。

我先前心里难过痛苦,多多的流泪,写信给你们。不是叫你们忧愁,乃是叫你们知道我格外的疼爱你们。

若有叫人忧愁的,他不但叫我忧愁,也是叫你们众人有几分忧愁,我说几分,恐怕说得太重。

这样的人,受了众人的责罚,也就够了。

倒不如赦免他,安慰他,免得他忧愁太过,甚至沉沦了。

所以我劝你们,要向他显出坚定不移的爱心来。

为此我先前也写信给你们,要试验你们,看你们凡事顺从不顺从。

你们赦免谁,我也赦免谁,我若有所赦免的,是在基督面前为你们赦免的。

免得撒但趁着机会胜过我们。因我们并非不晓得他的诡计。

我从前为基督的福音到了特罗亚,主也给我开了门。

那时因为没有遇见兄弟提多,我心里不安,便辞别那里的人往马其顿去了。

感谢神,常帅领我们在基督里夸胜,并藉着我们在各处显扬那因认识基督而有的香气。

因为我们在婉面前,无论在得救的人身上,或灭亡的人身上,都有基督馨香之气。

在这等人,就作了死的香气叫他死。在那等人,就作了活的香气叫他活。这事谁能当得起呢。

我们不像那许多人,为利混乱神的道。乃是由于诚实,由于神,在神面前凭着基督讲道。

\chapter{哥林多后书第3章}
我们岂是又举荐自己吗。岂像别人,用人的荐信给你们,或用你们的荐信给人吗。

你们就是我们的荐信,写在我们心里,被众人所知道所念诵的。

你们明显是基督的信,藉着我们修成的。不是用墨写的,乃是用永生神的灵写的。不是写在石版上,乃是写在心版上。

我们因基督所以在神面前才有这样的信心,

并不是我们凭自己能承担什么事,我们所能承担的,乃是出于神。

他叫我们能承当这新约的执事。不是凭着字句,乃是凭着精意。因为那字句是叫人死,精意是叫人活。(精意或作圣灵)。

那用字刻在石头上属死的职事,尚且有荣光,甚至以色列人因摩西面上的荣光,不能定睛看他的脸。这荣光原是渐渐退去的。

何况那属灵的职事,岂不更有荣光吗。

若是定罪的职事有荣光,那称义的职事,荣光就越发大了。

那从前有荣光的,因这极大的荣光,就算不得有荣光了。

若那废掉的有荣光,这长存的就有荣光了。

我们既有这样的盼望,就大胆讲说,

不像摩西将帕子蒙在脸上,叫以色列人不能定睛看到那将废者的结局。

但他们的心地刚硬。直到今日诵读旧约的时候,这帕子还没有揭去。这帕子在基督里已经废去了。

然而直到今日,每逢诵读摩西书的时候,帕子还在他们心上。

但他们的心几时归向主,帕子就几时除去了。

主就是那灵。主的灵在那里,那里就得以自由。

我们众人既然倘着脸,得以看见主的荣光,好像从镜子里反照,就变成主的形状,荣上加荣,如同从主的灵变成的。

\chapter{哥林多后书第4章}
我们既然蒙怜悯,受了这职分,就不丧胆。

乃将那些暗昧可耻的事弃绝了,不行诡诈,不谬称神的道理。只将真理表明出来,好在神面前把自己荐与各人的良心。

如果我们的福音蒙蔽,就是蒙蔽在灭亡的人身上。

此等不信之人,被这世界的神弄瞎了心眼,不叫基督荣耀福音的光照着他们。基督本是神的像。

我们原不是传自己,乃是传基督耶稣为主,并且自己因耶稣作你们的仆人。

那吩咐光从黑暗里照出来的神,已经照在我们心里,叫我们得知神荣耀的光,显在耶稣基督的面上。

我们有这宝贝放在瓦器里,要显明这莫大的力,是出于神,不是出于我们。

我们四面受敌,却不被困住。心里作难,却不至失望。

遭逼迫,却不被丢弃。打倒了,却不至死亡。

身上常带着耶稣的死,使耶稣的生,也显明在我们身上。

因为我们这活着的人,是常为耶稣被交于死地,使耶稣的生,在我们这必死的身上显明出来,

这样看来,死是在我们身上发动,生却在你们身上发动。

但我们既有信心,正如经上记着说,我因信,所以如此说话。我们也信,所以也说话。

自己知道,那叫主耶稣复活的,也必叫我们与耶稣一同复活,并且叫我们与你们一同站在他面前。

凡事都是为你们,好叫恩惠因人多越发加增,感谢格外显多,以致荣耀归与神。

所以我们不丧胆。外体虽然毁坏,内心却一天新似一天。

我们这至暂至轻的苦楚,要为我们成就极重无比永远的荣耀。

原来我们不是顾念所见的,乃是顾念所不见的。因为所见的是暂时的,所不见的是永远的。

\chapter{哥林多后书第5章}
我们原知道,我们这地上的帐棚若拆毁了,必得神所造,不是人手所造,在天上永存的房屋。

我们在这帐棚里叹息,深想得那从天上来的房屋,好像穿上衣服。

倘若穿上,被遇见的时候就不至于赤身了。

我们在这帐棚里,叹息劳苦,并非愿意脱下这个,乃是愿意穿上那个,好叫这必死的被生命吞灭了。

为此培植我们的就是神,他又赐给我们圣灵作凭据。(原文作质)。

所以我们时常坦然无惧,并且晓得我们住在身内,便与主相离。

因我们行事为人,是凭着信心,不是凭着眼见。

我们坦然无惧,是更愿意离开身体与主同住。

所以无论是住在身内,离开身外,我们立了志向,要得主的喜悦。

因为我们众人,必要在基督台前显露出来,叫各人按着本身所行的,或善或恶受报。

我们既知道主是可畏的,所以劝人,但我们在神面前是显明的,盼望在你们的良心里,也是显明的。

我们不是向你们再举荐自己,乃是叫你们因我们有可夸之处,好对那凭外貌不凭内心夸口的人,有言可答。

我们若果颠狂,是为神。若果谨守,是为你们。

原来基督的爱激励我们。因我们想一人既替众人死,众人就都死了。

并且他替众人死,是叫那些活着的人,不再为自己活,乃为替他们死而复活的主活。

所以我们从今以后,不凭着外貌(原文作肉体本节同)认人了。虽然凭着外貌认过基督,如今不再这样认他了。

若有人在基督里,他就是新造的人。旧事已过,都变成新的了。

一切都是出于神,他藉着基督使我们与他和好,又将劝人与他和好的职分赐给我们。

这就是神在基督里叫世人与自己和好,不将他们的过犯归到他们身上。并且将这和好的道理托付了我们。

所以我们作基督的使者,就好像神藉我们劝你们一般。我们替基督求你们与神和好。

神使那无罪的(无罪原文作不知罪),替我们成为罪。好叫我们在他里面成为神的义。

\chapter{哥林多后书第6章}
我们与神同工的,也劝你们,不可徒受他的恩典。

因为他说,在悦纳的时候,我应允了你。在拯救的日子,我搭救了你。看哪,现在正是悦纳的时候,现在正是拯救的日子。

我们凡事都不叫人有妨碍,免得这职分被人毁谤。

反倒在各样的事上,表明自己是神的用人,就如在许多的忍耐,患难,穷乏,困苦,

鞭打,监禁,扰乱,勤劳,儆醒,不食,

廉洁,知识,恒忍,恩慈,圣灵的感化,无伪的爱心,

真实的道理,神的大能。仁义的兵器在左在右。

荣耀羞辱,恶名美名。似乎是诱惑人的,却是诚实的。

似乎不为人所知,却是人所共知的。似乎要死却是活着的。似乎受责罚,却是不至丧命的。

似乎忧愁,却是常常快乐的。似乎贫穷,却是叫许多人富足的。似乎一无所有,却是样样都有的。

哥林多人哪,我们向你们,口是张开的,心是宽宏的。

你们狭窄,原不在乎我们,是在乎自己的心肠狭窄。

你们也要用宽宏的心报答我。我这话升像对自己的孩子说的。

你们和不信的原不相配,不要同负一轭。义和不义有什么相交呢。光明和黑暗有什么相通呢。

基督和彼列(彼列就是撒但的别名)有什么相和呢。信主的和不信主的有什么相干呢。

神的殿和偶像有什么相同呢。因为我们是永生神的殿。就如神曾说,我要在他们中间居住,在他们中间来往。我要作他们的神,他们要作我的子民。

又说,你们务要从他们中间出来,与他们分别,不要沾不洁净的物,我就收纳你们。

我要作你们的父,你们要作我的儿女。这是全能的主说的。

\chapter{哥林多后书第7章}
亲爱的弟兄阿,我们既有这等应许,就当洁净自己,除去身体灵魂一切的污秽,敬畏神,得以成圣。

你们要心地宽大收纳我们。我们未曾亏负谁,未曾败坏谁,未曾占谁的便宜。

我说这话,不是要定你们的罪。我已经说过,你们常在我们心里,情愿与你们同生同死。

我大大的放胆向你们说话。我因你们多多夸口,满得安慰。我们在一切患难中分外的快乐。

我们从前就是到了马其顿的时候,身体也不得安宁,周围遭患难,外有争战,内有惧怕。

但那安慰丧气之人的神,藉着提多来安慰了我们。

不但藉着他来,也藉着他从你们所得的安慰,安慰了我们。因他把你们的想念,哀恸,和向我的热心,都告诉了我,叫我更加欢喜。

我先前写信叫你们忧愁。我后来虽然懊悔,如今却不懊悔。因我知道那信叫你们忧愁,不过是暂时的。

如今我欢喜,不是因你们忧愁,是因你们从忧愁中生出懊悔来。你们依着神的意思忧愁,凡事就不至于因我们受亏损了。

因为依着神的意思忧愁,就生出没有后悔的懊悔来,以致得救。但世俗的忧愁,是叫人死。

你看,你们依着神的意思忧愁,从此就生出何等的殷勤,自诉,自恨,恐惧,想念,热心,责罚(或作自责),在这一切事上你们都表明自己是洁净的。

我虽然从前写信给你们却不是为那亏负人的,也不是为那受人亏负的,乃要在神面前,把你们顾念我们的热心,表明出来。

故此我们得了安慰,并且在安慰之中,因你们众人使提多心里畅快欢喜,我们就更加欢喜了。

我若对提多夸奖了你们什么,也觉得没有惭愧。因我对提多夸奖你们的话,成了真的。正如我对你们所说的话,也都是真的。

并且提多想起你们众人的顺服,是怎样恐惧战竞的接待他,他爱你们的心肠就越发热了。

我如今欢喜,能在凡事上为你们放心。

\chapter{哥林多后书第8章}
弟兄们,我把神赐给马其顿众教会的恩告诉你们。

就是他们在患难中受大试炼的时候,仍有满足的快乐,在极穷之间,还格外显出他们乐捐的厚恩。

我可以证明他们是按着力量,而且也过了力量,自己甘心乐意的捐助。

再三的求我们,准他们在这供给圣徒的恩情上有分。

并且他们所作的,不但照我们所想望的,更照神的旨意,先把自己献给主,又归附了我们。

因此我劝提多,既然在你们中间开办这慈惠的事,就当办成了。

你们既然在信心,口才,知识,热心,和待我们的爱心上,都格外显出满足来,就当在这慈惠的事上,也格外显出满足来。

我说这话,不是吩咐你们,乃是藉着别人的热心,试验你们爱心的实在。

你们知道我们主耶稣基督的恩典。他本来富足,却为你们成了贫穷,叫你们因他的贫穷,可以成为富足。

我在这事上把我的意见告诉你们,是与你们有益。因为你们下手办这事,而且起此心意,已经有一年了。

如今就当办成这事。既有愿作的心,也当照你们所有的去办成。

因为人若有愿作的心,比蒙悦纳,乃是照他所有的,并不是照他所无的。

我原不是要别人轻省,你们受累,

乃要均平。就是要你们的富馀,现在可以补他们的不足,使他们的富馀,将来也可以补你们的不足,这就均平了。

如经上所记,多收的也没有馀,少收的也没有缺。

多谢神,感动提多的心,叫他待你们殷勤,像我一样。

他固然是听了我的劝。但自己更是热心,情愿往你们那里去。

我们还打发一位兄弟和他同去。这人在福音上得了众教会的称赞。

不但这样,他也被众教会挑选,和我们同行,把所托与我们的捐赀送到了,可以荣耀主,又表明我们乐意的心。

这就免得有人因我们收的捐银很多,就挑我们的不是。

我们留心行光明的事,不但在主面前,就在人面前,也是这样。

我们又打发一位兄弟同去。这人的热心,我们在许多事上,屡次试验过,现在他因为深信你们,就更加热心了。

论到提多,他是我的同伴,一同为你们劳碌的。论到那两位兄弟,他们是教会的使者,是基督的荣耀。

所以你们务要在众教会面前,显明你们爱心的凭据,并我所夸奖你们的凭据。

\chapter{哥林多后书第9章}
论到供给圣徒的事,我不必写信给你们。

因为我知道你们乐意的心,常对马其顿人夸奖你们,说亚该亚人豫备好了,已经有一年了。并且你们的热心激动了许多人。

但我打发那几位弟兄去,要叫你们照我的话豫备妥当。免得我们在这事上夸奖你们的话落了空。

万一有马其顿人与我同去,见你们没有豫备,就叫我们所确信的,反成了羞愧。你们羞愧,更不用说了。

因此我想不得不求那几位弟兄,先到你们那里去,把从前所应许的捐赀,豫备妥当,就显出你们所捐的,是出于乐意,不是出于勉强。

少种的少收,多种的多收。这话是真的。

各人要随本心所酌定的。不要作难,不要勉强,因为捐得乐意的人,是神所喜爱的。

神能将各样的恩惠,多多的加给你们。使你们凡事常常充足,能行各样的善事。

如经上所记,他施舍钱财,周济贫穷。他的仁义存到永远。

那赐种给撒种的,赐粮给人吃的,必多多加给你们种地的种子,又增添你们仁义的果子。

叫你们凡事富足,可以多多施舍,就藉着我们使感谢归于神。

因为办这供给的事,不但补圣徒的缺乏,而且叫许多人越发感谢神。

他们从这供给的事上得了凭据,知道你们承认基督顺服他的福音,多多的捐钱给他们和众人,便将荣耀归与神。

他们也因神极大的恩赐,显在你们心里,就切切的想念你们,为你们祈祷。

感谢神,因他有说不尽的恩赐。

\chapter{哥林多后书第10章}
我保罗就是与你们见面的时候是谦卑的,不在你们那里的时候向你们是勇敢的,如今亲自藉着基督的温柔和平,劝你们。

有人以为我是凭着血气行事,我也以为必须用勇敢待这等人,求你们不要叫我在你们那里的时候,有这样的勇敢。

因为我们虽然在血气中行事,却不凭着血气争战。

我们争战的兵器,本不是属血气的,乃是在神面前有能力可以攻破坚固的营垒,

将各样的计谋,各样拦阻人认识神的那些自高之事,一概攻破了,又将人所有的心意夺回,使他都顺服基督。

并且我已经豫备好了,等你们十分顺服的时候,要责罚那一切不顺服的人。

你们是看眼前的吗。倘若有人自信是属基督的,他要再想想,他如何属基督,我们也是如何属基督的。

主赐给我们权柄是要造就你们,并不是要败坏你们。我就是为这权柄稍微夸口,也不至于惭愧。

我说这话免得你们以为我写信是要威吓你们。

因为有人说,他的信,又沉重,又利害。及至见面,却是气貌不扬,言语粗俗的。

这等人当想,我们不在那里的时候,信上的言语如何,见面的时候,行事也必如何。

因为我们不敢将自己和那自荐的人同列相比。他们用自己度量自己,用自己比较自己,乃是不通达的。

我们不愿意分外夸口,只要照神所量给我们的界限,构到你们那里。

我们并非过了自己的界限,好像构不到你们那里。因为我们早到你们那里,传了基督的福音。

我们不仗着别人所劳碌的,分外夸口。但指望你们信心增长的时候,所量给我们的界限,就可以因着你们更加开展,

得以将福音传到你们以外的地方,并不是在别人界限之内,藉着他现成的事夸口。

但夸口的当指着主夸口。

因为蒙悦纳的,不是自己称许的,乃是主所称许的。

\chapter{哥林多后书第11章}
但愿你们宽容我这一点愚妄。其实你们原是宽容我的。

我为你们起的愤恨,原是神那样的愤恨。因为我曾把你们许配一个丈夫,要把你们如同贞洁的童女,献给基督。

我只怕你们的心或偏于邪,失去那向基督所存纯一清洁的心,就像蛇用诡诈诱惑了夏娃一样。

假如有人来,另传一个耶稣,不是我们所传过的。或着你们另受一个灵,不是你们所受过的。或者另得一个福音,不是你们所得过的。你们容让他也就吧了。

但我想,我一点不在那些最大的使徒以下。

我的言语虽然粗俗,我的知识却不粗俗。这是我们在凡事上,向你们众人显明出来的。

我因为白白传神的福音给你们,就自居卑微,叫你们高升,这算是我犯罪吗。

我亏负了别的教会,向他们取了工价来,给你们效力。

我在你们那里缺乏的时候,并没有累着你们一个人。因我所缺乏的,那从马其顿来的弟兄们都补足了。我向来凡事谨守,后来也必谨守,总不至于累着你们。

既有基督的诚实在我里面,就无人能在亚该亚一带地方阻挡我这自夸。

为什么呢。是因我不爱你们吗。这有神知道。

我现在所作的,后来还要作,为要断绝那些寻机会人的机会,使他们在所夸的事上,也不过与我们一样。

那等人是假使徒,行事诡诈,装作基督使徒的模样。

这也不足为怪。因为连撒但也装作光明的天使。

所以他的差役,若装作仁义的差役,也不算希奇。他们的结局,必然照着他们的行为。

我再说,人不可把我看作愚妄的。纵然如此,也要把我当作愚妄人接纳,叫我可以略略自夸。

我说的话,不是奉主命说的,乃是像愚妄人放胆自夸。

既有好些人凭着血气自夸,我也要自夸了。

你们既是精明的人,就能甘心忍耐愚妄人。

假若有人强你们作奴仆,或侵吞你们,或掳掠你们,或悔慢你们,或打你们的脸,你们都能忍耐他。

我说这话,是羞辱自己。好像我们从前是软弱的。然而人在何事上勇敢,(我说句愚妄话)我也勇敢。

他们是希伯来人吗。我也是。他们是以色列人吗。我也是。他们是亚伯拉罕的后裔吗。我也是。

他们是基督的仆人吗。(我说句狂话)我更是。

我比他们多受劳苦,多下监牢,受鞭打是过重的,冒死是屡次有的。被犹太人鞭打五次,每次四十,减去一下。

被棍打了三次,被石头打了一次,遇着船坏三次,一昼一夜在深海里。

又屡次行远路,遭江河的危险,盗贼的危险,同族的危险,外邦人的危险,城里的危险,旷野的危险,海中的危险,假弟兄的危险。

受劳碌,受困苦,多次不得睡,又饥又渴,多次不得食。受寒冷,赤身露体。

除了这外面的事,还有为众教会挂心的事,天天压在我身上。

有谁软弱,我不软弱呢,有谁跌倒,我不焦急呢。

我若必须自夸,就夸那关乎我软弱的事便了。

那永远可称颂之主耶稣的父神,知道我不说谎。

在大马士革亚哩达王手下的提督,把守大马士革城要捉拿我。

我就从窗户中,在筐子里从城墙上被人缒下去,脱离了他的手。

\chapter{哥林多后书第12章}
我自夸固然无益,但我是不得已的。如今我要说到主的显现和启示。

我认得一个在基督里的人,他前十四年被提到第三层天上去。或在身内,我不知道。或在身外,我也不知道。只有神知道。

我认得这人,或在身内,或在身外,我都不知道。只有神知道。

他被提到乐园里,听到隐秘的言语,是人不可说的。

为这人,我要夸口。但是为我自己,除了我的软弱以外,我并不夸口。

我就是愿意夸口,也不算狂。因为我必说实话。只是我禁止不说,恐怕有人把我看高了,过于他在我身上所看见所听见的。

又恐怕我因所得的启示甚大,就过于自高,所以有一根刺加在我肉体上,就是撒但的差役,要攻击我,免得我过于自高。

为这事,我三次求过主,叫这刺离开我。

他对我说,我的恩典够你用的。因为我的能力,是在人的软弱上显得完全。所以我更喜欢夸自己的软弱,好叫基督的能力覆庇我。

我为基督的缘故,就以软弱,凌辱,急难,逼迫,困苦,为可喜乐的。因我什么时候软弱,什么时候就刚强了。

我成了愚妄人,是被你们强逼的。我本该被你们称许才是。我算不了什么,却没有一件事在那些最大的使徒以下。

我在你们中间,用百般的忍耐,藉着神迹奇事异能,显出使徒的凭据来。

除了我不累着你们这一件事,你们还有什么事不及别的教会呢。这不公之处,求你们饶恕我吧。

如今我打算第三次到你们那里去,也必不累着你们,因我所求的是你们,不是你们的财物。儿女不该为父母积财,父母该为儿女积财。

我也甘心乐意为你们的灵魂费财费力。难道我越发爱你们,就越发少得你们的爱吗。

吧了,我自己并没有累着你们,你们却有人说,我是诡诈,用心计牢笼你们。

我所差到你们那里去的人,我藉着他们一个人占过你们的便宜吗。

我劝了提多到你们那里去,又差那位兄弟与他同去。提多占过你们的便宜吗。我们行事,不同是一个心灵吗。不同是一个脚踪吗。(心灵或作圣灵)。

你们到如今,还想我们是向你们分诉。我们本是在基督里当神面前说话。亲爱的弟兄阿,一切的事,都是为造就你们。

我怕我再来的时候,见你们不合我所想望的,你们见我也不合你们所想望的。又怕有分争,嫉妒,恼怒,结党,毁谤,才言,狂傲,混乱的事。

且怕我来的时候,我的神叫我在你们面前渐愧。又因许多人从前犯罪,行污秽奸淫邪荡的事,不肯悔改,我就忧愁。

\chapter{哥林多后书第13章}
这是我第三次要到你们那里去。凭两三个人的口作见证,句句都要定准。

我从前说过,如今不在你们那里又说,正如我第二次见你们的时候所说的一样,就是对那犯罪的,和其馀的人说,我若再来必不宽容。

你们既然寻求基督在我里面说话的凭据,我必不宽容因为基督在你们身上,不是软弱的。在你们里面,是有大能的。

他因软弱被钉在十字架上,却因神的大能,仍然活着。我们也是这样同他软弱,但因神向你们所显的大能,也必与他同活。

你们总要自己省察有信心没有。也要自己试验。岂不知你们若不是可弃绝的,就有耶稣基督在你们心里吗。

我却盼望你们晓得我们不是可弃绝的人。

我们求神,叫你们一件恶事都不作。这不是要显明我们是蒙悦纳的,是要你们行事端正,任凭人看我们是被弃绝的吧。

我们凡事不能敌挡真理,只能扶助真理。

既使我们软弱,你们刚强,我们也欢喜。并且我们所求的,就是你们作完全人。

所以我不在你们那里的时候,把这话写给你们,好叫我见你们的时候,不用照主所给我的权柄,严厉的待你们。这权柄原是为造就人,并不是为败坏人。

还有未了的话,愿弟兄们都喜乐。要作完全人。要受安慰。要同心合意。要彼此和睦。如此仁爱和平的神,必常与你们同在。

你们亲嘴问安。彼此务要圣洁。

众圣徒都问你们安。

愿主耶稣基督的恩惠,神的慈爱,圣灵的感动,常与你们众人同在。

\chapter{加拉太书第1章}
作使徒的保罗,(不是由于人,也不是藉着人,乃是藉着耶稣基督,与叫他从死里复活的父神)

和一切与我同在的众弟兄,写信给加拉太的各教会。

愿恩惠平安,从父神与我们的主耶稣基督,归与你们。

基督照我们父神的旨意为我们的罪舍己,要救我们脱离这罪恶的世代。

但愿荣耀归于神直到永永远远。阿们。

我希奇你们这吗快离开那藉着基督之恩召你们的,去从别的福音。

那并不是福音不过有些人搅扰你们,要把基督的福音更改了。

但无论是我们,是天上来的使者,若传福音给你们,与我们所传给你们的不同,他就应当被咒诅。

我们已经说了,现在又说,若有人传福音给你们,与你们所领受的不同,他就应当被咒诅。

我现在是要得人的心呢,还是要得神的心呢。我岂是讨人的喜欢吗。若仍旧讨人的喜欢,我就不是基督的仆人了。

弟兄们,我告诉你们,我素来所传的福音,不是出于人的意思。

因为我不是从人领受的,也不是人教导我的,乃是从耶稣基督启示来的。

你们听见我从前在犹太教中所行的事,怎样极力逼迫残害神的教会。

我又在犹太教中,比我本国许多同岁的人更有长进,为我祖宗的遗传更加热心。

然而那把我从母腹里分别出来,又施恩召我的神,

既然乐意将他儿子启示在我心里,叫我把他传在外邦人中,我就没有与属血气的人商量,

也没有上耶路撒冷去,见那些比我先作使徒的。惟独往阿拉伯去。后又回到大马士革。

过了三年,才上耶路撒冷去见叽法,和他同住了十五天。

至于别的使徒,除了主的兄弟雅各,我都没有看见。

我写给你们的,不是谎话,这是我在神面前说的。

以后我到了叙利亚和基利家境内。

那时,犹太信基督的各教会都没有见过我的面。

不过听说,那从前逼迫我们的,现在传扬他原先所残害的真道。

他们就为我的缘故,归荣耀给神。

\chapter{加拉太书第2章}
过了十四年,我同巴拿巴又上耶路撒冷去,并带着提多同去。

我是奉启示上去的,把我在外邦人中所传的福音,对弟兄们陈说。却是背地里对那有名望之人说的。惟恐我现在,或是从前,徒然奔跑。

但与我同去的提多,虽是希腊人,也没有勉强他受割礼。

因为有偷着引进来的假弟兄,私下窥探我们在基督耶稣里的自由,要叫我们作奴仆。

我们就是一刻的工夫,也没有容让顺服他们,为要叫福音的真理仍存在你们中间。

至于那些有名望的,不论他是何等人,都与我无干。神不以外貌取人。那些有名望的,并没有加增我什么。

反倒看见了主托我传福音给那未受割礼的人,正如托彼得传福音给那受割礼的人。

(那感动彼得,叫他为受割礼之人作使徒的,也感动我,叫我为外邦人作使徒)

又知道所赐给我的恩典,那称为教会柱石的雅各,叽法,约翰,就向我和巴拿巴用右手行相交之礼,叫我们往外邦人那里去,他们往受割礼的人那里去。

只是愿意我们记念穷人。这也是我本来热心去行的。

后来叽法到了安提阿,因他有可责之处,我就当面抵挡他。

从雅各那里来的人,未到以先,他和外邦人一同吃饭。及至他们来到,他因怕奉割礼的人,就退去与外邦人隔开了。

其馀的犹太人,也都随着他装假。甚至连巴拿巴也随夥装假。

但我一看他们行的不正,与福音的真理不合,就在众人面前对矶法说,你既是犹太人,若随外邦人行事,不随犹太人行事,怎吗还勉强外邦人随犹太人呢。

我们这生来的犹太人,不是外邦的罪人,

既知道人称义,不是因行律法,乃是因信耶稣基督,连我们也信了基督耶稣,使我们因信基督称义,不因行律法称义,因为凡有血气的,没有一人因行律法称义。

我们若求在基督里称义,却仍旧是罪人,难道基督是叫人犯罪的吗。断乎不是。

我素来所拆毁的,若重新建造,这就证明自己是犯罪的人。

我因律法就向律法死了,叫我可以向神活着。

我已经与基督同钉十字架。现在活着的,不再是我,乃是基督在我里面活着。并且我如今在肉身活着,是因信神的儿子而活,他是爱我,为我舍己。

我不废掉神的恩。义若是藉着律法得的,基督就是徒然死了。

\chapter{加拉太书第3章}
无知的加拉太人哪,耶稣基督钉十字架,已经活画在你们眼前,谁又迷惑了你们呢。

我只要问你们这一件,你们受了圣灵,是因行律法呢,是因听信福音呢。

你们既靠圣灵入门,如今还靠肉身成全吗。你们是这样的无知吗。

你们受苦如此之多,都是徒然的吗。难道果真是徒然的吗。

那赐给你们圣灵,又在你们中间行异能的,是因你们行律法呢,是因你们听信福音呢。

正如,亚伯拉罕信神,这就算为他的义。

所以你们要知道那以信为本的人,就是亚伯拉罕的子孙。

并且圣经既然豫先看明,神要叫外邦人因信称义,就早已传福音给亚伯拉罕,说,万国都必因你得福。

可见那以信为本的人,和有信心的亚伯拉罕一同得福。

凡以行律法为本的,都是被咒诅的。因为经上记着,凡不常照律法书上所记一切之事去行的,就被咒诅。

没有一个人靠着律法在神面前称义,这是明显的。因为经上说,义人必因信得生。

律法原不本乎信,只说,行这些事的,就必因此活着。

基督既为我们受了咒诅,(受原文作成)就赎出我们脱离律法的咒诅。因为经上记着,凡挂在木头上都是被咒诅的。

这便叫亚伯拉罕的福,因基督耶稣可以临到外邦人,使我们因信得着所应许的圣灵。

弟兄们,我且照着人的常话说,虽然是人的文约,若已经立定了,就没有能废弃或加增的。

所应许的原是向亚伯拉罕和他子孙说的。神并不是说众子孙,指着许多人,乃是说你那一个子孙,指着一个人,就是基督。

我是这吗说,神豫先所立的约,不能被那四百三十年以后的律法废掉,叫应许归于虚空。

因为承受产业,若本乎律法,就不本乎应许。但神是凭着应许,把产业赐给亚伯拉罕。

这样说来,律法是为什么有的呢。原是为过犯添上的,等候那蒙应许的子孙来到。并且是藉天使经中保之手设立的。

但中保本不是为一面作的。神既是一位。

这样,律法是与神的应许反对吗。断乎不是。若曾传一个能叫人得生命的律法,义就诚然本乎律法了。

但圣经把众人都圈在罪里,使所应许的福因信耶稣基督,归给那信的人。

但这因信得救的理,还未来以先,我们被看守在律法之下,直圈到那将来的真道显明出来。

这样律法是我们训蒙的师傅,引我们到基督那里,使我们因信称义。

但这因信得救的理,既然来到,我们从此就不在师傅的手下了。

所以你们因信基督耶稣,都是神的儿子。

你们受洗归入基督的,都是披戴基督了。

并不分犹太人,希腊人,自主的,为奴的,或男或女。因为你们在基督耶稣里都成为一了。

你们既属乎基督,就是亚伯拉罕的后裔,是照着应许承受产业的了。

\chapter{加拉太书第4章}
我说那承受产业的,虽然是全业的主人,但为孩童的时候,却与奴仆毫无分别。

乃在师傅和管家的手下,直等他父亲豫定的时候来到。

我们为孩童的时候,受管于世俗小学之下,也是如此。

及至时候满足,神就差遣他的儿子,为女人所生,且生在律法以下,

要把律法以下的人赎出来,叫我们得着儿子的名分。

你们既为儿子,神就差他儿子的灵,进入你们(原文作我们)的心,呼叫阿爸,父。

可见,从此以后,你不是奴仆,乃是儿子了。既是儿子,就靠着神为后嗣。

但从前你们不认识神的时候,是给那些本来不是神的作奴仆。

现在你们既然认识神,更可说是被神所认识的,怎吗还要归回那懦弱无用的小学,情愿再给他作奴仆呢。

你们谨守日子,月分,节期,年分。

我为你们害怕。惟恐我在你们身上是枉费了工夫。

弟兄们,我劝你们要向我一样,因为我也像你们一样。你们一点没有亏负我。

你们知道我头一次传福音给你们,是因为身体有疾病。

你们为我身体的缘故受试炼,没有轻看我,也没有厌弃我。反倒接待我,如同神的使者,如同基督耶稣。

你们当日所夸的福气在那里呢。那时你们若能行,就是把自己的眼睛剜出来给我,也都情愿。这是我可以给你们作见证的。

如今我将真理告诉你们,就成了你们的仇敌吗。

那些人热心待你们,却不是好意,是要离间(原文作把你们关在外面)你们,叫你们热心待他们。

在善事上,常用热心待人,原是好的,却不单我与你们同在的时候才这样。

我小子阿,我为你们再受生产之苦,直等到基督成形在你们心里。

我巴不得现今在你们那里,改换口气,因我为你们,心里作难。

你们这愿意在律法以下的人,请告诉我,你们岂没有听见律法吗。

因为律法上记着,亚伯拉罕有两个儿子,一个是使女生的,一个是自主之妇人生的。

然而那使女所生的,是按着血气生的。那自主之妇人生的,是凭着应许生的。

这都是比方。那两个妇人,就是两约。一约是出于西奈山,生子为奴,乃是夏甲。

这夏甲二字是指着阿拉伯的西奈山,与现在的耶路撒冷同类。因耶路撒冷和他的儿女都是为奴的。

但那在上的耶路撒冷是自主的,他是我们的母。

因为经上记着,不怀孕不生养的,你要欢乐。未曾经过产难的,你要高声欢呼,因为没有丈夫的,比有丈夫的儿女更多。

弟兄们,我们是凭着应许作儿女,如同以撒一样。

当时那按着血气生的,逼迫了那按着圣灵生的。现在也是这样。

然而经上是怎吗说的呢。是说,把使女和他儿子赶出去,因为使女的儿子,不可与自主妇人的儿子一同承受产业。

弟兄们,这样看来,我们不是使女的儿女,乃是自主妇人的儿女了。

\chapter{加拉太书第5章}
基督释放了我们,叫我们得自由,所以要站立得稳,不要再被奴仆的轭挟制。

我保罗告诉你们,若受割礼,基督就与你们无益了。

我再指着凡受割礼的人确实的说,他是欠着行全律法的债。

你们这要靠律法称义的,是与基督隔绝,从恩典中坠落了。

我们靠着圣灵,凭着信心,等候所盼望的义。

原来在基督耶稣里,受割礼不受割礼,全无功效。惟独使人生发仁爱的信心,才有功效。

你们向来跑得好。有谁拦阻你们,叫你们不顺从真理呢。

这样的劝导,不是出于那召你们的。

一点面酵能使全团都发起来。

我在主里很信你们必不怀别样的心,但搅扰你们的,无论是谁,必担当他的罪名。

弟兄们,我若仍旧传割礼,为什么还受逼迫呢。若是这样,那十字架讨厌的地方就没有了。

恨不得那搅乱你们的人,把自己割绝了。

弟兄们,你们蒙召,是要得自由。只是不可将你们的自由当作放纵情欲的机会。总要用爱心互相服事。

因为全律法都包在爱人如己这一句话之内了。

你们要谨慎。若相咬相吞,只怕要彼此消灭了。

我说,你们当顺着圣灵而行,就不放纵肉体的情欲了。

因为情欲和圣灵相争,圣灵和情欲相争。这两个是彼此相敌,使你们不能作所愿意作的。

但你们若被圣灵引导,就不在律法以下。

情欲的事,都是显而易见的。就如奸淫,污秽,邪荡,

拜偶像,邪术,仇恨,争竞,忌恨,恼怒,结党,纷争,异端,

嫉妒,(有古卷在此有凶杀二字)醉酒,荒宴等类,我从前告诉你们,现在又告诉你们,行这样事的人,必不能承受神的国。

圣灵所结的果子,就是仁爱,喜乐,和平,忍耐,恩慈,良善,信实,

温柔,节制。这样的事,没有律法禁止。

凡属基督耶稣的人,是已经把肉体,连肉体的邪情私欲,同钉在十字架上了。

我们若是靠圣灵得生,就当靠圣灵行事。

不要贪图虚名,彼此惹气,互相嫉妒。

\chapter{加拉太书第6章}
弟兄们,若有人偶然被过犯所胜,你们属灵的人,就应当用温柔的心,把他挽回过来。又当自己小心,恐怕也被引诱。

你们各人的重担要互相担当,如此就完全了基督的律法。

人若无有,自己还以为有,就是自欺了。

各人应当察验自己的行为,这样,他所夸的就专在自己,不在别人了。

因为各人必担当自己的担子。

在道理上受教的,当把一切需用的供给施教的人。

不要自欺,神是轻慢不得的。人种的是什么,收的也是什么。

顺着情欲撒种的,必从情欲收败坏顺着圣灵撒种的,必从圣灵收永生。

我们行善,不可丧志。若不灰心,到了时候,就要收成。

所以有了机会,就当向众人行善。向信徒一家的人更当这样。

请看我亲手写给你们的字,是何等的大呢。

凡希图外貌体面的人,都勉强你们受割礼。无非是怕自己为基督的十字架受逼迫。

他们那些受割礼的,连自己也不守律法。他们愿意你们受割礼,不过要藉着你们的肉体夸口。

但我断不以别的夸口,只夸我们主耶稣基督的十字架。因这十字架,就我而论,世界已经钉在十字架上。就世界而论,我已经钉在十字架上。

受割礼不受割礼都无关紧要,要紧的就是作新造的人。

凡照此理而行的,愿平安怜悯加给他们,和神的以色列民。

从今以后,人都不要搅扰我。因为我身上带着耶稣的印记。

弟兄们,愿我主耶稣基督的恩常在你们心里。阿们。

\chapter{以弗所书第1章}
奉神旨意,作基督耶稣使徒的保罗,写信给在以弗所的圣徒,就是在基督耶稣里有忠心的人。

愿恩惠平安,从神我们的父,和主耶稣基督,归与你们。

愿颂赞归与我们主耶稣基督的父神,他在基督里,曾赐给我们天上各样属灵的福气。

就如神从创立世界以前,在基督里拣选了我们,使我们在他面前成为圣洁,无有瑕疵。

又因爱我们,就按着自己意旨所喜悦的,豫定我们,藉着耶稣基督得儿子的名分,

使他荣耀的恩典得着称赞。这恩典是他在爱子里所赐给我们的。

我们藉这爱子的血,得蒙救赎,过犯得以赦免,乃是照他丰富的恩典。

这恩典是神用诸般智慧聪明,充充足足赏给我们的,

都是照他自己所豫定的美意,叫我们知道他旨意的奥秘,

要照所安排的,在日期满足的时候,使天上地上一切所有的,都在基督里同归于一。

我们也在他里面得了基业,(得或作成)这原是那位随己意行作万事的,照着他旨意所豫定的。

叫他的荣耀,从我们这首先在基督里有盼望的人,可以得着称赞。

你们既听见真里的道,就是那叫你们得救的福音,也信了基督,既然信他,就受了所应许的圣灵为印记。

这圣灵,是我们得基业的凭据,(原文作质)直等到神之民(民原文作产业)被赎,使他的荣耀得着称赞。

因此,我既听见你们信从主耶稣,亲爱众圣徒,

就为你们不住的感谢神,祷告的时候,常题到你们。

求我们主耶稣基督的神,荣耀的父,将那赐人智慧和启示的灵,赏给你们,使你们真知道他。

并且照明你们心中的眼睛,使你们知道他的恩召有何等指望。他在圣徒中得的基业,有何等丰盛的荣耀。

并知道他向我们这信的人所显的能力,是何等浩大,

就是照他在基督身上,所运行的大能大力,使他从死里复活,叫他在天上坐在自己的右边,

远超过一切执政的,掌权的,有能的,主治的,和一切有名的。不但是今世的,连来世的也都超过了。

又将万有服在他的脚下,使他为教会作万有之首。

教会是他的身体,是那充满万有者所充满的。

\chapter{以弗所书第2章}
你们死在过犯罪恶之中,他叫你们活过来,

那时你们在其中行事为人随从今世的风俗,顺服空中掌权者的首领,就是现今在悖逆之子心中运行的邪灵。

我们从前也都在他们中间,放纵肉体的私欲,随着肉体和心中所喜好的去行,本为可怒之子,和别人一样。

然而神既有丰富的怜悯。因他爱我们的大爱,

当我们死在过犯中的时候,便叫我们与基督一同活过来。(你们得救是本乎恩)

他又叫我们与基督耶稣一同复活,一同坐在天上,

要将他极丰富的恩典,就是他在基督耶稣里向我们所施的恩慈,显明给后来的世代看。

你们得救是本乎恩,也因着信,这并不是出于自己,乃是神所赐的。

也不是出于行为,免得有人自夸。

我们原是他的工作,在基督耶稣里造成的,为要叫我们行善,就是神所豫备叫我们行的。

所以你们应当记念,你们从前按肉体是外邦人,是称为没受割礼的,这名原是那些凭人手在肉身上称为受割礼之人所起的。

那时你们与基督无关,在以色列国民以外,在所应许的诸约上是局外人。并且活在世上没有指望,没有神。

你们从前远离神的人,如今却在基督耶稣里,靠着他的血,已经得亲近了。

因他使我们和睦,(原文作因他是我们的和睦)将两下合而为一,拆毁了中间隔断的墙。

而且以自己的身体,废掉冤仇,就是那记在律法上的规条。为要将两下,藉着自己造成一个新人,如此便成就了和睦。

既在十字架上灭了冤仇,便藉这十字架,使两下归为一体,与神和好了。

并且来传和平的福音给你们远处的人,也给那近处的人。

因为我们两下藉着他被一个圣灵所感得以进到父面前。

这样,你们不再作外人,和客旅,是与圣徒同国,是神家里的人了。

并且被建造在使徒和先知的根基上,有基督耶稣自己为房角石。

各(或作全)房靠他联络得合式,渐渐成为主的圣殿。

你们也靠他同被建造成为神藉着圣灵居住的所在。

\chapter{以弗所书第3章}
因此,我保罗为你们外邦人作了基督耶稣被囚的,替你们祈祷(此句乃照对十四节所加)

掠必你们曾听见神赐恩给我,将关切你们的职分托付我,

用启示使我知道福音的奥秘,正如我以前略略写过的。

你们念了,就能晓得我是深知基督的奥秘。

这奥秘在以前的世代,没有叫人知道,像如今藉着圣灵启示他的圣使徒和先知一样。

这奥秘就是外邦人在基督耶稣里,藉着福音,得以同为后嗣,同为一体,同蒙应许。

我作了这福音的执事,是照神的恩赐。这恩赐是照他运行的大能赐给我的。

我本来比众圣徒中最小的还小。然而他还赐我这恩典,叫我把基督那测不透的丰富,传给外邦人。

又使众人都明白,这历代以来隐藏在创造万物之神里的奥秘,是如何安排的。

为要藉着教会使天上执政的,掌权的,现在得知神百般的智慧。

这是照神从万世以前,在我们主基督耶稣里所定的旨意。

我们因信耶稣,就在他里面放胆无惧,笃信不疑的来到神面前。

所以我求你们,不要因我为你们所受的患难丧胆。这原是你们的荣耀。

因此,我在父面前屈膝,

(天上地上的各(或作全)家,都是从他得名)

求他按着他丰盛的荣耀,藉着他的灵,叫你们心里的力量刚强起来,

使基督因你们的信,住在你们心里,叫你们的爱心,有根有基,

能以和众圣徒一同明白基督的爱,是何等长阔高深,

并且知道这爱是过于人所能测度的,便叫神一切所充满的,充满了你们。

神能照着运行在我们心里的大力,充充足足的成就一切超过我们所求所想的。

但愿他在教会中,并在基督耶稣里,得着荣耀,直到世世代代,永永远远。阿们。

\chapter{以弗所书第4章}
我为主被囚的劝你们,既然蒙召,行事为人就当与蒙召的恩相称。

凡事谦虚,温柔,忍耐,用爱心互相宽容,

用和平彼此联络,竭力保守圣灵所赐合而为一的心。

身体只有一个,圣灵只有一个,正如你们蒙召,同有一个指望,

一主,一信,一洗,

一神,就是众人的父,超乎众人之上,贯乎众人之中,也住在众人之内。

我们各人蒙恩,都是照基督所量给各人的恩赐。

所以经上说,他升上高天的时候,掳掠了仇敌,将各样的恩赐赏给人。

(既说升上,岂不是先降在地下吗。

那降下的,就是远升诸天之上要充满万有的

他所赐的有使徒,有先知。有传福音的。有牧师和教师。

为要成全圣徒,各尽其职,建立基督的身体。

直等到我们众人在真道上同归于一,认识神的儿子,得以长大成人,满有基督长成的身量。

使我们不再作小孩子,中了人的诡计,和欺骗的法术,被一切异教之风摇动,飘来飘去,就随从各样的异端。

惟用爱心说诚实话,凡事长进,连于元首基督。

全身都靠他联络得合式,百节各按各职,照着各体的功用,彼此相助,便叫身体渐渐增长,在爱中建立自己。

所以我说,且在主里确实的说,你们行事,不要再像外邦人存虚妄的心行事。

他们心地昏昧,与神所赐的生命隔绝了,都因自己无知,心里刚硬。

良心既然丧尽,就放纵私欲,贪行种种的污秽。

你们学了基督,却不是这样。

如果你们听过他的道,领了他的教,学了他的真理,

就要脱去你们从前行为上的旧人。这旧人是因私欲的迷惑,渐渐变坏的。

又要将你们的心志改换一新。

并且穿上新人。这新人是照着神的形像造的,有真里的仁义,和圣洁。

所以你们要弃绝谎言,各人与邻舍说实话。因为我们是互相为肢体。

生气却不要犯罪。不可含怒到日落。

也不可给魔鬼留地步。

从前偷窃的,不要再偷。总要劳力,亲手作正经事,就可有馀分给那缺少的人。我们工作不谨是为我们自己,也应当为别人。

污秽的言语,一句不可出口,只要随事说造就人的好话,叫听见的人得益处。

不要叫神的圣灵担忧。你们原是受了他的印记,等候得赎的日子来到。

一切苦毒,脑恨,忿怒,囔闹,毁谤,并一切的恶毒,(或作阴毒)都当从你们中间除掉。

并要以恩慈相待,存怜悯的心,彼此饶恕,正如神在基督里饶恕了你们一样。

\chapter{以弗所书第5章}
所以你们该效法神,好像蒙慈爱的儿女一样。

也要凭爱心行事,正如基督爱我们,为我们舍了自己,当作馨香的供物,和祭物,献与神。

至于淫乱,并一切污秽,或是贪婪,在你们中间连题都不可,方合圣徒的体统。

淫词,妄语,和戏笑的话,都不相宜,总要说感谢的话。

因为你们确实的知道,无论是淫乱的,是污秽的,是有贪心的,在基督和神的国里,都是无分的。有贪心的,就与拜偶像的一样。

不要被人虚浮的话欺哄。因这些事,神的忿怒必临到那悖逆之子。

所以你们不要与他们同夥。

从前你们是暗昧的,但如今在主里面是光明的,行事为人就当像光明的子女。

光明所结的果子,就是一切良善,公义,诚实。

总要察验何为主所喜悦的事。

那暗昧无益的事,不要与人同行,倒要责备行这事的人

因为他们暗中所行的,就是题起来,也是可耻的。

凡事受了责备,就被光显明出来。因为一切能显明的,就是光。

所以主说,你这睡着的人,当醒过来,从死里复活,基督就要光照你了。

你们要谨慎行事,不要像愚昧人,当像智慧人。

要爱惜光阴,因为现今的世代邪恶。

不要作糊涂人,要明白主的旨意如何。

不要醉酒,酒能使人放荡,乃要被圣灵充满。

当用诗章,颂词,灵歌,彼此对说,口唱心和的赞美主。

凡事要奉我们主耶稣基督的名,常常感谢父神。

又当存敬畏基督的心,彼此顺服。

你们作妻子的,当顺服自己的丈夫,如同顺服主。

因为丈夫是妻子的头,如同基督是教会的头。他又是教会全体的救主。

教会怎样顺服基督,妻子也要怎样凡事顺服丈夫。

你们作丈夫的,要爱你们的妻子,正如基督爱教会,为教会舍己。

要用水藉着道,把教会洗净,成为圣洁,

可以献给自己,作个荣耀的教会,毫无玷污皱纹等类的病,乃是圣洁没有瑕疵的。

丈夫也当照样爱妻子,如同爱自己的身子。爱妻子,便是爱自己了。

从来没有人恨恶自己的身子,总是保养顾惜,正像基督待教会一样。

因我们是他身上的肢体。(有古卷在此有就是他的骨他的肉)。

为这个缘故,人要离开父母,与妻子连合,二人成为一体。

这是极大的奥秘,但我是指着基督和教会说的。

然而你们各人都当爱妻子,如同爱自己一样。妻子也当敬重他的丈夫。

\chapter{以弗所书第6章}
你们作儿女的,要在主里听从父母,这是理所当然的。

要孝敬父母,使你们得福,在世长寿。

*这是第一条带应许的诫命。

你们作父亲的,不要惹儿女的气,只要照着主的教训和警戒,养育他们。

你们作仆人的,要惧怕战兢,用诚实的心听从你们肉身的主人,好像听从基督一般。

不要只在眼前事奉,像是讨人喜欢的,要像基督的仆人,从心里遵行神的旨意。

甘心事奉,好像服事主,不像服事人。

因为晓得各人所行的善事,不论是为奴的,是自主的,都必按所行的得主的赏赐

你们作主人的待仆人,也是一理,不要威吓他们。因为知道他们和你们,同有一位主在天上,他并不偏待人。

我还有末了的话,你们要靠着主,倚赖他的大能大力,作刚强的人。

要穿戴神所赐的全副军装,就能抵挡魔鬼的诡计。

因我们并不是与属血气的争战,乃是与那些执政的,掌权的,管辖这幽暗世界的,以及天空属灵气的恶魔争战。(两争战原都作摔跤)

所以要拿起神所赐的全副军装,好在磨难的日子,抵挡仇敌,并且成就了一切,还能站立得住。

所以要站稳了,用真理当作带子束腰,用公义当作护心镜遮胸。

又用平安的福音,当作豫备走路的鞋穿在脚上,

此外又拿着信德当作藤牌,可以灭尽那恶者一切的火箭。

并戴上救恩的头盔,拿着圣灵的宝剑,就是神的道。

靠着圣灵,随时多方祷告祈求,并要在此儆醒不倦,为众圣徒祈求,

也为我祈求,使我得着口才,能以放胆,开口讲明福音的奥秘,

(我为这福音的奥秘,作了带锁链的使者)并使我照着当尽的本分,放胆讲论。

今有所亲爱忠心事奉主的兄弟推基古,他要把我的事情并我的景况如何,全告诉你们叫你们知道。

我特意打发他到你们那里去,好叫你们知道我们的光景,又叫他安慰你们的心。

愿平安,仁爱,信心,从父神和主耶稣基督,归与弟兄。

并愿所有诚心爱我们主耶稣基督的人,都蒙恩惠。

\chapter{腓立比书第1章}
基督耶稣的仆人保罗,和提摩太,写信给凡住腓立比,在基督耶稣里的众圣徒,和诸位监督,诸位执事。

愿恩惠平安,从神我们的父,并主耶稣基督,归与你们。

我每逢想念你们,就感谢我的神。

(每逢为你们众人祈求的时候,常是欢欢喜喜的祈求)

因为从头一天直到如今,你们是同心合意的兴旺福音。

我深信那在你们心里动了善工的,必成全这工,直到耶稣基督的日子。

我为你们众人有这样的意念,原是应当的。因你们常在我心里,无论我是在捆锁之中,是辩明证实福音的时候,你们都与我一同得恩。

我体会基督耶稣的心肠,切切的想念你们众人。这是神可以给我作见证的。

我所祷告的,就是要你们的爱心,在知识和各样见识上,多而又多。

使你们能分别是非,(或作喜爱那美好的事)作诚实无过的人,直到基督的日子。

并靠着耶稣基督结满了仁义的果子,叫荣耀称赞归与神。

弟兄们,我愿意你们知道,我所遭遇的事,更是叫福音兴旺。

以致我受的捆锁,在御营全军,和其馀的人中,已经显明是为基督的缘故。

并且那在主里的弟兄,多半因我受的捆锁,就笃信不疑,越发放胆传神的道,无所惧怕。

有的传基督,是出于嫉妒分争。也有的是出于好意。

这一等是出于爱心,知道我是为辩明福音设立的。

那一等传基督是出于结党,并不诚实,意思要加增我捆锁的苦楚。

这有何防呢。或是假意,或是真心,无论怎样,基督究竟被传开了。为此我就欢喜,并且还要欢喜。

因为我知道这事藉着你们的祈祷,和耶稣基督之灵的帮助,终必叫我得救。

照着我所切慕所盼望的,没有一事教我羞愧,只要凡事放胆。无论是生,是死,总叫基督在我身上照常显大。

因我活着就是基督,我死了就有益处。

但我在肉身活着,若成就我工夫的果子,我就不知道该挑选什么。

我正在两难之间,情愿离世与基督同在。因为这是好得无比的。

然而我在肉身活着,为你们更是要紧的。

我既然这样深信,就知道仍要住在世间,且与你们众人同住,使你们腓01:25)在所信的道上。又长进又喜乐。

叫你们在基督耶稣里的欢乐,因我再到你们那里去,就越发加增。

只要你们行事为人与基督的福音相称。叫我或来见你们,或不在你们那里,可以听见你们的景况,知道你们同有一个心志,站立得稳,为所信的福音齐心努力。

凡事不怕敌人的惊吓。这是证明他们沉沦,你们得救,都是出于神。

因为你们蒙恩,不但得以信服基督,并要为他受苦。

你们的争战,就与你们在我身上从前所看见,现在所听见的一样。

\chapter{腓立比书第2章}
所以在基督里若有什么劝勉,爱心有什么安慰,圣灵有什么交通,心中有什么慈悲怜悯,

你们就要意念相同,爱心相同,有一样的心思,有一样的意念,使我们的喜乐可以满足。

凡事不可结党,不可贪图虚浮的荣耀。只要存心谦卑,各人看别人比自己强。

各人不要单顾自己的事,也要顾别人的事。

你们当以基督耶稣的心为心。

他本有神的行像,不以自己与神同等为强夺的。

凡倒虚己,取了奴仆的形像,成为人的样式。

既有人的样子,就自己卑微,存心顺服,以至于死,且死在十字架上。

所以神将他升为至高,又赐给他那超乎万名之上的名,

叫一切在天上的,地上的,和地底下的,因耶稣的名,无不屈膝,

无不口称耶稣基督为主,使荣耀归与父神。

这样看来,我亲爱的弟兄你们既是常顺服的,不但我在你们那里,就是我如今不在你们那里,更是顺服的,就当恐惧战兢,作成你们得救的工夫。

因为你们立志行事,都是神在你们心里运行,为要成就他的美意。

凡所行的,都不要发怨言,起争论,

使你们无可指摘,诚实无伪,在这弯曲悖谬的世代,作神无瑕无疵的儿女。你们显在这世代中,好像明光照耀,

将生命的道表明出来,叫我在基督的日子,好夸我没有空跑,也没有徒劳。

我以你们的信心为供献的祭物。我若被浇奠在其上,也是喜乐,并且与你们众人一同喜乐。

你们也要照样喜乐,并且与我一同喜乐。

我靠主耶稣指望快打发提摩太去见你们,叫我知道你们的事,心里就得着安慰。

因为我没有别人与我同心,实在挂念你们的事。

别人都求自己的事,并不求耶稣基督的事。

但你们知道提摩太的明证,他兴旺福音与我同劳,待我像儿子待父亲一样。

所以我一看出我的事要怎样了结,就盼望立刻打发他去。

但我靠着主,自信我也必快去。

然而我想必须打发以巴弗提到你们那里去。他是我的兄弟,与我一同作工,一同当兵,是你们所差遣的,也是供给我需用的。

他很想念你们众人,并且极其难过,因为你们听见他病了。

他实在是病了,几乎要死。然而神怜悯他,不但怜悯他,也怜悯我,免得我忧上加忧。

所以我越发急速打发他去,叫你们再见他,就可以喜乐,我也可以少些忧愁。

故此你们要在主里欢欢乐乐的接待他。而且要尊重这样的人。

因他为作基督的工夫,几乎至死,不顾性命,要补足你们供给我不及之处。

\chapter{腓立比书第3章}
弟兄们,我还有话说,你们要靠主喜乐。我把这话再写给你们,于我并不为难,于你们却是妥当。

应当防备犬类,防备作恶的,防备妄自行割的。

因为真受割礼的,乃是我们这以神的灵敬拜,在基督耶稣里夸口,不靠着肉体的。

其实我也可以靠肉体。若是别人想他可以靠肉体,我更可以靠着了。

我第八天受割礼,我是以色列族,便雅悯支派的人,是希伯来人所生的希伯来人。就律法说,我是法利赛人。

就热心说,我是逼迫教会的。就律法上的义说,我是无可指摘的。

只是我先前以为与我有益的,我现在因基督都当作有损的。

不但如此,我也将万事都当有损的,因我以认识我主基督耶稣为至宝。

我为他已经丢弃万事,看作粪土,为要得着基督。并且得以在他里面,不是有自己因律法而得的义,乃是有信基督的义,就是因信神而来的义。

使我认识基督,晓得他复活的大能,并且晓得和他一同受苦,效法他的死。

或者我也得以从死里复活。

这不是说,我已经得着了,已经完全了。我乃是竭力追求,或者可以得着基督耶稣所以得着我的。(所以得着我的或作所要我得的)

弟兄们,我不是以为自己已经得着了。我只有一件事,就是忘记背后努力面前的,

向着标竿直跑,要得神在基督耶稣里从上面召我来得的奖赏。

所以我们中间凡是完全人,总要存这样的心。若在什么事上,存别样的心,神也必以此指示你们。

然而我们到了什么地步,就当照着什么地步行。

弟兄们,你们要一同效法我,也当留意看那些照我们榜样行的人。

因为有许多人行事,是基督十字架的仇敌。我屡次告诉你们,现在又流泪的告诉你们。

他们的结局就是沉沦,他们的神就是自己的肚腹,他们以自己的羞辱为荣耀,专以地上的事为念。

我们却是天上的国民。并且等候救主,就是主耶稣基督,从天上降临。

他要按着那能叫万有归服自己的大能,将我们这卑贱的身体改变形状,和他自己荣耀的身体相似。

\chapter{腓立比书第4章}
我所亲爱所想念的弟兄们,你们就是我的喜乐,我的冠冕。我亲爱的弟兄,你们应当靠主站立得稳。

我劝友阿爹和循都基,要在主里同心。

我也求你这真实同负一轭的,帮助这两个女人,因为他们在福音上曾与我一同劳苦。还有革利免,并其馀和我一同作工的。他们的名字都在生命册上。

你们要靠主常常喜乐。我再说,你们要喜乐。

当叫众人知道你们谦让的心。主已经近了。

应当一无挂虑,只要凡事藉着祷告,祈求,和感谢,将你们所要的告诉神。

神所赐出人意外的平安,必在基督耶稣里,保守你们的心怀意念。

弟兄们,我还有未尽的话。凡是真实的,可敬的,公义的,清洁的,可爱的,有美名的。若有什么德行,若有什么称赞,这些事你们都要思念。

你们在我身上所学习的,所领受的,所听见的,所看见的,这些事你们都要去行。赐平安的神,就必与你们同在。

我靠主大大的喜乐,因为你们思念我的心,如今又发生。你们向来就思念我,只是没有机会。

我并不是因缺乏说这话,我无论在什么景况,都可以知足,这是我已经学会了。

我知道怎样处卑贱,也知道怎样处丰富,或饱足,或饥饿,或有馀,或缺乏,随事随在,我都得了秘诀。

我靠着那加给我力量的,凡事都能作。

然而你们和我同受患难,原是美事。

腓立比人哪,你们也知道我初传福音,离了马其顿的时候,论到授受的事,除了你们以外,并没有别的教会供给我。

就是我在帖撒罗尼迦,你们也一次两次的,打发人供给我的需用。

我并不求什么馈送,所求的就是你们的果子渐渐增多,归在你们的账上。

但我样样都有,并且有馀。我已经充足,因我从以巴弗提受了你们的馈送,当作极美的香气,为神所收纳所喜悦的祭物。

我的神必照他荣耀的丰富,在基督耶稣里,使你们一切所需用的都充足。

愿荣耀归给我们的父神,直到永永远远。阿们。

请问在基督耶稣里的各位圣徒安。在我这里的众弟兄都问你们安。

众圣徒都问你们安。在凯撒家里的人特特的问你们安。

愿主耶稣基督的恩常在你们心里。

\chapter{歌罗西书第1章}
奉神旨意,作基督耶稣使徒的保罗,和兄弟提摩太,

写信给歌罗西的圣徒,在基督里有忠心的弟兄。愿恩惠平安,从神我们的父,归与你们。

我们感谢神我们主耶稣基督的父,常常为你们祷告。

因听见你们在基督耶稣里的信心,并向众圣徒的爱心。

是为那给你们存在天上的盼望。这盼望就是你们从前在福音真理的道上所听见的。

这福音传到你们那里,也传到普天之下,并且结果增长,如同在你们中间,自从你们听见福音,真知道神恩惠的日子一样。

正如你们从我们所亲爱,一同作仆人的以巴弗所学的。他为我们(有古卷你们)作了基督忠心的执事。

也把你们因圣灵所存的爱心告诉了我们。

因此,我们自从听见的日子,也就为你们不住的祷告祈求,愿你们在一切属灵的智慧悟性上,满心知道神的旨意。

好叫你们行事为人对得起主,凡事蒙他喜悦,在一切善事上结果子,渐渐的多知道神。

照他荣耀的权能,得以在各样的力上加力,好叫你们凡事欢欢喜喜的忍耐宽容。

又感谢父,叫我们能与众圣徒在光明中同得基业。

他救了我们脱离黑暗的权势,把我们迁到他爱子的国里。

我们在爱子里得蒙救赎,罪过得以赦免。

爱子是那不能看见之神的像,是首生的,在一切被造的以先。

因为万有都是靠他造的,无论是天上的,地上的,能看见的,不能看见的,或是有位的,主治的,执政的,掌权的,一概都是藉着他造的,又是为他造的。

他在万有之先,万有也靠他而立。

他也是教会全体之首。他是元始,是从死里首先复生的,使他可已在凡事上居首位。

因为父喜欢叫一切的丰盛,在他里面居住。

既然藉着他在十字架上所流的血,成就了和平,便藉着他叫万有,无论是地上的,天上的,都与自己和好了。

你们从前与神隔绝,因着恶行,心里与他为敌。

但如今他藉着基督的肉身受死,叫你们与自己和好,都成了圣洁,没有瑕疵,无可责备,把你们引到自己面前。

只要你们在所信的道上恒心,根基稳固,坚定不移,不至被引动失去原文作离开福音的盼望。这福音就是你们所听过的,也是传与普天下万人听的。万人原文作凡受造的我保罗也作了这福音的执事。

现在我为你们受苦,倒觉欢乐,并且为基督的身体,就是为教会,要在我肉身上补满基督患难的缺欠。

我照神为你们所赐我的职分,作了教会的执事,要把神的道理传得全备。

这道理就是历世历代所隐藏的奥秘,但如今向他的圣徒显明了。

神愿意叫他们知道,这奥秘在外邦人中有何等丰盛的荣耀。就是基督在你们心里成了有荣耀的盼望。

我们传扬他,是用诸般的智慧,劝戒各人,教导各人。要把各人在基督里完完全全的引到神面前。

我也为此劳苦,照着他在我里面运用的大能,尽心竭力。

\chapter{歌罗西书第2章}
我愿意你们晓得我为你们和老底嘉人,并一切没有与我亲自见面的人,是何等的尽心竭力。

要叫他们的心得安慰,因爱心互相联络,以致丰丰足足在悟性中有充足的信心,使他们真知道神的奥秘,就是基督。

所积蓄的一切智慧知识,都在他里面藏着。

我说这话,免得有人用花言巧语迷惑你们。

我身子虽与你们相离,心却与你们同在,见你们循规蹈矩,信基督的心也坚固,我就欢喜了。

你们既然接受了主基督耶稣,就当遵他而行。

在他里面生根建造,信心坚固,正如你们所领的教训,感谢的心也更增长了。

你们要谨慎,恐怕有人用他的理学,和虚空的妄言,不照着基督,乃照人间的遗传,和世上的小学,就把你们掳去。

因为神本性一切的丰盛,都有形有体的居住在基督里面。

你们在他里面也得了丰盛。他是各样执政掌权者的元首。

你们在他里面,也受了不是人手所行的割礼,乃是基督使你们脱去肉体情欲的割礼。

你们既受洗与他一同埋葬,也就在此与他一同复活。都因信那叫他从死里复活神的功用

你们从前在过犯,和未受割礼的肉体中死了,神赦免了你们(或作我们)一切过犯,便叫你们与基督一同活过来。

又涂抹了在律例上所写,攻击我们有碍于我们的字据,把他撤去,钉在十字架上。

既将一切执政的掌权的掳来,明显给众人看,就仗着十字架夸胜。

所以不拘在饮食上,或节期,月朔,安息日,都不可让人论断你们。

这些原是后事的影儿。那形体却是基督。

不可让人因着故意谦虚,和敬拜天使,就夺去你们的奖赏。这等人拘泥在所见过的,(有古卷作这等人窥察所没有见过的)随着自己的欲心,无故的自高自大,

不持定元首,全身既然靠着他筋节得以相助联络,就因神大得长进。

你们若是与基督同死,脱离了世上的小学,为什么仍像在世俗中活着,

服从那不可拿,不可尝,不可摸,等类的规条呢。

这都是照人所吩咐所教导的。说到这一切正用的时候就都败坏了。

这些规条,使人徒有智慧之名,用私意崇拜,自表谦卑,苦待己身,其实在克制肉体的情欲上,是毫无功效。

\chapter{歌罗西书第3章}
所以你们若真与基督一同复活,就当求在上面的事。那里有基督坐在神的右边。

你们要思念上面的事,不要思念地上的事。

因为你们已经死了,你们的生命与基督一同藏在神里面。

基督是我们的生命,他显现的时候,你们也要与他一同显现在荣耀里。

所以要治死你们在地上的肢体。就如淫乱,污秽,邪情,恶欲,和贪婪,贪婪就与拜偶像一样。

因这些事,神的忿怒必临到那悖逆之子。

当你们在这些事中活着的时候,也曾这样行过。

但现在你们要弃绝这一切的事,以及恼恨,忿怒,恶毒,(或作阴毒)毁谤,并口中污秽的言语。

不要彼此说谎,因你们已经脱去旧人和旧人的行为,

穿上了新人。这新人在知识上渐渐更新,正如造他主的形像。

在此并不分希腊人,犹太人,受割礼的,未受割礼的,化外人,西古提人,为奴的,自主的,惟有基督是包括一切,又住在各人之内。

所以你们既是神的选民,圣洁蒙爱的人,就要存(原文作穿下同)怜悯,恩慈,谦虚,温柔,忍耐的心。

倘若这人与那人有嫌隙,总要彼此包容,彼此饶恕。主怎样饶恕了你们,你们也要怎样饶恕人。

在这一切之外,要存着爱心。爱心就是联络全德的。

又要叫基督的平安在你们心里作主。你们也为此蒙召,归为一体。且要存感谢的心。

当用各样的智慧,把基督的道里,丰丰富富的存在心里,(或作当把基督的道里丰丰富富的存在心里以各样的智慧)用诗章,颂词,灵歌,彼此教导,互相劝戒,心被恩感歌颂神。

无论作什么,或说话,或行事,都要奉主耶稣的名,藉着他感谢父神。

你们作妻子的,当顺服自己的丈夫,这在主里面是相宜的。

你们作丈夫的,要爱你们的妻子,不可苦待他们。

你们作儿女的,要凡事听从父母,因为这是主所喜悦的。

你们作父亲的,不要惹儿女的气,恐怕他们失去志气。

你们作仆人的,要凡事听从你们肉身的主人,不要只在眼前事奉,像是讨人喜欢的,总要存心诚实敬畏主。

无论作什么,都要从心里作,像是给主作的,不是给人作的。

因你们知道从主那里,必得着基业为赏赐。你们所事奉的乃是主基督。

那行不义的,必受不义的报应。主不偏待人。

\chapter{歌罗西书第4章}
你们作主人的,要公公平平的待仆人,因为知道你们也有一位主在天上。

你们要恒切祷告,在此儆醒感恩。

也要为我们祷告,求神给我们开传道的门,能以讲基督的奥秘,(我为此被捆锁)。

叫我按着所该说的话,将这奥秘发明出来。

你们要爱惜光阴,用智慧与外人交往。

你们的言语要常常带着和气,好像用盐调和,就可知道该怎样回答各人。

有我亲爱的兄弟推基古要将我一切的事都告诉你们。他是忠心的执事,和我一同作主的仆人。

我特意打发他到你们那里去,好叫你们知道我们的光景,又叫他安慰你们的心。

我又打发一位亲爱忠心的兄弟阿尼西母同去。他也是你们那里的人。他们要把这里一切的事都告诉你们。

与我一同坐监的亚里达古问你们安。巴拿巴的表弟马可也问你们安。(说到这马可,你们已经受了吩咐。他若到了你们那里,你们就接待他

耶数又称犹士都,也问你们安。奉割里的人中,只有这三个人,是为神的国与我一同作工的。也是叫我心里得安慰的。

有你们那里的人,作基督耶稣仆人的以巴弗问你们安。他在祷告之间,常为你们竭力的祈求,愿你们在神一切的旨意上,得以完全。信心充足,能站立得稳。

他为你们和老底嘉并希拉波立的弟兄,多多的劳苦。这是我可以给他作见证的。

所亲爱的医生路加,和底马问你们安。

请问老底嘉的弟兄和宁法,并他家里的教会安。

你们念了这书信,便交给老底嘉的教会,叫他们也念。你们也要念从老底嘉来的书信。

要对亚基布说,务要谨慎,尽你从主所受的职分。

我保罗亲笔问你们安。你们要记念我的捆锁。愿恩惠常与你们同在。

\chapter{帖撒罗尼迦前书第1章}
保罗,西拉,提摩太,写信给帖撒罗尼迦在父神和主耶稣基督里的教会。愿恩惠平安归与你们。

我们为你们众人常常感谢神,祷告的时候题到你们。

在神我们的父面前,不住的记念你们因信心所作的工夫,因爱心所受的劳苦,因盼望我们主耶稣基督所存的忍耐。

被神所爱的弟兄阿,我知道你们是蒙拣选的。

因为我们的福音传到你们那里,不独在乎言语,也在乎权能和圣灵,并充足的信心,正如你们知道我们在你们那里,为你们的缘故是怎样为人。

并且你们在大难之中,蒙了圣灵所赐的喜乐,领受真道,就效法我们,也效法了主。

甚至你们作了马其顿和亚该亚,所有信主之人的榜样。

因为主的道从你们那里已经传杨出来,你们向神的信心不但在马其顿和亚该亚,就是在各处,也都传开了。所以不用我们说什么话。

因为他们自己已经报明我们是怎样进到你们那里,你们是怎样离弃偶像归向神,要服事那又真又活的神,

等候他儿子从天降临,就是他从死里复活的,那位救我们脱离将来忿怒的耶稣。

\chapter{帖撒罗尼迦前书第2章}
弟兄们,你们自己原晓得我们进到你们那里,并不是徒然的。

我们从前在腓立比被害受辱,这是你们知道的。然而还是靠我们的神放开胆量,在大争战中把神的福音传给你们。

我们的劝勉,不是出于错误,不是出于污秽,也不是用诡诈。

但神既然验中了我们,把福音托付我们,我们就照样讲,不是要讨人喜欢,乃是要讨那察验我们心的神喜欢。

因为我们从来没有用过谄媚的话,这是你们知道的。也没有藏着贪心,这是神可以作见证的。

我们作基督的使徒,虽然可以叫人尊重,却没有向你们或向别人求荣耀,

只在你们中间存心温柔,如同母亲乳养自己的孩子。

我们既是这样爱你们,不但愿意将神的福音给你们,连自己的性命也愿意给你们,因你们是我们所疼爱的。

弟兄们,你们记念我们的辛苦劳碌,昼夜作工,传神的福音给你们,免得叫你们一人受累。

我们向你们信主的人,是何等圣洁,公义,无可指摘,有你们作见证,也有神作见证。

你们也晓得我们怎样劝勉你们,安慰你们,嘱咐你们各人,好像父亲待自己的儿女一样。

要叫你们行事对得起那召你们进他国得他荣耀的神。

为此,我们也不住的感谢神,因你们听见我们所传神的道,就领受了,不以为是人的道,乃以为是神的道。这道实在是神的,并且运行在你们信主的人心中。

弟兄们,你们曾效法犹太中,在基督耶稣里神的各教会。因为你们也受了本地人的苦害,像他们受了犹太人的苦害一样。

这犹太人杀了主耶稣和先知,又把我们赶出去。他们不得神的喜悦,且与众人为敌。

不许我们传道给外邦人使外邦人得救,常常充满自己的罪恶。神的忿怒临在他们身上已经到了极处。

弟兄们,我们暂时与你们离别,是面目离别,心里却不离别,我们极力的想法子,很愿意见你们的面。

所以我们有意到你们那里,我保罗有一两次要去,只是撒但阻挡了我们。

我们的盼望和喜乐,并所夸的冠冕,是什么呢。岂不是我们主耶稣来的时候你们在他面前删立得住吗。

因为你们就是我们的荣耀,我们的喜乐。

\chapter{帖撒罗尼迦前书第3章}
我们既不能再忍,就愿意独自等在雅典。

打发我们的兄弟在基督福音上作神执事的提摩太前去,(作神执事的有古卷作与神同工的)坚固你们,并在你们所信的道上劝慰你们。

免得有人被诸般患难摇动。因为你们自己知道我们受患难原是命定的。

我们在你们那里的时候,豫先告诉你们,我们必受患难,以后果然应验了,你们也知道。

为此,我既不能再忍,就打发人去,要晓得你们的信心如何,恐怕那诱惑人的到底诱惑了你们,叫我们的劳苦归于徒然。

但提摩太刚才从你们那里回来,将你们信心和爱心的好消息报给我们,又说你们常常记念我们,切切的想见我们,如同我们想见你们一样。

所以弟兄们,我们在一切困苦患难之中,因着你们的信心就得了安慰。

你们若靠主站立得稳,我们就活了。

我们在神面前,因着你们甚是喜乐,为这一切喜乐,可用何等的感谢,为你们报答神呢。

我们昼夜切切的祈求,要见你们的面,补满你们信心的不足。

愿神我们的父,和我们的主耶稣,一直引领我们到你们那里去。

又愿主叫你们彼此相爱的心,并爱众人的心,都能增长,充足,如同我们爱你们一样。

好使你们,当我们主耶稣同他众圣徒来的时候,在我们父神面前,心里坚固,成为圣洁,无可责备。

\chapter{帖撒罗尼迦前书第4章}
弟兄们,我还有话说。我们靠着主耶稣求你们,劝你们,你们既然受了我们的教训,知道该怎样行,可以讨神的喜悦,就要照你们现在所行的,更加勉励。

你们原晓得我们凭主耶稣传给你们什么命令。

神的旨意就是要你们成为圣洁,远避淫行。

要你们各人晓得怎样用圣洁尊贵,守着自己的身体。

不放纵私欲的邪情,像那不认识神的外邦人。

不要一个人在这事上越分,欺负他的弟兄。因为这一类的事,主必报应,正如我豫先对你们说过,又切切嘱咐你们的。

神召我们,本不是要我们沾染污秽,乃是要我们成为圣洁。

所以那弃绝的,不是弃绝人,乃是弃绝那赐圣灵给你们的神。

论到弟兄们相爱,不用人写信给你们,因为你们自己蒙了神的教训,叫你们彼此相爱。

你们向马其顿全地的众弟兄,固然是这样行,但我劝弟兄们要更加勉励。

又要立志作安静人,办自己的事,亲手作工,正如我们从前所吩咐你们的。

叫你们可以向外人行事端正,自己也就没有什么缺乏了。

论到睡了的人,我们不愿意弟兄们不知道,恐怕你们忧伤,像那些没有指望的人一样。

我们若信耶稣死而复活了,那已经在耶稣里睡了的人,神也必将他与耶稣一同带来。

我们现在照主的话告诉你们一件事。我们这活着还存留到主降临的人,断不能在那已经睡了的人之先。

因为主必亲自从天降临,有呼叫的声音,和天使长的声音,又有神的号吹响。那在基督里死了的人必先复活。

以后我们这活着还存留的人,必和他们一同被提到云里,在空中与主相遇。这样,我们就要和主永远同在。

所以你们当用这些话彼此劝慰。

\chapter{帖撒罗尼迦前书第5章}
弟兄们,论到时候日期,不用写信给你们。

因为你们自己明明晓得,主的日子来到,好像夜间的贼一样。

人正说平安稳妥的时候,灾祸忽然临到他们,如同叁难临到怀胎的妇人一样,他们绝不能逃脱。

弟兄们,你们却不在黑暗里,叫那日子临到你们像贼一样。

你们都是光明之子,都是白昼之子,我们不是属黑夜的,也不是属幽暗的。

所以我们不要睡觉,像别人一样,总要儆醒谨守。

因为睡了的人是在夜间睡。醉了的人是在夜间醉。

但我们既然属乎白昼,就应当谨守,把信和爱当作护心镜遮胸。把得救的盼望当作头盔戴上。

因为神不是豫定我们受刑,乃是豫定我们藉着我们主耶稣基督得救。

他替我们死,叫我们无论醒着睡着,都与他同活。

所以你们该彼此劝慰,互相建立,正如你们素常所行的。

弟兄们,我们劝你们敬重那在你们中间劳苦的人,就是在主里面治理你们,劝戒你们的。

又因他们所作的工,用爱心格外尊重他们,你们也要彼此和睦。

我们又劝弟兄们,要警戒不守规矩的人。勉励灰心的人。扶助软弱的人。也要向众人忍耐。

你们要谨慎,无论是谁都不可以恶报恶。或是彼此相待,或是待众人,常要追求良善。

要常常喜乐。

不住的祷告。

凡事谢恩。因为这是神在基督耶稣里向你们所定的旨意。

不要销灭圣灵的感动。

不要藐视先知的讲论。

但要凡事察验。善美的要持守。

各样的恶事要禁戒不作。

愿赐平安的神,亲自使你们全然成圣。又愿你们的灵,与魂,与身子,得蒙保守,在我主耶稣基督降临的时候,完全无可指摘。

那召你们的本是信实的,他必成就这事。

请弟兄们为我们祷告。

与众弟兄亲嘴问安务要圣洁。

我指着主嘱咐你们,要把这信念给众弟兄听。

愿我主耶稣基督的恩常与你们同在。

\chapter{帖撒罗尼迦后书第1章}
保罗,西拉,提摩太,写信给帖撒罗尼迦在神我们的父,与主耶稣基督里的教会。

愿恩惠平安,从父神和主耶稣基督,归与你们。

弟兄们,我们该为你们常常感谢神,这本是合宜的。因你们的信心格外增长,并且你们众人彼此相爱的心也都充足。

甚至我们在神的各教会里为你们夸口,都因你们在所受的一切逼迫患难中,仍旧存忍耐和信心。

这正是神公义判断的明证。叫你们可算配得神的国,你们就是为这国受苦。

神既是公义的,就必将患难报应那加患难给你们的人。

也必使你们这受患难的人,与我们同得平安。那时,主耶稣同他有能力的天使从天上在火焰中显现,

要报应那不认识神,和那不听从我主耶稣福音的人。

他们要受刑罚,就是永远沉沦,离开主的面和他权能的荣光。

这正是主降临要在他圣徒的身上得荣耀,又在一切信的人身上显为希奇的那日子。(我们对你们作的见证,你们也信了。)

因此,我们常为你们祷告,愿我们的神看你们配得过所蒙的召。又用大能成就你们一切所羡慕的良善,和一切因信心所作的工夫。

叫我们主耶稣的名在你们身上得荣耀,你们也在他们身上得荣耀,都照着我们的神并主耶稣基督的恩。

\chapter{帖撒罗尼迦后书第2章}
弟兄们,论到我们主耶稣基督降临,和我们到他那里聚集,

我劝你们,无论有灵有言语,有冒我名的书信,说主的日子现在到了,(现在或作就)不要轻易动心,也不要惊慌。

人不拘用什吗法子,你们总不要被他诱惑,因为那日子以前,必有离道反教的事,并有那大罪人,就是沈沦之子,显露出来。

他是抵挡主,高抬自己,超过一切称为神的,和一切受人敬拜的。甚至坐在神的殿里,自称是神。

我还在你们那里的时候,曾把这些事告诉你们,你们不记得吗。

现在你们也知道那拦阻他的是什么,是叫他到了的时候,才可以显露。

因为那不法的隐意已经发动。只是现在有一个拦阻的,等到那拦阻的被除去。

那时这不法的人,必显露出来。主耶稣要用口中的气灭绝他,用降临的荣光废掉他。

这不法的人来,是照撒但的运动,行各样的异能神迹,和一切虚假的奇事,

并且在那沉沦的人身上,行各样出于不义的诡诈。因为他们不领受真理的心,使他们得救。

故此,神就给他们一个生发错误的心,叫他们信从虚谎。

使一切不信真理,倒喜爱不义的人,都被定罪。

主所爱的弟兄哪,我们本该常为你们感谢神。因为他从起初拣选了你们,叫你们因信真道,又被圣灵感动,成为圣洁,能以得救。

神藉我们所传的福音,召你们到这地步,好得着我们主耶稣基督的荣光。

所以弟兄们,你们要站立得稳,凡所领受的教训,不拘是我们口传的,是信上写的,都要坚守。

但愿我们主耶稣基督,和那爱我们,开恩将永远的安慰,并美好的盼望,赐给我们的父神,

安慰你们的心,并且在一切善行善言上,坚固你们。

\chapter{帖撒罗尼迦后书第3章}
弟兄们,我还有话说,请你们为我们祷告,好叫主的道理快快行开,得着荣耀,正如在你们中间一样。

也叫我们脱离无理之恶人的手。因为人不都是有信心。

但主是信实的,要坚固你们,保护你们脱离那恶者。(或作脱离凶恶)

我们靠主深信你们现在是遵行我们所吩咐的,后来也必要遵行。

愿主引导你们的心,叫你们爱神并学基督的忍耐。

弟兄们,我们奉主耶稣基督的名吩咐你们,凡有弟兄不按规矩而行,不遵守从我们所受的教训,就当远离他。

你们自己原知道应当怎样效法我们。因为我们在你们中间,未尝不按规矩而行。

也未尝白吃人的饭。倒是辛苦劳碌,昼夜作工,免得叫你们一人受累。

这并不是因我们没有权柄,乃是要给你们作榜样,叫你们效法我们。

我们在你们那里的时候,曾吩咐你们说,若有人不肯作工,就不可吃饭。

因我们听说,在你们中间有人不按规矩而行,什么工都不作,反倒专管闲事。

我们靠主耶稣基督,吩咐劝戒这样的人,要安静作工,吃自己的饭。

弟兄们,你们行善不可丧志。

若有人不听从我们这信上的话,要记下他,不和他交往,叫他自觉羞愧。

但不要以他为仇人,要劝他如弟兄。

愿赐平安的主,随时随事亲自给你们平安。愿主常与你们众人同在。

我保罗亲笔问你们安。凡我的信都以此为记。我的笔迹就是这样。

愿我们主耶稣基督的恩,常与你们众人同在。

\chapter{提摩太前书第1章}
奉我们救主神,和我们的盼望基督耶稣之命,作基督耶稣使徒的保罗,

写信给那因信主作我真儿子的提摩太。愿恩惠怜悯平安,从父神和我们主基督耶稣,归与你。

我往马其顿去的时候,曾劝你仍住在以弗所,好嘱咐那几个人,不可传异教,

也不可听从荒渺无凭的话语,和无穷的家谱。这等事只生辩论,并不发明神在信上所立的章程。

但命令的总归就是爱。这爱是从清洁的心,和无亏的良心,无伪的信心,生出来的。

有人偏离这些,反去讲虚浮的话。

想要作教法师,却不明白自己所讲说的,所论定的。

我们知道律法原是好的,只要人用得合宜。

因为律法不是为义人设立的,乃是为不法和不服的,不虔诚和犯罪的,不圣洁和恋世俗的,弑父母和杀人的,

行淫和亲男色的,抢人口和说谎话的,并起假誓的,或是为别样敌正道的事设立的。

这是照着可称颂之神交托我荣耀福音说的。

我感谢那给我力量的,我们主基督耶稣,因他以我有忠心,派我服事他。

我从前是亵渎神的,逼迫人的,悔慢人的。然而我还蒙了怜悯,因我是不信不明白的时候而作的。

并且我主的恩是格外丰盛,使我在基督耶稣里有信心和爱心。

基督耶稣降世,为要拯救罪人。这话是可信的,是十分可佩服的。在罪人中我个是罪魁。

然而我蒙了怜悯,是因耶稣基督要在我这罪魁身上,显明他一切的忍耐,给后来信他得永生的人作榜样。

但愿尊贵,荣耀归与那不能朽坏不能看见永世的君王,独一的神,直到永永远远。阿们。

我儿提摩太阿,我照从前指着你的预言,将这命令交托你,叫你因此可以打那美好的杖。

常存信心,和无亏的良心。有人丢弃良心,就在真道上如同船破坏了一般。

其中有许米乃和亚历山大。我已经把他们交给撒但,使他们受责罚,就不再谤渎了。

\chapter{提摩太前书第2章}
我劝你第一要为万人恳求祷告,代求,祝谢。

为君王和一切在位的也该如此。使我们可以敬虔端正,平安无事的度日。

这是好的,在神我们救主面前可蒙悦讷。

他愿意万人得救,明白真道。

因为只有一位神,在神和人中间,只有一位中保,乃是降世为人的基督耶稣。

他舍自己作万人的赎价。到了时候,这事必证明出来。

我为此奉派,作传道的,作使徒,作外邦人的师傅,教导他们相信,学习涉道。我说的是真话,并不是谎言。

我愿男人无忿怒,无争论,(争论或作疑惑),举起圣洁的手,随处祷告。

又愿女人廉耻,自守,以正派衣裳为妆饰,不以编发,黄金,珍珠,和贵价的衣裳为妆饰。

只要有善行。这才与自称是敬神的女人相宜。

女人要沉静学道,一味的顺服。

我不许女人讲道,也不许他辖管男人,只要沉静。

因为先造的是亚当,后造的是夏娃。

且不是亚当被引诱,乃是女人被引诱,陷在罪里。

然而女人若常存信心爱心,又圣洁自守,就必在生产上得救。

\chapter{提摩太前书第3章}
人若想得监督的职分,就是羡慕善工。这话是可信的。

作监督的,必须无可指责,只作一个妇人的丈夫,有节制,自守,端正。乐意接待远人,善于教导。

不因酒滋事,不打人,只要温和,不争竞,不贪财。

好好管理自己的家,使儿女凡事端庄顺服。(或作端端庄庄的使儿女顺服)

人若不知道管理自己的家,焉能照管神的教会呢。

初入教的不可作监督,恐怕他自高自大,就落在魔鬼所受的刑罚里。

监督也必须在教外有好名声,恐怕被人毁谤,落在魔鬼的网罗里。

论执事的也是如此,必须端庄,不一口两舌,不好喝酒,不贪不义之财。

要存清洁的良心,固守真道的奥秘。

这等人也要先受试验。若没有可责之处,然后叫他们作执事。

女执事(原文作女人)也是如此,必须端庄,不说谗言,有节制,凡事忠心。

执事只要作一个妇人的丈夫,好好管理儿女和自己的家。

因为善作执事的,自己就得到美好的地步,并且在基督耶稣里的真道上大有胆量。

我指望快到你那里去,所以先将这些事写给你。

倘若我耽延日久,你也可以知道在神的家中当怎样行。这家就是永生神的教会,真理的柱石和根基。

大哉,敬虔的奥秘,无人不以为然,就是神在肉身显现,被圣灵称义,(或作在灵性称义),被天使看见,被传于外邦,被世人信服,被接在荣耀里。

\chapter{提摩太前书第4章}
圣灵明说,在后来的时候,必有人离弃真道,听从那引诱人的(邪)灵,和鬼魔的道理。

这是因为说谎之人的假冒。这等人的良心,如同被热铁烙惯了一般。

他们禁止嫁娶,又禁戒食物,(或作又叫人戒荤)就是神所造叫那信而明白真道的人,感谢着领受的。

凡神所造的物,都是好的。若感谢着领受,就没有一样可弃的。

都因神的道和人的祈求,成为圣洁了。

你若将这些事题醒弟兄们,便是基督耶稣的好执事,在真道的话语,和你向来所服从的善道上,得了教育。

只是要弃绝那世俗的言语,和老妇荒渺的话,在敬虔上操练自己。

操练身体,益处还少。惟独敬虔,凡事都有益处。因有今生和来生的应许。

这话是可信的,是十分可佩服的。

我们劳苦努力,正是为此。因我们的指望在乎永生的神。他是万人的救主,更是信徒的救主。

这些事你要吩咐人,也要教导人。

不可叫人小看你年轻。总要在言语,行为,爱心,信心,清洁上,都作信徒的榜样。

你要以宣读,劝勉,教导为念,直等到我来。

你不可轻忽所得的恩赐,就是从前藉着预言,在众长老按手的时候,赐给你的。

这些事你要殷勤去作,并要在此专心,使众人看出你的长进来。

你要谨慎自己和自己的教训,要在这些事上恒心。因为这样行,又能救自己,又能救听你的人。

\chapter{提摩太前书第5章}
不可严责老年人,只要劝他们如同父亲。劝少年人如同弟兄。

劝老年妇女如同母亲。劝少年妇女如同姐妹。总要清清洁洁的。

要尊敬那真为寡妇的。

若寡妇有儿女,或有孙子孙女,便叫他们先在自己的家中学着行孝,报答亲恩,因为这在神面前是可悦纳的。

那独居无靠真为寡妇的,是仰赖神,昼夜不住的祈求祷告。

但那好宴乐的寡妇,正活着的时候也是死的。

这些事你要嘱咐他们,叫他们无可指责。

人若不看顾亲属,就是背了真道,比不信的人还不好。不看顾自己家里的人,更是如此。

寡妇记在册上,必须年纪到六十岁,从来只作一个丈夫的妻子,

又有行善的名声,就如养育儿女,接待远人,洗圣徒的脚,救济遭难的人,竭力行各样善事。

至于年轻的寡妇,就可以辞他。因为他们的情欲发动,违背基督的时候,就想要嫁人。

他们被定罪,是因废弃了当初所许的愿。

并且他们又习惯懒惰,挨家闲游。不但是懒惰,又说长道短,好管闲事,说些不当说的话。

所以我愿意年轻的寡妇嫁人,生养儿女,治理家务,不给敌人辱骂的把柄。

因为已经有转去随从撒但的。

信主的妇女,若家中有寡妇,自己就当救济他们,不可累着教会,好使教会能救济那真无倚靠的寡妇。

那善于管理教会的长老,当以为配受加倍的敬奉。那劳苦传道教导人的,更当如此。

因为经上说,牛在场上踹谷的时候,不可笼住它的嘴。又说,工人得工价是应当的。

控告长老的呈子,非有两三个见证就不要收。

犯罪的人,当在众人面前责备他,叫其馀的人也可以惧怕。

我在神和基督耶稣并蒙拣选的天使面前嘱咐你,要遵守这些话,不可存成见,行事也不可有偏心。

给人行按手的礼,不可急促。不要在别人的罪上有分。要保守自己清洁。

因你胃口不清,屡次患病,再不要照常喝水,可以稍微用点酒。

有些人的罪是明显的,如同先到审判案前。有些人的罪是随后跟了去的。

这样,善行也有明显的。那不明显的,也不能隐藏。

\chapter{提摩太前书第6章}
凡在轭下作仆人的,当以自己主人配受十分的恭敬,免得神的名和道理,被人亵渎。

仆人有信道的主人,不可因为与他是弟兄就轻看他。更要加意服事他。因为得服事之益处的,是信道蒙爱的。你要以此教训人,劝勉人。

若有人传异教,不服从我们主耶稣基督纯正的话,与那合乎敬虔的道理。

他是自高自大,一无所知,专好问难争辩言词,从此就生出嫉妒,分争,毁谤,妄疑,

并那坏了心术,失丧真理之人的争竞。他们以敬虔为得利的门路。

然而敬虔加上知足的心便是大利了。

因为我们没有带什么到世上来,也不能带什么去。

只要有衣有食,就当知足。

但那些想发财的人,就陷在迷惑,落在网罗,和许多无知有害的私欲里,叫人沉在败坏和灭亡中。

贪财是万恶之根。有人贪恋钱财,就被引诱离了真道,用许多愁苦把自己刺透了。

但你这属神的人,要逃避这些事,追求公义,敬虔,信心,爱心,忍耐,温柔。

你要为真道打那美好的仗,持定永生。你为此被召,也在许多见证人面前,已经作了那美好的见证。

我在叫万物生活的神面前,并在向本丢彼拉多作过那美好见证的基督耶稣面前嘱咐你,

要守这命令,毫不玷污,无可指责,直到我们的主耶稣基督显现。

到了日期,那可称颂独有权能的,万王之王,万主之主,

就是那独一不死,住在人不能靠近的光里,是人未曾看见,也是不能看见的,要将他显明出来。但愿尊贵和永远的权能,都归给他。阿们。

你要嘱咐那些今世富足的人,不要渍高,也不要倚靠无定的钱财。只要倚靠那厚赐百物给我们享受的神。

又要嘱咐他们行善,在好事上富足,甘心施舍,乐意供给人,(供给或作体贴)

为自己积成美好的根基,豫备将来,叫他们持定那真正的生命。

提摩太阿,你要保守所托付你的,躲避世俗的虚谈,和那敌真道似是而非的学问。

已经有人自称有这学问,就偏离了真道。愿恩惠常与你们同在。

\chapter{提摩太后书第1章}
奉神旨意,照着在基督耶稣里生命的应许,作基督耶稣使徒的保罗,

写信给我亲爱的儿子提摩太。愿恩惠怜悯平安,从父神和我们主基督耶稣,归与你。

我感谢神,就是我接续祖先,用清洁的良心所事奉的神,祈祷的时候,不住的想念你,

记念你的眼泪,昼夜切切的想要见你,好叫我满心快乐。

想到你心里无伪之信。这信是先在你外祖母罗以,和你母亲友尼基心里的。我深信也在你的心里。

为此我题醒你,使你将神藉我按手所给你的恩赐,再如火挑旺起来。

因为神赐给我们,不是胆怯的心,乃是刚强,仁爱,谨守的心。

你不要以给我们的主作见证为耻,也不要以我这为主被囚的为耻。总要按神的能力,与我为福音同受苦难。

神救了我们,以圣召召我们,不是按我们的行为,乃是按他的旨意,和恩典。这恩典是万古之先,在基督耶稣里赐给我们的。

但如今藉着我们救主基督耶稣的显现,才表明出来了。他已经把死废去,藉着福音,将不能坏的生命彰显出来。

我为这福音奉派作传道的,作使徒,作师傅。

为这缘故,我也受这些苦难。然而我不以为耻。因为我知道我所信的是谁,也深信他能保全我所交付他的,(或作他所交托我的)直到那日。

你从我听的那纯正话语的规模,要用在基督耶稣里的信心和爱心,常常守着。

从前所交托你的善道,你要靠着那住在我们里面的圣灵,牢牢的守着。

凡在亚细亚的人都离弃我,这是你知道的。其中有腓吉路和黑摩其尼。

愿主怜悯阿尼色弗一家的人。因他屡次使我畅快,不以我的锁链为耻。

反倒在罗马的时候,殷勤的找我,并且找着了。

愿主使他在那日得主的怜悯。他在以弗所怎样多多的服事我,是你明明知道的。

\chapter{提摩太后书第2章}
我儿阿,你要在基督耶稣的恩典上刚强起来。

你在许多见证人面前听见我所教训的,也要交托那忠心能教导别人的人。

你要和我同受苦难,好像基督耶稣的精兵。

凡在军中当兵的,不将世物缠身,好叫那招他当兵的人喜悦。

人若在场上比武,非按规矩,就不能得冠冕。

劳力的农夫,理当先得粮食。

我所说的话你要思想。因为凡事主必给你聪明。

你要记念耶稣基督乃是大卫的后裔。他从死里复活,正合乎我所传的福音。

我为这福音受苦难,甚至被捆绑,像犯人一样。然而神的道,却不被捆绑。

所以我为选民凡事忍耐,叫他们也可以得着那在基督耶稣的救恩,和永远的荣耀。

有可信的话说,我们若与基督同死,也必与他同活。

我们若能忍耐,也必和他一同作王。我们若不认他,他也必不认我们。

我们纵然失信,他仍是可信的。因为他不能背乎自己。

你要使众人回想这些事,在主面前嘱咐他们,不可为言语争辩。这事没有益处的,只能败坏听见的人。

你当竭力,在神面前得蒙喜悦,作无愧的工人,按着正意分解真理的道。

但要远避世俗的虚谈。因为这等人必进到更不敬虔的地步。

他们的话如同毒疮,越烂越大。其中有许米乃和腓理徒。

他们偏离了真道,说复活的事已过,就败坏好些人的信心。

然而神坚固的根基立住了。上面有这印记说,主认识谁是他的人。又说,凡称呼主名的人,总要离开不义。

在大户人家,不但有金器银器,也有木器瓦器。有作为贵重的,有作为卑贱的。

人若自洁,脱离卑贱的事,就必作贵重的器皿,成为圣洁,合乎主用,豫备行各样的善事。

你要逃避少年的私欲,同那清心祷告主的人追求公义,信德,仁爱,和平。

惟有那愚拙无学问的辩论,总要弃绝。因为知道这等事是起争竞的。

然而主的仆人不可争竞,只要温温和和的待众人,善于教导,存心忍耐,

用温柔劝戒那抵挡的人。或者给他们悔改的心,可以明白真道。

叫他们这已经被魔鬼任意掳去的,可以醒悟,脱离他的网罗。

\chapter{提摩太后书第3章}
你该知道,末世必有危险的日子来到。

因为那时人要专顾自己,贪爱钱财,自夸,狂傲,谤??,违背父母,忘恩负义,心不圣洁,

无亲情,不解怨,好说才言,不能自约,性情凶暴,不爱良善,

卖主卖友,任意妄为,自高自大,爱宴乐不爱神。

有敬虔的外貌,却背了敬虔的实意。这等人你要躲开。

那偷进人家,牢笼无知妇女的,正是这等人。这些妇女担负罪恶,被各样的私欲引诱。

常常学习,终久不能明白真道。

从前雅尼和佯庇怎样敌挡摩西,这等人也怎样敌挡真道。他们的心地坏了,在真道上是可废弃的。

然而他们不能再这样敌挡,因为他们的愚昧,必在众人面前显露出来,像那二人一样。

但你已经服从了我的教训,品行,志向,信心,宽容,爱心,忍耐,

以及我在安提阿,以哥念,路司得,所遭遇的逼迫,苦难。我所忍受是何等的逼迫。但从这一切苦难中主都把我救出来了。

不但如此,凡立志在基督耶稣里敬虔度日的,也都要受逼迫。

只是作恶的,和迷惑人的,必越久越恶,他欺哄人也被人欺哄。

但你所学习的,所确信的,要存在心里。因为你知道是跟谁学的。

并且知道你是从小明白圣经。这圣经能使你因信基督耶稣有得救的智慧。

圣经都是神所默示的,(或作凡神默示的圣经)于教训,督责,使人归正,教导人学义,都是有益的。

叫属神的人得以完全,豫备行各样的善事。

\chapter{提摩太后书第4章}
我在神面前,并在将来审判活人死人的基督耶稣面前,凭着他的显现和他的国度嘱咐你。

务要传道。无论得时不得时,总要专心,并用百般的忍耐,各样的教训,责备人,警戒人,劝勉人。

因为时候要到,人必厌烦纯正的道理。耳朵发痒,就随从自己的情欲,增添好些师傅。

并且掩耳不听真道,偏向荒渺的言语。

你却要凡事谨慎,忍受苦难,作传道的工夫,尽你的职分。

我现在被浇奠,我离世的时候到了。

那美好的仗我已经打过了。当跑的路我已经跑尽了。所信的道我已经守住了。

从此以后,有公义的冠冕为我存留,就是按着公义审判的主到了那日要赐给我的。不但赐给我,也赐给凡爱慕他显现的人。

你要赶紧的到我这里来。

因为底马贪爱现今的世界,就离弃我往帖撒罗尼迦去了。革勒士往加拉太去。提多往挞马太去。

独有路加在我这里。你来的时候要把马可带来。因为他在传道的事上于我有益处。(传道或作服事我)

我已经打发推基古往以弗所去。

我在特罗亚留于加布的那件外衣,你来的时候可以带来。那些书也要带来。更要紧的是那些皮卷。

铜匠亚历山大多多的害我。主必照他所行的报应他。

你也要防备他。因为他极力敌挡了我们的话。

我初次申诉,没有人前来帮助,竟都离弃我。但愿这罪不归与他们。

惟有主站在我旁边,加给我力量,使福音被我尽都传明,叫外邦人都听见。我也从狮子口里被救出来。

主必救我脱离诸般的凶恶,也必救我进他的天国。愿荣耀归给他,直到永永远远。阿们。

问百基拉,亚居拉,和阿尼色弗一家的人安。

以拉都在哥林多住下了。特罗非摩病了,我就留他在米利都。

你要赶紧在冬天以前到我这里来。有友布罗,布田,利奴,革老底亚,和众弟兄,都问你安。

愿主与你的灵同在。愿恩惠常与你们同在。

\chapter{提多书第1章}
神的仆人,耶稣基督的使徒保罗,凭着神选民的信心,与敬虔真理的知识,

盼望那无谎言的神在万古之先所应许的永生,

到了日期,藉着传扬的工夫,把他的道显明了。这传扬的责任,是按着神我们救主的命令交托了我。

现在写信给提多,就是照着我们共信之道作我真儿子的。愿恩惠平安,从父神和我们的救主基督耶稣归与你。

我从前留你在克里特,是要你将没有辨完的事都辨整齐了,又照我所吩咐你的,在各城设立长老。

若有无可指责的人,只作一个妇人的丈夫,儿女也是信主的,没有人告诉他们是放荡不服约束的,就可以设立。

监督既是神的管家,必须无可指责,不任性,不暴躁,不因酒滋事,不打人,不贪无义之财,

乐意接待远人,好善,庄重,公平,圣洁,自持。

坚守所教真实的道理,就能将纯正的教训劝化人。又能把争辩的人驳倒了。

因为有许多人不服约束,说虚空话,欺哄人。那奉割礼的,更是这样。

这些人的口总要堵住。他们因贪不义之财,将不该教导的教导人,败坏人的全家。

有克里特人中的一个本地先知说,克里特人常说谎话,乃是恶兽,又馋又懒。

这个见证是真的。所以你要严严的责备他们,使他们在真道上纯全无疵。

不听犹太人荒渺的言语,和离弃真道之人的诫命。

在洁净的人,凡物都洁净。在污秽不信的人,什么都不洁净。连心地和天良,也都污秽了。

他们说是认识神行事却和他相背。本是可憎恶的,是悖逆的,在各样善事上是可废弃的。

\chapter{提多书第2章}
但你所讲的,总要合乎那纯正的道理。

劝老年人,要有节制,端庄,自守,在信心爱心忍耐上,都要纯全无疵。

又劝老年妇人,举止行动要恭敬,不说谗言,不给酒作奴仆,用善道教训人。

好指教少年妇人,爱丈夫,爱儿女,

谨守贞洁,料理家务,待人有恩,顺服自己的丈夫,免得神的道理被毁谤。

又劝少年人要谨守。

你自己凡事要显出善行的榜样,在教训上要正直,端庄,

言语纯全,无可指责,叫那反对的人,既无处可说我们的不是,便自觉羞愧。

劝仆人要顺服自己的主人,凡事讨他的喜欢。不可顶撞他。

不可私拿东西。要显为忠诚,以致凡事尊荣我们救主神的道。

因为神救众人的恩典,已经显明出来,

教训我们除去不敬虔的心,和世俗的情欲,在今世自守,公义,敬虔度日。

等候所盼望的福,并等候至大的神,和(或作无和字)我们救主耶稣基督的荣耀显现。

他为我们舍了自己,要赎我们脱离一切罪恶,又洁净我们,特作自己的子民,热心为善。

这些事你要讲明,劝戒人,用各等权柄责备人。不可叫人轻看你。

\chapter{提多书第3章}
你要题醒众人,叫他们顺服作官的,掌权的,遵他的命,豫备行各样的善事。

不要毁谤,不要争竞,总要和平,向众人大显温柔。

我们从前也是无知,悖逆,受迷惑,服事各样私欲和宴乐,常存恶毒(或作阴毒)嫉妒的心,是可恨的,又是彼此相恨。

但到了神我们救主的恩慈,和他向人所施的慈爱显明的时候,

他便救了我们,并不是因我们自己所行的义,乃是照他的怜悯,藉着重生的洗,和圣灵的更新。

圣灵就是神藉着耶稣基督我们救主,厚厚浇灌在我们身上的。

好叫我们因他的恩得称为义,可以凭着永生的盼望成为后嗣。(或作可以凭着盼望承受永生)。

这话是可信的,我也愿你把这些事,切切实实的讲明,使那些已信神的人,留心作正经事业。(或作留心行善)这都是美事,并且与人有益。

要远避无知的辩论,和家谱的空谈,以及分争,并因律法而起的争竞。因为这都是虚妄无益的。

分门结党的人,警戒过一两次,就要弃绝他。

因为知道这等人已经背道,犯了罪,自己明知不是,还是去作。

我打发亚提马,或是推基古,到你那里去的时候,你要赶紧往尼哥波立去见我。因为我已经定意在那里过冬。

你要赶紧给律师西纳,和亚波罗送行,叫他们没有缺乏。

并且我们的人要学习升经事业,(或作要学习行善)豫备所需用的,免得不结果子。

同我在一处的人都问你安。请代问那些因有信心爱我们的人安。愿恩惠常与你们众人同在。

\chapter{腓利门书第1章}
为基督耶稣被囚的保罗,同兄弟提摩太,写信给我们所亲爱的同工腓利门,

和妹子亚腓亚,并与我们同当兵的亚基布,以及在你家的教会。

愿恩惠平安,从神我们的父,和主耶稣基督,归与你们。

我祷告的时候题到你,常为你感谢我的神。

因听说你的爱心,并你向主耶稣和众圣徒的信心。(或作因听说你向主耶稣和众圣徒有爱心有信心)

愿你与人所同有的信心显出功效,使人知道你们各样善事都是为基督作的。

兄弟阿,我为你的爱心,大有快乐,大得安慰。因众圣徒的心从你得了畅快。

我虽然靠着基督能放胆吩咐你合宜的事。

然而像我这有年纪的保罗,现在又是为基督耶稣被囚的,宁可凭爱心求你。

就是为我在捆锁中所生的儿子阿尼西母(此名就是有益处的意思)求你。

他从前与你没有益处,但如今与你我都有益处。

我现在打发他亲自回你那里去。他是我心上的人。

我本来有意将他留下,在我为福音所受的捆锁中替你伺候我。

但不知道你的意思,我不愿意这样行,叫你的善行不是出于勉强,乃是出于甘心。

他暂时离开你,或者是叫你永远得着他。

不再是奴仆,乃是高过奴仆,是亲爱的兄弟,在我实在是如此,何况在你呢。这也不拘是按肉体说,是按主说。

你若以我为同伴,就收纳他,如同收纳我一样。

他若亏负你,或欠你什么,都归在我的账上。

我必偿还。这是我保罗亲笔写的。我并不用对你说,连你自己也亏欠于我。

兄弟阿,望你使我在主里因你得快乐。(或作益处)并望你使我的心在基督里得畅快。

我写信给你,深信你必顺服,知道你所要行的,必过于我所说的。

此外你还要给我豫备住处,因为我盼望藉着你们的祷告,必蒙恩到你们那里去。

为基督耶稣与我同坐监的以巴弗问你安。

与我同工的马可,亚里达古,底马,路加,也都问你安。

愿我们主耶稣基督的恩常在你的心里。阿们。

\chapter{希伯来书第1章}
神既在古时藉着众先知,多次多方的晓谕列祖,

就在这末世,藉着他儿子晓谕我们,又早已立他为承受万有的,也曾藉着他创造诸世界。

他是神荣耀所发的光辉,是神本体的真像,常用他权能的命令托住万有,他洗净了人的罪,就坐在高天至大者的右边。

他所承受的名,既比天使的名更尊贵,就远超过天使。

所有的天使,神从来对那一个说,你是我的儿子,我今日生你。又指着那一个说,我要作他的父,他要作我的子。

再者,神使长子到世上来的时候,(或作神再使长子到世上来的时候)就说,神的使者都要拜他。

论到使者,又说,神以风为使者,以火焰为仆役。

论到子却说,神阿,你的宝座是永永远远的,你的国权是正直的。

你喜爱公义,恨恶罪恶。所以神就是你的神,用喜乐油膏你,胜过膏你的同伴。

又说,主阿,你起初立了地的根基,天也是你手所造的。

天地都要灭没,你却要长存。天地都要像衣服渐渐旧了。

你要将天地卷起来,像一件外衣,天地都改变了。惟有你永不改变,你的年数没有穷尽。

所有的天使,神从来对那一个说,你坐在我的右边,等我使你仇敌作你的脚凳。

天使岂不都是服役的灵,奉差遣为那将要承受救恩的人效力吗。

\chapter{希伯来书第2章}
所以我们当越发郑重所听见的道理,恐怕我们随流失去。

那藉着天使所传的话,既是确定的,凡干犯悖逆的,都受了该受的报应。

我们若忽略这吗大的救恩,怎能逃罪呢。这救恩起先主亲自讲的,后来是听见的人给我们证实了。

神又按自己的旨意,用神迹奇事,和百般的异能,并圣灵的恩赐,同他们作见证。

我们所说将来的世界,神原没有交给天使管辖。

但有人在经上某处证明说,人算什么,你竟顾念他,世人算什么,你竟眷顾他。

你叫他比天使微小一点,(或作你叫他暂时比天使小)赐他荣耀尊贵为冠冕,并将你手所造的都派他管理。

叫万物都服在他的脚下。既叫万物都服他,就没有剩下一样不服他的。只是如今我们还不见万物都服他。

惟独见那成为比天使小一点的耶稣,(或作惟独见耶稣暂时比天使小)因为受死的苦,就得了尊贵荣耀为冠冕,叫他因着神的恩,为人人尝了死味。

原来那为万物所属,为万物所本的,要领许多的儿子进荣耀里去,使救他们的元帅,因受苦难得以完全,本是合宜的。

因那使人成圣的,和那些得以成圣的,都是出于一。所以他称他们为弟兄,也不以为耻,

说,我要将你的名传与我的弟兄,在会中我要颂杨你。

又说,我要倚赖他。又说,看哪,我与神所给我的儿女。

儿女既同有血肉之体,他也照样亲自成了血肉之体。特要藉着死,败坏那掌死权的,就是魔鬼。

并要释放那些一生因怕死而为奴仆的人。

他并不救拔天使,乃是救拔亚伯拉罕的后裔。

所以他凡事该与他的弟兄相同,为要在神的事上,成为慈悲忠信的大祭司,为百姓的罪献上挽回祭。

他自己既然被试探而受苦,就能搭救被试探的人。

\chapter{希伯来书第3章}
同蒙天召的圣洁弟兄阿,你们应当思想,我们所认为使者,为大祭司的耶稣。

他为那设立他的尽忠,如同摩西在神的全家尽忠一样。

他比摩西算是更配多得荣耀,好像建造房屋的比房屋更尊荣。

因为房屋都必有人建造。但建造万物的就是神。

摩西为仆人,在神的全家诚然尽忠,为要证明将来必传说的事。

但基督为儿子,治理神的家。我们若将可夸的盼望和胆量,坚持到底,便是他的家了。

圣灵有话说,你们今日若听他的话,

就不可硬着心,像在旷野惹他发怒,试探他的时候一样。

在那里,你们的祖宗试我探我,并且观看我的作为,有四十年之久。

所以我厌烦那世代的人,说,他们心里常常迷糊,竟不晓得我的作为。

我就在怒中起誓说,他们断不可进入我的安息。

弟兄们,你们要谨慎,免得你们中间,或有人存着不信的恶心,把永生神离弃了。

总要趁着还有今日,天天彼此相劝,免得你们中间,有人被罪迷惑,心里就刚硬了。

我们若将起初确实的信心,坚持到底,就在基督里有分了。

经上说,你们今日若听他的话,就不可硬着心,像惹他发怒的日子一样。

那时听见他话惹他发怒的是谁呢。岂不是跟着摩西从埃及出来的众人吗。

神四十年之久,又厌烦谁呢。岂不是那些犯罪尸首倒在旷野的人吗。

又向谁起誓,不容他们进入他的安息呢。岂不是向那些不信从的人吗。

这样看来,他们不能进入安息,是因为不信的缘故了。

\chapter{希伯来书第4章}
我们既蒙留下有进入他安息的应许,就当畏惧,免得我们中间,(我们原文作你们)或有人似乎是赶不上了。

因为有福音传给我们,像传给他们一样。只是所听见的道与他们无益,因为他们没有信心与所听见的道调和。

但我们已经相信的人,得以进入那安息,正如神所说,我在怒中起誓说,他们断不可进入我的安息。其实造物之工,从创世以来已经成全了。

论到第七日,有一处说,到第七日神就歇了他一切的工。

又有一处说,他们断不可进入我的安息。

既有必进安息的人,那先前听见福音的,因为不信从,不得进去。

所以过了多年,就在大卫的书上,又限定一日,如以上所引的说,你们今日若听他的话,就不可硬着心。

若是约书亚已叫他们享了安息,后来神就不再题别的日子了。

这样看来,必另有一安息日的安息,为神的子民存留。

因为那进入安息的,乃是歇了自己的工,正如神歇了他的工一样。

所以我们务必竭力进入那安息,免得有人学那不信从的样子跌倒了。

神的道是活泼的,是有功效的,比一切两刃的剑更快,甚至魂与灵,骨节与骨髓,都能刺入剖开,连心中的思念和主意,都能辨明。

并且被造的,没有一样在他面前不显然的。原来万物,在那与我们有关系的主眼前,都是赤露敞开的。

我们既然有一位已经升入高天尊荣的大祭司,就是神的儿子耶稣,便当持定所承认的道。

因我们的大祭司,并非不能体恤我们的软弱。他也曾凡事受过试探,与我们一样。只是他没有犯罪。

所以我们只管坦然无惧的,来到施恩的宝座前,为要得怜恤,蒙恩惠作随时的帮助。

\chapter{希伯来书第5章}
凡从人间挑选的大祭司,是奉派替人辨理属神的事,为要献上礼物,和赎罪祭。(或作要为罪献上礼物和祭物)

他能体谅那愚蒙的,和失迷的人,因为他自己也是被软弱所困。

故此他理当为百姓和自己献祭赎罪。

这大祭司的尊荣,没有人自取,惟要蒙神所召,像亚伦一样。

如此,基督也不是自取荣耀作大祭司,乃是在乎向他说,你是我的儿子,我今日生你。的那一位。

就如经上又有一处说,你是照着麦基洗德的等次永远为祭司。

基督在肉体的时候,既大声哀哭,流泪祷告恳求那能救他免死的主,就因他的虔诚,蒙了应允。

他虽然为儿子,还是因所受的苦难学了顺从。

他既得以完全,就为凡顺从的的人,成了永远得救的根源。

并蒙神照着麦基洗德的等次称他为大祭司。

论到麦基洗德,我们有好些话,并且难以解明,因为你们听不进去。

看你们学习的工夫,本该作师傅,谁知还得有人将神圣言小学的开端,另教导你们。并且成了那必须吃奶,不能吃乾粮的人。

凡只能吃奶的,都不熟练仁义的道理。因为他是婴孩。

惟独长大成人的,才能吃乾粮,他们的心窍,习练得通达,就能分辨好歹了。

\chapter{希伯来书第6章}
所以我们应当离开基督道理的开端,竭力进到完全的地步。不必再立根基,就如那懊悔死行,信靠神,

各样洗礼,按手之礼,死人复活,以及永远审判,各等教训。

神若许我们,我们必如此行。

论到那些已经蒙了光照,尝过天恩的滋味,又于圣灵有分,

并尝过神善道的滋味,觉悟来世权能的人,

若是离弃道理,就不能叫他们从新懊悔了。因为他们把神的儿子重钉十字架,明明的羞辱他。

就如一块田地,吃过屡次下的雨水,生长菜蔬合乎耕种的人用,就从神得福。

若长荆棘和疾??,必被废弃,近于咒诅,结局就是焚烧。

亲爱的弟兄们,我们虽是这样说,却深信你们的行为强过这些,而且近乎得救。

因为神并非不公义,竟忘记你们所作的工,和你们为他名所显的爱心,就是先前伺候圣徒,如今还是伺候。

我们愿你们各人都显出这样的殷勤,使你们有满足的指望,一直到底。

并且不懈怠。总要效法那些凭信心和忍耐承受应许的人。

当初神应许亚伯拉罕的时候,因为没有比自己更大可以指着起誓的,就指着自己起誓说,

论福,我必赐大福给你。论子孙,我必叫你的子孙多起来。

这样,亚伯拉罕既恒久忍耐,就得了所应许的。

人都是指着比自己大的起誓。并且以起誓为实据,了结各样的争论

照样,神愿意为那承受应许的人,格外显明他的旨意是不更改的,就起誓为证。

藉这两件不更改的事,神决不能说谎,好叫我们这逃往避难所,持定摆在我们前头指望的人,可以大得勉励。

我们有这指望如同灵魂的锚,又坚固又牢靠,且通入幔内。

作先锋的耶稣,既照着麦基洗德的等次,成了永远的大祭司,就为我们进入幔内。

\chapter{希伯来书第7章}
这麦基洗德,就是撒冷王,又是至高神的祭司,本是长远为祭司的。他当亚伯拉罕杀败诸王回来的时候,就迎接他,给他祝福。

亚伯拉罕也将自己所得来的取十分之一给他。他头一个名翻出来,就是仁义王,他又名撒冷王,就是平安王的意思。

他无父,无母,无族谱,无生之始,无命之终,乃是与神的儿子相似。

你们想一想,先祖亚伯拉罕,将自己所掳来上等之物取十分之一给他,这人是何等尊贵呢。

那是祭司职任的利未子孙,领命照例向百姓取十分之一,这百姓是自己的弟兄,虽是从亚伯拉罕身中生的,(身原文作腰),还是照例取十分之一

独有麦基洗德,不与他们同谱,倒收纳亚伯拉罕的十分之一,为那蒙应许的亚伯拉罕祝福。

从来位分大的给位分小的祝福,这是驳不倒的理。

在这里收十分之一的都是必死的人。但在那里收十分之一的,有为他作见证的说,他是活的。

并且可说,那受十分之一的利未,也是藉着亚伯拉罕纳了十分之一。

因为麦基洗德迎接亚伯拉罕的时候,利未已经在他先祖的身中。(身原文作腰)

从前百姓在利未人祭司职任以下受律法,倘若藉这职任能得完全,又何用另外兴起一位祭司,照麦基洗德的等次,不照亚伦的等次呢。

祭司的职任既已更改,律法也必须更改。

因为这话所指的人,本属别的支派,那支派里从来没有一人伺候祭坛。

我们的主分明是从犹大出来的。但这支派,摩西并没有题到祭司。

倘若照麦基洗德的样式,另外兴起一位祭司来,我的话更是显而易见的了。

他成为祭司,并不是照属肉体的条例,乃是照无穷之生命的大能。(无穷原文作不能毁坏)。

因为有给他作见证的说,你是照着麦基洗德的等次永远为祭司。

先前的条例,因软弱无益,所以废掉了。

(律法原来一无所成)就引进了更美的指望,靠这指望我们便可以进到神面前。

再者,耶稣为祭司,并不是不起誓立的。

至于那些祭司,原不是起誓立的,只有耶稣是起誓立的。因为那立他的对他说,主起了誓决不后悔,你是永远为祭司。

既是起誓立的,耶稣就作了更美之约的中保。

那些成为祭司的,数目本来多,是因为有死阻隔不能长久。

这位既是永远长存的,他祭司的职任,就长久不更换。

凡靠着他进到神面前的人,他都能拯救到底。因为他是长远活着,替他们祈求。

像这样圣洁,无邪恶,无玷污,远离罪人,高过诸天的大祭司,原是与我们合宜的。

他不像那些大祭司,每日必须先为自己的罪,后为百姓的罪献祭,因为他只一次将自己献上,就把这事成全了。

律法本是立软弱的人为大祭司。但在律法以后起誓的话,是立儿子为大祭司,乃是成全到永远的。

\chapter{希伯来书第8章}
我们所讲的事,其中第一要紧的,就是我们有这样的大祭司,已经坐在天上至大者宝座的右边,

在圣所,就是真帐幕里,作执事。这帐幕是主所支的,不是人所支的。

凡大祭司都是为献礼物和祭物设立的。所以这位大祭司也必须有所献的。

他若在地上,必不得为祭司,因为已经有照律法献礼物的祭司。

他们供奉的事,本是天上事的形状和影像,正如摩西将要造帐幕的时候,蒙神警戒他,说,你要谨慎,作各样的物件,都要照着在山上指示你的样式。

如今耶稣所得的职任是更美的,正如他作更美之约的中保。这约原是凭更美之应许立的。

那前约若没有瑕疵,就无处寻求后约了。

所以主指责他的百姓说,(或作所以主指前约的缺欠说)日子将到,我要与以色列家,和犹大家,另立新约。

不像我拉着他们祖宗的手,领他们出埃及的时候,与他们所立的约。因为他们不恒心守我的约,我也不理他们。这是主说的。

主又说,那些日子以后,我与以色列家所立的约乃是这样。我要将我的律法放在他们里面,写在他们心上,我要作他们的神,他们要作我的子民。

他们不用各人教导自己的乡邻,和自己的弟兄,说,你该认识主。因为他们从最小的到至大的,都必认识我。

我要宽恕他们的不义,不再记念他们的罪愆。

既说新约,就以前约为旧了。但那渐旧渐衰的,就必快归无有了。

\chapter{希伯来书第9章}
原来前约有礼拜的条例,和属世界的圣幕。

因为有豫备的帐幕,头一层叫作圣所。里面有灯台,桌子,和陈设饼。

第二幔子后,又有一层帐幕,叫作至圣所。

有金香炉,(炉或作坛)有包金的约柜,柜里有盛吗哪的金罐,和亚伦发过芽的杖,并两块约版。

柜上面有荣耀基路伯的影芍着施恩座。(施恩原文作蔽罪)这几件我现在不能一一细说。

这些物件既如此豫备齐了,众祭司就常进头一层帐幕,行拜神的礼。

至于第二层帐幕,惟有大祭司一年一次独自进去,没有不带着血,为自己和百姓的过错献上。

圣灵用此指明,头一层帐幕仍存的时候,进入至圣所的路还未显明。

那头一层帐幕作现今的一个表样,所献上的礼物和祭物,就着良心说,都不能叫礼拜的人得以完全。

这些事连那饮食和诸般洗濯的规矩,都不过是属肉体的条例,命定到振兴的时候为止。

但现在基督已经来到,作了将来美事的大祭司,经过那更大更全备的帐幕,不是人手所造也不是属乎这世界的。

并且不用山羊和牛犊的血,乃用自己的血,只一次进入圣所,成了永远赎罪的事。

若山羊和公牛的血,并母牛犊的灰洒在不洁的人身上,尚且叫人成圣,身体洁净。

何况基督藉着永远的灵,将自己无瑕无疵献给神,他的血岂不更能洗净你们的心。(原文作良心)除去你们的死行,使你们事奉那永生神吗。

为此他作了新约的中保。既然受死赎了人在前约之时所犯的罪过,便叫蒙召之人得着所应许永远的产业。

凡有遗命,必须等到留遗命的人死了。(遗命原文与约字同)

因为人死了,遗命才有效力,若留遗命的尚在,那遗命还有用处吗。

所以前约也不是不用血立的。

因为摩西当日照着律法,将各样诫命传给众百姓,就拿朱红色绒和牛膝草,把牛犊山羊的血和水,洒在书上,又洒在众百姓身上,说,

这血就是神与你们立约的凭据。

他又照样把血洒在帐幕,和各样器皿上。

按着律法,凡物差不多都是用血洁净的,若不流血,罪就不得赦免了。

照着天上样式作的物件,必须用这些祭物去洁净。但那天上的本物,自然当用更美的祭物去洁净。

因为基督并不是进了人手所造的圣所,(这不过是真圣所的影像)乃是进了天堂,如今为我们显在神面前。

也不是多次将自己献上,像那大祭司每年带着牛羊血进入圣所。(牛羊的血原文作不是自己的血)

如果这样,他从创世以来,就必多次受苦了。但如今在这末世显现一次,把自己献为祭,好除掉罪。

按着定命,人人都有一死,死后且有审判。

这样,基督既然一次被献,担当了多人的罪,将来要向那等候他的人第二次显现,并与罪无关,乃是为拯救他们。

\chapter{希伯来书第10章}
律法既是将来美事的影儿,不是本物的真像,总不能藉着每年常献一样的祭物,叫那近前来的人得以完全。

若不然,献祭的事岂不早已止住了吗。因为礼拜的人,良心既被洁净,就不再觉得有罪了。

但这些祭物是叫人每年想起罪来。

因为公牛和山羊的血,断不能除罪。

所以基督到世上来的时候,就说,神阿祭物和礼物是你不愿意的,你曾给我豫备了身体。

燔祭和赎罪祭是你不喜欢的。

那时我说,神阿,我来了为要照你的旨意行。我的事在经卷上已经记载了。

以上说,祭物和礼物,燔祭和赎罪祭,是你不愿意的,也是你不喜欢的,(这都是按着律法献的)。

后又说我来了为要照你的旨意行。可见他是除去在先的,为要立定在后的。

我们凭这旨意,靠耶稣基督只一次献上他的身体,就得以成圣。

凡祭司天天站着事奉神,履次献上一样的祭物。这祭物永不能除罪。

但基督献了一次永远的赎罪祭,就在神的右边坐下了。

从此等候他仇敌成了他的脚凳。

因为他一次献祭,便叫那得以成圣的人永远完全。

圣灵也对我们作见证。因为他既已说过,

主说,那些日子以后,我与他们所立的约乃是这样。我要将我的律法写在他们心上,又要放在他们的里面。

以后就说,我不再记念他们的罪愆,和他们的过犯。

这些罪过既已赦免,就不用再为罪献祭了。

弟兄们,我们既因耶稣的血,得以坦然进入至圣所,

是藉着他给我们开了一条又新又活的路从幔子经过,这幔子就是他的身体。

又有一位大祭司治理神的家。

并我们心中天良的亏欠已经洒去,身体用清水洗净了,就当存着诚心,和充足的信心,来到神面前。

也要坚守我们所承认的指望,不至摇动。因为那应许我们的是信实的。

又要彼此相顾,激发爱心,勉励行善。

你们不可停止聚会,好像那些停止惯了的人,倒要彼此劝勉。既知道(原文作看见)那日子临近,就更当如此。

因为我们得知真道以后,若故意犯罪,赎罪的祭就再没有了。

惟有战惧等候审判和那烧灭众人的烈火。

人干犯摩西的律法,凭两三个见证人,尚且不得怜恤而死。

何况人践踏神的儿子,将那使他成圣之约的血当作平常,又亵慢施恩的圣灵,你们想,他要受的刑罚该怎样加重呢。

因为我们知道谁说,伸冤在我,我必报应。又说,主要审判他的百姓

落在永生神的手里真是可怕的。

你们要追念往日,蒙了光照以后,所忍受大争战的各样苦难。

一面被毁谤,遭患难,成了戏景,叫众人观看。一面陪伴那些受这样苦难的人。

因为你们体恤了那些被捆锁的人,并且你们的家业被人抢去,也甘心忍受,知道自己有更美常存的家业。

所以你们不可丢弃勇敢的心,存这样的心必得大赏赐。

你们必须忍耐,使你们行完了神的旨意,就可以得着所应许的。

因为还有一点点时候,那要来的就来,并不迟延。

只是义人必因信得生。(义人有古卷作我的义人)他若退后,我心里就不喜欢他。

我们却不是退后入沉沦的那等人,乃是有信心以致灵魂得救的人。

\chapter{希伯来书第11章}
信就是所望之事的实底,是未见之事的确据。

古人在这信上得了美好的证据。

我们因着信,就知道诸世界是藉神话造成的。这样,所看见的,并不是从显然之物造出来的。

亚伯因着献祭与神,比该隐所献的更美,因此便得了称义的见证,就是神指他礼物作的见证。他虽然死了,却因这信仍旧说话。

以诺因着信被接去,不至于见死。人也找不着他,因为神已经把他接去了。只是他被接去以先,已经得了神喜悦他的明证。

人非有信就不能得神的喜悦。因为到神面前来的人,必须信有神,且信他赏赐那寻求他的人。

挪亚因着信,既蒙神指示他未见的事,动了敬畏的心,豫备了一只方舟,使他全家得救。因此就定了那世代的罪,自己也承受了那从信而来的义。

亚伯拉罕因着信,蒙召的时候,就遵命出去,往将来要得为业的地方去。出去的时候,还不知往那里去。

他因着信,就在所应许之地作客,好像在异地居住帐棚,与那同蒙一个应许的以撒,雅各一样。

因为他等候那座有根基的城,就是神所经营所建造的。

因着信,连撒拉自己,虽然过了生育的岁数,还能怀孕。因他以为那应许他的是可信的。

所以从一个彷佛已死的人就生出子孙,如同天上的星那样众多,海边的沙那样无数。

这些人都是存着信心死的,并没有得着所应许的,却从远处望见,且欢喜迎接,又承认自己在世上是客旅,是寄居的。

说这样话的人,是表明自己要找一个家乡。

他们若想念所离开的家乡,还有可以回去的机会。

他们却羡慕一个更美的家乡,就是在天上的。所以神被称为他们的神,并不以为耻。因为他已经给他们豫备了一座城。

亚伯拉罕因着信,被试验的时候,就把以撒献上。这便是那欢喜领受应许的,将自己独生的儿子献上。

论到这儿子曾有话说,从以撒生的才要称为你的后裔。

他以为神还能叫人从死里复活。他也彷佛从死中得回他的儿子来。

以撒因着信,就指着将来的事,给雅各以扫祝福。

雅各因着信,临死的时候,给约瑟的两个儿子各自祝福,扶着杖头敬拜神。

约瑟因着信,临终的时候,题到以色列族将来要出埃及,并为自己的骸骨留下遗命。

摩西生下来,他的父母见他是个俊美的孩子,就因着信把他藏了三个月,并不怕王命。

摩西因着信,长大了就不肯称为法老女儿之子。

他宁可和神的百姓同受苦害,也不愿暂时享受罪中之乐。

他看为基督受的凌辱,比埃及的财物更宝贵。因他想望所要得的赏赐。

他因着信就离开埃及,不怕王怒。因为他恒心忍耐,如同看见那不能看见的主)。

他因着信,就守逾越节,(守或作立)行洒血的礼,免得那灭长子的临近以色列人。

他们因着信,过红海如行乾地。埃及人试着要过去,就被吞灭了。

以色列人因着信,围绕耶利哥城七日,城墙就倒塌了。

妓女喇合因着信,曾和和平平的接待探子,就不与那些不顺从的人一同灭亡。

我又何必再说呢。若一一细说,基甸,巴拉,叁孙,耶弗他,大卫,撒母耳,和众先知的事,时候就不够了。

他们因着信,制伏了敌国,行了公义,得了应许,堵了狮子的口。

灭了烈火的猛势,脱了刀剑的锋刃,软弱变为刚强,争战显出勇敢,打退外邦的全军。

有妇人得自己的死人复活,又有人忍受严刑,不肯苟且得释放,(释放原文作赎)为要得着更美的复活。

又有人忍受戏弄,鞭打,捆锁,监禁,各等的磨炼。

被石头打死,被锯锯死,受试探,被刀杀。披着绵羊山羊的皮各处奔跑,受穷乏,患难,苦害。

在旷野,山岭,山洞,地穴,飘流无定。本是世界不配有的人。

这些人都是因信得了美好的证据,却仍未得着所应许的。

因为神给我们豫备了更美的事,叫他们若不与我们同得,就不能完全。

\chapter{希伯来书第12章}
我们既有这许多的见证人,如同云彩围着我们,就当放下各样的重担,脱去容易缠累我们的罪,存心忍耐,奔那摆在我们前头的路程,

仰望为我们信心创始成终的耶稣。(或作仰望那将真道创始成终的耶稣)他因那摆在前头的喜乐,就轻看羞辱,忍受了十字架的苦难,便坐在神宝座的右边。

那忍受罪人这样顶撞的,你们要思想,免得疲倦灰心。

你们与罪恶相争,还没有抵挡到留血的地步。

你们又忘了那劝你们如同劝儿子的话,说,我儿,你不可轻看主的管教,被他责备的时候,也不可灰心。

因为主所爱的他必管教,又鞭打凡所收纳的儿子。

你们所忍受的,是神管教你们,待你们如同待儿子。焉有儿子不被父亲管教的呢。

管教原是众子所共受的,你们若不受管教,就是私子,不是儿子了。

再者,我们曾有生身的父管教我们,我们尚且敬重他,何况万灵的父,我们岂不更当顺服他得生吗。

生身的父都是暂随己意管教我们。惟有万灵的父管教我们,是要我们得益处,使我们在他的圣洁上有分。

凡管教的事,当时不觉得快乐,凡觉得愁苦。后来却为那经练过的人,结出平安的果子,就是义。

所以你们要把下垂的手,发酸的腿,梃起来。

也要为自己的脚把道路修直了,使瘸子不至歪脚,反得痊愈。(歪脚或作差路)

你们要追求与众人和睦,并要追求圣洁。非圣洁没有人能见主。

又要谨慎,恐怕有人失了神的恩。恐怕有毒根生出来扰乱你们,因此叫众人沾染污秽。

恐怕有淫乱的,有贪恋世俗如以扫的。他因一点食物把自己长子的名分卖了。

后来想要承受父所祝的福,竟被弃绝,虽然号哭切求,却得不着门路,使他父亲的心意回转,这是你们知道的。

你们原不是来到那能摸的山,此山有火焰,密云,黑暗,暴风,

角声与说话的声音。那些听见这声音的。都求不要再向他们说话。

因为他们当不起所命他们的话说,靠近这山的,既便是走兽,也要用石头打死。

所见的极其可怕,甚至摩西说,我甚是恐惧战竞。

你们乃是来到锡安山,永生神的城邑,就是天上的耶路撒冷。那里有千万的天使,

有名录在天上诸长子之会所共聚的总会,有审判众人的神,和被成全之义人的灵魂。

并新约的中保耶稣,以及所洒的血。这血所说的比亚伯的血所说的更美。

你们总要谨慎,不可弃绝那向你们说话的。因为那些弃绝在地上警戒他们的,尚且不能逃罪,何况我们违背那从天上警戒我们的呢。

当时他的声音震动了地。但如今他应许说,再一次我不单要震动地,还要震动天。

这再一次的话,是指明被震动的,就是受造之物,都要挪去,使那不被震动的常存。

所以我们既得了不能震动的国,就当感恩,照神所喜悦的,用虔诚敬畏的心事奉神。

因为我们的神乃是烈火。

\chapter{希伯来书第13章}
你们务要存弟兄相爱的心。

不可忘记用爱心接待客旅。因为曾有接待客旅的,不知不觉就接待了天使。

你们要记念被捆绑的人,好像与他们同受捆绑,也要记念遭苦害的人,想到自己也在肉身之内。

婚姻,人人都当尊重,床也不可污秽。因为苟合行淫的人神必要审判。

你们存心不可贪爱钱财。要以自己所有的为足。因为主曾说,我总不撇下你,也不丢去你。

所以我们可以放胆说,主是帮助我的,我必不惧怕。人能把我怎吗样呢。

从前引导你们,传神之道给你们的人,你们要想念他们,效法他们的信心,留心看他们为人的结局。

耶稣基督,昨日今日一直到永远是一样的。

你们不要被那诸般怪异的教训勾引了去。因为人心靠恩得坚固才是好的。并不是靠饮食。那在饮食上专心的,从来没有得着益处。

我们有一祭坛,上面的祭物是那些在帐幕中供职的人不可同吃的。

原来牲畜的血,被大祭司带入圣所作赎罪祭,牲畜的身子,被烧在营外。

所以耶稣,要用自己的血叫百姓成圣,也就在城门外受苦。

这样,我们也当出到营外就了他去,忍受他所受的凌辱。

我们在这里本没有常存的城,乃是寻求那将来的城。

我们应当靠着耶稣,常常以颂赞为祭献给神,这就是那承认主名之人嘴唇的果子。

只是不可忘记行善,和捐输的事。因为这样的祭,是神所喜悦的。

你们要依从那些引导你们的,且要顺服。因他们为你们的灵魂时刻儆醒,好像那将来交账的人。你们要使他们交的时候有快乐,不至忧愁。若忧愁就与你们无益了。

请你们为我们祷告。因我们自觉良心无亏,愿意凡事按正道而行。

我更求你们为我祷告,使我快些回到你们那里去。

但愿赐平安的神,就是那凭永约之血使群羊的大牧人我主耶稣,从死里复活的神,

在各样善事上,成全你们,叫你们遵行他的旨意,又藉着耶稣基督在你们心里行他所喜悦的事。愿荣耀归给他,直到永永远远。阿们。

弟兄们,我略略写给你们,望你们听我劝勉的话。

你们该知道我们的弟兄提摩太已经释放了。他若快来,我必同他夸见你们。

请你们问引导你们的诸位和众圣徒安。从意大利来的人也问你们安。

愿恩惠常与你们众人同在。阿们。

\chapter{雅各书第1章}
作神和主耶稣基督仆人的雅各,请散住十二个支派之人的安。

我的弟兄们,你们落在百般的试炼中,都要以为大喜乐。

因为知道你们的信心经过试验就生忍耐。

但忍耐也当成功,使你们成全完备,毫无缺欠。

你们中间若有缺少智慧的,应当求那厚赐与众人,也不斥责人的神,主就必赐给他。

只要凭着信心求,一点不疑惑。因为那疑惑的人,就像海中的波浪,被风吹动翻腾。

这样的人,不要想从主那里得什么。

心怀二意的人,在他一切所行的路上,都没有定见。

卑微的弟兄升高,就该喜乐。

富足的降卑,也该如此。因为他必要过去,如同草上的花一样。

太阳出来,热风刮起,草就枯乾,花也凋谢,美容就消没了。那富足的人,在他所行的事上,也要这样衰残。

忍受试探的人是有福的。因为他经过试验以后,必得生命的冠冕,这是主应许给那些爱他之人的。

人被试探,不可说,我是被神试探。因为神不能被恶试探,他也不试探人。

但各人被试探,乃是被自己的私欲牵引诱惑的。

私欲既怀了胎,就生出罪来。罪既长成,就生出死来。

我亲爱的弟兄们,不要看错了。

各样美善的恩赐,和各样全备的赏赐,都是从上头来的。从众光之父那里降下来的。在他并没有改变,也没有转动的影儿。

他按自己的旨意,用真道生了我们,叫我们在他所造的万物中,好像初熟的果子。

我亲爱的弟兄们,这是你们所知道的。但你们各人要快快的听,慢慢的说,慢慢的动怒。

因为人的怒气,并不成就神的义。

所以你们要脱去一切的污秽,和盈馀的邪恶,存温柔的心领受那所栽种的道,就是能救你们灵魂的道。

只是你们要行道,不要单单听道,自己欺哄自己。

因为听道而不行道的,就像人对着镜子看自己本来的面目。

看见,走后,随即忘了他的相貌如何。

惟有那详细察看那全备使人自由之律法的,并且时常如此,这人既不是听了就忘,乃是实在行出来,就在他所行的事上必然得福。

若有人自以为虔诚,却不勒住他的舌头,反欺哄自己的心,这人的虔诚是虚的。

在神我们的父面前,那清洁没有玷污的虔诚,就是看雇在患难中的孤儿寡妇,并且保守自己不沾染世俗。

\chapter{雅各书第2章}
我的弟兄们,你们信奉我们荣耀的主耶稣基督,便不可按着外貌待人。

若有一个人带着金戒指,穿着华美衣服,进你们的会堂去。又有一个穷人,穿着肮脏衣服也进去。

你们就重看那穿华美衣服的人,说,请坐在这好位上。又对那穷人说,你站在那里,或坐在我脚凳下边。

这岂不是你们用偏心待人,用恶意断定人吗。

我亲爱的弟兄们请听,神岂不是拣选了世上的贫穷人,叫他们在信上富足,并承受他所应许给那些爱他之人的国吗。

你们反倒羞辱贫穷人。那富足人岂不是欺压你们,拉你们到公堂去吗。

他们不是亵渎你们所敬奉的尊名吗。(所敬奉或作被称)

经上记着说,要爱人如己。你们若全守这至尊的律法才是好的。

但你们若按外貌待人,便是犯罪,被律法定为犯法的。

因为凡遵守全律法的,只在一条上跌倒,他就是犯了众条。

原来那说不可奸淫的,也说不可杀人。你就是不奸淫,却杀人,仍是成了犯律法的。

你们既然要按使人自由的律法受审判,就该照这律法说话行事。

因为那不怜悯人的,也要受无怜悯的审判。怜悯原是向审判夸胜。

我的弟兄们,若有人说,自己有信心,却没有行为,有什么益处呢。这信心能救他吗。

若是弟兄,或是姐妹,赤身露体,又缺了日用的饮食,

你们中间有人对他们说,平平安安的去吧,愿你们穿得暖吃得饱。却不给他们身体所需用的,这有什么益处呢。

这样信心若没有行为就是死的。

必有人说,你有信心,我有行为。你将你没有行为的信心指给我看,我便藉着我的行为,将我的信心指给你看。

你信神只有一位,你信的不错。鬼魔也信,却是战兢。

虚浮的人哪,你愿意知道没有行为的信心是死的吗。

我们的祖宗亚伯拉罕,把他儿子以撒献在坛上,岂不是因行为称义吗。

可见信心是与他的行为并行,而且信心因着行为才得成全。

这就应验经上所说,亚伯拉罕信神,这就算为他的义。他又得称为神的朋友。

这样看来,人称义是因着行为,不是单因着信。

妓女喇合接待使者,又放他们从别的路出去,不也是一样因行为称义吗。

身体没有灵魂是死的,信心没有行为也是死的。

\chapter{雅各书第3章}
我的弟兄们,不要多人作师父,因为晓得我们要受更的判断。

原来我们在许多事上都有过失。若有人在话语上没有过失,他就是完全人,也能勒住自己的全身。

我们若把嚼环放在马嘴里,叫他顺服,就能调动他的全身。

看哪,船只虽然甚大,又被大风催逼,只用小小的舵,就随着掌舵的意思转动。

这样,舌头在百体里也是最小的,却能说大话。看哪,最小的火能点着最大的树林。

舌头就是火,在我们百体中,舌头是个罪恶的世界,能污秽全身,也能把生命的轮子点起来,并且是从地狱里点着的。

各类的走兽,飞禽,昆虫,水族,本来都可以制伏,也以经被人制伏了。

惟独舌头没有人能制伏,是不止息的恶物,满了害死人的毒气。

我们用舌头颂赞那为主为父的,又用舌头咒诅那照着神形像被造的人。

颂赞和咒诅从一个口里出来,我的弟兄们,这是不应当的。

泉源从一个眼里能发出甜苦两样的水吗。

我的弟兄们,无花果树能生橄榄吗,葡萄树能结无花果吗。咸水里也不能发出甜水来。

你们中间有智慧有见识的呢。他就当在智慧的温柔上,显出他的善行来。

你们心里若怀着苦毒的嫉妒和分争,就不可自夸,也不可说谎话抵挡真道。

这样的智慧,不是从上头来的,乃是属地的,属情欲的,属鬼魔的。

在何处有嫉妒分争,就在何处有扰乱,和各样的坏事。

惟独从上头来的智慧,先是清洁,后是和平,温良柔顺,满有怜悯,多结善果,没有偏见,没有假冒。

并且使人和平的,是用和平所栽种的义果。

\chapter{雅各书第4章}
你们中间的争战斗殴,是从那里来的呢。不是从你们百体中战斗之私欲来的吗。

你们贪恋,还是得不着。你们杀害嫉妒,又斗殴争战,也不能得。你们得不着,是因为你们不求。

你们求也得不着,是因为你们妄求,要浪费在你们的宴乐中。

你们这些淫乱的人哪,(淫乱的人原文作淫妇)岂不知与世俗为友,就是与神为敌吗。所以凡想要与世俗为友的,就是与神为敌了。

你们想经上所说的是徒然的吗。神所赐住在我们里面的灵,是恋爱至于嫉妒吗。

但他赐更多的恩典。所以经上说,神阻挡骄傲的人,赐恩给谦卑的人。

故此你们要顺服神,务要抵挡魔鬼,魔鬼就必离开你们逃跑了。

你们亲近神,神就必亲近你们。有罪的人哪,要洁净你们的手。心怀二意的人哪,要清洁你们的心。

你们要愁苦,悲哀,哭泣。将喜笑变作悲哀,欢乐变作愁闷。

务要在主面前自卑,主就比叫你们升高。

弟兄们,你们不可彼此批评。人若批评弟兄,论断弟兄,就是批评律法,论断律法。你若论断律法,就不是遵行律法,乃是判断人的。

设立律法和判断人的,只有一位,就是那能救人也能灭人的。你是谁,竟敢论断别人呢。

??,你们有话说,今天明天我们要往某城里去,在那里住一年,作买卖得利。

其实明天如何,你们还不知道。你们的生命是什么呢。你们原来是一片云雾,出现少时就不见了。

你们只当说,主若愿意,我们就可以活着,也可以作这事,或作那事。

现今你们竟以张狂跨口。凡这样跨口都是恶的。

人若知道行善,却不去行,这就是他的罪了。

\chapter{雅各书第5章}
??,你们这些富足人哪,应当哭泣,号??,因为将有苦难临到你们身上。

你们的财物坏了,衣服也被虫子咬了。

你们的金银都长了锈。那锈要证明你们的不是,又要吃你们的肉,如同火烧。你们在这末世,只知积攒钱财。

工人给你们收割庄稼,你们亏欠他们的工钱。这工钱有声音呼叫。并且那收割之人的冤声,已经入了万军之主的耳了。

你们在世上享美福,好宴乐,当宰杀的日子竟娇养你们的心。

你们定了义人的罪,把他杀害,他也不抵挡你们。

弟兄们哪,你们要忍耐直到主来。看哪,农夫忍耐等候地里宝贵的出产,直到得了秋雨春雨。

你们也当忍耐,坚固你们的心。因为主来的日子近了。

弟兄们,你们不要彼此埋怨,免得受审判。看哪,审判的主站在门前了。

弟兄们,你们要把那先前奉主名说话的众先知,当作能受苦能忍耐的榜样。

那先前忍耐的人,我们称他们是有福的,你们听见过约伯的忍耐,也知道主给他的结局,明显主是满有怜悯,大有慈悲。

我的弟兄们,最要紧的是不可起誓。不可指着天起誓,也不可指着地起誓,无论何誓都不可起。你们说话,是就说是,不是就说不是,免得你们落在审判之下。

你们中间有受苦的呢,他就该祷告。有喜乐的呢,他就该歌颂。

你们中间有病了的呢,他就该请教会的长老来。他们可以奉主的名用油抹他,为他祷告。

出于信心的祈祷,要救那病人,主必叫他起来。他若犯了罪,也必蒙赦免。

所以你们要彼此认罪,互相代求,使你们可以得医治。义人祈祷所发的力量,是大有功效的。

以利亚与我们是一样性情的人,他恳切祷告,求不要下雨,雨就三年零六个月不下在地上。

他又祷告,天就降下雨来,地也生出土产。

我的弟兄们,你们中间若有失迷真道的,有人使他回转。

这人该知道叫一个罪人从迷路上转回,便是救一个灵魂不死,并且遮盖许多的罪。

\chapter{彼得前书第1章}
耶稣基督的使徒彼得,写信给那分散在本都,加拉太,加帕多家,亚细亚,庇推尼寄居的。

就是照父神的先见被拣选,藉着圣灵得成圣洁,以致顺服耶稣基督,又蒙他血所洒的人。愿恩惠平安,多多的加给你们。

愿颂赞归与我们的主耶稣基督的父神,他曾照自己的大怜悯,藉耶稣基督从死里复活重生了我们,教我们有活泼的盼望,

可以得着不能朽坏,不能玷污,不能衰残,为你们存留在天上的基业。

你们这因信蒙神能力保守的人,必能得着所豫备,到末世要显现的救恩。

因此,你们是大有喜乐,但如今,在百般的试炼中暂时忧愁。

叫你们的信心既被试验,就比那被火试验,仍然能坏的金子,更显宝贵。可以在耶稣基督显现的时候,得着称赞,荣耀,尊贵。

你们虽然没有见过他,却是爱他。如今虽不得看见,却因信他就有说不出来,满有荣光的大喜乐。

并且得着你们信心的果效,就是灵魂的救恩。

论到这救恩,那豫先说你们要得恩典的众先知,早已详细的寻求考察。

就是考察在他们心里基督的灵,豫先证明基督受苦难,后来得荣耀,是指着什么时候,并怎样的时候。

他们得了启示,知道他们所传讲的一切事(传讲原文作服事),不是为自己,乃是为你们。那靠着从天上差来的圣灵,传福音给你们的人,现在将这些事报给你们。天使也愿意详细察看这些事。

所以要约束你们的心(原文作束上你们心中的腰),谨慎自守,专心盼望耶稣基督显现的时候所带来给你们的恩。

你们既作顺命的儿女,就不要效法从前蒙昧无知的时候,那放纵私欲的样子。

那召你们的既是圣洁,你们在一切所行的事上也要圣洁。

因为经上记着说,你们要圣洁,因为我是圣洁的。

你们既称那不偏待人,按各人行为审判人的主为父,就当存敬畏的心,度你们在世寄居的日子。

知道你们得救赎,脱去你们祖宗所传流虚妄的行为,不是凭着能坏的金银等物。

乃是凭着基督的宝血,如同无瑕疵无玷污的羔羊之血。

基督在创世以前,是豫先被神知道的,却在这末世,才为你们显现。

你们也因信着他,信那叫他从死里复活,又给他荣耀的神,叫你们的信心,和盼望,都在于神。

你们既因顺从真理,洁净了自己的心,以致爱弟兄没有虚假,就当从心里彼此切实相爱。从心里有古卷作从清洁的心

你们蒙了重生,不是由于能坏的种子,乃是由于不能坏的种子,是藉着神活泼常存的道。

因为凡有血气的,尽都如草,他的美荣,都像草上的花。草必枯乾,花必凋谢。

惟有主的道是永存的。所传给你们的福音就是这道。

\chapter{彼得前书第2章}
所以你们既除去一切的恶毒(或作阴毒),诡诈,并假善,嫉妒,和一切毁谤的话,

就要爱慕那纯净的灵奶,像才生的婴孩爱慕奶一样,叫你们因此渐长,以致得救。

你们若尝过主恩的滋味,就必如此

主乃活石。固然是被人所弃的,却是被神所拣选所宝贵的。

你们来到主面前,也就像活石,被建造成为灵宫,作圣洁的祭司,藉着耶稣基督奉献神所悦纳的灵祭。

因为经上说,看哪,我把所拣选所宝贵的房角石,安放在锡安。信靠他的人,必不至于羞愧。

所以他在你们信的人就为宝贵,在那不信的人有话说,匠人所弃的石头,已作了房角的头块石头。

又说,作了绊脚的石头,跌人的磐石。他们既不顺从,就在道理上绊跌。(或作他们绊跌都因不顺从道理)他们这样绊跌也是豫定的。

惟有你们是被拣选的族类,是有君尊的祭司,是圣洁的国度,是属神的子民,要叫你们宣扬那召你们出黑暗入奇妙光明者的美德。

你们从前算不得子民,现在却作了神的子民。从前未曾蒙怜恤,现在却蒙了怜恤。

亲爱的弟兄阿你们是客旅,是寄居的。我劝你们要禁戒肉体的私欲。这私欲是与灵魂争战的。

你们在外邦人中,应当品行端正,叫那些毁谤你们是作恶的,因看见你们的好行为,便在鉴察的日子(鉴察或作眷顾),归荣耀给神。

你们为主的缘故,要顺服人的一切制度,或是在上的君王,

或是君王所派罚恶赏善的臣宰。

因为神的旨意原是要你们行善,可以堵住那糊涂无知人的口。

你们虽是自由的,却不可藉着自由遮盖恶毒(或作阴毒),总要作神的仆人。

务要尊敬众人。亲爱教中的弟兄。敬畏神。尊敬君王。

你们作仆人的,凡事要存敬畏的心顺服主人。不但顺服那善良温和的,就是那乖僻的也要顺服。

倘若人为叫良心对得住神,就忍受冤屈的苦楚,这是可喜爱的。

你们若因犯罪受责打,能忍耐,有什么可夸的呢。但你们若因行善受苦,若能忍耐,这在神看是可喜爱的。

你们蒙恩原是为此。因基督也为你们受过苦,给你们留下榜样,叫你们跟随他的脚踪行。

他并没有犯罪,口里也没有诡诈。

他被骂不还口。受害不说威吓的话。只将自己交托那按公义审判人的主。

他被挂在木头上亲身担当了我们的罪,使我们既然在罪上死,就得以在义上活。因他受的鞭伤,你们便得医治。

你们从前好像迷路的羊。如今却归到你们灵魂的牧人监督了。

\chapter{彼得前书第3章}
你们作妻子的,要顺服自己的丈夫。这样,若有不信从道理的丈夫,他们虽然不听道,也可以因妻子的品行被感化过来。

这是因为看见你们有贞洁的品行,和敬畏的心。

你们不要以外面的辫头发,戴金饰,穿美衣,为妆饰,

只要以里面存着长久温柔安静的心为妆饰。这在神面前是极宝贵的。

因为古时仰赖神的圣洁妇人,正是以此为妆饰,顺服自己的丈夫。

就如撒拉听从亚伯拉罕,称他为主。你们若行善,不因恐吓而害怕,便是撒拉的女儿了。

你们作丈夫的,也要按情理和妻子同住(情理原文作知识)。因他比你软弱(比你软弱原文是软弱的器皿),与你一同承受生命之恩的,所以要敬重他。这样便叫你们的祷告没有阻碍。

总而言之,你们都要同心,彼此体恤,相爱如弟兄,存慈怜谦卑的心。

不要以恶报恶,以辱骂还辱骂,倒要祝福。因你们是为此蒙召,好叫你们承受福气。

因为经上说,人若爱生命,愿享美福,须要禁止舌头不出恶言,嘴唇不说诡诈的话。

也要离恶行善。寻求和睦,一心追赶。

因为主的眼看顾义人,主的耳听他们的祈祷。惟有行恶的人,主向他们变脸。

你们若是热心行善,有谁害你们呢。

你们就是为义受苦,也是有福的。不要怕人的威吓,也不要惊慌(的威吓或作所怕的)。

只要心里尊主基督为圣。有人问你们心中盼望的缘由,就要常作准备,以温柔敬畏的心回答各人。

存着无亏的良心,叫你们在何事上被毁谤,就在何事上,可以叫那诬赖你们在基督里有好品行的人,自觉羞愧。

神的旨意若叫你们因行善受苦,总强如因行恶受苦。

因基督也曾一次为罪受苦(受苦有古卷作受死),就是义的代替不义的,为要引我们到神面前。按着肉体说他被治死。按着灵性说他复活了。

他藉这灵,曾去传道给那些在监狱里的灵听。

就是那从前在挪亚豫备方舟,神容忍等待的时候,不信从的人。当时进入方舟,藉着水得救的不多,只有八个人。

这水所表明的洗礼,现在藉着耶稣基督复活。也拯救你们。这洗礼本不在乎除掉肉体的污秽,只求在神面前有无亏的良心。

耶稣已经进入天堂,在神的右边。众天使和有权炳的,并有能力的,都服从了他。

\chapter{彼得前书第4章}
基督既在肉身受苦,你们也当将这样的心志作为兵器。因为在肉身受过苦的,就已经与罪断绝了。

你们存这样的心,从今以后,就可以不从人的情欲,只从神的旨意,在世度馀下的光阴。

因为往日随从外邦人的心意,行邪淫,恶欲,醉酒,荒宴,群饮,并可恶拜偶像的事,时候已经够了。

他们在这些事上,见你们不与他们同奔那放荡无度的路,就以为怪毁谤你们。

他们必在那将要审判活人死人的主面前交账。

为此,就是死人也曾有福音传给他们,要叫他们的肉体按着人受审判,他们的灵性却靠神活着。

万物的结局近了。所以你们要谨慎自守,儆醒祷告。

最要紧的是彼此切实相爱。因为爱能遮掩许多的罪。

你们要互相款待,不发怨言。

各人要照所得的恩赐彼此服事,作神百般恩赐的好管家。

若有讲道的,要按着神的圣言讲。若有服事人的,要按着神所赐的力量服事。叫神在凡事上因耶稣基督得荣耀。原来荣耀权能都是他的,直到永永远远。阿们。

亲爱的弟兄阿,有火炼的试验临到你们,不要以为奇怪,(似乎是遭遇非常的事)

倒要欢喜。因为你们是与基督一同受苦,使你们在他荣耀显现的时候,也可以欢喜快乐。

你们若为基督的名受辱骂,便是有福的。因为神荣耀的灵,常住在你们身上。

你们中间却不可有人,因为杀人,偷窃,作恶,好管闲事而受苦。

若为作基督徒受苦,却不要羞耻。倒要因这名归荣耀给神。

因为时候到了,审判要从神的家起首。若是先从我们起首,那不信从神福音的人,将有何等的结局呢。

若是义人仅仅得救,那不虔敬和犯罪的人,将有何地可站呢。

所以那照神旨意受苦的人,要一心为善,将自己灵魂交与那信实的造化之主。

\chapter{彼得前书第5章}
我这作长老,作基督受苦的见证,同享后来所要显现之荣耀的,劝你们中间与我同作长老的人。

务要牧养在你们中间神的群羊,按着神旨意照管他们。不是出于勉强,乃是出于甘心。也不是因为贪财,乃是出于乐意。

也不是辖制所托付你们的,乃是作群羊的榜样。

到了牧长显现的时候,你们必得那永不衰残的荣耀冠冕。

你们年幼的,也要顺服年长的。就是你们众人,也都要以谦卑束腰,彼此顺服。因为神阻挡骄傲的人,赐恩给谦卑的人。

所以你们要自卑,服在神大能的手下,到了时候他必叫你们升高。

你们要将一切的忧虑卸给神,因为他顾念你们。

务要谨守,儆醒。因为你们的仇敌魔鬼,如同吼叫的狮子,遍地游行,寻梢可吞吃的人。

你们要用坚固的信心抵挡他,因为知道你们在世上的众弟兄,也是经历这样的苦难。

那赐诸般恩典的神,曾在基督里召你们,得亨他永远的荣耀,等你们暂受苦难之后,必要亲自成全你们,坚固你们,赐力量给你们。

愿权能归给他,直到永永远远。阿们。

我略略的写了这信,托我所看为忠心的兄弟西拉转交你们,劝勉你们,又证明这恩是神的真恩。你们务要在这恩上站立得住。

在巴比伦与你们同蒙拣选的教会问你们安。我儿子马可也问你们安。

你们要用爱心彼此亲嘴问安。愿平安归与你们凡在基督里的人。

\chapter{彼得后书第1章}
作耶稣基督仆人和使徒的西门彼得,写信给那因我们的神,和(有古卷无和字)救主耶稣基督之义,与我们同得一样宝贵信心的人。

愿恩惠平安,因你们认识神和我们主耶稣,多多的加给你们。

神的神能已经将一切关乎生命和虔敬的事赐给我们,皆因我们认识那用自己荣耀和美德召我们的主。

因此他已将又宝贵又极大的应许赐给我们,叫我们既脱离世上从情欲来的败坏,就得与神的性情有分。

正因这缘故,你们要分外的殷勤。有了信心,又要加上德行。有了德行,又要加上知识。

有了知识,又要加上节制。有了节制,又要加上忍耐。有了忍耐,又要加上虔敬。

有了虔敬,又要加上爱弟兄的心。有了爱弟兄的心,又要加上爱众人的心

你们若充充足足的有这几样,就必使你们在认识我们的主耶稣基督上,不至于闲懒不结果子了。

人若没有这几样,就是眼瞎,只看见近处的,忘了他旧日的罪已经得了洁净。

所以弟兄们,应当更加殷勤,使你们所蒙的恩召和拣选坚定不移。你们若行这几样,就永不失脚。

这样,必叫你们丰丰富富的,得以进入我们主救主耶稣基督永远的国。

你们虽然晓得这些事,并且在你们已有的真道上坚固,我却要将这些事常常题醒你们。

我以为应当趁我还在这帐棚的时候题醒你们,激发你们。

因为知道我脱离这帐棚的时候快到了,正如我们主耶稣基督所指示我的。

并且我尽心竭力,使你们在我去世以后,时常记念这些事。

我们从前,将我们主耶稣基督的大能,和他降临的事,告诉你们,并不是随从乖巧捏造的虚言,乃是亲眼见过他的威荣。

他从父神得尊贵荣耀的时候,从极大荣光之中,有声音出来向他说,这是我的爱子,我所喜悦的。

我们同他在圣山的时候,亲自听见这声音从天上出来。

我们并有先知更确的预言,如同灯照在暗处。你们在这预言上留意,直等到天发亮晨星在你们心里出现的时候,才是好的。

第一要紧的,该知道经上所有的预言,没有可随私意解说的。

因为预言从来没有出于人意的,乃是人被圣灵感动说出神的话来。

\chapter{彼得后书第2章}
从前在百姓中有假先知起来,将来在你们中间,也必有假师傅,私自引进害人的异端,连买他们的主他们也不承认,自取速速的灭亡。

将有许多人随从他们邪淫的行为,便叫真道,因他们的缘故被毁谤。

他们因有贪心,要用捏造的言语,在你们身上取利。他们的刑罚,自古以来并不迟延,他们的灭亡也必速速来到(原文作也不打盹)。

就是天使犯了罪,神也没有宽容,曾把他们丢在地狱,交在黑暗坑中,等候审判。

神也没有宽容上古的世代,曾叫洪水临到那不敬虔的世代,却保护了传义道的挪亚一家八口。

又判定所多玛,蛾摩拉,将二城倾覆,焚烧成灰,作为后世不敬虔人的监戒。

只搭救了那常为恶人淫行忧伤的义人罗得。

因为那义人住在他们中间,看见听见他们不法的事,他的义心就天天伤痛。

主知道搭救敬虔的人脱离试探,把不义的人留在刑罚之下,等候审判的日子。

那些随肉身,纵污秽的情欲,轻慢主治之人的,更是如此。他们胆大任性,毁谤在尊位的也不知惧怕。

就是天使,虽然力量权能更大,还不用毁谤的话在主面前告他们。

但这些人好像没有灵性,生来就是畜类,以备捉拿宰杀的。他们毁谤所不晓得的事,正在败坏人的时候,自己必遭遇败坏。

行的不义,就得了不义的工价。这些人喜爱白昼宴乐,他们已被玷污,又有瑕疵,正与你们一同坐席,就以自己的诡诈为快乐。

他们满眼是淫色(淫色原文作淫妇),止不住犯罪。引诱那心不坚固的人,心中习惯了贪婪,正是被咒诅的种类。

他们离弃正路,就走差了,随从比珥之子巴兰的路,巴兰就是那贪爱不义之工价的先知。

他们却为自己的过犯受了责备。那不能说话的驴,以人言拦阻先知的狂妄。

这些人是无水的井,是狂风催逼的雾气,有墨黑的幽暗为他们存留。

他们说虚妄矜夸的大话,用肉身的情欲,和邪淫的事,引诱那些刚才脱离妄行的人。

他们应许人得以自由,自己却作败坏的奴仆。因为人被谁制伏就是谁的奴仆。

倘若他们因认识主救主耶稣基督,得以脱离世上的污秽,后来又在其中被缠住制伏,他们末后的景况,就比先前更不好了。

他们晓得义路,竟背弃了传给他们的圣命,倒不如不晓得为妙。

俗语说得真不错,狗所吐的他转过来又吃。猪洗净了又回到泥里去辊。这话在他们身上正合式。

\chapter{彼得后书第3章}
亲爱的兄弟阿,我现在写给你们的是第二封信。这两封都是题醒你们,激发你们诚实的心。

叫你们记念圣先知豫先所说的话,和主救主的命令,就是使徒所传给你们的。

第一要紧的,该知道在末世必有好讥诮的人,随从自己的私欲出来讥诮说,

主要降临的应许在那里呢。因为从列祖睡了以来,万物与起初创造的时候仍是一样。

他们故意忘记,从太古凭神的命有了天,并从水而出藉水而成的地。

故此,当时的世界被水淹没就消灭了。

但现在的天地,还是凭着那命存留,直留到不敬虔之人受审判遭沈沦的日子,用火焚烧。

亲爱的弟兄阿,有一件事你们不可忘记,就是主看一日如千年,千年如一日。

主所应许的尚未成就,有人以为他是耽延,其实不是耽延,乃是宽容你们,不愿有一人沈沦,乃愿人人都悔改。

但主的日子要像贼来到一样。那日天必大有响声废去,有形质的都要被烈火销化。地和其上的物都要烧尽了。

这一切既都要如此销化,你们为人该当怎样圣洁,怎样敬虔,

切切仰望神的日子来到。在那日天被烧就销化了,有形质的都要被烈火熔化。

但我们照他的应许,盼望新天新地,有义居在其中。

亲爱的弟兄阿,你们既盼望这些事,就当殷勤,使自己没有玷污,无可指摘,安然见主。

并且要以我主长久忍耐为得救的因由,就如我们所亲爱的兄弟保罗,照着所赐给他的智慧,写了信给你们。

他一切的信上,也都是讲论这事。信中有些难明白的,那无学问不坚固的人强解,如强解别的经书一样,就自取沈沦。

亲爱的弟兄阿,你们既然豫先知道这事,就当防备,恐怕被恶人的错谬诱惑,就从自己坚固的地步上坠落。

你们却要在我们主救主耶稣基督的恩典和知识上有长进。愿荣耀归给他,从今直到永远。阿们。

\chapter{约翰壹书第1章}
论到从起初原有的生命之道,就是我们所听见所看见的,亲眼看过,亲手摸过的。

(这生命已经显现出来,我们也看见过,现在又作见证,将原与父同在,且显现与我们那永远的生命,传给你们)。

我们将所看见,所听见的,传给你们,使你们与我们相交。我们乃是与父并他儿子耶稣基督相交的。

我们将这些话写给你们,使你们(有古卷作我们)的喜乐充足。

神就是光,在他毫无黑暗。这是我们从主所听见,又报给你们的信息。

我们若说是与神相交,却仍在黑暗里行,就是说谎话,不行真理了。

我们若在光明中行,如同神在光明中,就彼此相交,他儿子耶稣的血也洗净我们一切的罪。

我们若说自己无罪,便是自欺,真里不在我们心里了。

我们若认自己的罪,神是信实的,是公义的,必要赦免我们的罪,洗净我们一切的不义。

我们若说自己没有犯过罪,便是以神为说谎的。他的道也不在我们心里了。

\chapter{约翰壹书第2章}
我小子们哪,我将这些话写给你们,是要叫你们不犯罪。若有人犯罪,在父那里我们有一位中保,就是那义者耶稣基督。

他为我们的罪作了挽回祭。不是单为我们的罪,也是为普天下人的罪。

我们若遵守他的诫命,就晓得是认识他。

人若说我认识他,却不遵守他的诫命,便是说谎话的。真里也不在他心里了。

凡遵守主道的,爱神的心在他里面实在是完全的,从此我们知道我们是在主里面。

人若说他住在主里面,就该自己照主所行的去行。

亲爱的弟兄阿,我写给你们的,不是一条新命令,乃是你们从起初所受的旧命令。这旧命令就是你们所听见的道。

再者,我写给你们的,是一条新命令,在主是真的,在你们也是真的。因为黑暗渐渐过去,真光已经照耀。

人若说自己在光明中。却恨他的弟兄,他到如今还是在黑暗里。

爱弟兄的就是住在光明中,在他并没有绊跌的缘由。

唯独恨弟兄的是在黑暗里,且在黑暗里行,也不知道往那里去,因为黑暗叫他眼睛瞎了。

小子们哪,我写信给你们,因为你们的罪藉着主名得了赦免。

父老阿,我写信给你们,因为你们认识那从起初原有的。少年人哪,我写信给你们,因为你们胜了那恶者。小子们哪,我曾写信给你们,因为你们认识父。

父老阿,我曾写信给你们,因为你们认识那从起初原有的。少年人哪,我曾写信给你们,因为你们刚强,神的道常存在你们心里,你们也胜了那恶者。

不要爱世界,和世界上的事。人若爱世界,爱父的心就不在他里面了。

因为凡世界上的事,就像肉体的情欲,眼目的情欲,并今生的骄傲,都不是从父来的,乃是从世界来的。

这世界,和其上的情欲,都要过去。唯独遵行神旨意的,是永远常存。

小子们哪,如今是末时了。你们曾听见说,那敌基督的要来现在已经有好些敌基督的出来了。从此我们就知道如今是末时了。

他们从我们中间出去,却不是属我们的。若是属我们的,就必仍旧与我们同在。他们出去,显明都不是属我们的。

你们从那圣者受了恩膏,并且知道这一切的事。(或作都有知识)

我写信给你们,不是因你们不知道真理,正是因你们知道,并且知道没有虚谎是从真理出来的。

谁是说谎话的呢。不是那不认耶稣为基督的吗。不认父与子的,这就是敌基督的。

凡不认子的就没有父。认子的连父也有了。

论到你们,务要将那从起初所听见的常存在心里。若将从起初所听见的存在心里,你们就必住在子里面,也必住在父里面。

主所应许我们的就是永生。

我将这些话写给你们,是指着那引诱你们的人说的。

你们从主所受的恩膏,常存在你们心里,并不用人教训你们。自有主的恩膏在凡事上教训你们。这恩膏是真的,不是假的。你们要按这恩膏的教训,住在主里面。

小子们哪,你们要住在主里面。这样,他若显现,我们就可以坦然无惧。当他来的时候,在他面前也不至于惭愧。

你们若知道他是公义的,就知道凡行公义之人都是他所生的。

\chapter{约翰壹书第3章}
你看父赐给我们是何等的慈爱,使我们得称为神的儿女。我们也真是他的儿女。世人所以不认识我们,是因未曾认识他。

亲爱的弟兄阿,我们现在是神的儿女,将来如何,还未显明。但我们知道主若显现,我们必要像他。因为必得见他的真体。

凡向他有这指望的,就洁净自己,像他洁净一样。

凡犯罪的,就是违背律法。违背律法就是罪。

你们知道主曾显现,是要除掉人的罪。在他并没有罪。

凡住在他里面的,就不犯罪。凡犯罪的,是未曾看见他,也未曾认识他。

小子们哪,不要被人诱惑,行义的才是义人。正如主是义的一样。

犯罪的是属魔鬼,因为魔鬼从起初就犯罪。神的儿子显现出来,为要除灭魔鬼的作为。

凡从神生的就不犯罪,因神的道(原文作种)存在他心里。他也不能犯罪,因为他是由神生的。

从此就显明出谁是神的儿女,谁是魔鬼的儿女。凡不行义的,就不属神。不爱弟兄的也是如此。

我们应当彼此相爱。这就是你们从起初所听见的命令。

不可像该隐。他是属那恶者,杀了他的兄弟。为什么杀了他呢。因为自己的行为是恶的,兄弟的行为是善的。

弟兄们,世人若恨你们,不要以为希奇。

我们因为爱弟兄,就晓得是已经出死入生了。没有爱心的,仍住在死中。

凡恨他弟兄的,就是杀人的。你们晓得凡杀人的,没有永生存在他里面。

主为我们舍命,我们从此就知道何为爱。我们也应当为弟兄舍命。

凡有世上财物的,看见弟兄穷乏,却塞住怜恤的心,爱神的心怎能存在他里面呢。

小子们哪,我们相爱,不要只在言语和舌头上。总要在行为和诚实上。

从此就知道我们是属真理的,并且我们的心在神面前可以安稳。

我们的心若责备我们,神比我们的心大,一切事没有不知道的。

亲爱的弟兄阿,我们的心若不责备我们,就可以向神坦然无惧了。

并且我们一切所求的,就从他得着。因为我们遵守他的命令,行他所喜悦的事。

神的命令就是叫我们信他儿子耶稣基督的名,且照他所赐给我们的命令彼此相爱。

遵守神命令的,就住在神里面。神也住在他里面。我们所以知道神住在我们里面,是因他所赐给我们的圣灵。

\chapter{约翰壹书第4章}
亲爱的弟兄阿,一切的灵,你们不可都信。总要试验那些灵是出于神的不是。因为世上有许多假先知已经出来了。

凡灵认耶稣基督是成了肉身来的,就是出于神的。从此你们可以认出神的灵来。

凡灵不认耶稣,就不是出于神。这是敌基督者的灵。你们从前听见他要来。现在已经在世上了。

小子们哪,你们是属神的,并且胜了他们。因为那在你们里面的,比那在世界上的更大。

他们是属世界的。所以论世界的事,世人也听从他们。

我们是属神的。认识神的就听从我们。不属神的就不听从我们。从此我们可以认出真理的灵,和谬妄的灵来。

亲爱的弟兄阿,我们应当彼此相爱。因为爱是从神来的。凡有爱心的,都是由神而生,并且认识神。

没有爱心的,就不认识神。因为神就是爱。

神差他独生子到世间来,使我们藉着他得生,神爱我们的心,在此就显明了。

不是我们爱神,乃是神爱我们,差他的儿子,为我们的罪作了挽回祭,这就是爱了。

亲爱的弟兄阿,神既是这样爱我们,我们也当彼此相爱。

从来没有人见过神。我们若彼此相爱,神就住在我们里面,爱他的心在我们里面得以完全了。

神将他的灵赐给我们,从此就知道我们是住在他里面,他也住在我们里面。

父差子作世人的救主,这是我们所看见且作见证的。

凡认耶稣为神儿子的,神就住在他里面。

神爱我们的心,我们也知道也信。神就是爱。住在爱里面的,就住在神里面,神也住在他里面。

这样爱在我们里面得以完全,我们就可以在审判的日子,坦然无惧。因为他如何,我们在这世上也如何。

爱里没有惧怕。爱既完全,就把惧怕除去。因为惧怕理含着刑罚。惧怕的人在爱里未得完全。

我们爱,因为神先爱我们。

人若说,我爱神,却恨他的弟兄,就是说谎话的。不爱他所看见的弟兄,就不能爱没有看见的神。(有古卷作怎能爱没有看见的神呢)。

爱神的,也当爱弟兄,这是我们从神所受的命令。

\chapter{约翰贰书第1章}
作长老的写信给蒙拣选的太太,(太太作教会下同),和他的儿女,就是我诚心所爱的。不但我爱,也是一切知道真理之人所爱的。

爱你们是为真理的缘故,这真理存在我们里面,也必永远与我们同在。

恩惠,怜悯,平安,从父神和他儿子耶稣基督,在真理和爱心上,必常与我们同在。

我见你的儿女,有照我们从父所受之命令遵行真理的,就甚欢喜。

太太阿,我现在劝你,我们大家要彼此相爱。这并不是我写一条新命令给你,乃是我们从起初所受的命令。

我们若照他的命令行,这就是爱。你们从起初所听见当行的,就是这命令。

因为世上有许多迷惑人的出来,他们不认耶稣基督是成了肉身来的。这就是那迷惑人的,敌基督的。

你们要小心,不要失去你们(有古卷作我们)所作的工,乃要得着满足的赏赐。

凡越过基督教训,不常守着的,就没有神。常守这教训的,就有父又有子。

若有人到你们那里,不是传这教训,不要接他到家里,也不要问他的安。

因为问他安的,就在他的恶行上有分。

我还有许多事要写给你们,却不愿意用纸墨写出来。但盼望到你们那里,与你们当面谈论,使你们的喜乐满足。

你那蒙拣选之姐妹的儿女都问你安。

\chapter{约翰叁书第1章}
作长老的写信给亲爱的该犹,就是我诚心所爱的。

亲爱的兄弟阿,我愿你凡事兴盛,身体健壮,如你的灵魂兴盛一样。

有弟兄来证明你心里存的真理,正如你按真理而行,我就甚喜乐。

我听见我的儿女们按真理而行,我的喜乐就没有比这个大的。

亲爱的弟兄阿,凡你向作客旅之弟兄所行的,都是忠心的;

他们在教会面前证明了你的爱。你若配得过神,帮住他们往前行,这就好了。

因他们是为主的名(原文作那名)出外,对于外邦人一无所取。

所以我们应该接待这样的人,叫我们与他们一同为真理作工。

我曾略略的写信给教会。但那在教会中好为首的丢特腓不接待我们。

所以我若去,必要题说他所行的事。就是他用恶言妄论我们。还不以此为足,他自己不接待弟兄,有人愿意接待,他也禁止,并且将接待弟兄的人赶出教会。

亲爱的兄弟阿,不要效法恶,只要效法善。行善的属乎神。行恶的未曾见过神。

低米丢行善有众人给他作见证。又有真理给他作见证。就是我们也给他作见证。你也知道我们的见证是真的。

我原有许多事要写给你,却不愿意用笔墨写给你。

但盼望快快的见你,我们就当面谈论。

愿你平安。众位朋友都问你安。请你替我按着姓名问众位朋友安。

\chapter{犹大书第1章}
耶稣基督的仆人,雅各的弟兄犹大,写信给那被召,在父神里蒙爱,为耶稣基督保守的人。

愿怜恤,平安,慈爱,多多的加给你们。

亲爱的弟兄阿,我想尽心写信给你们,论我们同得救恩的时候,就不得不写信劝你们,要为从前一次交付圣徒的真道,竭力的争辩。

因为有些人偷着进来,就是自古被定受刑罚的,是不虔诚的,将我们神的恩变作放纵情欲的机会,并且不认独一的主宰我们(我们或作和我们)主耶稣基督。

从前主救了他的百姓出埃及地,后来就把那些不信的灭绝了。这一切的事,你们虽然都知道,我却仍要题醒你们。

又有不守本位,离开自己住处的天使,主用锁链把他们永远拘留在黑暗里,等候大日的审判。

又如所多玛,蛾摩拉,和周围城邑的人,也照他们一味的行淫,随从逆性的情欲,就受永火的刑罚,作为监戒。

这些作梦的人,也像他们污秽身体,轻慢主治的,毁谤在尊位的。

天使长米迦勒,为摩西的尸首与魔鬼争辩的时候,尚且不敢用毁谤的话罪责他,只说,主责备你吧。

但这些人毁谤他们所不知道的。他们本性所知道的事与那没有灵性的畜类一样,在这事上竟败坏了自己。

他们有祸了。因为走了该隐的道路,又为利往巴兰的错谬里直奔,并在可拉的背判中灭亡了。

这样的人,在你们的爱席上,与你们同吃的时候,正是礁石。(或作玷污)他们作牧人,只知喂养自己,无所惧怕。是没有雨的云彩,被风飘荡,是秋天没有果子的树,死而又死,连根被拔出来。

是海里的狂浪,涌出自己可耻的沫子来。是流荡的星,有墨黑的幽暗为他们永远存留。

亚当的七世孙以诺,曾预言这些人说,看哪,主带着他的千万圣者降临,

要在众人身上行审判,证实那一切不敬虔的人,所妄行一切不敬虔的事,又证实不敬虔之罪人所说顶撞他的刚愎话。

这些人是私下议论,常发怨言的,随从自己的情欲而行,口中说夸大的话,为得便宜谄媚人。

亲爱的弟兄阿,你们要记念我们主耶稣基督之使徒从前所说的话。

他们曾对你们说过,末世必有好讥诮的人,随从自己不敬虔的私欲而行。

这就是那些引人结党,属乎血气,没有圣灵的人。

亲爱的弟兄阿,你们却要在至圣的真道上造就自己,在圣灵里祷告,

保守自己常在神的爱中,仰望我们主耶稣基督的怜悯,直到永生。

有些人存疑心,你们要怜悯他们。

有些人你们要从火中抢出来搭救他们。有些人你们要存惧怕的心怜悯他们。连那被情欲沾染的衣服也当厌恶。

那能保守你们不失脚,叫你们无瑕无疵,欢欢喜喜站在他荣耀之前的,我们的救主独一的神。

愿荣耀,威严,能力,权柄,因我们的主耶稣基督,归与他,从万古以前,并现在,直到永永远远。阿们。

\chapter{启示录第1章}
耶稣基督的启示,就是神赐给他,叫他将必要成的事指示他的仆人。他就差遣使者,晓谕他的仆人约翰。

约翰便将神的道,和耶稣基督的见证,凡自己所看见的,都证明出来。

念这书上预言的,和那些听见又遵守其中所记载的,都是有福的。因为日期近了。

约翰写信给亚细亚的七个教会。但愿从那昔在今在以后永在的神和他宝座前的七灵。

并那诚实作见证的,从死里首先复活,为世上君王元首的耶稣基督。有恩惠平安归与你们。他爱我们,用自己的血使我们脱离罪恶。(脱离有古卷作洗去)

又使我们成为国民,作他父神的祭司。但愿荣耀权能归给他,直到永永远远。阿们。

看哪,他驾云降临。众目要看见他,连刺他的人也要看见他。地上的万族都要因他哀哭。这话是真实的。阿们。

主神说,我是阿拉法,我是俄梅戛(阿拉法俄梅戛乃希腊字母首末二字),是昔在今在以后永在的全能者。

我约翰就是你们的弟兄,和你们在耶稣的患难,国度,忍耐里一同有分。为神的道,并为给耶稣作的见证,曾在那名叫拔摩的海岛上。

当主日我被圣灵感动,听见在我后面有大声如吹号说,

你所看见的,当写在书上,达与以弗所,士每拿,别迦摩,推雅推喇,撒狄,非拉铁非,老底嘉,那七个教会。

我转过身来,要看是谁发声与我说话。既转过来,就看见七个金灯台。

灯台中间,有一位好像人子,身穿长衣,直垂到脚,胸间束着金带。

他的头与发皆白,如白羊毛,如雪。眼目如同火焰。

脚好像在炉中锻炼光明的铜。声音如同众水的声音。

他右手拿着七星。从他口中出来一把两刃的利剑。面貌如同烈日放光。

我一看见,就仆倒在他脚前,像死了一样。他用右手按着我说,不要惧怕。我是首先的,我是末后的,

又是那存活的。我曾死过,现在又活了,直活到永永远远。并且拿着死亡和阴间的钥匙。

所以你要把所看见的,和现在的事,并将来必成的事,都写出来。

论到你所看见在我右手中的七星,和七个金灯台的奥秘。那七星就是七个教会的使者。七个灯台就是七个教会。

\chapter{启示录第2章}
你要写信给以弗所教会的使者,说,那右手拿着七星,在七个金灯台中间行走的,说,

我知道你的行为,劳碌,忍耐,也知道你不能容忍恶人,你也曾试验那自称为使徒却不是使徒的,看出他们是假的来。

你也能忍耐,曾为我的名劳苦,并不乏倦。

然而有一件事我要责备你,就是你把起初的爱心离弃了。

所以应当回想你是从那里坠落的,并要悔改,行起初所行的事。你若不悔改,我就临到你那里,把你的灯台从原处挪去。

然而你还有一个可取的事,就是你恨恶尼哥拉一党人的行为,这也是我所恨恶的。

圣灵向众教会所说的话,凡有耳的,就应当听。得胜的,我必将神乐园中生命树的果子赐给他吃。

你要写信给士每拿教会的使者说,那首先的,末后的,死过又活的说,

我知道你的患难,你的贫穷,(你却是富足的)也知道那自称是犹太人所说的悔谤话,其实他们不是犹太人,乃是撒但一会的人。

你将要受的苦你不用怕。魔鬼要你们中间几个人下在监里,叫你们被试炼。你们必受患难十日。你务要至死忠心,我就赐给你那生命的冠冕。

圣灵向众教会所说的话,凡有耳的,就应当听。得胜的,必不受第二次死的害。

你要写信给别迦摩教会的使者,说,那有两刃利剑的说,

我知道你的居所,就是有撒但座位之处。当我忠心的见证人安提帕在你们中间,撒但所住的地方被杀之时,你还坚守我的名,没有弃绝我的道。

然而有几件事我要责备你,因为在你那里,有人服从巴兰的教训。这巴兰曾教导巴勒将绊脚石放在以色列人面前,叫他们吃祭偶像之物,行奸淫的事。

你那里也有人照样服从了尼哥拉一党人的教训。

所以你当悔改。若不悔改,我就快临到你那里,用我口中的剑,攻击他们。

圣灵向众教会所说的话,凡有耳的就应当听。得胜的,我必将那隐藏的吗哪赐给他。并赐他一块白石,石上写着新名。除了那领受的以外。没有人能认识。

你要写信给推雅推喇教会的使者,说,那眼目如火焰,脚像光明铜的神之子,说,

我知道你的行为,爱心,信心,勤劳,忍耐。又知道你末后所行的善事,比起初所行的更多。

然而有一件事我要责备你,就是你容让那自称是先知的妇人耶洗别教导我的仆人,引诱他们行奸淫,吃祭偶像之物。

我曾给他悔改的机会,他却不肯悔改他的淫行。

看哪,我要叫他病卧在床,那些与他行淫的人,若不悔改所行的,我也要叫他们同受大患难。

我又要杀死他的党类(党类原文作儿女),叫众教会知道,我是那察看人肺腑心肠的。并要照你们的行为报应你们各人。

至于你们推雅推喇其馀的人,就是一切不从那教训,不晓得他们素常所说撒但深奥之理的人。我告诉你们,我不将别的担子放在你们身上。

但你们已经有的,总要持守,直等到我来。

那得胜又遵守我命令到底的,我要赐给他权柄制伏列国。

他必用铁杖辖管他们(辖管原文作牧),将他们如同窑户的瓦器打得粉碎。像我从我父领受的权柄一样。

我又要把晨星赐给他。

圣灵向众教会所说的话,凡有耳的就应当听。

\chapter{启示录第3章}
你要写信给撒狄教会的使者,说,那有神的七灵和七星的,说,我知道你的行为,按名你是活的,其实是死的。

你要儆醒,坚固那剩下将要衰微的(衰微原文作死)。因我见你的行为,在我神面前,没有一样是完全的。

所以要回想你是怎样领受,怎样听见的。又要遵守,并要悔改。若不儆醒,我必临到你那里如同贼一样。我几时临到,你也决不能知道。

然而在撒狄你还有几名是未曾污秽自己衣服的。他们要穿白衣与我同行。因为他们是配得过的。

凡得胜的,必这样穿白衣。我也必不从生命册上涂抹他的名。且要在我父面前,和我父众使者面前认他的名。

圣灵向众教会所说的话,凡有耳的,就应当听。

你要写信给非拉铁非教会的使者,说,那圣洁,真实,拿着大卫的钥匙,开了就没有人能关,关了就没有人能开的,说,

我知道你的行为,你略有一点力量,也曾遵守我的道,没有弃绝我的名。看哪,我在你面前给你一个敞开的门,是无人能关的。

那撒但一会的,自称是犹太人,其实不是犹太人,乃是说谎话的,我要使他们来在你脚前下拜,也使他们知道我是已经爱你了。

你既遵守我忍耐的道,我必在普天下人受试炼的时候,保守你免去你的试炼。

我必快来,你要持守你所有的,免得人夺去你的冠冕。

得胜的,我要叫他在我神殿中作柱子,他也必不再从那里出去。我又要将我神的名,和我神城的名,(这城就是从天上从我神那里降下来的新耶路撒冷)并我的新名,都写在他上面。

圣灵向众教会所说的话,凡有耳的,就应当听。

你要写信给老底嘉教会的使者,说,那为阿们的,为诚信真实见证的,在神创造万物之上为元首的,说,

我知道的行为,你也不冷也不热。我巴不得你或冷或热。

你既如温水,也不冷也不热,所以我必从我口中把你吐出去。

你说,我是富足,已经发了财,一样都不缺。却不知道你是那困苦,可怜,贫穷,瞎眼,赤身的。

我劝你向我买火炼的金子,叫你富足。又买白衣穿上,叫你赤身的羞耻不露出来。又买眼药擦你的眼睛,使你能看见。

凡我所疼爱的,我就责备管教他。所以你要发热心,也要悔改。

看哪,我站在门外叩门。若有听见我声音就开门的,我要进到他那里去,我与他,他与我一同坐席。

得胜的,我要赐他在我宝座上与我同坐,就如我得了胜,在我父的宝座上与他同坐一般。

圣灵向众教会所说的话,凡有耳的,就应当听。

\chapter{启示录第4章}
此后,我观看,见天上有门开了,我初次听见好像吹号的声音,对我说,你上到这里来,我要将以后必成的事指示你。

我立刻被圣灵感动,见有一个宝座安置在天上,又有一位坐在宝座上。

看那坐着的,好像碧玉和红宝石。又有虹围着宝座,好像绿宝石。

宝座的周围,又有二十四个座位,其上坐着二十四位长老,身穿白衣,头上戴着金冠冕。

有闪电,声音,雷轰,从宝座中发出。又有七盏火灯在宝座前点着,这七灯就是神的七灵。

宝座前好像一个玻璃海如同水晶。宝座中,和宝座周围有四个活物,前后遍体都满了眼睛。

第一个活物像狮子,第二个像牛犊,第三个脸面像人,第四个像飞鹰。

四活物各有六个翅膀,遍体内外都满了眼睛。他们昼夜不住的说,圣哉,圣哉,圣哉,主神。是昔在今在以后永在的全能者。

每逢四活物将荣耀,尊贵,感谢,归给那坐在宝座上,活到永永远远者的时候,

那二十四位长老,就俯伏在坐宝座的面前,敬拜那活到永永远远的,又把他们的冠冕放在宝座前,说,

我们的主,我们的神,你是配得荣耀尊贵权柄的。因为你创造了万物,并且万物是因你的旨意被创造而有的。

\chapter{启示录第5章}
我看见坐宝座的右手中有书卷,里外都写着字,用七印封严了。

我又看见一位大力的天使,大声宣传说,有谁配展开那书卷,揭开那七印呢。

在天上,地上,地底下,没有能展开能观看那书卷的。

因为没有配展开,配观看那书卷的,我就大哭。

长老中有一位对我说,不要哭。看哪,犹大支派中的狮子,大卫的根,他已得胜,能以展开那书卷,揭开那七印。

我又看见宝座与四活物并长老之中,有羔羊站立,像是被杀过的,有七角七眼,就是神的七灵,奉差遣往普天下去的。

这羔羊前来,从坐宝座的右手里拿了书卷。

他既拿了书卷,四活物和二十四位长老,就俯伏在羔羊面前,各拿着琴,和盛满了香的金炉。这香就是众圣徒的祈祷。

他们唱新歌,说,你配拿书卷,配揭开七印。因为你曾被杀,用自己的血从各族各方,各民各国中买了人来,叫他们归于神,

又叫他们成为国民,作祭司,归于神。在地上执掌王权。

我又看见,且听见,宝座与活物并长老的周围,有许多天使的声音。他们的数目有千千万万。

大声说,曾被杀的羔羊,是配得权柄,丰富,智慧,能力,尊贵,荣耀,颂赞的。

我又听见,在天上,地上,地底下,沧海里,和天地间一切所有被造之物,都说,但愿颂赞,尊贵,荣耀,权势,都归给坐宝座的和羔羊,直到永永远远。

四活物就说,阿们。众长老也俯伏敬拜。

\chapter{启示录第6章}
我看见羔羊揭开七印中第一印的时候,就听见四活物中的一个活物,声音如雷,说,你来。

我就观看,见有一匹白马,骑在马上的拿着弓。并有冠冕赐给他。他便出来,胜了又胜。

揭开第二印的时候,我听见第二个活物说,你来。

就另有一匹马出来,是红的。有权柄给了那骑马的,可以从地上夺去太平,使人彼此相杀。又有一把大刀赐给他。

揭开第三印的时候,我听见第三个活物说,你来。我就观看,见有一匹黑马。骑在马上的手里拿着天平。

我听见在四活物中,似乎有声音说,一钱银子买一升麦子,一钱银子买三升大麦。油和酒不可糟蹋。

揭开第四印的时候,我听见第四活物说,你来。

我就观看,见有一匹灰马。骑在马上的,名字叫死。阴府也随着他。有权柄赐给他们,可以用刀剑,饥荒,瘟疫(瘟疫或作死亡),野兽,杀害地上四分之一的人。

揭开第五印的时候,我看见在祭坛底下,有为神的道,并为作见证,被杀之人的灵魂。

大声喊着说,圣洁真实的主阿,你不审判住在地上的人给我们伸流血的冤,要到几时呢。

于是有白衣赐给他们各人。又有话对他们说,还要安息片时,等着一同作仆人的,和他们的弟兄,也像他们被杀,满足了数目。

揭开第六印的时后,我又看见地大震动。日头变黑像毛布,满月变红像血。

天上的星辰坠落于地,如同无花果树被大风摇动,落下未熟的果子一样。

天就挪移,好像书卷被卷起来。山岭海岛都被挪移离开本位。

地上的君王,臣宰,将军,富户,壮士,和一切为奴的,自主的,都藏在山洞,和岩石穴里。

向山和岩石说,倒在我们身上吧,把我们藏起来,躲避坐宝座者的面目,和羔羊的忿怒。

因为他们忿怒的大日到了,谁能站得住呢。

\chapter{启示录第7章}
此后我看见四位天使站在地的四角,执掌地上四方的风,叫风不吹在地上,海上,和树上。

我又看见另有一位天使,从日出之地上来,拿着永生神的印。他就向那得着权柄能伤害地和海的四位天使,大声喊着说,

地与海并树木,你们不可伤害,等我们印了我们神众仆人的额。

我听见以色列人,各支派中受印的数目,有十四万四千。

犹大支派中受印的有一万二千。流便支派中有一万二千。迦得支派中有一万二千。

亚设支派中有一万二千。拿弗他利支派中有一万二千。玛拿西支派中有一万二千

西缅支派中有一万二千。利未支派中有一万二千。以萨迦支派中有一万二千。

西布伦支派中有一万二千。约瑟支派中有一万二千。便雅悯支派中有一万二千。

此后,我观看,见有许多的人,没有人能数过来,是从各国各族各民各方来的,站在宝座和羔羊面前,身穿白衣,手拿棕树枝。

大声喊着说,愿救恩归与坐在宝座上我们的神,也归与羔羊。

众天使都站在宝座和众长老并四活物的周围,在宝座前,面伏于地,敬拜神,

说,阿们。颂赞,荣耀,智慧,感谢,尊贵,权柄,大力,都归与我们的神,直到永永远远。阿们。

长老中有一位问我说,这些穿白衣的是谁,是从那里来的。

我对他说,我主,你知道。他向我说,这些人是从大患难中出来的,曾用羔羊的血,把衣裳洗白净了。

所以他们在神宝座前,昼夜在他殿中事奉他。坐宝座的要用帐幕覆庇他们。

他们不再饥,不再渴。日头和炎热,也必不伤害他们。

因为宝座中的羔羊必牧养他们,领他们到生命水的泉源。神也必擦去他们一切的眼泪。

\chapter{启示录第8章}
羔羊揭开第七印的时候,天上寂静约有二刻。

我看见那站在神面前的七位天使,有七枝号赐给他们。

另有一位天使拿着金香炉,来站在祭坛旁边。有许多香赐给他,要和众圣徒的祈祷一同献在宝座前的金坛上。

那香的烟,和众圣徒的祈祷,从天使的手中一同升到神面前。

天使拿着香炉,盛满了坛上的火,倒在地上。随有雷轰,大声,闪电,地震。

拿着七枝号的七位天使,就豫备要吹。

第一位天使吹号,就有雹子与火搀着血丢在地上。地的三分之一和树的三分之一被烧了,一切的青草也被烧了。

第二位天使吹号,就有彷佛火烧着的大山扔在海中。海的三分之一变成血。

海中的活物死了三分之一。船只也坏了三分之一。

第三位天使吹号,就有烧着的大星,好像火把从天上落下来,落在江河的三分之一,和众水的泉源上。

这星名叫茵??。众水的三分之一便为茵??。因水变苦,就死了许多人。

第四位天使吹号,日头的三分之一,月亮的三分之一,星辰的三分之一,都被击打。以致日月星的三分之一黑暗了,白昼的三分之一没有光,黑夜也是这样。

我又看见一个鹰飞在空中,并听见他大声说,三位天使要吹那其馀的号,你们住在地上的民,祸哉,祸哉,祸哉。

\chapter{启示录第9章}
第五位天使吹号,我就看见一个星从天落到地上。有无底坑的钥匙赐给他。

他开了无底坑,便有烟从坑里往上冒,好像大火炉的烟。日头和天空,都因这烟昏暗了。

有蝗虫从烟中出来飞到地上。有能力赐给他们,好像地上蝎子的能力一样。

并且吩咐他们说,不可伤害地上的草,和各样青物,并一切树木,惟独要伤害额上没有神印记的人。

但不许蝗虫害死他们,只叫他们受痛苦五个月。这痛苦就像蝎子螫人的痛苦一样。

在那些日子,人要求死,决不得死。愿意死,死却远避他们。

蝗虫的形状,好像豫备出战的马一样,头上戴的好像金冠冕,脸面好像男人的脸面。

头发像女人的头发,牙齿像狮子的牙齿。

胸前有甲,好像铁甲。他们翅膀的声音,好像许多车马奔跑上阵的声音。

有尾巴像蝎子。尾巴上的毒钩能伤人五个月。

有无底坑的使者作他们的王。按着希伯来话,名叫亚巴顿,希腊话,名叫亚玻伦。

第一样灾祸过去,还有两样灾祸要来。

第六位天使吹号,我就听见有声音,从神面前金坛的四角出来,

吩咐那吹号的第六位天使,说,把那捆绑在伯拉大河的四个使者释放了。

那四个使者就被释放。他们原是豫备好了,到某年某月某日某时,要杀人的三分之一。

马军有二万万。他们的数目我听见了。

我在异象中看见那些马和骑马的,骑马的胸前有甲如火,与紫玛瑙,并硫磺。马的头好像狮子头,有火,有烟,有硫磺,从马的口中出来。

口中所出来的火,与烟,并硫磺,这三样灾杀了人的三分之一。

这马的能力,是在口里,和尾巴上。因这尾巴像蛇,并且有头用以害人。

其馀未曾被这灾所杀的人,仍旧不悔改自己手所作的,还是去拜鬼魔,和那些不能看,不能听,不能走,金,银,铜,木,石,的偶像。

又不悔改他们那些凶杀,邪术,奸淫,偷窃的事。

\chapter{启示录第10章}
我又看见另有一位大力的天使,从天降下,披着云彩,头上有虹。脸面像日头,两脚像火柱。

他手里拿着小书卷是展开的。他用右脚踏海,左脚踏地。

大声呼喊,好像狮子吼叫,呼喊完了,就有七雷发声。

七雷发声之后,我正要写出来,就听见从天上有声音说,七雷所说的你要封上,不可写出来。

我所看见的那踏海踏地的天使,向天举起右手来,

指着那创造天和天上之物,地和地上之物,海和海中之物,直活到永永远远的,起誓说,不再有时日了。(或作不再耽延了)

但在第七位天使吹号发声的时候,神的奥秘,就成全了,正如神所传给他仆人众先知的佳音。

我先前从天上所听见的那声音,又吩咐我说,你去把那踏海踏地之天使手中展开的小书卷取过来。

我就走到天使那里,对他说,请你把小书卷给我。他对我说,你拿着吃尽了,便叫你肚子发苦,然而在你口中要甜如蜜。

我从天使手中把小书卷接过来,吃尽了。在我口中果然甜如蜜。吃了以后,肚子觉得发苦了。

天使(原文作他们)对我说,你必指着多民多国多方多王说预言。

\chapter{启示录第11章}
有一根苇子赐给我,当作量度的杖。且有话说,起来,将神的殿,和祭坛,并在殿中礼拜的人,都量一量。

只是殿外的院子,要留下不用量。因为这是给了外邦人的。他们要践踏圣城四十二个月。

我要使我那两个见证人,穿着毛衣,传道一千二百六十天。

他们就是那两棵橄榄树,两个灯台,立在世界之主面前的。

若有人想要害他们就有火从他们口中出来,烧灭仇敌。凡想要害他们的,都必这样被杀。

这二人有权柄,在他们传道的日子叫天闭塞不下雨,叫水变为血。并且能随时随意用各样的灾殃攻击世界。

他们作完见证的时候,那从无底坑里上来的兽,必与他们交战,并且得胜,把他们杀了。

他们的尸首就倒在大城里的街上。这城按着灵意叫所多玛,又叫埃及,就是他们的主钉十字架之处。

从各民各族各方各国中,有人观看他们的尸首三天半,又不许把尸首放在坟墓里。

住在地上的人,就为他们欢喜快乐,互相馈送礼物。因这两位先知曾叫住在地上的人受痛苦。

过了三天半,有生气从神那里进入他们里面,他们就站起来。看见他们的人甚是害怕。

两位先知听见有大声音从天上来,对他们说,上到这里来。他们就驾着云上了天。他们的仇敌也看见了。

正在那时候,地大震动,城就倒塌了十分之一。因地震而死的有七千人。其馀的都恐惧,归荣耀给天上的神。

第二样灾祸过去。第三灾祸快到了。

第七位天使吹号,天上就有大声音说,世上的国,成了我主和主基督的国。他要作王,直到永永远远。

在神面前,坐在自己位上的二十四位长老,就面伏于地敬拜神,

说,昔在今在的主神,全能者阿,我们感谢你,因你执掌大权作王了。

外邦发怒,你的忿怒也临到了。审判死人的时候也到了。你的仆人众先知,和众圣徒,凡敬畏你名的人连大带小得赏赐的时候也到了。你败坏那些败坏世界之人的时候也就到了。

当时神天上的殿开了。在他殿中现出他的约柜后有闪电,声音,雷轰,地震,大雹。

\chapter{启示录第12章}
天上现出大异象来。有一个妇人,身披日头,脚踏月亮,头戴十二星的冠冕。

他怀了孕,在生产的艰难中疼痛呼叫。

天上又现出异象来。有一条大红龙,七头十角,七头上戴着七个冠冕。

他的尾巴拖拉着天上星辰的三分之一,摔在地上。龙就站在那将要生产的妇人面前,等他生产之后,要吞吃他的孩子。

妇人生了一个男孩,是将来要用铁杖辖管万国的。(辖管原文作牧)他的孩子被提到神宝座那里去了。

妇人就逃到旷野,在那里有神给他豫备的地方,使他被养活一千二百六十天。

在天上就有了争战。米迦勒同他的使者与龙争战。龙也同他的使者去争战。

并没有得胜,天上再没有他们的地方。

大龙就是那古蛇,名叫魔鬼,又叫撒但,是迷惑普天下的。他被摔在地上,他的使者也一同被摔下去。

我听见在天上有大声音说,我神的救恩,能力,国度,并他基督的权柄,现在都来到了。因为那在我们神面前昼夜控告我们弟兄的,已经被摔下去了。

弟兄胜过他,是因羔羊的血,和自己所见证的道。他们虽至于死,也不爱惜性命。

所以诸天和住在其中的,你们都快乐吧。只是地与海有祸了,因为魔鬼知道自己的时候不多,就气忿忿的下到你们那里去了。

龙见自己被摔在地上,就逼迫那生男孩子的妇人。

于是有大鹰的两个翅膀赐给妇人,叫他能飞到旷野,到自己的地方,躲避那蛇。他在那里被养活一载二载半载。

蛇就在妇人身后,从口中吐出水来像河一样,要将妇人冲去。

地却帮助妇人,开口吞了从龙口吐出来的水。(原文作河)

龙向妇人发怒,去与他其馀的儿女争战,这儿女就是那守神诫命,为耶稣作见证的。那时龙就站在海边的沙上。

\chapter{启示录第13章}
我又看见一个兽从海中上来,有十角七头,在十角上戴着十个冠冕,七头上有亵渎的名号。

我所看见的兽,形状像豹,脚像熊的脚,口像狮子的口。那龙将自己的能力,座位,和大权柄,都给了他。

我看见兽的七头中,有一个似乎受了死伤。那死伤却医好了。全地的人,都希奇跟从那兽。

又拜那龙,因为他将自己的权柄给了兽。也拜兽说,谁能比这兽,谁能与他交战呢。

又赐给他说夸大亵渎话的口。又有权柄赐给他,可以任意而行四十二个月。

兽就开口向神说亵渎的话,亵渎神的名,并他的帐幕,以及那些住在天上的。

又任凭他与圣徒争战,并且得胜。也把权柄赐给他,制伏各族各民各方各国。

凡住在地上,名字从创世以来,没有记在被杀之羔羊生命册上的人,都要拜他。

凡有耳的,就应当听。

掳掠人的必被掳掠。用刀杀人的,必被刀杀。圣徒的忍耐和信心,就是在此。

我又看见另有一个兽从地中上来。有两角如同羊羔,说话好像龙。

他在头一个兽面前,施行头一个兽所有的权柄。并且叫地和住在地上的人,拜那死伤医好的头一个兽。

又行大奇事,甚至在人面前,叫火从天降在地上。

他因赐给他们权柄在兽面前能行奇事,就迷惑住在地上的人,说,要给那受刀伤还活着的兽作个像。

又有权柄赐给他叫兽像有生气。并且能说话,又叫所有不拜兽像的人都被杀害。

他又叫众人,无论大小贫福,自主的,为奴的,都在右手上,或是在额上,受一个印记。

除了那受印记,有了兽名,或有兽名数目的,都不得作买卖。

在这里有智慧。凡有聪明的,可以算计兽的数目,因为这是人的数目。他的数目是六百六十六。

\chapter{启示录第14章}
我又观看,见羔羊站在锡安山,同他又有十四万四千人,都有他的名,和他父的名,写在额上。

我听见从天上有声音,像众水的声音,和大雷的声音。并且我所听见的好像弹琴的所弹的琴声。

他们在宝座前,并在四活物和众长老前唱歌,彷佛是新歌。除了从地上买来的那十四万四千人以外,没有人能学这歌。

这些人未曾沾染妇女,他们原是童身。羔羊无论往那里去,他们都跟随他。他们是从人间买来的,作初熟的果子归与神和羔羊。

在他们口中察不出谎言来。他们是没有瑕疵的。

我又看见另一位天使飞在空中,有永远的福音要传给住在地上的人,就是各国各族各方各民。

他大声说,应当敬畏神,将荣耀归给他。因他施行审判的时候已经到了。应当敬拜那创造天地海和众水泉源的。

又有第二位天使,接着说,叫万民喝邪淫大怒之酒的巴比伦大城倾倒了,倾倒了。

又有第三位天使,接着他们,大声说,若有人拜兽和兽像,在额上,或在手上,受了印记,

这人也必喝神大怒的酒,此酒斟在神忿怒的杯中纯一不杂。他要在圣天使和羔羊面前,在火与硫磺之中受痛苦。

他受痛苦的烟往上冒,直到永永远远。那些拜兽和兽像受他名之印记的,昼夜不得安宁。

圣徒的忍耐就在此。他们是守神诫命,和耶稣真道的。

我听见从天上有声音说,你要写下,从今以后,在主里面而死的人有福了。圣灵说,是的,他们息了自己的劳苦,作工的果效也随着他们。

我又观看,见有一片白云,云上坐着一位好像人子,头上戴着金冠冕,手里拿着快镰刀。

又有一位天使从殿中出来,向那坐在云上的大声喊着说,伸出你的镰刀来收割。因为收割的时候已经到了,地上的庄稼已经熟透了。

那坐在云上的,就把镰刀扔在地上。地上的庄稼就被收割了。

又有一位天使从天上的殿中出来,他也拿着快镰刀。

又有一位天使从祭坛中出来,是有权柄管火的,向拿着快镰刀的大声喊着说,伸出快镰刀来收取地上葡萄树的果子。因为葡萄熟透了。

那天使就把镰刀扔在地上,收取了地上的葡萄,丢在神忿怒的大酒榨中。

那酒榨踹在城外,就有血从酒榨里流出来,高到马的嚼环,远有六百里。

\chapter{启示录第15章}
我又看见在天上有异象,大而且奇,就是七位天使掌管末了的七灾。因为神的大怒在这七灾中发尽了。

我看见彷佛有玻璃海,其中有火搀杂。又看见那些胜了兽和兽的像,并他名字数目的人,都站在玻璃海上,拿着神的琴。

唱神仆人摩西的歌,和羔羊的歌,说,主神,全能者阿,你的作为大哉,奇哉。万世之王阿,(世或作国)你的道途义哉,诚哉。

主阿,谁敢不敬畏你,不将荣耀归与你的名呢。因为独有你是圣的。万民都要来在你面前敬拜。因你公义的作为已经显出来了。

此后,我看见在天上那存法柜的殿开了。

那掌管七灾的七位天使,从殿中出来,穿着洁白光明的细麻衣,(细麻衣有古卷作宝石)胸间束着金带。

四活物中有一个把盛满了活到永永远远之神大怒的七个金碗给了那七位天使。

因神的荣耀,和能力,殿中充满了烟。于是没有人能以进殿,直等到那七位天使所降的七灾完毕了。

\chapter{启示录第16章}
我听见有大声音从殿中出来,向那七位天使说,你们去,把盛神大怒的七碗倒在地上。

第一位天使便去,把碗倒在地上,就有恶而且毒的疮,生在那些有兽印记,拜兽像的人身上。

第二位天使把碗倒在海里,海就变成血,好像死人的血,海中的活物都死了。

第三位天使把碗倒在江河与众水的泉源里,水就变成血了。

我听见掌管众水的天使说,昔在今在的圣者阿,你这样判断是公义的。

他们曾流圣徒与先知的血,现在你给他们血喝。这是他们所该受的。

我又听见祭坛中有声音说,是的,主神,全能者阿,你的判断义哉,诚哉。

第四位天使把碗倒在日头上,叫日头能用火烤人。

人被大热所烤,就亵渎那有权掌管这些灾的神之名,并不悔改将荣耀归给神。

第五位天使把碗倒在兽的座位上,兽的国就黑暗了。人因疼痛就咬自己的舌头。

又因所受的疼痛,和生的疮,就亵渎天上的神。并不悔改所行的。

第六位天使把碗倒在伯拉大河上,河水就乾了,要给那从日出之地所来的众王豫备道路。

我又看见三个污秽的灵,好像青蛙,从龙口兽口并假先知的口中出来。

他们本是鬼魔的灵,施行奇事,出去到普天下众王那里,叫他们在神全能者的大日聚集争战。

(看哪,我来像贼一样。那儆醒,看守衣服,免得赤身而行,叫人见他羞耻的,有福了

那三个鬼魔便叫众王聚集在一处,希伯来话叫哈米吉多顿。

第七位天使把碗倒在空中,就有大声音从殿中的宝座上出来,说,成了。

又有闪电,声音,雷轰,大地震,自从地上有人以来,没有这样大这样利害的地震。

那大城裂为三段,列国的城也都倒塌了。神也想起巴比伦大城来,要把那盛自己烈怒的酒杯递给他。

各海岛都逃避了,众山也不见了。

又有大雹子从天落在人身上,每一个约重一他连得。(一他连得约有九十斤)为这雹子的灾极大,人就亵渎神。

\chapter{启示录第17章}
拿着七碗的七位天使中,有一位前来对我说,你到这里来,我将坐在众水上的大淫妇所要受的刑罚指给你看。

地上的君王与他行淫。住在地上的人喝醉了他淫乱的酒。

我被圣灵感动,天使带我到旷野去。我就看见一个女人骑在朱红色的兽上。那兽有七头十角,遍体有亵渎的名号。

那女人穿着紫色和朱红色的衣服,用金子宝石珍珠为妆饰。手拿金杯,杯中盛满了可憎之物,就是他淫乱的污秽。

在他额上有名写着说,奥秘哉,大巴比伦,作世上的淫妇和一切可憎之物的母。

我又看见那女人喝醉了圣徒的血,和为耶稣作见证之人的血。我看见他,就大大的希奇。

天使对我说,你为什么希奇呢。我要将这女人和驮着他的那七头十角兽的奥秘告诉你。

你所看见的兽,先前有,如今没有。将要从无底坑里上来,又要归于沉沦。凡住在地上名字从创世以来没有记在生命册上的,见先前有,如今没有,以后再有的兽,就必希奇。

智慧的心在此可以思想。那七头就是女人所坐的七座山。

又是七位王。五位已经倾倒了,一位还在,一位还没有来到。他来的时候,必须暂时存留。

那先前有,如今没有的兽,就是第八位。他也和那七位同列,并且归于沉沦。

你所看见的那十角,就是十王。他们还没有得国。但他们一时之间,要和兽同得权柄与王一样。

他们同心合意,将自己的能力权柄给那兽。

他们与羔羊争战,羔羊必胜过他们,因为羔羊是万主之主,万王之王。同着羔羊的,就是蒙召被选有忠心的,也必得胜。

天使又对我说,你所看见那淫妇坐的众水,就是多民多人多国多方。

你所看见的那十角,与兽,必恨这淫妇,使他冷落赤身。又要吃他的肉,用火将他烧尽。

因为神使诸王同心合意,遵行他的旨意,把自己的国给那兽,直等到神的话都应验了。

你所看见的那女人,就是管辖地上众王的大城。

\chapter{启示录第18章}
此后,我看见另有一位有大权柄的天使从天降下。地就因他的荣耀发光。

他大声喊着说,巴比伦大城倾倒了,成了鬼魔的住处,和各样污秽之灵的巢穴,(或作牢狱下同)并各样污秽可憎之雀鸟的巢穴。

因为列国都被他邪淫大怒的酒倾倒了。地上的君王与他行淫,地上的客商,因他奢华太过就发了财。

我又听见从天上有声音说,我的民哪,你们要从那城出来,免得与他一同有罪,受他所受的灾殃。

因他的罪恶滔天他的不义神已经想起来了。

他怎样待人,也要怎样待他,按他所行的加倍的报应他。用他调酒的杯,加倍的调给他喝。

他怎样荣耀自己,怎样奢华,也当叫他照样痛苦悲哀。因他心里说,我坐了皇后的位,并不是寡妇,决不至于悲哀。

所以在一天之内,他的灾殃要一齐来到,就是死亡,悲哀,饥荒,他又要被火烧尽了。因为审判他的主神大有能力。

地上的君王,素来与他行淫一同奢华的,看见烧他的烟,就必为他哭泣哀号。

因怕他的痛苦,就远远的站着说,哀哉,哀哉,巴比伦大城,坚固的城阿,一时知间你的刑罚就来到了。

地上的客商也都为他哭泣悲哀,因为没有人再买他们的货物了。

这货物就是金,银,宝石,珍珠,细麻布,紫色料,绸子,朱红色料,各样香木,各样象牙的器皿,各样极宝贵的木头和铜,铁,汉白玉的器皿,

并肉桂,豆蔻,香料,香膏,乳香,酒,油,细面,麦子,牛,羊,车,马,和奴仆,人口。

巴比伦哪,你所贪爱的果子离开了你。你一切的珍馐美味,和华美的物件,也从你中间毁灭,决不能再见了。

贩卖这些货物,藉着他发了财的客商,因怕他的痛苦,就远远的站着哭泣悲哀,说,

哀哉,哀哉,这大城阿,素常穿着细麻,紫色,朱红色的衣服,又用金子,宝石,和珍珠为妆饰。

一时之间,这吗大的富厚就归于无有了。凡船主,和坐船往各处去的,并众水手,连所有靠海为业的,都远远的站着,

看见烧他的烟,就喊着说,有何城能比这大城呢。

他们又把尘土撒在头上,哭泣悲哀,喊着说,哀哉,哀哉,这大城阿。凡有船在海中的,都因他的珍宝成了富足。他在一时之间就便成了荒场。

天哪,众圣徒众使徒众先知阿,你们都要因他欢喜。因为神已经在他身上伸了你们的冤。

有一位大力的天使举起一块石头,好像大磨石,扔在海里,说,巴比伦大城,也必这样猛力的被扔下去,不能再见了。

弹琴,作乐,吹笛,吹号的声音,在你们中间决不能再听见。各行手艺人在你中间决不能再遇见。推磨的声音在你们中间决不能再听见。

灯光在你们中间决不能再照耀。新郎和新妇的声音,在你们中间决不能再听见。你的客商原来是地上的尊贵人。万国也被你的邪术迷惑了。

先知和圣徒,并地上一切被杀之人的血,都在这城里看见了。

\chapter{启示录第19章}
此后,我听见好像群众在天上大声说,哈利路亚,(就是赞美耶和华的意思)救恩,荣耀,权能,都属乎我们的神。

他的判断是真实公义的。因他判断了那用淫行败坏世界的大淫妇,并且向淫妇讨流仆人血的罪,给他们伸冤。

又说,哈利路亚。烧淫妇的烟往上冒,直到永永远远。

那二十四位长老与四活物,就俯伏敬拜坐宝座的神,说,阿们,哈利路亚。

有声音从宝座出来说,神的众仆人哪,凡敬畏他的,无论大小,都要赞美我们的神。

我听见好像群众的声音,众水的声音,大雷的声音,说,哈利路亚。因为主我们的神,全能者,作王了。

我们要欢喜快乐,将荣耀归给他。因为羔羊婚娶的时候到了,新妇也自己豫备好了。

就蒙恩得穿光明洁白的细麻衣,这细麻衣就是圣徒所行的义。

天使吩咐我说,你要写上,凡被请赴羔羊之婚筵的有福了。又对我说,这是神真实的话。

我就俯伏在他脚前要拜他。他说,千万不可。我和你并你那些为耶稣作见证的弟兄同是作仆人的。你要敬拜神。因为预言中的灵意,乃是为耶稣作见证。

我观看,见天开了。有一匹白马。骑在马上的,称为诚信真实。他审判争战都按着公义。

他的眼睛如火焰,他头上戴着许多冠冕。又有写着的名字,除了他自己没有人知道。

他穿着溅了血的衣服。他的名称为神之道。

在天上的众军,骑着白马,穿着细麻衣,又白又洁,跟随他。

有利剑从他口中出来,可以击杀列国。他必用铁杖辖管他们。(辖管原文作牧)并要踹全能神烈怒的酒榨。

在他衣服和大腿上,有名写着说,万王之王,万主之主。

我又看见一位天使站在日头中,向天空所飞的鸟,大声喊着说,你们聚集来赴神的大筵席。

可以吃君王与将军的肉,壮士与马和骑马者的肉,并一切自主的为奴的,以及大小人民的肉。

我看见那兽,和地上的君王,并他们的众军,都聚集,要与骑白马的并他的军兵争战。

那兽被擒拿,那在兽面前曾行奇事,迷惑受兽印记,和拜兽像之人的假先知,也与兽同被擒拿。他们两个就活活的被扔在烧着硫磺的火湖里。

其馀的被骑白马者口中出来的剑杀了。飞鸟都吃饱了他们的肉。

\chapter{启示录第20章}
我又看见一位天使从天降下,手里拿着无底坑的钥匙,和一条大链子。

他捉住那龙,就是古蛇,又叫魔鬼,也叫撒但,把他捆绑一千年,

扔在无底坑里,将无底坑关闭,用印封上,使他不得再迷惑列国,等到那一千年完了。以后必须暂时释放他。

我又看见几个宝座,也有坐在上面的,并有审判的权柄赐给他们。我又看见那些因为给耶稣作见证,并为神之道被斩者的灵魂,和那没有拜过兽像,也没有在额上和手上受过他印记之人的灵魂,他们都复活了,与基督一同作王一千年。

这是头一次的复活。其馀的死人还没有复活,直等那一千年完了。

在头一次复活有分的,有福了,圣洁了。第二次的死在他们身上没有权柄。他们必作神和基督的祭司,并要与基督一同作王一千年。

那一千年完了,撒但必从监牢里被释放,

出来要迷惑地上四方的列国,(方原文作角)就是歌革和玛各,叫他们聚集争战。他们的人数多如海沙。

他们上来遍满了全地,围住圣徒的营,与蒙爱的城。就有火从天上降下,烧灭了他们。

那迷惑他们的魔鬼,被扔在硫磺的火湖里,就是兽和假先知所在的地方。他们必昼夜受痛苦,直到永永远远。

我又看见一个白色的大宝座,与坐在上面的。从他面前天地都逃避,再也无可见之处了。

我又看见死了的人,无论大小,都站在宝座前。案卷展开了。并且另有一卷展开,就是生命册。死了的人都凭着这些案卷所记载的,照他们所行的受审判。

于是海交出其中的死人。死亡和阴间也交出其中的死人。他们都照各人所行的受审判。

死亡和阴间也被扔在火湖里。这火湖就是第二次的死。

若有人名字没有记在生命册上,他就被扔在火湖里。

\chapter{启示录第21章}
我又看见一个新天新地。因为先前的天地已经过去了。海也不再有了。

我又看见圣城新耶路撒冷由神那里从天而降,豫备好了,就如新妇壮饰整齐,等候丈夫。

我听见有大声音从宝座出来说,看哪,神的帐幕在人间。他要与人同住,他们要作他的子民,神要亲自与他们同在,作他们的神。

神要擦去他们一切的眼泪。不再有死亡,也不再有悲哀,哭号,疼痛,因为以前的事都过去了。

坐宝座的说,看哪,我将一切都更新了。又说,你要写上。因这些话是可信的,是真实的。

他又对我说,都成了。我是阿拉法,我是俄梅戛,我是初,我是终。我要将生命泉的水白白赐给那口渴的人喝。

得胜的,必承受这些为业。我要作他的神,他要作我的儿子。

惟有胆怯的,不信的,可憎的,杀人的,淫乱的,行邪术的,拜偶像的,和一切说谎话的,他们的分就在烧着硫磺的火湖里。这是第二次的死。

拿着七个金碗,盛满末后七灾的七位天使中,有一位来对我说,你到这里来,我要将新妇,就是羔羊的妻,指给你看。

我被圣灵感动,天使就带我到一座高大的山,将那由神那里从天而降的圣城耶路撒冷指示我。

城中有神的荣耀。城的光辉如同极贵的宝石,好像碧玉,明如水晶。

有高大的墙。有十二个门,门上有十二位天使。门上又写着以色列十二个支派的名字。

东边有三门。北边有三门。南边有三门。西边有三门。

城墙有十二根基,根基上有羔羊十二使徒的名字。

对我说话的拿着金苇子当尺,要量那城,和城门城墙。

城的四方的,长宽一样。天使用苇子量那城,共有四千里。长宽高都是一样。

又量了城墙,按着人的尺寸,就是天使的尺寸,共有一百四十四肘。

墙是碧玉造的。城是精金的,如同明净的玻璃。

城墙的根基是用各样宝石修饰的。第一根基是碧玉。第二是蓝宝石。第三是绿玛瑙。第四是绿宝石。

第五是红玛瑙。第六是红宝石。第七是黄璧玺。第八是水苍玉。第九是红璧玺。第十是翡翠。第十一是紫玛瑙。第十二是紫晶。

十二个门是十二颗珍珠。每门是一颗珍珠。城内的街道是精金,好像明透的玻璃。

我未见城内有殿,因主神全能者,和羔羊,为城的殿。

那城内又不用日月光照。因有神的荣耀光照。又有羔羊为城的灯。

列国要在城的光里行走。地上的君王必将自己的荣耀归与那城。

城门白昼总不关闭。在那里原没有黑夜。

人必将列国的荣耀尊贵归与那城。

凡不洁净的,并那行可憎与虚谎之事的,总不得进那城。只有名字写在羔羊生命册上的才得进去。

\chapter{启示录第22章}
天使又指示我在城内街道当中一道生命水的河,明亮如水晶,从神和羔羊的宝座流出来。

在河这边与那边有生命树,结十二样果子,(样或作回)每月都结果子。树上的叶子乃为医治万民。

以后再没有咒诅。在城里有神和羔羊的宝座。他的仆人都要事奉他。

也要见他的面。他的名字必写在他们的额上。

不再有黑夜。他们也不用灯光日光。因为主神要光照他们。他们要作王,直到永永远远。

天使又对我说,这些话是真实可信的。主就是众先知被感之灵的神,差遣他的使者,将那必要快成的事指示他仆人。

看哪,我必快来。凡遵守这书上预言的有福了。

这些事是我约翰所听见所看见的。我既听见看见了。就在指示我的天使脚前俯伏要拜他。

他对我说,千万不可。我与你,和你的弟兄众先知,并那些守这书上言语的人,同是作仆人的。你要敬拜神。

他又对我说,不可封了这书上的预言。因为日期近了。

不义的,叫他仍旧不义。污秽的,叫他仍旧污秽。为义的,叫他仍旧为义。圣洁的,叫他仍旧圣洁。

看哪,我必快来。赏罚在我,要照各人所行的报应他。

我是阿拉法,我是俄梅戛,我是首先的,我是末后的,我是初,我是终。

那些洗净自己衣服的有福了,可得权柄能到生命树那里,也能从门进城。

城外有那些犬类,行邪术的,淫乱的,杀人的,拜偶像的,并一切喜好说谎编造虚谎的。

我耶稣差遣我的使者为众教会将这些事向你们证明。我是大卫的根,又是他的后裔。我是明亮的晨星。

圣灵和新妇都说来。听见的人也该说来。口渴的人也当来。愿意的都可以白白取生命的水喝。

我向一切听见这书上预言的作见证,若有人在这预言上加添什么,神必将在这书上的灾祸加在他身上。

这书上的预言,若有人删去什么,神必从这书上所写的生命树,和圣城,删去他的分。

证明这事的说,是了。我必快来。阿们。主耶稣阿,我愿你来。

愿主耶稣的恩惠,常与众圣徒同在。阿们。

\backmatter
\end{document}